% ======================================================================
% appG_ddw.tex — Appendix G: De Donder–Weyl (multisymplectic field version)
% ======================================================================

\section{De Donder–Weyl (multisymplectic field version)}
\label{app:ddw}

This appendix records the field analogue of the rate (Poincar\'e--Cartan) 1‑form
used in the body for particles (cf.\ \eqref{eq:pc-form}–\eqref{eq:canonical}).
Statements are \emph{Stokes‑first}: the closure of a multisymplectic form yields
the covariant De Donder–Weyl (DDW) Hamilton equations. \emph{No least‑action
postulate is assumed}; sector mappings are specified outside the axioms.

% ----------------------------- G.1 -------------------------------------
\subsection{Fields, polymomenta, and the DW generator}
\label{appG:fields}

Let $\varphi^a(x)$ denote fields on a spacetime manifold $M$ with local
coordinates $x^\mu$ ($\mu=0,\dots,n$) and volume form $d^{\,n+1}\!x$.
Introduce \emph{polymomenta} $P_a^{\ \mu}(x)$ and a DDW ``rate generator''
(also called a covariant Hamiltonian density)
\begin{equation}
\mathcal H(\varphi,P,x)
\;:=\;
P_a^{\ \mu}\,\partial_\mu \varphi^a \;-\; \mathcal R(\varphi,\partial\varphi,x),
\label{appG:eq:DW-H}
\end{equation}
where $\mathcal R$ is a convex rate density in the first derivatives
$\partial_\mu\varphi^a$. The generalized Legendre map
$P_a^{\ \mu}=\partial\mathcal R/\partial(\partial_\mu\varphi^a)$ is assumed
regular on the sector considered.

% ----------------------------- G.2 -------------------------------------
\subsection{Poincar\'e--Cartan (multisymplectic) form}
\label{appG:pc-form}

Let $d^{\,n}x_\mu:=\iota_{\partial_\mu}d^{\,n+1}\!x$ denote the contracted
$n$‑forms. The field Poincar\'e--Cartan $(n{+}1)$‑form is
\begin{equation}
\Theta_{\!\mathrm{DW}}
\;=\;
P_a^{\ \mu}\, d\varphi^a \wedge d^{\,n}x_\mu
\;-\;
\mathcal H(\varphi,P,x)\, d^{\,n+1}\!x.
\label{appG:eq:thetaDW}
\end{equation}
Its exterior derivative $d\Theta_{\!\mathrm{DW}}$ encodes the dynamics.

% ----------------------------- G.3 -------------------------------------
\subsection{Closure and DDW equations from Stokes}
\label{appG:closure}

Evaluating Stokes on any $(n{+}2)$‑chain $\Sigma$ whose boundary lies on two
nearby field histories,
\[
\int_{\partial\Sigma}\Theta_{\!\mathrm{DW}}
\;=\;
\int_{\Sigma} d\Theta_{\!\mathrm{DW}}
\;=\; 0,
\]
and requiring the coefficients of independent variations to vanish yields the
covariant Hamilton (De Donder--Weyl) equations:
\begin{equation}
\partial_\mu \varphi^a
\;=\;
\frac{\partial \mathcal H}{\partial P_a^{\ \mu}},
\qquad
\partial_\mu P_a^{\ \mu}
\;=\;
-\,\frac{\partial \mathcal H}{\partial \varphi^a}.
\label{appG:eq:DW-eqs}
\end{equation}
These are the field counterpart of the canonical rate equations
\eqref{eq:canonical} derived in the particle case.

% ----------------------------- G.4 -------------------------------------
\subsection{Quadratic kinetic example}
\label{appG:quadratic}

With a positive operator $M^{-1\,ab}_{\mu\nu}(x)$ and a scalar rate
field $U(\varphi,x)$, define
\begin{equation}
\mathcal H(\varphi,P,x)
\;=\;
\tfrac12\,P_a^{\ \mu}\,M^{-1\,ab}_{\mu\nu}\,P_b^{\ \nu}
\;+\;
U(\varphi,x).
\label{appG:eq:quad}
\end{equation}
Then \eqref{appG:eq:DW-eqs} gives
$\partial_\mu \varphi^a = M^{-1\,ab}_{\mu\nu} P_b^{\ \nu}$ and
$\partial_\mu P_a^{\ \mu} = -\,\partial U/\partial\varphi^a$.
Sector scales (if any) live in $M^{-1}$ and are specified externally (see App.~\ref{app:calibrations}).

% ----------------------------- G.5 -------------------------------------
\subsection{Minimal coupling to an informational 1‑form (Info–EM)}
\label{appG:min-coupling}

When an informational 1‑form $A_I$ is specified (Info--EM specialization),
minimal coupling enters by
\begin{equation}
P_a^{\ \mu}
\;\longmapsto\;
P_a^{\ \mu}
\;-\;
\mathcal Q_{a}{}^{b}\,A_I^{\mu}\,\varphi_b,
\label{appG:eq:min-coupling}
\end{equation}
with $\mathcal Q$ an operational ``charge'' matrix (sector mapping as in App.~\ref{app:calibrations}).
The multisymplectic form \eqref{appG:eq:thetaDW} remains first‑order; $F_I=dA_I$ appears in $d\Theta_{\!\mathrm{DW}}$.

% ----------------------------- G.6 -------------------------------------
\subsection{Conservation law (multisymplectic current)}
\label{appG:conservation}

If $\mathcal H$ carries no explicit $x^\mu$ dependence in a region, then
$d\Theta_{\!\mathrm{DW}}=0$ implies the closure of the multisymplectic current
there, yielding a local identity for energy–momentum‑like flow. This is the field
analogue of the boundary \emph{cut identity} (cf.\ \eqref{eq:cut-identity}) and echoes the
informational identity \eqref{eq:stokes-identity}.

% ----------------------------- G.7 -------------------------------------
\subsection{Boundary conditions and sources}
\label{appG:boundary-sources}

On a fixed region $U\subset M$ over a time interval $[t_0,t_1]$, source terms
(production) enter by adding an $(n{+}1)$‑form $\Sigma_{\rm src}$ to
$d\Theta_{\!\mathrm{DW}}$:
\begin{equation}
\int_{\partial(U\times[t_0,t_1])}\!\Theta_{\!\mathrm{DW}}
\;=\;
\int_{U\times[t_0,t_1]}\!\Sigma_{\rm src},
\label{appG:eq:DW-source}
\end{equation}
mirroring the informational identity \eqref{eq:stokes-identity}.

% ----------------------------- G.8 -------------------------------------
\subsection{\texorpdfstring{$3{+}1$}{3+1} split and match to the particle case}
\label{appG:split}

Choose $x^0=t$ and write $P_a^{\ 0}=P_a$ and $P_a^{\ i}$ ($i=1,2,3$). For
configurations homogeneous in space, \eqref{appG:eq:DW-eqs} reduces to
\[
\dot\varphi^a=\frac{\partial\mathcal H}{\partial P_a},
\qquad
\dot P_a=-\,\frac{\partial\mathcal H}{\partial\varphi^a},
\]
which are the $(q,P)$ canonical rates \eqref{eq:canonical} under the
identifications $q\!\leftrightarrow\!\varphi$, $P\!\leftrightarrow\!P_a$.

% ----------------------------- G.9 -------------------------------------
\subsection{Dependencies (no unit restatement)}
\label{appG:deps}

The constructions above depend only on: (i) Stokes applied to $\Theta_{\!\mathrm{DW}}$,
(ii) regularity of the Legendre map for the chosen sector, and (iii) the usual differential‑forms
conventions summarized in App.~\ref{app:forms-stokes}. Sector‑specific scales, when used,
are treated externally (App.~\ref{app:calibrations}); no unit policy is restated here.

\medskip
\noindent\textit{Summary.}
The field Poincar\'e--Cartan form $\Theta_{\!\mathrm{DW}}$ closes ($d\Theta_{\!\mathrm{DW}}=0$)
and, by Stokes, generates the covariant De Donder--Weyl equations. Minimal coupling to an
informational 1‑form preserves the first‑order structure. A $3{+}1$ split recovers the canonical
rate equations used in the body.

\medskip
\noindent\emph{Literature note.}
Background on multisymplectic/DDW formulations appears in Kanatchikov~\cite{Kanatchikov1998},
Gotay--Isenberg--Marsden~\cite{GotayIsenbergMarsden1998}, and Marsden--Patrick--Shkoller~\cite{MarsdenPatrickShkoller1998}.
For the particle Poincar\'e--Cartan/Hamiltonian structure see Arnold~\cite{Arnold1989}. General differential‑forms/Stokes
conventions are summarized in App.~\ref{app:forms-stokes}.
