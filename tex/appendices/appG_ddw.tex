% ======================================================================
% appG_ddw.tex — Appendix G: De Donder–Weyl (multisymplectic field version)
% ======================================================================

\section{De Donder–Weyl (multisymplectic field version)}
\label{app:ddw}

This appendix records the field analogue of the rate (Poincar\'e–Cartan) 1‑form
used in the body for particles (cf.\ \eqref{axioms:path-rate:eq:pc-form}–\eqref{axioms:path-rate:eq:canonical}).
Statements are \emph{Stokes‑first}: the closure of a multisymplectic form yields
the covariant De Donder–Weyl (DDW) Hamilton equations. No least‑action premise is required;
sector mappings appear outside the axioms.

% ----------------------------- G.1 -------------------------------------
\subsection{Fields, polymomenta, and the DW generator}
\label{app:ddw:fields}

Let \(\varphi^a(x)\) denote fields on a spacetime manifold \(M\) with local
coordinates \(x^\mu\) (\(\mu=0,\dots,n\)) and volume form \(d^{\,n+1}\!x\).
Introduce \emph{polymomenta} \(P_a^{\ \mu}(x)\) and a DDW generator
(covariant Hamilton density)
\begin{equation}
\mathcal H(\varphi,P,x)
\;:=\;
P_a^{\ \mu}\,\partial_\mu \varphi^a \;-\; \mathcal R(\varphi,\partial\varphi,x),
\label{app:ddw:eq:H}
\end{equation}
where \(\mathcal R\) is a convex rate density in the first derivatives
\(\partial_\mu\varphi^a\).
Assume regularity of the generalized Legendre map
\(P_a^{\ \mu}=\partial\mathcal R/\partial(\partial_\mu\varphi^a)\) on the sector considered.

% ----------------------------- G.2 -------------------------------------
\subsection{Poincar\'e–Cartan (multisymplectic) form}
\label{app:ddw:pc-form}

Let \(d^{\,n}x_\mu:=\iota_{\partial_\mu}d^{\,n+1}\!x\) denote the contracted
\(n\)-forms. The field Poincar\'e–Cartan \((n{+}1)\)-form is
\begin{equation}
\Theta_{\!\mathrm{DW}}
\;=\;
P_a^{\ \mu}\, d\varphi^a \wedge d^{\,n}x_\mu
\;-\;
\mathcal H(\varphi,P,x)\, d^{\,n+1}\!x.
\label{app:ddw:eq:theta}
\end{equation}
Its exterior derivative \(d\Theta_{\!\mathrm{DW}}\) encodes the dynamics.

% ----------------------------- G.3 -------------------------------------
\subsection{Closure and DDW equations from Stokes}
\label{app:ddw:closure}

Evaluating Stokes on any \((n{+}2)\)-chain \(\Sigma\) whose boundary lies on two
nearby field histories,
\[
\int_{\partial\Sigma}\Theta_{\!\mathrm{DW}}
\;=\;
\int_{\Sigma} d\Theta_{\!\mathrm{DW}}
\;=\; 0,
\]
and requiring the coefficients of independent variations to vanish yields the
covariant Hamilton (De Donder–Weyl) equations:
\begin{equation}
\partial_\mu \varphi^a
\;=\;
\frac{\partial \mathcal H}{\partial P_a^{\ \mu}},
\qquad
\partial_\mu P_a^{\ \mu}
\;=\;
-\,\frac{\partial \mathcal H}{\partial \varphi^a}.
\label{app:ddw:eq:dw}
\end{equation}
These are the field counterpart of the canonical rate equations
\eqref{axioms:path-rate:eq:canonical} derived in the particle case.

% ----------------------------- G.4 -------------------------------------
\subsection{Quadratic kinetic example}
\label{app:ddw:quadratic}

With a positive‑definite operator \(M^{-1\,ab}_{\mu\nu}(x)\) and a scalar rate
field \(U(\varphi,x)\), define
\begin{equation}
\mathcal H(\varphi,P,x)
\;=\;
\tfrac12\,P_a^{\ \mu}\,M^{-1\,ab}_{\mu\nu}\,P_b^{\ \nu}
\;+\;
U(\varphi,x).
\label{app:ddw:eq:quad}
\end{equation}
Then \eqref{app:ddw:eq:dw} gives
\(\partial_\mu \varphi^a = M^{-1\,ab}_{\mu\nu} P_b^{\ \nu}\) and
\(\partial_\mu P_a^{\ \mu} = -\,\partial U/\partial\varphi^a\).
Sector scales (if any) live in \(M^{-1}\) and are fixed once by the one‑datum mapping (App.~\ref{app:calibrations}).

% ----------------------------- G.5 -------------------------------------
\subsection{Minimal coupling to an informational 1‑form (Info–EM)}
\label{app:ddw:min-coupling}

When an informational 1‑form \(A_I\) is specified (Info–EM specialization),
minimal coupling enters by
\begin{equation}
P_a^{\ \mu}
\ \longmapsto\
P_a^{\ \mu}
\;-\;
\mathcal Q_{a}{}^{b}\,A_I^{\mu}\,\varphi_b,
\label{app:ddw:eq:min-coupling}
\end{equation}
with \(\mathcal Q\) an operational “charge’’ matrix (sector mapping as in App.~\ref{app:calibrations}).
The multisymplectic form \eqref{app:ddw:eq:theta} remains first‑order; \(F_I=dA_I\) appears in \(d\Theta_{\!\mathrm{DW}}\).
Wave transport and envelopes are centralized in §\ref{waves}; this subsection only records the covariant coupling.

% ----------------------------- G.6 -------------------------------------
\subsection{Conservation law (multisymplectic current)}
\label{app:ddw:conservation}

If \(\mathcal H\) carries no explicit \(x^\mu\) dependence in a region, then
\(d\Theta_{\!\mathrm{DW}}=0\) implies the closure of the associated multisymplectic current,
yielding a local balance identity for energy–momentum‑like flow. This is the field
analogue of the boundary cut identity (cf.\ \eqref{axioms:stokes:identity:eq:cut}) and echoes
the informational continuity \eqref{axioms:stokes:identity:eq}.

% ----------------------------- G.7 -------------------------------------
\subsection{Boundary conditions and sources}
\label{app:ddw:boundary-sources}

On a fixed region \(U\subset M\) over a time interval \([t_0,t_1]\), source terms
(production) enter by adding an \((n{+}1)\)-form \(\Sigma_{\rm src}\) to \(d\Theta_{\!\mathrm{DW}}\):
\begin{equation}
\int_{\partial(U\times[t_0,t_1])}\!\Theta_{\!\mathrm{DW}}
\;=\;
\int_{U\times[t_0,t_1]}\!\Sigma_{\rm src},
\label{app:ddw:eq:source}
\end{equation}
mirroring the informational identity \eqref{axioms:stokes:identity:eq}.

% ----------------------------- G.8 -------------------------------------
\subsection{\texorpdfstring{$3{+}1$}{3+1} split and match to the particle case}
\label{app:ddw:split}

Choose \(x^0=t\) and write \(P_a^{\ 0}=P_a\) and \(P_a^{\ i}\) (\(i=1,2,3\)).
For configurations homogeneous in space, \eqref{app:ddw:eq:dw} reduces to
\[
\dot\varphi^a=\frac{\partial\mathcal H}{\partial P_a},
\qquad
\dot P_a=-\,\frac{\partial\mathcal H}{\partial\varphi^a},
\]
which are the \((q,P)\) canonical rates \eqref{axioms:path-rate:eq:canonical} under the
identifications \(q\!\leftrightarrow\!\varphi\), \(P\!\leftrightarrow\!P_a\).

% ----------------------------- G.9 -------------------------------------
\subsection{Dependencies and summary}
\label{app:ddw:deps}

The constructions above depend only on: (i) Stokes applied to \(\Theta_{\!\mathrm{DW}}\),
(ii) regularity of the Legendre map for the chosen sector, and (iii) the differential‑forms
conventions summarized in App.~\ref{app:forms-stokes}. Sector‑specific scales, when used,
are treated externally (App.~\ref{app:calibrations}); no unit policy is restated here.

\medskip
\noindent\textit{Summary.}
The field Poincar\'e–Cartan form \(\Theta_{\!\mathrm{DW}}\) closes (\(d\Theta_{\!\mathrm{DW}}=0\))
and, by Stokes, generates the covariant De Donder–Weyl equations. Minimal coupling to an
informational 1‑form preserves the first‑order structure. A \(3{+}1\) split recovers the canonical
rate equations used in the body.

\medskip
\noindent\emph{Literature note.}
Multisymplectic/DDW formulations: Kanatchikov~\cite{Kanatchikov1998};
Gotay–Isenberg–Marsden~\cite{GotayIsenbergMarsden1998};
Marsden–Patrick–Shkoller~\cite{MarsdenPatrickShkoller1998}.
Particle Poincar\'e–Cartan/Hamiltonian structure: Arnold~\cite{Arnold1989}.
Forms/Stokes conventions: App.~\ref{app:forms-stokes}.
