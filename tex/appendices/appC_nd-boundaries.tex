% ======================================================================
% appC_nd-boundaries.tex — Appendix C: Boundaries in n dimensions
% ======================================================================

\section{Boundaries in \(n\) dimensions: areas, harmonics, Green functions}
\label{app:nD-boundaries}

This appendix collects geometric identities for \(n\)-dimensional \emph{boundaries} and
their consequences for frame amplitudes under the Stokes identity (\eqref{axioms:stokes:identity:eq})
and the \emph{frame} linear closure (\eqref{axioms:stokes:frame:eq:closure}). All amplitudes here are
informational; measured sectors follow from a one‑datum calibration (App.~\ref{app:kick:calibrations}).

% ----------------------------- C.1 -------------------------------------
\subsection{Sphere areas and ball volumes}
\label{app:nD-boundaries:areas}

Let \(S_{n-1}\) denote the area of the unit \((n{-}1)\)-sphere in \(\mathbb R^{n}\) and
\(B_n(r)\) the ball of radius \(r\). Let
\[
A_n(r)=S_{n-1}\,r^{\,n-1}
\]
denote the area of a radius‑\(r\) \emph{boundary} in \(n\) spatial dimensions. Then
\begin{equation}
S_{n-1}=\frac{2\,\pi^{n/2}}{\Gamma(n/2)}, 
\qquad
A_n(r)=S_{n-1}\,r^{\,n-1},
\qquad
V_n(r)=\frac{\pi^{n/2}}{\Gamma(\tfrac{n}{2}+1)}\,r^{\,n}.
\label{app:nD-boundaries:eq:areas}
\end{equation}
Checks: \(S_0=2\), \(S_1=2\pi\), \(S_2=4\pi\), hence
\(A_1(r)=2\), \(A_2(r)=2\pi r\), \(A_3(r)=4\pi r^2\).

% ----------------------------- C.2 -------------------------------------
\subsection{Radial harmonics and the frame boundary law}
\label{app:nD-boundaries:radial-harmonic}

On a frame with \(\sigma=0\) in an annulus and with the frame closure
\(\mathbf j=-\kappa\,\mathbf E_I\), \(\mathbf E_I:=-\nabla\Phi_I\) (\eqref{axioms:stokes:frame:eq:closure}),
the local identity (\eqref{axioms:stokes:identity:eq}) gives \(\nabla\!\cdot\mathbf j=0\), hence \(\Delta\Phi_I=0\) away from sources.
For radial \(\Phi_I(r)\),
\begin{equation}
\frac{d}{dr}\!\left(r^{\,n-1}\,\frac{d\Phi_I}{dr}\right)=0
\ \Longrightarrow\
\frac{d\Phi_I}{dr}=\frac{C}{r^{\,n-1}},
\qquad
|\mathbf E_I(r)|=\frac{|C|}{r^{\,n-1}} .
\label{app:nD-boundaries:eq:radial-harmonic}
\end{equation}
Let \(\Phi(r):=\int_{\Sigma(r)}\mathbf j\!\cdot n\,dA\) be the directed flux through a spherical boundary \(\Sigma(r)\).
With \(\mathbf j=-\kappa\,\mathbf E_I\),
\begin{equation}
\Phi(r)=-\kappa\,|\mathbf E_I(r)|\,A_n(r)
\ \Rightarrow\
|\mathbf E_I(r)|=\frac{1}{\kappa}\,\frac{|\Phi|}{A_n(r)}
=\frac{|\Phi|}{\kappa\,S_{n-1}}\,\frac{1}{r^{\,n-1}} .
\label{app:nD-boundaries:eq:boundary-law}
\end{equation}
Equation \eqref{app:nD-boundaries:eq:boundary-law} is the full‑boundary version of the cone/tube rule
used in the main text (\S\ref{corollary:info-gas}, \S\ref{corollary:frame-boundary}).

% ----------------------------- C.3 -------------------------------------
\subsection{Green functions of the Laplacian (normalization by \(S_{n-1}\))}
\label{app:nD-boundaries:greens}

Let \(G_n(x)\) solve \(\Delta G_n=\delta\) in \(\mathbb R^n\) with the normalization compatible
with \eqref{app:nD-boundaries:eq:boundary-law}:
\begin{equation}
G_n(x)=
\begin{cases}
-\dfrac{1}{2}\,|x|, & n=1,\\[6pt]
\dfrac{1}{2\pi}\,\ln |x|, & n=2,\\[8pt]
-\dfrac{1}{(n-2)\,S_{n-1}}\,|x|^{\,2-n}, & n\ge 3.
\end{cases}
\label{app:nD-boundaries:eq:greens}
\end{equation}
Then \(\Phi_I=G_n * \rho\) solves \(\Delta\Phi_I=\rho\) and \(\mathbf E_I=-\nabla\Phi_I\).
The factors of \(S_{n-1}\) ensure that integrating \(\nabla\!\cdot\mathbf j=0\) on a ball and using \(A_n(r)\)
reproduces the \(1/r^{\,n-1}\) amplitude.

% ----------------------------- C.4 -------------------------------------
\subsection{Sector (cone) and tube forms}
\label{app:nD-boundaries:sector}

For a conical sector of fixed solid angle \(\Delta\Omega\),
\begin{equation}
A_\Delta(r)=\Delta\Omega\,r^{\,n-1},\qquad
\big\langle j_n\big\rangle(r)=\frac{\Phi_\Delta}{A_\Delta(r)},
\qquad
|\mathbf E_I(r)|=\frac{1}{\kappa}\,\frac{\Phi_\Delta}{\Delta\Omega}\,\frac{1}{r^{\,n-1}} .
\label{app:nD-boundaries:eq:sector}
\end{equation}
For a narrow tube whose side surface is tangent to \(\mathbf j\), the directed tube flux is constant along the tube;
cross‑sectional averages satisfy \(\langle j_n\rangle=\Phi/A_\perp\) with fixed \(A_\perp\).

% ----------------------------- C.5 -------------------------------------
\subsection{Multipoles on \(S^{n-1}\) (outside bounded sources)}
\label{app:nD-boundaries:multipole}

If \(\sigma=0\) outside a bounded region, \(\Phi_I\) is harmonic there and admits the expansion
\begin{equation}
\Phi_I(r,\Omega)=\sum_{\ell=0}^{\infty} r^{-(\ell+n-2)}
\sum_{m} a_{\ell m}\,Y_{\ell m}(\Omega)\qquad (n\ge 3),
\label{app:nD-boundaries:eq:multipole}
\end{equation}
with \(Y_{\ell m}\) the spherical harmonics on \(S^{n-1}\).
The leading term (\(\ell=0\)) reproduces \eqref{app:nD-boundaries:eq:radial-harmonic}; higher multipoles decay
faster and average to zero on full boundaries.

% ----------------------------- C.6 -------------------------------------
\subsection{Notes for \(n=1\) and \(n=2\)}
\label{app:nD-boundaries:notes-n12}

For \(n=2\), harmonic potentials are logarithmic and the frame field decays as \(1/r\), consistent with
\eqref{app:nD-boundaries:eq:boundary-law}. For \(n=1\), the boundary “area’’ is \(A_1(r)=2\) (two points),
so frame amplitudes are \(r\)‑independent along a line/tube. These are the confinement limits discussed in
\S\ref{corollary:dimensional}.

% ----------------------------- C.7 -------------------------------------
\subsection{Conventions and units}
\label{app:nD-boundaries:units}

The results above are informational. Measured sector fields follow from the one‑datum calibrations
\(\mathbf g=c_M\mathbf E_I\), \(\mathbf E=c_Q\mathbf E_I\) (App.~\ref{app:kick:calibrations}). Rates are expressed in Kz (App.~\ref{app:kick}).
No additional constants appear beyond the single calibration per sector.

\medskip
\noindent\emph{Cross‑references.}
Frame boundary laws: \S\ref{corollary:info-gas}, \S\ref{corollary:frame-boundary}. 
Dimensional flows and confinement: \S\ref{corollary:dimensional}.
Wave transport \& envelopes (far field): \S\ref{waves}.

\medskip
\noindent\emph{Literature note.}
Background on differential geometry and forms: Frankel~\cite{Frankel2011}.
Harmonic/potential theory and Green functions: Axler--Bourdon--Ramey~\cite{Axler2001} and Evans~\cite{Evans2010}.
These are provenance pointers; the identities used here follow directly from
\eqref{axioms:stokes:identity:eq} and \eqref{axioms:stokes:frame:eq:closure}, together with the formulas above.
