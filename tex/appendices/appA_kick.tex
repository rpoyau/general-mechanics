% appendices/appA_kick.tex — Units: the Compton energy Kick

\section{Units: The Compton Energy Kick \texorpdfstring{\(\Kk\)}{Kk}}
\label{sec:kick_units}\label{sec:kick}

% ------------------------------------------------------------------
\subsection*{Frequency \(\equiv\) energy}
The mass–energy identity
\[
  E \;=\; m c^{2} \;=\; h\,f_{\mathrm C}(m)
  \tag{2.1}
\]
equates every rest mass with a \emph{frequency}. We therefore \emph{define} the \textbf{Kick}~\cite{Compton1923}:
\[
  1\,\Kk \;\coloneqq\; 1~\text{Hz}.
\]
\noindent\emph{Remark.} Kick is a \textbf{representation} of energy as frequency (Hz). It is independent of the entropy gauge \(C\) (nats/bits/SI).

% ------------------------------------------------------------------
\subsection{Kilogram relegated}\label{subsec:kick-kg}
The SI kilogram is a derived label in Kick units:
\[
  1~\text{kg} \;=\; \frac{c^{2}}{h} \;\Kk
  \;=\;
  1.356392489652\times 10^{50}\;\Kk,
\]
retained for commercial continuity; all calculations may proceed in \(\Kk\).

% ------------------------------------------------------------------
\subsection{Planck frequency: a natural ceiling}\label{subsec:kick-planck}
The Planck tick \(t_{P}\approx 5.391\times 10^{-44}\,\text{s}\) sets a maximal natural frequency:
\[
  f_{P}=t_{P}^{-1}\approx 1.854\times 10^{43}\;\text{Hz}
  \;=\; 1.854\times 10^{43}\;\Kk.
\]

% ------------------------------------------------------------------
\subsection{Prefix ladder}\label{subsec:kick-prefixes}
\[
\begin{array}{@{}llc@{}}
\text{Name} & \text{Symbol} & \text{Definition (Hz)} \\ \hline
\text{Kick}          & \Kk      & 10^{0} \\
\text{Kilokick}      & \text{k}\Kk & 10^{3} \\
\text{Megakick}      & \text{M}\Kk & 10^{6} \\
\text{Gigakick}      & \text{G}\Kk & 10^{9} \\
\text{Terakick}      & \text{T}\Kk & 10^{12} \\
\text{Planck freq.}  & f_{P}    & 1.854\times 10^{43}
\end{array}
\]

% ------------------------------------------------------------------
\subsection{Quick reference}\label{subsec:kick-quick-ref}
\[
\begin{aligned}
f_{\mathrm C}(m_{e}) &\approx 1.2356\times 10^{20}\;\Kk,\\
f_{\mathrm C}(m_{p}) &\approx 2.2687\times 10^{23}\;\Kk,\\
f_{\mathrm C}\bigl(1~\text{g}\bigr) &\approx 1.3564\times 10^{47}\;\Kk.
\end{aligned}
\]

\begin{tcolorbox}[title=Kick–SI round-trip,width=\linewidth]
\begin{enumerate}\itemsep0pt
  \item \textbf{Electron mass.}\quad
        \(m_{e}=1.2356\times 10^{20}\,\Kk
        \;\longrightarrow\;
        9.11\times 10^{-31}\,\text{kg}
        \;\longrightarrow\;
        1.2356\times 10^{20}\,\Kk\).
  \item \textbf{One joule.}\quad
        \(1~\text{J}=\tfrac{1}{h}\,\Kk
        \;\approx\;
        1.50919045\times 10^{33}\,\Kk\);\;
        the inverse map is \(E[\text{J}]=h\,E[\Kk]\).
\end{enumerate}
\end{tcolorbox}

% ------------------------------------------------------------------
\subsection{Temperature in Kick units}\label{subsec:kick-temperature}
Temperature shares the same unit when expressed as a frequency:
\[
  T[\Kk] \;=\; \frac{k_B}{h}\,T[\text{K}]
  \;\approx\; 2.08366191\times 10^{10}\;\Kk/\text{K}\;\times T[\text{K}].
\]
(Thus \(k_B\) and \(h\) appear only in conversions; formulas written entirely in \(\Kk\) need neither.)

% ------------------------------------------------------------------
\paragraph{Practical consequence.}
Representing energy (and temperature) in \(\Kk\):
\begin{itemize}\itemsep3pt
  \item removes \(h\) and \(\hbar\) from expressions written in Kick units; constants appear only in conversions (this appendix);
  \item keeps the speed of light \(c\) explicit as the bridge between space and time;
  \item makes the joule a fixed multiplier: \(1~\text{J}=\tfrac{1}{h}\,\Kk\) (exact by SI definition of \(h\));
  \item allows temperature to use the same unit: \(T[\Kk]=\tfrac{k_B}{h}\,T[\text{K}]\);
  \item lets entropy remain a pure Shannon tally (nats by default; bits add a factor \(\ln 2\)).
\end{itemize}

Thus kilogram, joule, and kelvin become veneers; once stripped away, the thermodynamic
identities (e.g.\ Sec.~\ref{sec:landauer}) reduce to straight arithmetic in \(\Kk\).
