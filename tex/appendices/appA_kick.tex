% ======================================================================
% appA_kz-kram.tex — Appendix A: Kick (Kz) units & Kram anchors
% ======================================================================

\section{Kick (Kz) units \& Kram anchors}
\label{app:kick}

This appendix states the unit convention used here: energy‑like quantities are expressed as \emph{rates}
by dividing SI units by \(h\). This keeps identities and bounds in rate form and avoids introducing
additional constants in intermediate equations. Pointers: boundary geometry (App.~\ref{app:nD-screens});
noise kernels and covariance (App.~\ref{app:noise-kernel}); the one‑datum sector mapping within this appendix
(§\ref{app:kick:calibrations}).

% ---------------------------- A.1 --------------------------------------
\subsection{Kz (Kick) convention}
\label{app:kick:kz-conv}

Define Kz (\emph{Kick}) units by dividing by Planck’s constant \(h\):
\begin{equation}
E^{(\mathrm{Kz})}=\frac{E}{h},\qquad
M^{(\mathrm{Kz})}=\frac{mc^{2}}{h},\qquad
T^{(\mathrm{Kz})}=\frac{k_{B}T}{h},\qquad
P^{(\mathrm{Kz})}=\frac{P_{\rm SI}}{h}.
\label{app:kick:eq:kz-map}
\end{equation}
Thus \(E^{(\mathrm{Kz})}\), \(M^{(\mathrm{Kz})}\), \(T^{(\mathrm{Kz})}\), and \(P^{(\mathrm{Kz})}\) have units of s\(^{-1}\).
In these units, erasure power divided by temperature is a rate, as used in the time–information bounds
(§\ref{corollary:time-info}).

\paragraph{Consistency check.}
In Kz, the Landauer/erasure rate bound for reliable, durable records (trits/s) reads
\begin{equation}
\frac{\Delta I_{\rm trits}}{\Delta t}\ \le\ \frac{P^{(\mathrm{Kz})}}{T^{(\mathrm{Kz})}\,\ln 3}.
\label{app:kick:eq:erasure-rate}
\end{equation}
The same statement in nats/s drops the factor \(\ln 3\).

% ---------------------------- A.2 --------------------------------------
\subsection{Trits, nats, and entropic flux}
\label{app:kick:trits-nats}

If boundary counts are recorded in trits,
\begin{equation}
j_{\text{nats}}=(\ln 3)\,j_{\text{trits}},\qquad
S_{\text{nats}}=(\ln 3)\,S_{\text{trits}}.
\label{app:kick:eq:trits-nats}
\end{equation}
All boundary/tube relations and bounds are invariant under this unit change; only calibration
prefactors shift by \(\ln 3\).

% ---------------------------- A.3 --------------------------------------
\subsection{Kram anchors (optional decimal ladder)}
\label{app:kick:kram}

When magnitudes span many decades, it can be useful to define logarithmic anchors for Kz rates:
\begin{equation}
1\,k=10^{47}\,\mathrm{Kz},\qquad
1\,kk=10^{50}\,\mathrm{Kz},\qquad
1\,Mk=10^{53}\,\mathrm{Kz},\qquad
1\,Gk=10^{56}\,\mathrm{Kz},\qquad
1\,Tk=10^{59}\,\mathrm{Kz}.
\label{app:kick:eq:kram}
\end{equation}

% ---------------------------- A.4 --------------------------------------
\subsection{One‑datum sector calibrations (frame readouts)}
\label{app:kick:calibrations}

The informational field is \(\mathbf E_I:=-\nabla\Phi_I\), with frame linear closure
\(\mathbf j=-\kappa\,\mathbf E_I\) (see \eqref{axioms:stokes:frame:eq:closure}). Measured sectors
use a single linear calibration (in Kz):
\begin{equation}
\mathbf g=c_M\,\mathbf E_I,\qquad
\mathbf E=c_Q\,\mathbf E_I,\qquad
\mathbf j_{\rm th}=\kappa_T\,\mathbf E_I .
\label{app:kick:eq:calib-map}
\end{equation}
Each constant \((c_M,c_Q,\kappa_T)\) is fixed by one datum in that sector.

\paragraph{How one datum fixes the scale.}
On a frame, source‑free annulus the directed flux \(\Phi\) through a radius‑\(r\) boundary is constant and
\begin{equation}
\big\langle j_n\big\rangle(r)=\frac{\Phi}{A_n(r)},\qquad
\big|\mathbf E_I(r)\big|=\frac{1}{\kappa}\,\frac{\Phi}{A_n(r)},\qquad
A_n(r)=S_{n-1}\,r^{\,n-1}\ \ \text{(App.~\ref{app:nD-screens})}.
\label{app:kick:eq:boundary-law}
\end{equation}
Given a reference boundary \(r_\star\) around a source with flux \(\Phi_\star\) and a measured field
\(Y_\star\in\{g_\star,E_\star,j_{{\rm th},\star}\}\),
\begin{equation}
c_{\rm sec}
=\frac{Y_\star}{|\mathbf E_I(r_\star)|}
=Y_\star\,\frac{\kappa\,A_n(r_\star)}{\Phi_\star},
\qquad \text{(sec}\in\{M,Q,{\rm th}\}).
\label{app:kick:eq:one-datum}
\end{equation}
No additional constants appear beyond this datum.

\paragraph{Ratio predictions (constant‑free).}
Once \(c_{\rm sec}\) is fixed,
\begin{equation}
Y(r)=c_{\rm sec}\,|\mathbf E_I(r)|
=Y_\star\,\frac{\Phi(r)}{\Phi_\star}\,\frac{A_n(r_\star)}{A_n(r)}.
\label{app:kick:eq:ratio}
\end{equation}
In lossless frame statics \(\Phi(r)=\Phi_\star\), so \(Y(r)=Y_\star\,(r_\star/r)^{\,n-1}\).

% ---------------------------- A.5 --------------------------------------
\subsection{Capacity and erasure in Kz (operational knobs)}
\label{app:kick:capacity-erasure}

With a channel class and budgets specified, a reliable capacity (trits/s) can be represented as
\begin{equation}
C_{\rm chan}^{(\text{trits})}(\varepsilon)\ =\ \mathcal C\big[R^{(a)};\,\mathcal N,\,\mathcal B;\,\varepsilon\big],
\label{app:kick:eq:capacity}
\end{equation}
where \(R^{(a)}\) are boundary controls (apertures, alphabets), \(\mathcal N\) is a noise kernel
(App.~\ref{app:noise-kernel}), and \(\mathcal B\) collects budgets (power \(P^{(\mathrm{Kz})}\),
temperature \(T^{(\mathrm{Kz})}\), area \(A\), time window \(\Delta t\), reliability \(\varepsilon\)).
The time–information bounds used in the body (§\ref{corollary:time-info}) combine three independent limits:
\begin{equation}
\frac{\Delta I_{\text{trits}}}{\Delta t}
\ \le\
\min\!\left\{
\frac{\overline{J}_{A}}{\ln 3},\quad
C_{\rm chan}^{(\text{trits})}(\varepsilon),\quad
\frac{P^{(\mathrm{Kz})}}{T^{(\mathrm{Kz})}\,\ln 3}
\right\}.
\label{app:kick:eq:time-info}
\end{equation}

% ---------------------------- A.6 --------------------------------------
\subsection{Quick conversions and notes}
\label{app:kick:conversions}

For any SI quantity \(X\), \(X^{(\mathrm{Kz})}=X/h\).
Examples (units shown as s\(^{-1}\)):
\[
E\,[\mathrm{J}]\mapsto E/h,\qquad
E\,[\mathrm{eV}]\mapsto \frac{E\,[\mathrm{eV}]\cdot e}{h},\qquad
T\,[\mathrm{K}]\mapsto \frac{k_B T}{h},\qquad
P\,[\mathrm{W}]\mapsto \frac{P}{h}.
\]
Predictions in the main text are expressed in Kz; numerical evaluations (if any) convert back to SI at the final step.

\medskip
\noindent\emph{Summary.}
Kz expresses energies, masses, temperatures, and powers as rates. A one‑datum, sector‑specific calibration maps
the informational field to measured readouts without introducing additional constants; capacity and erasure appear
as independent operational limits in the same units.

\medskip
\noindent\emph{Literature note.}
Information measures and reliable communication: Shannon and Cover–Thomas~\cite{Shannon1948,CoverThomas2006};
erasure cost and reversibility: Landauer and Bennett~\cite{Landauer1961,Bennett2003}. Pointers to internal material:
Apps.~\ref{app:forms-stokes}, \ref{app:nD-screens}, \ref{app:noise-kernel}.
