% ======================================================================
% appD_jacobson.tex — Appendix D: Local-horizon (Jacobson-style) calibrated sketch
% ======================================================================

\section{Local-horizon (Jacobson-style) calibrated sketch}
\label{app:jacobson}

This appendix records a calibrated derivation that relates local geometric focusing to a stress–geometry
relation used in the main text.

% ----------------------------- D.1 -------------------------------------
\subsection{Setup: local wedge, generators, and kinematics}
\label{app:jacobson:setup}

At any event \(p\in M\), choose a local Rindler wedge with future-directed generators \(k^a\)
(null in the reversible limit, or timelike and highly boosted in practice). Let \(\lambda\) be an affine
parameter along \(k^a\), \(\theta:=\nabla_a k^a\) the expansion, and \(A(\lambda)\) the cross-sectional
area of a thin pencil orthogonal to \(k^a\). The basic identities are
\begin{equation}
\frac{d}{d\lambda}\ln A(\lambda) \;=\; \theta(\lambda),
\qquad
\frac{d\theta}{d\lambda}
\;=\; -\tfrac12\,\theta^2 \;-\; \sigma_{ab}\sigma^{ab} \;-\; R_{ab}\,k^a k^b
\quad(\omega_{ab}=0),
\label{app:jacobson:eq:ray}
\end{equation}
with shear \(\sigma_{ab}\) and vanishing vorticity \(\omega_{ab}\) for hypersurface‑orthogonal generators.
Positive \(R_{ab}k^a k^b\) increases focusing (drives \(\theta\) downward).

% ----------------------------- D.2 -------------------------------------
\subsection{Area change in the reversible limit}
\label{app:jacobson:area}

Take \(\theta(0)=0\) at \(p\) (stationary cross‑section) and neglect \(\sigma_{ab}\sigma^{ab}\) in the
reversible limit. Integrating \eqref{app:jacobson:eq:ray} to first nontrivial order gives
\begin{equation}
\delta A
\;=\;
-\!\int\!\!\int R_{ab}\,k^a k^b \;\lambda\; d\lambda\, dA,
\label{app:jacobson:eq:dA}
\end{equation}
where \(dA\) is the area element on the transverse slice through \(p\).

% ----------------------------- D.3 -------------------------------------
\subsection{Energy (heat) flux across the local horizon}
\label{app:jacobson:flux}

Let \(\mathcal T_{ab}^{(\mathrm{cal})}\) denote the \emph{calibrated} stress tensor of the informational
channel (same one‑datum sector calibration as in the \emph{frame} setting). The energy/heat flux through the horizon patch is
\begin{equation}
\delta Q
\;=\;
\int\!\!\int \mathcal T_{ab}^{(\mathrm{cal})}\,k^a k^b \;\lambda\; d\lambda\, dA .
\label{app:jacobson:eq:dQ}
\end{equation}

% ----------------------------- D.4 -------------------------------------
\subsection{Clausius with calibrated densities and the focusing equality}
\label{app:jacobson:clausius}

Assign a calibrated entropy density \(s_H^{(\mathrm{cal})}\) (nats per area) to the local horizon and a
calibrated temperature \(T_H^{(\mathrm{cal})}\) (Kz) for the wedge. The reversible Clausius statement reads
\begin{equation}
\delta Q \;=\; T_H^{(\mathrm{cal})}\,\delta S,
\qquad
\delta S \;=\; s_H^{(\mathrm{cal})}\,\delta A.
\label{app:jacobson:eq:clausius}
\end{equation}
Combining \eqref{app:jacobson:eq:dA}–\eqref{app:jacobson:eq:clausius} and requiring equality for all patches and for all
generators \(k^a\) through \(p\) yields the local calibrated relation
\begin{equation}
R_{ab}\,k^a k^b
\;\calibeq\;
8\pi\,\mathcal T^{(\mathrm{cal})}_{ab}\,k^a k^b ,
\qquad
\text{(same one‑datum sector calibration as in the \emph{frame} setting).}
\label{app:jacobson:eq:calibrated}
\end{equation}
The numerical factor is a dimensionless normalization absorbed by the one‑datum calibration (choice of
\(s_H^{(\mathrm{cal})}\) and \(T_H^{(\mathrm{cal})}\)).
Equation \eqref{app:jacobson:eq:calibrated} is the focusing equality used in the body
(see \eqref{corollary:geometry:eq:info-einstein}).

% ----------------------------- D.5 -------------------------------------
\subsection{Remarks and scope}
\label{app:jacobson:remarks}

\begin{itemize}\itemsep2pt
\item \textbf{Dissipation.} Away from the reversible limit, shear contributes via
\(-\sigma_{ab}\sigma^{ab}\) in \eqref{app:jacobson:eq:ray}; the corresponding inequality appears as positive
production \(\sigma\ge 0\) in the Stokes identity of the main text (cf. \eqref{axioms:stokes:identity:eq}–\eqref{axioms:stokes:identity:eq:cut}
and App.~\ref{app:forms-stokes}, worldtube equations \eqref{app:forms-stokes:eq:cap-side}–\eqref{app:forms-stokes:eq:cut}).
\item \textbf{Calibration unity.} The same one‑datum calibration that ties \emph{frame} boundary amplitudes to the informational field
is used here to fix \((s_H^{(\mathrm{cal})},T_H^{(\mathrm{cal})})\), so \eqref{app:jacobson:eq:calibrated} introduces no additional constants.
\item \textbf{Use.} The equality \eqref{app:jacobson:eq:calibrated} is applied in \S\ref{corollary:geometry} on the same region and boundary.
\end{itemize}

\medskip
\noindent\emph{Summary.}
Local Clausius, together with Raychaudhuri focusing, yields a calibrated equality between
\(R_{ab}k^a k^b\) and \(\mathcal T^{(\mathrm{cal})}_{ab}k^a k^b\). This mirrors the \emph{frame} boundary laws on the
same region and supports the geometry tools used in \S\ref{corollary:geometry}.

\medskip
\noindent\emph{Literature note.}
Raychaudhuri equation and focusing background: Raychaudhuri~\cite{Raychaudhuri1955}; Wald~\cite{Wald1984}; Poisson~\cite{Poisson2004}.
Local‑horizon Clausius picture: Jacobson~\cite{Jacobson1995}. Unruh temperature for accelerated frames: \cite{Unruh1976}.
These are provenance pointers.
