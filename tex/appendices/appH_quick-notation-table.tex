% ======================================================================
% appendices/appH_notation-table.tex — Notation quick reference (v1.0.2)
% ======================================================================

\section{Notation quick reference}
\label{app:notation-table}

\paragraph{Unit convention.}
All energy-like quantities are expressed in Kick (\(\mathrm{Kz}\)), i.e.\ frequency units:
\(1~\mathrm{Kz}=1~\mathrm{Hz}\). Hence we write \(PV=NT\), \(u=\tfrac{3}{2}nT\), etc.
To revert to SI, multiply by \(h\). Kram anchors (no SI collision):
\(1\,k=10^{47}\,\mathrm{Kz}\), \(1\,kk=10^{50}\,\mathrm{Kz}\), \(1\,Mk=10^{53}\,\mathrm{Kz}\),
\(1\,Gk=10^{56}\,\mathrm{Kz}\), \(1\,Tk=10^{59}\,\mathrm{Kz}\).

% --- Units & anchors
\begin{table}[h]
  \centering
  \renewcommand{\arraystretch}{1.13}
  \begin{tabular}{@{}ll@{}}
    \toprule
    Symbol & Meaning \\
    \midrule
    \(\mathrm{Kz}\) & Kick (frequency unit), \(1~\mathrm{Kz}=1~\mathrm{Hz}\). \\
    \(E/h,\; mc^2/h,\; k_BT/h,\; P/h\) & Energy, rest mass, temperature, pressure in Kz. \\
    \(k,\; kk,\; Mk,\; Gk,\; Tk\) & Kram anchors (fixed decades in Kz; do not append “Kz”). \\
    \bottomrule
  \end{tabular}
\end{table}

% --- Currents, balances, static closure
\begin{table}[h]
  \centering
  \renewcommand{\arraystretch}{1.13}
  \begin{tabular}{@{}ll@{}}
    \toprule
    Symbol & Meaning \\
    \midrule
    \(i,\ \mathbf j\) & Information density and spatial flux. \\
    \(J_I^\mu=(i,\mathbf j)\) & Information 4-current. \\
    \(\sigma\) & Information production (nonnegative). \\
    \(j:=J_{I\mu}dx^\mu,\ \ j_3:=*j\) & Current 1-form and its Hodge dual 3-form. \\
    \(\nabla_\mu J_I^\mu=\sigma\) & Continuity; equivalently \(dj_3=\sigma\,\mathrm{vol}\). \\
    \(\mathbf j=\kappa\,\nabla\Phi_I\) & Static closure (constitutive). \\
    \(\Phi_I,\ \mathbf F_I:=-\nabla\Phi_I\) & Informational potential and field. \\
    \(\dot I_{\mathrm{enc}}\) & Enclosed information rate (Kz). \\
    \(S_{n-1}=\frac{2\pi^{n/2}}{\Gamma(n/2)}\) & Area of the unit \((n{-}1)\)-sphere. \\
    \(\displaystyle |\nabla\Phi_I|=\frac{\dot I_{\mathrm{enc}}}{S_{n-1}\kappa r^{\,n-1}}\) & Static radial law (\(n\)D). \\
    \bottomrule
  \end{tabular}
\end{table}

% --- Worldtube IED (Information Electrodynamics)
\begin{table}[h]
  \centering
  \renewcommand{\arraystretch}{1.13}
  \begin{tabular}{@{}ll@{}}
    \toprule
    Symbol & Meaning \\
    \midrule
    \(\mathcal A_I\) & Information potential 1-form. \\
    \(\mathcal F_I:=d\mathcal A_I\) & Information field 2-form. \\
    \(\mathcal H_I:=\chi\!\cdot\!\mathcal F_I\) & Excitation (medium response \(\chi\)). \\
    \(d\mathcal F_I=0,\ \ d\mathcal H_I=j_3\) & IED field equations (worldtube Stokes). \\
    \(\mathbf E_I,\mathbf B_I,\mathbf D_I,\mathbf H_I\) & \(3{+}1\) split of \(\mathcal F_I,\mathcal H_I\). \\
    \(\mathbf D_I=\varepsilon_I\mathbf E_I,\ \mathbf B_I=\mu_I\mathbf H_I\) & Constitutive laws (isotropic). \\
    \(v_I=\dfrac{c}{\sqrt{\varepsilon_I\mu_I}}\) & Signal speed (causal, \(v_I\le c\)). \\
    \(Z_I=\sqrt{\mu_I/\varepsilon_I}\) & IED impedance. \\
    \(\mathbf S_I=\mathbf E_I\times\mathbf H_I\) & Poynting-type flux; \(u_I=\tfrac12(\mathbf E_I\!\cdot\!\mathbf D_I+\mathbf B_I\!\cdot\!\mathbf H_I)\). \\
    \bottomrule
  \end{tabular}
\end{table}

% --- Calibrations & geometric response
\begin{table}[h]
  \centering
  \renewcommand{\arraystretch}{1.13}
  \begin{tabular}{@{}ll@{}}
    \toprule
    Symbol & Meaning \\
    \midrule
    \(\mathbf g=c_M\,\mathbf F_I,\; \mathbf E=c_Q\,\mathbf F_I,\; j_{\rm th}=\kappa_T\,\mathbf F_I\) & Sector calibrations (gravity/EM/heat). \\
    \(G_{\mu\nu}+\Lambda g_{\mu\nu}=\chi_M T_{\mu\nu}\) & Metric response (info–Einstein). \\
    \(T_{\mu\nu}=(\rho+p/c^2)u_\mu u_\nu + pg_{\mu\nu} + \Pi_{\mu\nu}\) & Stress tensor in Kz. \\
    \(\nabla^2\psi=\kappa_N\,\rho_{\rm eff}\) & Weak-field/Poisson form; \(\mathbf g=-\nabla\psi\). \\
    \(N_{\mu\nu\alpha\beta}\) & Noise kernel (cov. two-point of stress fluctuations). \\
    \(\tau_{\mu\nu}\) & Fluctuating stress with \(\langle\tau\rangle=0\). \\
    \bottomrule
  \end{tabular}
\end{table}

% --- Path-rate mechanics
\begin{table}[h]
  \centering
  \renewcommand{\arraystretch}{1.13}
  \begin{tabular}{@{}ll@{}}
    \toprule
    Symbol & Meaning \\
    \midrule
    \(\mathcal R(q,\dot q,t)\) & Information \emph{rate} (primitive; Kz). \\
    \(P=\partial_{\dot q}\mathcal R\) & Polymomentum (point mechanics). \\
    \(\Phi=P\!\cdot\!\dot q-\mathcal R\) & Convex dual (rate-dual). \\
    \(\Theta_{\mathcal R}=P\!\cdot dq-\Phi\,dt\) & Information Poincaré–Cartan 1-form. \\
    \(\omega=d\Theta_{\mathcal R}\) & Symplectic 2-form on time slices. \\
    \(\dot q=\partial_P\Phi,\ \dot P=-\partial_q\Phi\) & Canonical equations (strip–Stokes). \\
    \(\partial_t S+\Phi(q,\partial_q S,t)=0\) & Hamilton–Jacobi in Kz (information eikonal \(S\)). \\
    \bottomrule
  \end{tabular}
\end{table}

% --- De Donder–Weyl (multisymplectic)
\begin{table}[h]
  \centering
  \renewcommand{\arraystretch}{1.13}
  \begin{tabular}{@{}ll@{}}
    \toprule
    Symbol & Meaning \\
    \midrule
    \(\mathcal R(\phi,\partial\phi,x)\) & Rate density (Kz per volume). \\
    \(P_a^{\ \mu}=\partial\mathcal R/\partial(\partial_\mu\phi^a)\) & Polymomenta. \\
    \(\Phi=P_a^{\ \mu}\partial_\mu\phi^a-\mathcal R\) & Dual density. \\
    \(\Theta=P_a^{\ \mu}d\phi^a\wedge d^{\,n}x_\mu-\Phi\,d^{\,n+1}x\) & Cartan \(n\)-form. \\
    \(\Omega=d\Theta\) & Multisymplectic \((n{+}1)\)-form. \\
    \(\partial_\mu\phi^a=\partial\Phi/\partial P_a^{\ \mu}\) & DDW field equations. \\
    \(\partial_\mu P_a^{\ \mu}=-\partial\Phi/\partial\phi^a\) & (continued). \\
    \(D_\mu \phi^a=\partial_\mu\phi^a-\alpha\,\mathcal A_{I,\mu}\phi^a\) & Minimal coupling to \(\mathcal A_I\). \\
    \(\omega^\mu=\delta P_a^{\ \mu}\wedge\delta \phi^a\) & Multisymplectic current (conserved on shell). \\
    \bottomrule
  \end{tabular}
\end{table}

% --- Geometry & measures
\begin{table}[h]
  \centering
  \renewcommand{\arraystretch}{1.13}
  \begin{tabular}{@{}ll@{}}
    \toprule
    Symbol & Meaning \\
    \midrule
    \(V_n(r)=\dfrac{S_{n-1}}{n}\,r^n\) & Volume of the \(n\)-ball. \\
    \(G_n(\mathbf r)=\dfrac{1}{(n{-}2)S_{n-1}}\,\dfrac{1}{|\mathbf r|^{\,n-2}}\) & Free-space Green’s function (\(n\ge3\)). \\
    \(G_3(\mathbf r)=-\dfrac{1}{4\pi |\mathbf r|}\) & 3D Green’s function (Poisson). \\
    \(\mathrm{vol}=\sqrt{|g|}\,dt\wedge d^n x\) & Volume form (mostly-plus signature). \\
    \(\epsilon_{ijk}\) & Spatial Levi–Civita symbol. \\
    \bottomrule
  \end{tabular}
\end{table}

\paragraph{Reading guide.}
Static boundary laws and radial solutions: App.~\ref{app:d-dim-gauss}; forms/Stokes and worldtube bookkeeping:
App.~\ref{app:forms-stokes}; Kram/Kick units and conversions: App.~\ref{sec:kick}; worldtube IED: \S\ref{sec:worldtube-IED};
geometry response: \S\ref{sec:geometry-response}; path‑rate: \S\ref{sec:path-rate}; time–information bounds and noise:
\S\ref{sec:time-info-bounds}, \S\ref{sec:examples-noise-decoherence}.
