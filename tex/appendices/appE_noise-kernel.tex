% ======================================================================
% appE_noise-kernel.tex — Appendix E: Noise kernel & stochastic response
% ======================================================================

\section{Noise kernel \& stochastic response}
\label{app:noise-kernel}

We collect constant–free formulas for fluctuations about the mean on the same region
\(U\times[t_0,t_1]\): (i) a \emph{noise kernel} for current/stress fluctuations,
(ii) a linear stochastic response (Einstein–Langevin–type, abstracted),
(iii) propagation of covariances via Green functions, and
(iv) a minimal OU/Langevin toy for a scalar readout. Readout mappings to measured sectors
are supplied once in \S\ref{sec:operationalization} and are not repeated here.

% ------------------------- E.1: Noise kernel ---------------------------

\subsection{Noise kernel}
\label{appE:noise}
Decompose a symmetric tensor channel (e.g.\ a stress–like object on \(U\)) into mean and
fluctuation,
\begin{equation}
\mathcal T_{ab}(x)
\;=\;
\big\langle \mathcal T_{ab}(x) \big\rangle \;+\; \tau_{ab}(x),
\qquad
\langle \tau_{ab}\rangle=0 .
\label{appE:eq:stress-split}
\end{equation}
Define the (symmetrized) two–point \emph{noise kernel}
\begin{equation}
N_{ab\,cd}(x,x')
\;:=\;
\tfrac12\,\big\langle \tau_{ab}(x)\,\tau_{cd}(x')+\tau_{cd}(x')\,\tau_{ab}(x) \big\rangle .
\label{appE:eq:N}
\end{equation}
On a boundary patch \(A\subset\partial U\) with outward unit normal \(n\), one may equivalently
use a projected flux–flux kernel \(N_{A}(t,t')\) built from \((\mathbf j\!\cdot n)_-\).

% ------------------------- E.2: Stochastic response --------------------

\subsection{Stochastic linear response (Einstein–Langevin–type)}
\label{appE:EL}
Linearize the channel’s geometric/constitutive response around a background that satisfies the
mean balance (App.~\ref{app:forms-stokes}, \eqref{appB:eq:stokes}–\eqref{appB:eq:local}). Let \(X\) denote
the small response field (e.g.\ a metric–like perturbation \(h_{ab}\), or a scalar potential fluctuation).
A constant–free stochastic equation is
\begin{equation}
\mathcal L[X](x) \;=\; S(x) \;+\; \xi(x),
\qquad
\big\langle \xi(x)\,\xi(x') \big\rangle \;=\; \mathcal N(x,x'),
\label{appE:eq:EL}
\end{equation}
where \(\mathcal L\) is the linearized deterministic operator, \(S\) the mean source (from
\(\langle\mathcal T\rangle\) or mean current divergence), and \(\xi\) is an auxiliary classical field
whose two–point \(\mathcal N\) matches the physical noise kernel (e.g.\ \(\mathcal N\!\equiv\!N\) or an
appropriate projection). No universal coupling is introduced; sector scales, if needed, are handled
once in \S\ref{sec:operationalization}.

% ------------------------- E.3: Covariance propagation -----------------

\subsection{Green functions and covariance propagation}
\label{appE:greens}
Let \(G(x,x')\) be the retarded Green function of \(\mathcal L\),
\(\mathcal L_x\,G(x,x')=\delta(x,x')\).
Then the two–point covariance of \(X\) is
\begin{equation}
\mathrm{Cov}\!\big[X(x),X(y)\big]
\;=\;
\int d^{n+1}x'\!\int d^{n+1}y'\;
G(x,x')\,\mathcal N(x',y')\,G(y,y') .
\label{appE:eq:cov-prop}
\end{equation}
For boundary estimators built from patch integrals (e.g.\ of \(\mathbf j\!\cdot n\)),
\eqref{appE:eq:cov-prop} reduces to patch–filtered versions with the corresponding projected kernel.

% ------------------------- E.4: Screen variance floor ------------------

\subsection{Screen/patch estimator: variance floor}
\label{appE:screen-floor}
Specializing to inward Poisson arrivals across a patch \(A\) over a window \(\Delta t\), the
counting variance implies the local bound used in Corollary~\ref{corollary:noise} (C6),
\begin{equation}
\mathrm{Var}(\widehat{E}_I)
\;\gtrsim\;
\frac{\overline{j}_A}{\kappa^{2}\,|A|\,\Delta t},
\qquad
\overline{j}_A
:=\frac{1}{|A|}\int_{A}(\mathbf j\!\cdot n)_-\,dA ,
\label{appE:eq:EI-bound}
\end{equation}
identical to \eqref{eq:c6-var-EI}. Via the single readout mapping in \S\ref{sec:operationalization},
the same scaling transfers to measured sectors.

% ------------------------- E.5: OU/Langevin toy ------------------------

\subsection{OU/Langevin toy for a scalar readout}
\label{appE:OU}
Let \(Y(t)\) denote a scalar readout along a chosen tube (e.g.\ an acceleration or field
sample). A minimal stationary model is the Ornstein–Uhlenbeck process
\begin{equation}
\dot Y(t) \;=\; -\gamma\big(Y(t)-\bar Y\big)\;+\;\zeta(t),
\qquad
\langle \zeta(t)\rangle=0,\quad
\langle \zeta(t)\,\zeta(t')\rangle=2D_Y\,\delta(t-t').
\label{appE:eq:OU}
\end{equation}
For \(t\gg\gamma^{-1}\),
\(
\mathrm{Var}[Y]=D_Y/\gamma
\)
and the one–sided PSD is
\(
S_Y(\omega)=\frac{2D_Y}{\omega^2+\gamma^2}.
\)
In a shot–noise regime, \(D_Y\) inherits the same \(\propto \overline{j}_A/|A|\) scaling as
\eqref{appE:eq:EI-bound}, up to the (single) sector factor supplied in \S\ref{sec:operationalization}.

% ------------------------- E.6: Correlations & spectra -----------------

\subsection{Correlated arrivals and spectral view}
\label{appE:correlated}
If arrivals exhibit an effective correlation time \(\tau>0\) (dead time, coherence, batching), the
number of independent samples in \(\Delta t\) is \(\approx \Delta t/\tau\), and the floor becomes
\(
\mathrm{Var}(\widehat{E}_I)\gtrsim \frac{\overline{j}_A}{\kappa^2|A|}\,\frac{\tau}{\Delta t}.
\)
Equivalently, with a colored noise kernel \(\mathcal N\), \eqref{appE:eq:cov-prop} yields the spectral
density of \(X\) by replacing \(\mathcal N\) with its spectrum and \(G\) with the transfer function.

% ------------------------- E.8: Energy-rate → acceleration noise --------

\subsection{Energy-rate fluctuations $\Rightarrow$ acceleration noise (static, far field)}
\label{appE:energy-to-accel}

\paragraph{Scope (Stokes + static closure only).}
Work on a time-frozen, source-free annulus around a compact source in $n{=}3$.
By Stokes with \(\sigma=0\) (App.~\ref{app:forms-stokes}, \eqref{appB:eq:stokes}) and the tube/screen argument
(Corollaries~\ref{corollary:info-gas} and \ref{corollary:static-screen}), the directed flux $\Phi$
across any spherical screen of radius $r$ is constant and
\begin{equation}
\big\langle j_n\big\rangle(r)=\frac{\Phi}{A_3(r)},\qquad
|\mathbf E_I(r)|=\frac{1}{\kappa}\,\frac{\Phi}{A_3(r)},\qquad
A_3(r)=4\pi r^2 .
\label{appE:eq:stokes-static}
\end{equation}
Any boundary readout $Y$ proportional to $|\mathbf E_I|$ therefore satisfies $Y(r)\propto \Phi/A_3(r)$.
No bulk Gauss/Poisson postulate is introduced; this is a boundary statement from Stokes.

\paragraph{Energy as a source label.}
For a source family, take the screen flux to carry a linear label with the total source energy,
\begin{equation}
\Phi\ \propto\ E_{\rm src}.
\label{appE:eq:label}
\end{equation}
This is a labeling convention on the family (as in \S\ref{sec:operationalization}); it introduces no new constants.

\paragraph{Fractional mapping (ratio form).}
At fixed geometry $(r,A_3)$, small changes of the source label induce proportional changes of the flux and of any readout:
\begin{equation}
\frac{\delta Y(r,t)}{Y(r)}=\frac{\delta\Phi}{\Phi}
=\frac{\delta E_{\rm src}(t)}{E_{\rm src}} .
\label{appE:eq:ratio}
\end{equation}
For an acceleration readout $Y\equiv g$,
\begin{equation}
\boxed{\ \ \delta g(r,t)=g(r)\,\frac{\delta E_{\rm src}(t)}{E_{\rm src}}\ .\ }
\label{appE:eq:delta-g}
\end{equation}
Because this is a ratio statement, any overall scale (including $\kappa$) cancels.

\paragraph{Spectral form.}
With one-sided PSDs,
\begin{equation}
\boxed{\ \ S_g(\omega)=\left(\frac{g(r)}{E_{\rm src}}\right)^{\!2}\,S_{E}(\omega)\ .\ }
\label{appE:eq:psd}
\end{equation}

\paragraph{Thermal floor (canonical ensemble).}
In equilibrium, \(\mathrm{Var}(E)=k_B T^2 C\) for heat capacity \(C\). Over any band slower than the source’s thermal
relaxation (white \(S_E\) in-band),
\begin{equation}
\boxed{\ \ g_{\mathrm{rms}}(r)\ \approx\ g(r)\,\frac{\sqrt{k_B T^2 C}}{E_{\rm src}}\ .\ }
\label{appE:eq:rms}
\end{equation}
If the source has a single thermal time \(\tau\), replace \(S_E\) by \(S_E/(1+\omega^2\tau^2)\) in \eqref{appE:eq:psd}.

\paragraph{Assumptions.}
(i) Far-field screen outside the source; (ii) quasistatic band: \(\omega\ll c/r\) so wave export (C4, \S\ref{corollary:info-em}) is negligible;
(iii) independent subsystems add their PSDs.

\medskip
\noindent\emph{Remark.}
The mapping \eqref{appE:eq:ratio}–\eqref{appE:eq:rms} is a direct corollary of Stokes on the region and the inverse–area rule
(Corollaries~\ref{corollary:info-gas}–\ref{corollary:static-screen}). (If desired elsewhere, SI forms can be introduced by an external
identification; they are not used here.)

\medskip
\noindent\emph{Literature note.}
The stochastic–response scaffold in \S\ref{appE:EL} follows the Einstein–Langevin program (see Hu–Verdaguer~\cite{HuVerdaguer2008});
Poisson counting and renewal processes background appear in Kingman~\cite{Kingman1993}; Cramér–Rao estimator floors in
Kay and van Trees~\cite{Kay1993,VanTrees2001}; OU/Fokker–Planck tools in Gardiner and Risken~\cite{Gardiner2004,Risken1989}.
The balance/continuity identities used here are recalled in App.~\ref{app:forms-stokes}; geometric screen factors in App.~\ref{app:nD-screens}.
% ======================================================================
