% ======================================================================
% appE_noise-kernel.tex — Appendix E: Noise kernel & stochastic response
% ======================================================================

\section{Noise kernel \& stochastic response}
\label{app:noise-kernel}

This appendix records constant–free formulas for fluctuations about the mean on the same region
\(U\times[t_0,t_1]\): (i) a \emph{noise kernel} for current/stress fluctuations,
(ii) a linear stochastic response (Einstein–Langevin–type),
(iii) propagation of covariances via Green functions, and
(iv) a minimal OU/Langevin toy for a scalar readout.
Readout mappings to measured sectors are stated once in \S\ref{operationalization} and are not repeated here.

% ------------------------- E.1: Noise kernel ---------------------------

\subsection{Noise kernel}
\label{app:noise-kernel:noise}

Decompose a symmetric tensor channel (e.g., a stress–like object on \(U\)) into mean and
fluctuation,
\begin{equation}
\mathcal T_{ab}(x)
\;=\;
\big\langle \mathcal T_{ab}(x) \big\rangle \;+\; \tau_{ab}(x),
\qquad
\langle \tau_{ab}\rangle=0 .
\label{app:noise-kernel:eq:stress-split}
\end{equation}
Define the (symmetrized) two–point \emph{noise kernel}
\begin{equation}
N_{ab\,cd}(x,x')
\;:=\;
\tfrac12\,\big\langle \tau_{ab}(x)\,\tau_{cd}(x')+\tau_{cd}(x')\,\tau_{ab}(x) \big\rangle .
\label{app:noise-kernel:eq:N}
\end{equation}
On a \emph{boundary block} \(A\subset\partial U\) with outward unit normal \(n\), one may equivalently
use a projected flux–flux kernel \(N_{A}(t,t')\) built from \((\mathbf j\!\cdot n)_-\).

% ------------------------- E.2: Stochastic response --------------------

\subsection{Stochastic linear response (Einstein–Langevin–type)}
\label{app:noise-kernel:EL}

Linearize the channel’s geometric/constitutive response around a background that satisfies the
mean continuity (from the Stokes identity; App.~\ref{app:forms-stokes},
\eqref{app:forms-stokes:eq}–\eqref{app:forms-stokes:eq:local}). Let \(X\) denote
the small response field (e.g., a metric–like perturbation \(h_{ab}\), or a scalar potential fluctuation).
A constant–free stochastic equation is
\begin{equation}
\mathcal L[X](x) \;=\; S(x) \;+\; \xi(x),
\qquad
\big\langle \xi(x)\,\xi(x') \big\rangle \;=\; \mathcal N(x,x'),
\label{app:noise-kernel:eq:EL}
\end{equation}
where \(\mathcal L\) is the linearized deterministic operator, \(S\) the mean source (from
\(\langle\mathcal T\rangle\) or mean current divergence), and \(\xi\) is an auxiliary classical field
whose two–point \(\mathcal N\) matches the physical noise kernel (e.g., \(\mathcal N\!\equiv\!N\) or an
appropriate projection). Sector scales, if needed, are handled once in \S\ref{operationalization}.

% ------------------------- E.3: Covariance propagation -----------------

\subsection{Green functions and covariance propagation}
\label{app:noise-kernel:greens}

Let \(G(x,x')\) be the retarded Green function of \(\mathcal L\),
\(\mathcal L_x\,G(x,x')=\delta(x,x')\).
Then the two–point covariance of \(X\) is
\begin{equation}
\mathrm{Cov}\!\big[X(x),X(y)\big]
\;=\;
\int d^{n+1}x'\!\int d^{n+1}y'\;
G(x,x')\,\mathcal N(x',y')\,G(y,y') .
\label{app:noise-kernel:eq:cov-prop}
\end{equation}
For \emph{boundary‑block} estimators built from integrals over \(A\subset\partial U\) (e.g., of \(\mathbf j\!\cdot n\)),
\eqref{app:noise-kernel:eq:cov-prop} reduces to block–filtered versions with the corresponding projected kernel.

% ------------------------- E.4: Boundary variance floor ----------------

\subsection{Boundary estimator (block): variance floor}
\label{app:noise-kernel:boundary-floor}

Specializing to inward Poisson arrivals across a \emph{boundary block} \(A\) over a window \(\Delta t\), the
counting variance implies the local bound used in \S\ref{corollary:noise},
\begin{equation}
\mathrm{Var}(\widehat{E}_I)
\;\gtrsim\;
\frac{F_A\,\overline{j}_A}{\kappa^{2}\,|A|\,\Delta t_{\!\mathrm{eff}}},
\qquad
\overline{j}_A
:=\frac{1}{|A|}\int_{A}(\mathbf j\!\cdot n)_-\,dA ,
\label{app:noise-kernel:eq:EI-bound}
\end{equation}
identical to \eqref{corollary:noise:eq:var-EI}. Shot–noise specialization: \(F_A\to1\) and
\(\Delta t_{\!\mathrm{eff}}\to\Delta t\).
Via the one–datum mapping in \S\ref{operationalization},
the same scaling transfers to measured sectors.

% ------------------------- E.5: OU/Langevin toy ------------------------

\subsection{OU/Langevin toy for a scalar readout}
\label{app:noise-kernel:OU}

Let \(Y(t)\) denote a scalar readout along a chosen tube (e.g., an acceleration or field
sample). A minimal stationary model is the Ornstein–Uhlenbeck process
\begin{equation}
\dot Y(t) \;=\; -\gamma\big(Y(t)-\bar Y\big)\;+\;\zeta(t),
\qquad
\langle \zeta(t)\rangle=0,\quad
\langle \zeta(t)\,\zeta(t')\rangle=2D_Y\,\delta(t-t').
\label{app:noise-kernel:eq:OU}
\end{equation}
For \(t\gg\gamma^{-1}\),
\(
\mathrm{Var}[Y]=D_Y/\gamma
\)
and the one–sided PSD is
\(
S_Y(\omega)=\frac{2D_Y}{\omega^2+\gamma^2}.
\)
In a shot–noise regime, \(D_Y\) inherits the same \(\propto \overline{j}_A/|A|\) scaling as
\eqref{app:noise-kernel:eq:EI-bound}, up to the (single) sector factor supplied in \S\ref{operationalization}.

% ------------------------- E.6: Correlations & spectra -----------------

\subsection{Correlated arrivals and spectral view}
\label{app:noise-kernel:correlated}

If arrivals exhibit an effective correlation time \(\tau>0\) (dead time, coherence, batching), the
number of independent samples in \(\Delta t\) is \(\approx \Delta t/\tau\), and the floor becomes
\(
\mathrm{Var}(\widehat{E}_I)\gtrsim \frac{\overline{j}_A}{\kappa^2|A|}\,\frac{\tau}{\Delta t}.
\)
Equivalently, with a colored noise kernel \(\mathcal N\), \eqref{app:noise-kernel:eq:cov-prop} yields the spectral
density of \(X\) by replacing \(\mathcal N\) with its spectrum and \(G\) with the transfer function.
This matches \eqref{corollary:noise:eq:var-EI} upon identifying \(\Delta t_{\!\mathrm{eff}}\!\sim\!\Delta t/\tau\) up to order‑one factors set by the measured autocorrelation.

% ------------------------- E.7: Energy-rate → acceleration noise --------

\subsection{Energy‑rate fluctuations \(\Rightarrow\) acceleration noise (frame, far field)}
\label{app:noise-kernel:energy-to-accel}

\paragraph{Scope (Stokes + frame linear closure).}
Work on a time‑frozen, source‑free annulus around a compact source in \(n{=}3\).
By Stokes with \(\sigma=0\) (App.~\ref{app:forms-stokes}, \eqref{app:forms-stokes:eq}) and the tube/boundary argument
(\S\ref{corollary:info-gas}, \S\ref{corollary:frame-boundary}), the directed flux \(\Phi\)
across any spherical boundary of radius \(r\) is constant and
\begin{equation}
\big\langle j_n\big\rangle(r)=\frac{\Phi}{A_3(r)},\qquad
|\mathbf E_I(r)|=\frac{1}{\kappa}\,\frac{\Phi}{A_3(r)},\qquad
A_3(r)=4\pi r^2 .
\label{app:noise-kernel:eq:stokes-frame}
\end{equation}
Any boundary readout \(Y\) proportional to \(|\mathbf E_I|\) therefore satisfies \(Y(r)\propto \Phi/A_3(r)\).

\paragraph{Energy as a source label.}
For a source family, take the boundary flux to carry a linear label with the total source energy,
\begin{equation}
\Phi\ \propto\ E_{\rm src}.
\label{app:noise-kernel:eq:label}
\end{equation}
This is a labeling convention on the family (as in \S\ref{operationalization}); it introduces no new constants.

\paragraph{Fractional mapping (ratio form).}
At fixed geometry \((r,A_3)\), small changes of the source label induce proportional changes of the flux and of any readout:
\begin{equation}
\frac{\delta Y(r,t)}{Y(r)}=\frac{\delta\Phi}{\Phi}
=\frac{\delta E_{\rm src}(t)}{E_{\rm src}} .
\label{app:noise-kernel:eq:ratio}
\end{equation}
For an acceleration readout \(Y\equiv g\),
\begin{equation}
\boxed{\ \ \delta g(r,t)=g(r)\,\frac{\delta E_{\rm src}(t)}{E_{\rm src}}\ .\ }
\label{app:noise-kernel:eq:delta-g}
\end{equation}
Because this is a ratio statement, any overall scale (including \(\kappa\)) cancels.

\paragraph{Spectral form.}
With one‑sided PSDs,
\begin{equation}
\boxed{\ \ S_g(\omega)=\left(\frac{g(r)}{E_{\rm src}}\right)^{\!2}\,S_{E}(\omega)\ .\ }
\label{app:noise-kernel:eq:psd}
\end{equation}

\paragraph{Thermal floor (canonical ensemble).}
In equilibrium, \(\mathrm{Var}(E)=k_B T^2 C\) for heat capacity \(C\). Over any band slower than the source’s thermal
relaxation (white \(S_E\) in-band),
\begin{equation}
\boxed{\ \ g_{\mathrm{rms}}(r)\ \approx\ g(r)\,\frac{\sqrt{k_B T^2 C}}{E_{\rm src}}\ .\ }
\label{app:noise-kernel:eq:rms}
\end{equation}
If the source has a single thermal time \(\tau\), replace \(S_E\) by \(S_E/(1+\omega^2\tau^2)\) in \eqref{app:noise-kernel:eq:psd}.

\paragraph{Assumptions.}
(i) Full boundary surrounding the source; (ii) a \emph{frame window} where the W‑FluxConst witness holds (no significant export on the window; cf.\ \S\ref{operationalization}); (iii) independent subsystems add their PSDs.

\medskip
\noindent\emph{Remark.}
The mapping \eqref{app:noise-kernel:eq:ratio}–\eqref{app:noise-kernel:eq:rms} is a direct corollary of Stokes on the region and the inverse–area rule
(\S\ref{corollary:info-gas}, \S\ref{corollary:frame-boundary}). SI forms can be introduced by an external identification if needed.

\medskip
\noindent\emph{Literature note.}
Stochastic response in the Einstein–Langevin sense: Hu–Verdaguer~\cite{HuVerdaguer2008}.
Poisson counting and renewal processes: Kingman~\cite{Kingman1993}.
Stokes/continuity statements used here: App.~\ref{app:forms-stokes}. Boundary geometry factors: App.~\ref{app:nD-boundaries}.
