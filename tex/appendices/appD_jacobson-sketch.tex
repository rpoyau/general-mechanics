% ======================================================================
% appendices/appD_jacobson-sketch.tex — Jacobson-style sketch in Kz (v1.0.2)
% ======================================================================

\section{Jacobson-style local horizon sketch (Kz units)}
\label{app:jacobson-sketch}

This appendix follows the thermodynamic route from a local Rindler horizon to a calibrated
metric response \(G_{\mu\nu}+\Lambda g_{\mu\nu}=\chi_M T_{\mu\nu}\), recast entirely in Kick (Kz) units.
See also \S\ref{sec:geometry-response} for the resulting field equation.

\subsection*{Setup: local Rindler wedge, Unruh temperature in Kz}
Pick any spacetime point \(p\) and a future-directed null direction there. Let \(\ell^\mu\) be the
tangent to the corresponding null geodesics (affine parameter \(\lambda\), with \(\lambda=0\) at \(p\)),
generating a local causal horizon \(\mathcal H\).
Near \(p\), choose the local boost Killing vector
\[
\chi^\mu \;=\; -\,\kappa\,\lambda\,\ell^\mu \,+\,\mathcal O(\lambda^2),
\]
with surface gravity \(\kappa\) (dimension of acceleration). An observer just inside \(\mathcal H\) with proper
acceleration \(a=\kappa\) detects the Unruh temperature
\begin{equation}
T \;=\; \frac{a}{4\pi^2 c},
\label{eq:unruh-kz-appD}
\end{equation}
in Kz (since \(T_{\rm SI}=\hbar a/(2\pi c k_B)\) and \(T=T_{\rm SI}\,k_B/h=\tfrac{a}{4\pi^2 c}\)).

\subsection*{Clausius relation and heat flow across the horizon}
Let \(d\Sigma^\nu=\ell^\nu\,d\lambda\,dA\) be the horizon area–parameter element.
The boost energy flux (“heat’’ in this context) through a small patch of \(\mathcal H\) is
\begin{equation}
\delta Q \;=\; \int_{\mathcal H} T_{\mu\nu}\,\chi^\mu\,d\Sigma^\nu
\;=\; -\,\kappa \int \lambda\, T_{\mu\nu}\,\ell^\mu \ell^\nu\,d\lambda\,dA.
\label{eq:deltaQ}
\end{equation}
Clausius then demands, for a reversible process,
\begin{equation}
\delta Q \;=\; T\,dS,
\label{eq:clausius}
\end{equation}
with \(dS\) the entropy change of the horizon patch (dimensionless; nats by default).

\subsection*{Raychaudhuri and area change}
Let \(\theta(\lambda)\) be the expansion of the null congruence. The Raychaudhuri equation reads
\begin{equation}
\frac{d\theta}{d\lambda} \;=\; -\,\tfrac12 \theta^2 \;-\; \sigma_{\mu\nu}\sigma^{\mu\nu}
\;-\; R_{\mu\nu}\,\ell^\mu\ell^\nu.
\end{equation}
With local equilibrium initial data \(\theta(0)=0=\sigma(0)\), one has, to first order in \(\lambda\),
\begin{equation}
\theta(\lambda)\;=\; -\,\lambda\,R_{\mu\nu}\,\ell^\mu\ell^\nu \;+\; \mathcal O(\lambda^2).
\end{equation}
Hence the horizon area change is
\begin{equation}
\delta A \;=\; \int \theta\,d\lambda\,dA
\;=\; -\,\int \lambda\,R_{\mu\nu}\,\ell^\mu\ell^\nu\,d\lambda\,dA.
\label{eq:deltaA}
\end{equation}

\subsection*{Entropy–area proportionality and the field equation}
Assume an entropy density \(\eta\) per unit area (nats/m\(^2\)):
\begin{equation}
dS \;=\; \eta\,\delta A.
\label{eq:entropy-area}
\end{equation}
Combine \eqref{eq:deltaQ}, \eqref{eq:clausius}, \eqref{eq:deltaA}, and the Unruh temperature \eqref{eq:unruh-kz-appD}.
The common factor \(\int \lambda(\cdot)\,d\lambda\,dA\) cancels, yielding, for \emph{every} null \(\ell^\mu\) at \(p\),
\begin{equation}
T_{\mu\nu}\,\ell^\mu\ell^\nu
\;=\; \frac{\eta}{4\pi^2 c}\,R_{\mu\nu}\,\ell^\mu\ell^\nu.
\end{equation}
By the standard polarization argument (holding for all null \(\ell^\mu\)), there exists a scalar \(\Phi\) such that
\begin{equation}
R_{\mu\nu} + \Phi\,g_{\mu\nu} \;=\; \chi_M\,T_{\mu\nu},
\qquad
\chi_M \;:=\; \frac{4\pi^2 c}{\eta}.
\label{eq:pre-einstein}
\end{equation}
Taking the divergence and using \(\nabla^\mu G_{\mu\nu}=0\) forces \(\Phi=-\tfrac12 R + \Lambda\), with \(\Lambda\) constant,
so \eqref{eq:pre-einstein} is equivalent to the calibrated Einstein equation
\begin{equation}
\boxed{\quad G_{\mu\nu} + \Lambda\,g_{\mu\nu} \;=\; \chi_M\,T_{\mu\nu}. \quad}
\label{eq:einstein-calibrated}
\end{equation}
Thus the metric response constant \(\chi_M\) is \emph{fixed} by the entropy‑per‑area calibration \(\eta\); conversely,
choosing \(\chi_M\) determines \(\eta\). In conventional GR normalization one would have
\(\eta = 4\pi^2 c/\chi_M \to k_B c^3/(4G\hbar)\) (not assumed here).

\subsection*{Remarks}
\begin{enumerate}[itemsep=0.25em,leftmargin=1.3em]
\item \textbf{Kz clarity.} With all energy-like quantities in Kz, \(\delta Q/T=dS\) is dimensionless without extra constants; Unruh is \(T=a/(4\pi^2 c)\).
\item \textbf{Locality.} The derivation is pointwise and uses only (i) local Rindler structure, (ii) Raychaudhuri, (iii) Clausius with horizon entropy density \(\eta\).
\item \textbf{Calibration.} The single empirical knob \(\eta\) (or equivalently \(\chi_M\)) carries all non‑universal information; no \(G\) or \(k_B\) appear in the algebra.
\item \textbf{Scope.} Non‑equilibrium corrections (shear/viscosity terms, entropy production) lead to higher‑derivative or dissipative additions; see \S\ref{sec:geometry-response} and App.~\ref{app:noise-kernel} for stochastic extensions.
\end{enumerate}

% Bibliographic note: This recasts the argument of Jacobson (1995) in Kz units.
