% ======================================================================
% appendices/appG_de-donder-weyl.tex — De Donder–Weyl multisymplectic mechanics (v1.0.2)
% ======================================================================

\section{De Donder--Weyl multisymplectic rate mechanics}
\label{appG_de-donder-weyl}

All energy-like quantities are in Kick (Kz), cf.\ \S\ref{sec:conventions}. A \emph{rate density}
\(\mathcal R(\phi,\partial\phi,x)\) (SI Lagrangian density divided by \(h\)) plays the role of the field-theoretic
primitive; no separate \(L/H\) are assumed. Spacetime is \((n{+}1)\)-dimensional with volume form
\(\mathrm{vol} = \sqrt{|g|}\, \mathrm dx^0\wedge\cdots\wedge\mathrm dx^n\). Define
\( \mathrm d^{\,n}x_\mu := \iota_{\partial_\mu}\mathrm{vol} \) and \( \mathrm d^{\,n+1}x := \mathrm{vol} \).

\subsection*{Polymomenta, dual density, and Cartan form}
Given fields \(\phi^a(x)\) and \(\mathcal R(\phi,\partial_\mu\phi,x)\) in Kz per unit volume, define the polymomenta
\begin{equation}
P^{\ \mu}_a \;:=\; \frac{\partial \mathcal R}{\partial(\partial_\mu \phi^a)} \quad [\text{Kz}], 
\qquad
\Phi(\phi,P,x) \;:=\; P^{\ \mu}_a\,\partial_\mu \phi^a - \mathcal R \quad [\text{Kz}].
\label{eq:ddw-basic}
\end{equation}
The (multisymplectic) Cartan \(n\)-form and \((n{+}1)\)-form are
\begin{equation}
\Theta \;=\; P^{\ \mu}_a\,\mathrm d\phi^a \wedge \mathrm d^{\,n}x_\mu \;-\; \Phi\,\mathrm d^{\,n+1}x,
\qquad
\Omega \;=\; \mathrm d\Theta \;=\; \mathrm dP^{\ \mu}_a \wedge \mathrm d\phi^a \wedge \mathrm d^{\,n}x_\mu \;-\; \mathrm d\Phi \wedge \mathrm d^{\,n+1}x.
\label{eq:ddw-cartan}
\end{equation}
For a variation vector field \(X\) tangent to a solution section, Stokes on a worldtube \(\Omega=V\times[t_0,t_1]\) gives
\(\int_{\partial\Omega} \Theta = \int_{\Omega} \iota_X \Omega\). Stationarity for fixed boundary data yields the DDW equations.

\subsection*{Field equations (De Donder--Weyl)}
The DDW equations are the field‑theoretic analogue of the canonical pair in \S\ref{sec:path-rate}:
\begin{equation}
\partial_\mu \phi^a \;=\; \frac{\partial \Phi}{\partial P^{\ \mu}_a},
\qquad
\partial_\mu P^{\ \mu}_a \;=\; -\,\frac{\partial \Phi}{\partial \phi^a}.
\label{eq:ddw-eq-app}
\end{equation}
Equations \eqref{eq:ddw-eq-app} are equivalent to \( \iota_X \Omega = 0 \) for the evolutionary vector field \(X\).

\subsection*{Noether currents (multisymplectic form)}
Let \(\Xi\) be an infinitesimal symmetry generator with \(\delta\phi^a = \Xi^a(\phi,x)\), \(\delta x^\mu=0\),
and suppose \( \delta \mathcal R = \partial_\mu K^\mu \) (a total divergence).
The associated Noether \(n\)-form current is
\begin{equation}
J_\Xi \;=\; \iota_\Xi \Theta \;-\; K^\mu\,\mathrm d^{\,n}x_\mu,
\qquad
\mathrm d J_\Xi \;\doteq\; 0,
\label{eq:ddw-noether}
\end{equation}
where \( \doteq \) denotes equality on shell. In components, a common form is
\( J^\mu_\Xi = P^{\ \mu}_a\,\Xi^a - K^\mu \), with \( \partial_\mu J^\mu_\Xi \doteq 0 \).
Time translations give an energy current; spatial translations give momentum currents, etc.

\subsection*{Example: real scalar field (Klein--Gordon) in Kz}
Let \(M:=mc^2/h\) be the rest–Kick (Hz) and \(\omega_M := 2\pi M\) the angular Compton frequency.
Take the rate density (Kz/m\(^n\))
\begin{equation}
\mathcal R_{\rm KG}(\phi,\partial \phi) \;=\; 
\frac{\kappa_\phi}{2}\,\bigg( \frac{1}{c^2}(\partial_t \phi)^2 - |\nabla \phi|^2 \bigg) \;-\;
\frac{\kappa_\phi}{2}\,\frac{\omega_M^2}{c^2}\,\phi^2,
\label{eq:Rkg}
\end{equation}
where \(\kappa_\phi>0\) is a (channel) calibration ensuring \(\mathcal R\) is in Kz per volume.
Then
\begin{equation}
P^{\ 0} = \frac{\partial \mathcal R_{\rm KG}}{\partial(\partial_t \phi)} = \kappa_\phi\,\frac{\partial_t \phi}{c^2},
\qquad
P^{\ i} = \frac{\partial \mathcal R_{\rm KG}}{\partial(\partial_i \phi)} = -\,\kappa_\phi\,\partial_i \phi,
\end{equation}
and the dual density is \( \Phi_{\rm KG} = P^{\ \mu}\partial_\mu \phi - \mathcal R_{\rm KG} \).
Equations \eqref{eq:ddw-eq-app} give the KG equation in Kz units,
\begin{equation}
\frac{1}{c^2}\,\partial_t^2 \phi \;-\; \nabla^2 \phi \;+\; \frac{\omega_M^2}{c^2}\,\phi \;=\; 0,
\label{eq:KG-Kz}
\end{equation}
since \( (mc/\hbar)^2 = (\omega_M/c)^2 \). Any other free field is handled analogously by choosing the appropriate \(\mathcal R\).

\subsection*{Minimal coupling to the informational 1-form}
To couple a (charged) field to the informational gauge sector of \S\ref{sec:worldtube-IED},
use the covariant replacement
\begin{equation}
\partial_\mu \phi^a \;\longrightarrow\; D_\mu \phi^a := \partial_\mu \phi^a - \alpha\,\mathcal A_{I,\mu}\,\phi^a,
\label{eq:min-coupling}
\end{equation}
in \(\mathcal R(\phi,\partial\phi,x)\), with a calibration \(\alpha\) (Abelian case shown).
Equivalently, one may shift the Cartan form by \(\alpha\,\mathcal A_I\):
\(
\Theta \mapsto \Theta + \alpha\,\mathcal A_I \wedge \Upsilon
\),
where \(\Upsilon\) is an \( (n{-}1) \)-form whose divergence reproduces the source \(j_3\) in IED.
The DDW equations \eqref{eq:ddw-eq-app} pick up the usual gauge terms, while the IED sector evolves by
\(\mathrm d\mathcal F_I=0,\ \mathrm d\mathcal H_I=j_3\).

\subsection*{Multisymplectic conservation law}
Let \(\delta_1,\delta_2\) be two linearized solutions about a background.
The De Donder–Weyl multisymplectic current
\begin{equation}
\omega^\mu(\delta_1,\delta_2) \;=\;
\delta_1 P^{\ \mu}_a\,\delta_2 \phi^a \;-\; \delta_2 P^{\ \mu}_a\,\delta_1 \phi^a
\label{eq:ms-current}
\end{equation}
is conserved on shell: \( \partial_\mu \omega^\mu = 0 \).
Equivalently, the \(n\)-form \( \iota_{\delta_2}\iota_{\delta_1}\Omega \) is closed: \( \mathrm d(\iota_{\delta_2}\iota_{\delta_1}\Omega)\doteq 0 \).
Integrating \( \omega^\mu \) over a Cauchy hypersurface gives an invariant symplectic structure on the linearized solution space.

\subsection*{Reduction to point mechanics}
For spatially homogeneous fields on a domain \(V\) (or after Galerkin truncation),
define collective coordinates \(q^a(t):=|V|^{-1}\int_V \phi^a\,\mathrm d^n x\) and momenta
\(P_a(t):=\int_V P^{\ 0}_a\,\mathrm d^n x\).
The effective rate is \(\mathcal R_{\rm eff}(q,\dot q,t)=\int_V \mathcal R(\phi,\partial\phi)\,\mathrm d^n x\),
and the Cartan 1-form is precisely the path‑rate \(\Theta_{\mathcal R}\) of \S\ref{sec:path-rate}.
Thus point‑mechanics \(\leftrightarrow\) field‑mechanics are connected by spatial integration.

\paragraph{Remarks.}
(1) The DDW formalism shows that our “no \(L/H\) assumption’’ carries seamlessly to fields: the only primitives are \(\mathcal R\) and its convex dual \(\Phi\).  
(2) Kz units make mass parameters appear as \(\omega_M=2\pi M\), i.e.\ as pure frequencies, keeping dimension book‑keeping transparent.  
(3) Gauge couplings to the informational sector are natural via \eqref{eq:min-coupling}; the worldtube IED equations then carry the dynamics of “surprise’’ with causal speed \(v_I\).
