% ======================================================================
% appendices/appB_forms-stokes.tex — Differential forms & Stokes (v1.0.2)
% ======================================================================

\section{Differential forms and Stokes conventions}
\label{app:forms-stokes}

\paragraph{Manifold, metric, orientation.}
Work on an oriented Lorentzian manifold \((\mathcal M^{n+1},g)\) with \emph{mostly-plus} signature \((-,+,\dots,+)\).
Choose coordinates \((x^\mu)=(t,x^1,\dots,x^n)\) with orientation
\[
\mathrm{vol}:=\sqrt{|g|}\;\mathrm dt\wedge\mathrm dx^1\wedge\cdots\wedge \mathrm dx^n.
\]
For a \(p\)-form \(\alpha=\tfrac1{p!}\alpha_{\mu_1\cdots\mu_p}\,\mathrm dx^{\mu_1}\!\wedge\!\cdots\!\wedge\!\mathrm dx^{\mu_p}\),
the pointwise inner product is
\(\langle \alpha,\beta\rangle=\tfrac1{p!}\alpha_{\mu_1\cdots\mu_p}\beta^{\mu_1\cdots\mu_p}\).

\paragraph{Hodge star \(*\).}
The Hodge dual \(*:\Lambda^p\to\Lambda^{n+1-p}\) is defined by
\[
\alpha\wedge *\beta = \langle \alpha,\beta\rangle\,\mathrm{vol}\qquad(\alpha,\beta\in\Lambda^p).
\]
We will not need explicit \(*^2\) signs; all identities below are written without squaring \(*\).
In \(3{+}1\) with our orientation one has, in components,
\[
*(\mathrm dt)=\sqrt{|g|}\,\mathrm dx^1\wedge\mathrm dx^2\wedge\mathrm dx^3,\qquad
*(\mathrm dx^i)= -\,\sqrt{|g|}\,\mathrm dt\wedge\epsilon_{ijk}\,\mathrm dx^j\wedge\mathrm dx^k/2,
\]
and similarly for higher degrees (these are sufficient to recover the \(3{+}1\) split below).

\paragraph{Exterior derivative and Stokes.}
The exterior derivative \(\mathrm d:\Lambda^p\to\Lambda^{p+1}\) obeys \(\mathrm d(\alpha\wedge\beta)=\mathrm d\alpha\wedge\beta+(-1)^p\alpha\wedge\mathrm d\beta\).
For any smooth region \(\Omega\subset\mathcal M\) with smooth boundary \(\partial\Omega\),
\[
\boxed{\quad \displaystyle \int_{\partial\Omega}\omega = \int_{\Omega}\mathrm d\omega\qquad(\omega\in\Lambda^p). \quad}
\]

\paragraph{Worldtube cap–side decomposition.}
Let \(\Omega=V\times[t_0,t_1]\) be a fixed worldtube with \(V\subset\mathbb R^n\).
With the standard product orientation,
\[
\partial\Omega \;=\; \underbrace{(V\times\{t_1\})}_{\text{top cap}}
\;\;\sqcup\;\; \underbrace{-(V\times\{t_0\})}_{\text{bottom cap}}
\;\;\sqcup\;\; \underbrace{(\partial V)\times[t_0,t_1]}_{\text{lateral side}}.
\]
Thus any Stokes identity over \(\Omega\) splits into “cap” and “side” contributions, which we used for the Poynting‑type balance.

\subsection*{Continuity as a 3‑form identity}
Let \(J_I^\mu=(i,\mathbf j)\) be the information 4‑current and \(j:=J_{I\mu}\,\mathrm dx^\mu\) the associated \(1\)-form.
Define the dual \(3\)-form
\[
j_3:=*j.
\]
Then the continuity equation with production \(\sigma\) is exactly
\[
\boxed{\quad \nabla_\mu J_I^\mu=\sigma \qquad\Longleftrightarrow\qquad \mathrm d j_3=\sigma\,\mathrm{vol}. \quad}
\]
Integrating over a worldtube \(\Omega=V\times[t_0,t_1]\) gives the master balance
\[
\int_{\partial\Omega}\!\!*j=\int_{\Omega}\sigma\,\mathrm d^{\,n+1}x
\quad\Longleftrightarrow\quad
\int_{V_{t_1}}\!\!*j-\int_{V_{t_0}}\!\!*j+\int_{\partial V\times[t_0,t_1]}\!\!*j=\int_{\Omega}\sigma\,\mathrm{vol}.
\]
In a static snapshot (\(\partial_t i=0\)) with \(\sigma\) time‑independent, this reduces to the Gauss law used in \S\ref{sec:static-gauss-nd}.

\subsection*{Information electrodynamics (IED) in forms}
Introduce the information potential \(1\)-form \(\mathcal A_I\), field \(2\)-form \(\mathcal F_I:=\mathrm d\mathcal A_I\), and excitation \(2\)-form \(\mathcal H_I:=\chi\!\cdot\!\mathcal F_I\) (linear medium map \(\chi\) in the simplest cases).
IED is the pair
\[
\boxed{\quad \mathrm d\mathcal F_I=0,\qquad \mathrm d\mathcal H_I=j_3. \quad}
\]
In a \(3{+}1\) split with \(\mathcal A_I=-\phi_I\,\mathrm dt+\mathbf A_I\cdot\mathrm d\mathbf x\),
\[
\mathcal F_I=\mathbf E_I\cdot \mathrm dt\wedge\mathrm d\mathbf x
+\mathbf B_I\cdot \tfrac12\,\epsilon_{ijk}\,\mathrm dx^i\wedge\mathrm dx^j,
\]
and writing \(\mathcal H_I=\mathbf D_I\cdot \mathrm d\mathbf x\wedge\mathrm dt
+\mathbf H_I\cdot \tfrac12\,\epsilon_{ijk}\,\mathrm dx^i\wedge\mathrm dx^j\),
the equations become the Maxwell‑type system
\[
\nabla\cdot \mathbf D_I=i,\qquad \nabla\cdot \mathbf B_I=0,\qquad
\nabla\times \mathbf H_I=\mathbf j+\partial_t\mathbf D_I,\qquad
\nabla\times \mathbf E_I=-\partial_t\mathbf B_I,
\]
used in \S\ref{sec:worldtube-IED}. The static limit with \(\mathbf j=0\) recovers \(\mathbf E_I=-\nabla\Phi_I\) and the Laplace/Poisson forms of \S\ref{sec:static-gauss-nd}.

\subsection*{Variation strip for the information 1‑form}
Let \(\Theta\) be any \(1\)-form on the extended space \((q,t)\), and let \(\Sigma\) be the \emph{variation strip} swept by a 1‑parameter family of nearby curves between fixed endpoints at \(t_0,t_1\).
Stokes on \(\Sigma\) gives
\[
\int_{\partial\Sigma}\Theta=\int_{\Sigma}\mathrm d\Theta.
\]
Taking \(\Theta=\Theta_{\mathcal R}=P\!\cdot\mathrm dq-\Phi\,\mathrm dt\) with \(\mathcal R(q,\dot q,t)\) convex in \(\dot q\), one finds
\[
\mathrm d\Theta_{\mathcal R}= \mathrm dP\wedge \mathrm dq - \mathrm d\Phi\wedge \mathrm dt
\quad\Rightarrow\quad
\iota_X(\mathrm d\Theta_{\mathcal R})=0 \;\;\Longleftrightarrow\;\; \dot q=\partial_P\Phi,\ \dot P=-\partial_q\Phi,
\]
the canonical equations of \S\ref{sec:path-rate}. Here \(X=\partial_t+\dot q^i\partial_{q^i}+\dot P_i\partial_{P_i}\) is the strip tangent.
No action functional is assumed; the result is an immediate corollary of boundary calculus.

\paragraph{Remarks.}
(1) All identities above are \((n{+}1)\)-covariant and metric‑compatible; only the choice of \(\chi\) (constitutive law) and calibrations to measured sectors are empirical.  
(2) The worldtube cap–side split is the bookkeeping device behind our fixed‑boundary balances (e.g.\ the Poynting‑type identity).  
(3) The same machinery applies to other conserved/distinguished currents by replacing \(j\) and \(\chi\) accordingly.
