% ======================================================================
% appB_forms-stokes.tex — Appendix B: Differential forms & generalized Stokes
% ======================================================================

\section{Differential forms \& generalized Stokes}
\label{app:forms-stokes}

\subsection{Manifolds, orientation, and volume forms}
\label{appB:manifolds}
Let $M$ be an $(n{+}1)$–dimensional oriented manifold with local coordinates $(x^1,\dots,x^n,t)$.
Fix a region $U\subset M$ with smooth spatial slices $V_t:=U\cap\{t=\mathrm{const}\}$ whose boundary
is $\partial V_t$. The product region $U_{[t_0,t_1]}:=U\cap\{t\in[t_0,t_1]\}$ has oriented boundary
\[
\partial U_{[t_0,t_1]}
=\underbrace{V_{t_1}}_{\text{top cap}}
\;\cup\;
\underbrace{(-V_{t_0})}_{\text{bottom cap}}
\;\cup\;
\underbrace{\big(\partial V\times[t_0,t_1]\big)}_{\text{side}}.
\]
Write $\mathrm{vol}_{n}$ for the $n$–volume form on $V_t$ and $\mathrm{vol}_{n+1}$ for the $(n{+}1)$–volume
form on $U$ induced by the orientation.

\subsection{Current 1–form, Hodge dual, and production}
\label{appB:current}
Let $j:=J^\mu dx_\mu$ denote the informational/entropic current $1$–form on $U$ with components
$J^\mu=(i,\mathbf j)$ in adapted coordinates. Its Hodge dual $j^{(n)}:=\!*j$ is an $n$–form that
integrates naturally over spatial caps and side surfaces.\footnote{Legacy sections may write
$j^{(3)}$ for the dual $n$–form. Here we use the dimension‑agnostic $j^{(n)}:=*j$.}
Production is the $(n{+}1)$–form $\Sigma:=\sigma\,\mathrm{vol}_{n+1}$ with nonnegative density $\sigma$
in the sector where dissipation is present.

\subsection{Global balance (generalized Stokes)}
\label{appB:global-balance}
The Stokes balance is the exterior–derivative statement
\begin{equation}
d\,j^{(n)}=\Sigma
\quad\Longleftrightarrow\quad
\partial_\mu J^\mu=\sigma .
\label{appB:eq:stokes}
\end{equation}
% ---- Legacy alias for backward compatibility (no duplicate labels in the same equation)
\begingroup
\makeatletter
\edef\@currentlabel{\theequation}%
\makeatother
\label{eq:stokes-balance}
\endgroup
% ----------------------------------------------------------------------

Equivalently, for any product region $U_{[t_0,t_1]}$,
\begin{equation}
\int_{\partial U_{[t_0,t_1]}} j^{(n)} \;=\; \int_{U_{[t_0,t_1]}} \Sigma .
\label{appB:eq:stokes-int}
\end{equation}

\subsection{Cap/side decomposition and the cut balance}
\label{appB:cap-side}
Using the decomposition of $\partial U_{[t_0,t_1]}$,
\begin{align}
\int_{\partial U_{[t_0,t_1]}}\!\! j^{(n)}
&=
\underbrace{\int_{V_{t_1}}\! j^{(n)}}_{\text{top cap}}
-
\underbrace{\int_{V_{t_0}}\! j^{(n)}}_{\text{bottom cap}}
+
\underbrace{\int_{\partial V\times[t_0,t_1]}\!\! j^{(n)}}_{\text{side}}
\label{appB:eq:cap-side}
\end{align}
with
\[
I_{\mathrm{enc}}(t):=\int_{V_t} i\,\mathrm{vol}_{n},\qquad
\Phi_{\mathrm{in/out}}(t):=\int_{\partial V_t}(\mathbf j\!\cdot n)_{\mp}\,dA,
\qquad
\Pi(t):=\int_{V_t}\sigma\,d^n x.
\]
Differentiating \eqref{appB:eq:stokes-int} in $t$ yields the \emph{cut balance}
\begin{equation}
\dot I_{\mathrm{enc}}(t)\;=\;\Phi_{\mathrm{in}}(t)-\Phi_{\mathrm{out}}(t)\;+\;\Pi(t),
\label{appB:eq:cut}
\end{equation}
identical to \eqref{eq:cut-balance} in the main text.

\subsection{Local form (continuity)}
\label{appB:local}
In adapted coordinates $(x,t)$ the differential statement \eqref{appB:eq:stokes} becomes
\begin{equation}
\partial_t i(x,t)\;+\;\nabla\!\cdot \mathbf j(x,t)\;=\;\sigma(x,t),
\label{appB:eq:local}
\end{equation}
the usual $(n{+}1)$–dimensional continuity law (cf.\ \eqref{eq:local-continuity}).

\subsection{Static snapshot and tube/screen invariants}
\label{appB:static}
For a time–frozen snapshot with $\sigma=0$ in a region, $d j^{(n)}=0$ implies that the flux through any two
homologous screens is equal. If $\mathcal T$ is a narrow tube whose side surface is tangent to $j$, then the
directed tube flux $\Phi=\int_{\Sigma}j\!\cdot n\,dA$ is constant along the tube and cross–sectional averages obey
\[
\langle j_n\rangle=\frac{\Phi}{A}.
\]
With the static linear closure $\mathbf E_I:=-\nabla\Phi_I$ and $\mathbf j=-\kappa\,\mathbf E_I$
(\eqref{eq:static-closure}), amplitudes follow the inverse–area rule used in C0–C1
(see \S\ref{corollary:info-gas} and \S\ref{corollary:static-screen}).

\subsection{Optional specialization: informational electrodynamics and Poynting export}
\label{appB:poynting}
On segments that support wave‑like transport, adopt the informational electrodynamics (Info–EM) closure
\[
F_I=dA_I,\qquad dF_I=0,\qquad dH_I={*}j,\qquad H_I=\chi:F_I,
\]
where $A_I$ is a $1$‑form potential, $F_I$ the $2$‑form field, $H_I$ the excitation $2$‑form via a (linear, local)
constitutive map $\chi$ on the segment considered. \emph{Causality of $\chi$ implies Kramers–Kronig dispersion relations;}
see \S\ref{appB:KK}, eq.~\eqref{appB:eq:KK}.

The informational energy–rate density and Poynting vector are
\[
u_I=\tfrac12\,F_I\!:\!H_I,
\qquad
\mathbf S_I\ \text{(spatial Poynting vector)},
\]
and they satisfy the Poynting identity on any spatial slice $V_t$ with outward normal $n$:
\begin{equation}
\frac{d}{dt}\int_{V_t} u_I\,d^n x\;+\;\int_{\partial V_t}\mathbf S_I\!\cdot n\,dA
\;=\; -\int_{V_t} \mathbf j\!\cdot\!\mathbf E_I\,d^n x .
\label{appB:eq:poynting}
\end{equation}
% ---- Legacy alias for backward compatibility (avoid duplicate label in the same equation)
\begingroup
\makeatletter
\edef\@currentlabel{\theequation}%
\makeatother
\label{eq:poynting}
\endgroup
% ----------------------------------------------------------------------

In a source‑free subregion ($\mathbf j=0$), the surface term equals the negative time‑rate of change of field energy
in $V_t$, so exported power across homologous screens is equal. This statement underlies the far‑field envelope laws
used in the wave corollaries, in particular C4 (\S\ref{corollary:info-em}).

\subsection{Causality note (Kramers--Kronig for $\chi$)}
\label{appB:KK}
For linear, time–invariant media the (temporal) Fourier transform $\chi(\omega)$ obeys Hilbert–transform relations
when $\chi$ is causal:
\begin{align}
\Re\,\chi(\omega)
&=\frac{1}{\pi}\,\mathrm{P}\!\!\int_{-\infty}^{\infty}\frac{\Im\,\chi(\omega')}{\omega'-\omega}\,d\omega',
\qquad
\Im\,\chi(\omega)
=-\frac{1}{\pi}\,\mathrm{P}\!\!\int_{-\infty}^{\infty}\frac{\Re\,\chi(\omega')}{\omega'-\omega}\,d\omega' .
\label{appB:eq:KK}
\end{align}
This provides a consistency check for any chosen $\chi$ in Info–EM specializations.

\subsection{Sign conventions and directional split}
\label{appB:signs}
Reversing the orientation of $V_t$ or the time direction flips signs consistently in
\eqref{appB:eq:cap-side}–\eqref{appB:eq:local}. The inward/outward split is defined by
$(a)_\pm:=\tfrac12\,(a\pm|a|)$ and is independent of those choices.

\subsection{Summary}
\label{appB:summary}
With $j$ a $1$–form and $j^{(n)}=\!*j$, the single identity $d j^{(n)}=\sigma\,\mathrm{vol}_{n+1}$ yields:
(i) the local continuity equation \eqref{appB:eq:local}, and (ii) the cut balance \eqref{appB:eq:cut} from the
cap/side decomposition of $\partial U_{[t_0,t_1]}$. Static inverse–area rules (C0–C1) and dynamic Poynting export (C4)
used in the main text are direct applications once closures are specified.

\medskip
\noindent\emph{Literature note.}
Background on differential forms/Stokes appears in Frankel~\cite{Frankel2011};
a standard EM Poynting treatment is Jackson~\cite{Jackson1999} or Landau–Lifshitz–Pitaevskii~\cite{LandauLifshitzEDCM1984};
dispersion/causality background is Nussenzveig~\cite{Nussenzveig1972}.
