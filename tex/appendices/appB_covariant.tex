% appendices/appB_covariant.tex  —  Covariant bookkeeping on (M,g)

\section{Covariant Formulation}
\label{app:covariant}

\subsection*{B.1  Space–time background}

Let \((M^{n+1},g_{\mu\nu})\) be a smooth Lorentz (or Riemann) manifold with Levi-Civita connection \(\nabla_\mu\) and volume form \(\mathrm d^{\,n+1}x\sqrt{|g|}\).  
Indices are raised/lowered with \(g_{\mu\nu}\) and its inverse \(g^{\mu\nu}\).

\subsection*{B.2  Informational current}

\[
J^{\mu} = (\rho,\mathbf J),
\qquad
\mathcal J = \star (J^{\mu}\mathrm d x_{\mu}),
\qquad
\nabla_{[\alpha}\mathcal J_{\beta_1\!\dots\beta_n]} = 0.
\]

Covariant continuity:

\[
\nabla_{\mu} J^{\mu} = \sigma.
\tag{B.1}
\]

\subsection*{B.3  Differential REOS}

\[
\boxed{%
\nabla_{\mu}\rho
   \;=\;
   -\Lambda_{a}\,\nabla_{\mu} R^{(a)}
   - R^{(a)}\,\nabla_{\mu}\Lambda_{a}.
}
\tag{B.2}
\]

\subsection*{B.4  Informational force}

\[
f_{\mu}
  = -\,R^{(a)}\,\nabla_{\mu}\Lambda_{a},
\qquad
\nabla^{\mu}T_{\mu\nu}=f_{\nu},
\]
where \(T_{\mu\nu}=A_{\mu\alpha}J^{\alpha}J_{\nu}-g_{\mu\nu}\mathcal L\) is the information–stress tensor and \(A_{\mu\nu}\) is any positive-definite mobility tensor.

\subsection*{B.5  Resource–clock bound}

For each resource \(R^{(a)}\) define a conjugate angular variable
\(\Theta_{(a)}\in S^{1}\) such that \((R^{(a)},\Theta_{(a)})\) is canonical with respect to the extended symplectic form

\[
\Omega = \mathrm dJ^{\mu}\!\wedge\! \mathrm d x_{\mu}
         + \sum_{a}\mathrm d R^{(a)}\!\wedge\! \mathrm d\Theta_{(a)} .
\]

The Cramér–Rao inequality on \((\widehat M, \Omega)\) gives

\[
\boxed{%
\Delta R^{(a)}\,\Delta\Theta_{(a)}
   \;\gtrsim\;
   \frac{1}{2\ln 2}\ .
}
\tag{B.3}
\]

\subsection*{B.6  Stokes on extended tubes}

For any codimension-1 hypersurface \(\widehat\Sigma\subset\widehat M\)
with interior \(\widehat V\),

\[
\int_{\widehat\Sigma}\!\widehat{\mathcal J}
  \;=\;
  \int_{\widehat V}\!\widehat{\sigma},
\qquad
\widehat{\mathcal J}
  = \star_{\widehat M}\!(J^{\mu},J^{\Theta_{(a)}})\mathrm dX_{\mu},
\]
where \(J^{\Theta_{(a)}}=-\Lambda_{a}\).  
Projection onto ordinary space–time recovers Eq.\,(B.1); projection onto any resource–clock plane recovers Eq.\,(B.3).

\paragraph{Summary.}
The four Tier-0 pillars (continuity, differential REOS, informational force, least-action) retain their boxed form under arbitrary curvature and for an arbitrarily large resource list \(\{R^{(a)}\}\); only the Hodge-star and covariant derivatives pick up the metric.  All Tier-1 corollaries therefore hold on curved space–time provided local clocks and mobility tensors are chosen consistently.
