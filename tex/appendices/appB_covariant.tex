% appendices/appB_covariant.tex  —  Covariant bookkeeping on (M,g)

\section{Covariant Formulation}
\label{app:covariant}

\subsection{Corollary: Einstein with \texorpdfstring{\(\kappa\)}{kappa} and informational stress}

\paragraph{Direct GR form (null-tube corollary).}
Starting from REOS + clock bound + Clausius,
\[
R_{\mu\nu}-\tfrac12 R\,g_{\mu\nu}
\;=\;
8\pi G\,T_{\mu\nu}
\;+\;
\underbrace{(\nabla_\mu\nabla_\nu - g_{\mu\nu}\,\Box)\,\ln \kappa}_{\text{clock-bound deviation}}
\;+\;
8\pi G\,T^{\mathrm{info}}_{\mu\nu}[\Lambda_E],
\]
with $\kappa(x)$ the local deviation from exact $\Delta E\,\Delta t = 1/(2\ln 2)$
and $T^{\mathrm{info}}_{\mu\nu}$ built from the informational force
$f_\mu = -\,R^{(E)}\,\nabla_\mu \Lambda_E$.
For $\kappa\!\to\!1$ and $\nabla\Lambda_E\!\to\!0$ the pure Einstein equation is recovered.

\paragraph{Conservation check.}
Taking $\nabla^\mu$ of
$G_{\mu\nu}=8\pi G\,(T_{\mu\nu}+T^{\mathrm{info}}_{\mu\nu})+(\nabla_\mu\nabla_\nu-g_{\mu\nu}\Box)\ln\kappa$
and using $\nabla^\mu G_{\mu\nu}=0$ gives
\[
8\pi G\,\nabla^\mu\!\bigl(T_{\mu\nu}+T^{\mathrm{info}}_{\mu\nu}\bigr)
= -\,R_{\nu}{}^{\mu}\,\nabla_\mu \ln\kappa .
\]
With $f_\nu=-\,R^{(E)}\nabla_\nu\Lambda_E$ acting on matter, the informational field stress obeys the EM‑style balance
\[
\nabla^\mu T^{\mathrm{info}}_{\mu\nu} = -\,f_\nu = +\,R^{(E)}\,\nabla_\nu\Lambda_E,
\]
so matter $+$ informational field are conserved together. In the GR limit
$\kappa\!\to\!1$ and $\nabla\Lambda_E\!\to\!0$, both extra terms vanish.

% ------------------------------------------------------------
\subsection*{A.1\quad Space–time background}

Let \((M^{n+1},g_{\mu\nu})\) be a smooth Lorentz (or Riemann) manifold with Levi‑Civita connection \(\nabla_\mu\) and volume form \(\mathrm d^{\,n+1}x\,\sqrt{|g|}\).
Indices are raised/lowered with \(g_{\mu\nu}\) and its inverse \(g^{\mu\nu}\).

\subsection*{A.2\quad Informational current}

\[
J^{\mu} = (\rho,\mathbf J),
\qquad
\mathcal J = \star (J^{\mu}\,\mathrm d x_{\mu}),
\qquad
\nabla_{[\alpha}\mathcal J_{\beta_1\!\dots\beta_n]} = 0.
\]

Covariant continuity:
\[
\nabla_{\mu} J^{\mu} = \sigma.
\tag{A.1}
\]

\subsection*{A.3\quad Differential REOS}

\[
\boxed{%
\nabla_{\mu}\rho
   \;=\;
   -\Lambda_{a}\,\nabla_{\mu} R^{(a)}
   - R^{(a)}\,\nabla_{\mu}\Lambda_{a}.
}
\tag{A.2}
\]

\subsection*{A.4\quad Informational force}

\[
f_{\mu}
  = -\,R^{(a)}\,\nabla_{\mu}\Lambda_{a},
\qquad
\nabla^{\mu}T^{\mathrm{info}}_{\mu\nu} = -\,f_{\nu},
\]
where \(T^{\mathrm{info}}_{\mu\nu}=A_{\mu\alpha}J^{\alpha}J_{\nu}-g_{\mu\nu}\mathcal L\) is an information–stress tensor and \(A_{\mu\nu}\) is any positive‑definite mobility tensor.

\subsection*{A.5\quad Resource–clock bound}

For each resource \(R^{(a)}\) define a conjugate angular variable
\(\Theta_{(a)}\in S^{1}\) such that \((R^{(a)},\Theta_{(a)})\) is canonical with respect to the extended symplectic form
\[
\Omega = \mathrm dJ^{\mu}\!\wedge\! \mathrm d x_{\mu}
         + \sum_{a}\mathrm d R^{(a)}\!\wedge\! \mathrm d\Theta_{(a)} .
\]
The Cramér–Rao inequality on \((\widehat M, \Omega)\) gives
\[
\boxed{%
\Delta R^{(a)}\,\Delta\Theta_{(a)}
   \;\gtrsim\;
   \frac{1}{2\ln 2}\ .
}
\tag{A.3}
\]

\subsection*{A.6\quad Stokes on extended tubes}

For any codimension‑1 hypersurface \(\widehat\Sigma\subset\widehat M\) with interior \(\widehat V\),
\[
\int_{\widehat\Sigma}\!\widehat{\mathcal J}
  \;=\;
  \int_{\widehat V}\!\widehat{\sigma},
\qquad
\widehat{\mathcal J}
  = \star_{\widehat M}\!\big(J^{\mu},J^{\Theta_{(a)}}\big)\,\mathrm dX_{\mu},
\]
where \(J^{\Theta_{(a)}}=-\Lambda_{a}\).
Projection onto ordinary space–time recovers Eq.\,(A.1);
projection onto any resource–clock plane recovers Eq.\,(A.3).

\paragraph{Summary.}
The Tier‑0 pillars (continuity, differential REOS, informational force, least‑action) retain their boxed form under arbitrary curvature and for an arbitrarily large resource list \(\{R^{(a)}\}\); only the Hodge‑star and covariant derivatives pick up the metric.  Tier‑1 corollaries therefore hold on curved space–time provided local clocks and mobility tensors are chosen consistently.

% ------------------------------------------------------------
\subsection{Corollary: REOS Poisson correction}

\paragraph{Corollary.}
In the static/weak‑field regime the REOS correction to Poisson reads
\begin{equation}
\nabla^2\Phi \;=\; 4\pi G\,\rho_m \;-\; \nabla\!\cdot\!\bigl(R^{(E)}\,\nabla\Lambda_E\bigr),
\label{eq:poisson_info}
\end{equation}
where $R^{(E)}=m+T+U_{\text{field}}$ (Kicks) and $\Lambda_E$ is the local price field for an energy bit.

\paragraph{Relativistic form.}
On a fixed background, the Newtonian result lifts to
\begin{equation}
\Box_{g}\,\Phi
\;=\;
4\pi G\,\rho_m
\;-\;
\nabla_\mu\!\left(R^{(E)}\,\nabla^\mu \Lambda_E\right),
\end{equation}
reducing to \eqref{eq:poisson_info} in the weak/static limit.

\paragraph{Spherical symmetry.}
If $\Lambda_E=\Lambda_E(r)$ then
\begin{equation}
\frac{1}{r^2}\frac{d}{dr}\!\left(r^2\frac{d\Phi}{dr}\right)
\;=\; 4\pi G\,\rho_m
\;-\; \frac{1}{r^2}\frac{d}{dr}\!\left(r^2 R^{(E)} \frac{d\Lambda_E}{dr}\right).
\label{eq:sph_info}
\end{equation}
The second term is negative for outward‑rising $\Lambda_E$ (mild screening) and vanishes when $\nabla\Lambda_E=0$.

\paragraph{Remark.}
Eq.~\eqref{eq:poisson_info} is the REOS counterpart of common “entropic‑correction’’ ansätze:
the correction is the divergence of the informational force density
$f_\mu=-\,R^{(E)}\nabla_\mu\Lambda_E$ and \emph{does not} assume an Unruh‑temperature postulate.

% ------------------------------------------------------------
\subsection{Maxwell‑style informational field form}

\paragraph{Definitions.}
Let the covariant price 1‑form be $\Lambda_\mu := (\Lambda_E,\boldsymbol{\Lambda})$ and
\[
F_{\mu\nu} := \nabla_\mu \Lambda_\nu - \nabla_\nu \Lambda_\mu .
\]
Gauge symmetry: $\Lambda_\mu \mapsto \Lambda_\mu + \nabla_\mu \chi$.

\paragraph{Field equations (inhomogeneous and Bianchi).}
\[
\nabla_\mu F^{\mu\nu} = J^\nu_{\mathrm{info}},
\qquad
\nabla_{[\lambda} F_{\mu\nu]} = 0 .
\]
Here $J^\nu_{\mathrm{info}}$ is the informational source 4‑current consistent with the continuity balance in the main text.

\paragraph{Force and stress.}
The informational force is
\[
f_\nu = -\,R^{(E)}\,\nabla_\nu \Lambda_E
      = -\,R^{(E)}\,F_{\nu 0}
      \quad\text{(static gauge $\boldsymbol{\Lambda}=0$)} .
\]
A standard stress tensor is
\[
T^{\mathrm{info}}_{\mu\nu}[F]
= F_{\mu\lambda}F_{\nu}{}^{\lambda}
  - \tfrac{1}{4}\,g_{\mu\nu}\,F_{\alpha\beta}F^{\alpha\beta},
\]
which satisfies $\nabla^\mu T^{\mathrm{info}}_{\mu\nu} = -\,F_{\nu\mu} J^\mu_{\mathrm{info}}$; matching $J^\mu_{\mathrm{info}}$ to the price–clock current gives the balance used above.

\paragraph{Action (Kick units).}
A minimal unit‑agnostic Lagrangian is
\[
\mathcal L[\Lambda,J]
= -\tfrac{1}{4} F_{\mu\nu}F^{\mu\nu}
  + \Lambda_\nu J^\nu_{\mathrm{info}},
\]
whose Euler–Lagrange equations give $\nabla_\mu F^{\mu\nu}=J^\nu_{\mathrm{info}}$.
In strictly source‑free regions ($\sigma=0$) this matches the continuity identity;
with local production one may use the price–clock extension of §\ref{sec:multi_clock}
to make the extended current divergence‑free.
