% ======================================================================
% appendices/appC_d-dim-gauss.tex — nD Gauss & potentials (v1.0.2)
% ======================================================================

\section{nD Gauss identities and radial potentials}
\label{app:d-dim-gauss}

All energy-like quantities are in Kick (Kz); see \S\ref{sec:conventions}. This appendix collects the
geometric facts and distributional identities behind the static results of \S\ref{sec:static-gauss-nd}.

\subsection*{Sphere area and ball volume}
Let \(n\) be the number of spatial dimensions. The area of the unit \((n{-}1)\)-sphere and the volume
of the radius-\(r\) ball are
\begin{equation}
S_{n-1}=\frac{2\pi^{n/2}}{\Gamma(n/2)},\qquad
V_n(r)=\frac{S_{n-1}}{n}\,r^{\,n}.
\label{eq:s-area-vn}
\end{equation}
Useful values: \(S_1=2\pi\), \(S_2=4\pi\), \(S_3=2\pi^2\), \(S_4=\tfrac{8}{3}\pi^2\).

\subsection*{Radial Gauss law}
For any spherically symmetric field \(\mathbf X(r)=X_r(r)\,\hat{\mathbf r}\),
\begin{equation}
\oint_{S_r}\mathbf X\cdot d\mathbf A \;=\; S_{n-1}\,r^{\,n-1}\,X_r(r).
\label{eq:gauss-radial}
\end{equation}
With the informational closure \(\mathbf j=\kappa\nabla\Phi_I\) and \(\mathbf F_I=-\nabla\Phi_I\),
the static balance \(\oint_{\partial V}\mathbf j\!\cdot d\mathbf A=\dot I_{\rm enc}(V)\) gives (for a sphere)
\begin{equation}
|\nabla\Phi_I(r)|=\frac{\dot I_{\rm enc}}{S_{n-1}\,\kappa\,r^{\,n-1}},
\qquad
\mathbf F_I(r)=-\frac{\dot I_{\rm enc}}{S_{n-1}\,\kappa\,r^{\,n-1}}\,\hat{\mathbf r},
\label{eq:nd-inverse-power-app}
\end{equation}
which is \S\ref{sec:static-gauss-nd} in radial form.

\subsection*{Fundamental solutions of Poisson/Laplace}
Let \(\Delta\) be the Euclidean Laplacian on \(\mathbb R^n\setminus\{0\}\).
For \(n\ge 3\),
\begin{equation}
\Delta\big(r^{\,2-n}\big) \;=\; -\,(n{-}2)\,S_{n-1}\,\delta^{(n)}(\mathbf x).
\label{eq:fund-nge3}
\end{equation}
For \(n=2\),
\begin{equation}
\Delta(\ln r) \;=\; 2\pi\,\delta^{(2)}(\mathbf x).
\label{eq:fund-n2}
\end{equation}
Thus for a pointlike interior production \(\dot I_{\rm enc}=I_0\), the Poisson equation
\(\Delta\Phi_I=-(I_0/\kappa)\delta\) is solved by
\begin{equation}
\Phi_I(r)=
\begin{cases}
\dfrac{I_0}{(n{-}2)\,S_{n-1}\,\kappa}\;r^{\,2-n}+C, & n\ge 3,\\[1.0ex]
\dfrac{I_0}{2\pi\,\kappa}\,\ln\!\dfrac{r}{r_0}, & n=2,
\end{cases}
\quad\Rightarrow\quad
\mathbf F_I=-\nabla\Phi_I
=\begin{cases}
-\dfrac{I_0}{S_{n-1}\,\kappa}\,\dfrac{\hat{\mathbf r}}{r^{\,n-1}}, & n\ge 3,\\[1.0ex]
-\dfrac{I_0}{2\pi\,\kappa}\,\dfrac{\hat{\mathbf r}}{r}, & n=2,
\end{cases}
\label{eq:point-sol}
\end{equation}
agreeing with \eqref{eq:nd-inverse-power-app}.

\subsection*{Uniform source on a ball}
Let \(\sigma\) be constant production density on the ball \(B_R\).
The enclosed rate is \(\dot I_{\rm enc}(r)=\sigma\,V_n(r)\) for \(r<R\) and \(\dot I_{\rm enc}(r)=\sigma\,V_n(R)\) for \(r\ge R\).
From \eqref{eq:nd-inverse-power-app},
\begin{equation}
\mathbf F_I(r)=
\begin{cases}
-\dfrac{\sigma}{n\,\kappa}\,r\,\hat{\mathbf r}, & r<R,\\[1.0ex]
-\dfrac{\sigma\,V_n(R)}{S_{n-1}\,\kappa}\,\dfrac{\hat{\mathbf r}}{r^{\,n-1}}, & r\ge R,
\end{cases}
\label{eq:uniform-ball}
\end{equation}
i.e.\ linear inside, inverse-power outside. A continuous potential follows by integrating \(\partial_r\Phi_I=-F_{I,r}\).

\subsection*{The \(n=2\) special case}
Because \(\Phi_I\sim \ln r\), one must choose a reference scale \(r_0\).
Fields still follow \(|\mathbf F_I| \propto 1/r\), and the Gauss balance \(\oint\mathbf F_I\cdot d\mathbf A\) remains fixed by the total enclosed rate.

\subsection*{Multipoles and harmonic expansions}
On any source-free exterior region, \(\Delta\Phi_I=0\) and \(\Phi_I\) admits a harmonic expansion.
In \(n=3\),
\begin{equation}
\Phi_I(r,\Omega) \;=\; \sum_{\ell=0}^{\infty}\sum_{m=-\ell}^{\ell}
\left(\frac{A_{\ell m}}{r^{\ell+1}} + B_{\ell m}\,r^{\ell}\right) Y_{\ell m}(\Omega),
\qquad \mathbf F_I=-\nabla\Phi_I.
\label{eq:multipole-3d}
\end{equation}
Exterior fields use the decaying branch \(A_{\ell m}/r^{\ell+1}\); \(\ell=0\) is the monopole determined by \eqref{eq:gauss-radial}. In general \(n\), the decaying branch behaves as \(r^{-(\ell+n-2)}\) times hyperspherical harmonics.

\subsection*{Green and Poisson formulas (free space)}
The free-space Green function for \(n\ge 3\) is
\begin{equation}
G_n(\mathbf x,\mathbf x')=\frac{1}{(n{-}2)\,S_{n-1}}\;\frac{1}{|\mathbf x-\mathbf x'|^{\,n-2}},
\qquad
\Delta G_n(\cdot,\mathbf x')=-\delta^{(n)}(\cdot-\mathbf x').
\label{eq:greens-free}
\end{equation}
For a compactly supported source \(s(\mathbf x)=-(1/\kappa)\,i(\mathbf x)\),
\begin{equation}
\Phi_I(\mathbf x)=\int_{\mathbb R^n} G_n(\mathbf x,\mathbf x')\,s(\mathbf x')\,d^n x',
\qquad
\mathbf F_I(\mathbf x)=-\nabla\Phi_I(\mathbf x).
\label{eq:greens-solution}
\end{equation}
Boundary-value problems on a ball use the standard Poisson kernel; we omit it here since \S\ref{sec:static-gauss-nd} needs only the radial case.

\subsection*{Calibration remarks}
Equations above are purely geometric identities tied to the informational closure.
Measured sectors are obtained by the linear calibrations of \S\ref{sec:static-gauss-nd}:
\(\mathbf g=c_M\,\mathbf F_I\), \(\mathbf E=c_Q\,\mathbf F_I\), heat flux \(=\kappa_T\,\mathbf F_I\).
No \(G\) or \(k_B\) appear in the algebra; one datum per sector fixes the mapping.

\paragraph{Summary.}
Inverse-power fields \(\propto r^{1-n}\) (or \(1/r\) in \(n=2\)) follow from Gauss and the closure \(\mathbf j=\kappa\nabla\Phi_I\).
Potentials are \(r^{2-n}\) (or \(\ln r\)), and interior fields of uniform sources are linear in \(r\). These are the static building blocks used throughout §\ref{sec:static-gauss-nd} and the entropic-screen analysis in §\ref{sec:entropic-gravity-screen}.
