% ======================================================================
% appF_calibrations.tex — Appendix F: One‑datum calibrations (static sectors)
% ======================================================================

\section{One‑datum calibrations (static sectors)}
\label{app:calibrations}

We record how a \emph{single} datum per sector maps the informational field
$\mathbf E_I:=-\nabla\Phi_I$ to measured quantities, using only the static closure
\eqref{eq:static-closure} and the screen/sector balance (via \eqref{eq:stokes-identity}).

% ------------------------- F.1 Recap -----------------------------------
\subsection{Recap: static screen/sector law}
\label{appF:recap}

On a time‑frozen snapshot in a source‑free annulus ($\sigma=0$), the directed flux
\[
\Phi(r)\;:=\;\int_{\Sigma(r)} \mathbf j\!\cdot n\,dA
\]
is independent of $r$. With the static closure $\mathbf j=-\kappa\,\mathbf E_I$,
the mean normal current on a radius‑$r$ screen in $n$ spatial dimensions is
\begin{equation}
\big\langle j_n\big\rangle(r)=\frac{\Phi}{A_n(r)},\qquad
|\mathbf E_I(r)|=\frac{1}{\kappa}\,\frac{\Phi}{A_n(r)},
\qquad A_n(r)=S_{n-1}\,r^{\,n-1},
\label{appF:eq:screen}
\end{equation}
with $S_{n-1}$ given in Appendix~\ref{app:nD-screens}, eq.~\eqref{appC:eq:areas}. For a conical sector of
solid angle $\Delta\Omega$, replace $A_n(r)$ by $A_\Delta(r)=\Delta\Omega\,r^{\,n-1}$.

% ------------------------- F.2 Definition ------------------------------
\subsection{Calibration definition (one datum per sector)}
\label{appF:def}

A measured sector is a linear readout of $\mathbf E_I$ set by a \emph{single} datum:
\begin{equation}
\boxed{\qquad
\mathbf g=c_M\,\mathbf E_I,\qquad
\mathbf E=c_Q\,\mathbf E_I,\qquad
\mathbf j_{\rm th}=\kappa_T\,\mathbf E_I
\qquad}
\label{appF:eq:cal-def}
\end{equation}
The sector scales $(c_M,c_Q,\kappa_T)$ are fixed once (units as stated in Appendix~\ref{app:kick}).

% ------------------------- F.3 One datum fixes scale -------------------
\subsection{How one datum fixes the scale}
\label{appF:one-datum}

Pick a reference screen $\Sigma(r_\star)$ around a source with directed flux $\Phi_\star$.
Measure a sector field $Y_\star\in\{g_\star,E_\star,j_{{\rm th},\star}\}$ at $r_\star$.
Using \eqref{appF:eq:screen},
\begin{equation}
c_{\rm sec}
\;=\;
\frac{Y_\star}{|\mathbf E_I(r_\star)|}
\;=\;
Y_\star\,\frac{\kappa\,A_n(r_\star)}{\Phi_\star},
\qquad
\text{(sec}\in\{M,Q,{\rm th}\}).
\label{appF:eq:one-datum}
\end{equation}
No further inputs are required.

% ------------------------- F.4 Ratio predictions -----------------------
\subsection{Ratio predictions (same source family)}
\label{appF:ratios}

After fixing $c_{\rm sec}$, any screen at $r$ gives
\begin{equation}
Y(r)\;=\;c_{\rm sec}\,|\mathbf E_I(r)|
\;=\;
Y_\star\,
\frac{\Phi(r)}{\Phi_\star}\,
\frac{A_n(r_\star)}{A_n(r)}.
\label{appF:eq:ratio}
\end{equation}
In lossless statics ($\Phi(r)=\Phi_\star$),
\begin{equation}
\boxed{\qquad
Y(r)=Y_\star\,\frac{A_n(r_\star)}{A_n(r)}
\;=\;
Y_\star\left(\frac{r_\star}{r}\right)^{\!n-1}.
\qquad}
\label{appF:eq:one-datum-ratio}
\end{equation}
In $n{=}3$ this specializes to $Y(r)=Y_\star\,(r_\star^2/r^2)$.

% ------------------------- F.5 Operational label -----------------------
\subsection{Operational source label (optional)}
\label{appF:Qop}

Define an operational source label by the flux ratio
\begin{equation}
\frac{\mathcal Q}{\mathcal Q_\star}:=\frac{\Phi}{\Phi_\star}.
\label{appF:eq:Qop}
\end{equation}
Then \eqref{appF:eq:ratio} becomes
\begin{equation}
Y(r)\;=\;Y_\star\,\frac{\mathcal Q}{\mathcal Q_\star}\,
\frac{A_n(r_\star)}{A_n(r)}.
\label{appF:eq:ratio-Q}
\end{equation}

% ------------------------- F.6 Sector notes ----------------------------
\subsection{Sector notes}
\label{appF:sectors}

\begin{itemize}\itemsep2pt
\item \textbf{Gravitational readout.} $Y\!=\!g$ with $c_{\rm sec}\!\equiv\!c_M$ yields inverse‑power profiles via \eqref{appF:eq:one-datum-ratio}.
\item \textbf{Electrostatic readout.} $Y\!=\!E$ with $c_{\rm sec}\!\equiv\!c_Q$ uses the same $A_n(r)$ geometry; only the datum changes.
\item \textbf{Thermal/particle channel.} $Y\!=\!j_{\rm th}$ with $c_{\rm sec}\!\equiv\!\kappa_T$ maps a measured heat/particle flux density to $\mathbf E_I$.
\end{itemize}

% ------------------------- F.7 Apertures & confinement -----------------
\subsection{Cones/apertures and confinement}
\label{appF:apertures}

For a conical sector of solid angle $\Delta\Omega$, replace $A_n(r)$ by
$A_\Delta(r)=\Delta\Omega\,r^{\,n-1}$ in \eqref{appF:eq:one-datum}--\eqref{appF:eq:ratio-Q}.
In guides/sheets (Corollary~\ref{corollary:dimensional}), use the effective cross‑section or
circumference in place of $A_n(r)$.

% ------------------------- F.8 Far‑field waves -------------------------
\subsection{Far‑field waves (transfer of envelopes)}
\label{appF:waves}

In homogeneous, source‑free segments the exported power is constant and the far‑field envelope \emph{follows}
$\langle|\mathbf E_I|\rangle\propto 1/\sqrt{A_n(r)}$ (Corollary~\ref{corollary:info-em}).
Readouts inherit:
\begin{equation}
\langle|Y(r)|\rangle
\;=\; c_{\rm sec}\,\langle|\mathbf E_I(r)|\rangle
\;\propto\; \frac{1}{\sqrt{A_n(r)}}.
\label{appF:eq:waves}
\end{equation}

\label{appF:Tritdensity}

One may write $N(r)=\alpha\,A_n(r)$ (\emph{trit} density on a screen) and a per‑trit hit rate $f(r)$ with
$\Phi(r)=f(r)\,N(r)$. Conservation then sets $f(r)\propto 1/A_n(r)$, so only the \emph{product} $f\alpha$ matters
and can be absorbed into the single calibration via \eqref{appF:eq:one-datum}. Hence $\alpha$ is a convenient
parameterization, not an extra constant.

% ------------------------- F.10 Dependencies ---------------------------
\subsection{Dependencies}
\label{appF:units}

Equations \eqref{appF:eq:screen}--\eqref{appF:eq:waves} depend only on:
(i) the static closure \eqref{eq:static-closure}, (ii) the balance identity \eqref{eq:stokes-identity},
and (iii) the geometric areas $A_n(r)$ from Appendix~\ref{app:nD-screens} (eq.~\eqref{appC:eq:areas}).
No bare constants appear beyond the single datum per sector.

\medskip
\noindent\emph{Literature note.}
Geometry and screen areas are collected in Appendix~\ref{app:nD-screens}; a standard reference is Frankel~\cite{Frankel2011}.
For a Poynting‑type export identity in an EM realization, see Appendix~\ref{app:forms-stokes} and e.g.\ Jackson~\cite{Jackson1999}.
