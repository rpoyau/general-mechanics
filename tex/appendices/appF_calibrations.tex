% ======================================================================
% appF_calibrations.tex — Appendix F: One‑datum calibrations (static sectors)
% ======================================================================

\section{One‑datum calibrations (static sectors)}
\label{app:calibrations}

This appendix records how a \emph{single} datum per sector maps the informational field
\(\mathbf E_I:=-\nabla\Phi_I\) to measured quantities, using only the \emph{frame} linear closure
\eqref{axioms:stokes:frame:eq:closure} and the Stokes identity on the same region
\eqref{axioms:stokes:identity:eq}. Units follow App.~\ref{app:kick}.

% ------------------------- F.1 Recap -----------------------------------
\subsection{Recap: frame boundary/sector law}
\label{app:calibrations:recap}

On a frame (time‑frozen) in a source‑free annulus (\(\sigma=0\)), the directed flux through any
radius–\(r\) boundary \(\Sigma(r)\) is independent of \(r\):
\[
\Phi(r)\;:=\;\int_{\Sigma(r)} \mathbf j\!\cdot n\,dA\quad\text{is constant.}
\]
With the frame closure \(\mathbf j=-\kappa\,\mathbf E_I\), the mean normal current on a radius–\(r\)
boundary in \(n\) spatial dimensions is
\begin{equation}
\big\langle j_n\big\rangle(r)=\frac{\Phi}{A_n(r)},\qquad
\big|\mathbf E_I(r)\big|=\frac{1}{\kappa}\,\frac{\Phi}{A_n(r)},
\qquad
A_n(r)=S_{n-1}\,r^{\,n-1}\ \ \text{(App.~\ref{app:nD-screens}, eq.~\eqref{app:nD-boundaries:eq:areas}).}
\label{app:calibrations:eq:frame-law}
\end{equation}
For a conical sector of solid angle \(\Delta\Omega\), replace \(A_n(r)\) by
\(A_\Delta(r)=\Delta\Omega\,r^{\,n-1}\).

% ------------------------- F.2 Definition ------------------------------
\subsection{Calibration definition (one datum per sector)}
\label{app:calibrations:def}

A measured sector is a linear readout of \(\mathbf E_I\) set by a \emph{single} datum:
\begin{equation}
\boxed{\qquad
\mathbf g=c_M\,\mathbf E_I,\qquad
\mathbf E=c_Q\,\mathbf E_I,\qquad
\mathbf j_{\rm th}=\kappa_T\,\mathbf E_I
\qquad}
\label{app:calibrations:eq:def}
\end{equation}
The sector scales \((c_M,c_Q,\kappa_T)\) are fixed once (Kz conventions in App.~\ref{app:kick}).

% ------------------------- F.3 One datum fixes scale -------------------
\subsection{How one datum fixes the scale}
\label{app:calibrations:one-datum}

Pick a reference boundary \(\Sigma(r_\star)\) around a source with directed flux \(\Phi_\star\).
Measure a sector field \(Y_\star\in\{g_\star,E_\star,j_{{\rm th},\star}\}\) at \(r_\star\).
Using \eqref{app:calibrations:eq:frame-law},
\begin{equation}
c_{\rm sec}
\;=\;
\frac{Y_\star}{|\mathbf E_I(r_\star)|}
\;=\;
Y_\star\,\frac{\kappa\,A_n(r_\star)}{\Phi_\star},
\qquad \text{(sec}\in\{M,Q,{\rm th}\}).
\label{app:calibrations:eq:one-datum}
\end{equation}
Orientation signs may be absorbed into the choice of outward normal.

% ------------------------- F.4 Ratio predictions -----------------------
\subsection{Ratio predictions (same source family)}
\label{app:calibrations:ratios}

After fixing \(c_{\rm sec}\), any boundary at \(r\) gives
\begin{equation}
Y(r)\;=\;c_{\rm sec}\,|\mathbf E_I(r)|
\;=\;
Y_\star\,
\frac{\Phi(r)}{\Phi_\star}\,
\frac{A_n(r_\star)}{A_n(r)}.
\label{app:calibrations:eq:ratio}
\end{equation}
In lossless statics (\(\Phi(r)=\Phi_\star\)),
\begin{equation}
\boxed{\qquad
Y(r)=Y_\star\,\frac{A_n(r_\star)}{A_n(r)}
\;=\;
Y_\star\left(\frac{r_\star}{r}\right)^{\!n-1}.
\qquad}
\label{app:calibrations:eq:frame-ratio}
\end{equation}
For \(n{=}3\), \(Y(r)=Y_\star\,(r_\star^2/r^2)\).

% ------------------------- F.5 Operational label -----------------------
\subsection{Operational source label (optional)}
\label{app:calibrations:Qop}

Define an operational source label by the flux ratio
\begin{equation}
\frac{\mathcal Q}{\mathcal Q_\star}:=\frac{\Phi}{\Phi_\star}.
\label{app:calibrations:eq:Qop}
\end{equation}
Then \eqref{app:calibrations:eq:ratio} becomes
\begin{equation}
Y(r)\;=\;Y_\star\,\frac{\mathcal Q}{\mathcal Q_\star}\,
\frac{A_n(r_\star)}{A_n(r)}.
\label{app:calibrations:eq:ratio-Q}
\end{equation}

% ------------------------- F.6 Sector notes ----------------------------
\subsection{Sector notes}
\label{app:calibrations:sectors}

\begin{itemize}\itemsep2pt
\item \textbf{Gravitational readout.} \(Y=g\) with \(c_{\rm sec}\equiv c_M\) yields inverse‑power profiles via \eqref{app:calibrations:eq:frame-ratio}.
\item \textbf{Electric readout.} \(Y=E\) with \(c_{\rm sec}\equiv c_Q\) uses the same \(A_n(r)\) geometry; only the datum changes.
\item \textbf{Thermal/particle channel.} \(Y=j_{\rm th}\) with \(c_{\rm sec}\equiv\kappa_T\) maps a measured heat/particle flux density to \(\mathbf E_I\).
\end{itemize}

% ------------------------- F.7 Apertures & confinement -----------------
\subsection{Cones/apertures and confinement}
\label{app:calibrations:apertures}

For a conical sector of solid angle \(\Delta\Omega\), replace \(A_n(r)\) by
\(A_\Delta(r)=\Delta\Omega\,r^{\,n-1}\) in \eqref{app:calibrations:eq:one-datum}–\eqref{app:calibrations:eq:ratio-Q}.
In guides/sheets (§\ref{corollary:dimensional}), use the effective cross‑section or
circumference in place of \(A_n(r)\).

% ------------------------- F.8 Far‑field waves (pointer) ---------------
\subsection{Far‑field waves (pointer)}
\label{app:calibrations:waves}

Wave transport and envelopes are collected in §\ref{waves}. Readouts inherit wave envelopes via
\eqref{app:calibrations:eq:def}; no additional constants are introduced.

% ------------------------- F.9 Dependencies ----------------------------
\subsection{Dependencies}
\label{app:calibrations:units}

Equations \eqref{app:calibrations:eq:frame-law}–\eqref{app:calibrations:eq:ratio-Q} depend only on:
(i) the frame linear closure \eqref{axioms:stokes:frame:eq:closure}, (ii) the Stokes identity \eqref{axioms:stokes:identity:eq},
and (iii) the geometric areas \(A_n(r)\) from App.~\ref{app:nD-screens} (eq.~\eqref{app:nD-boundaries:eq:areas}).
No bare constants appear beyond the single datum per sector.
