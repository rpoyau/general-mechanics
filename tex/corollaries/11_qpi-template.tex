% ======================================================================
% 03c12_qpi-template.tex — Quantum path–integral template (from fluctuations)
% ======================================================================

\subsection{Quantum path–integral template (from informational fluctuations)}
\label{corollary:qpi-template}

\paragraph{Scope and stance.}
This records a \emph{template} for wave–like evolution that can be overlaid on the framework
when one models small fluctuations of boundary counting by a quadratic local kernel.
It is \emph{not} an axiom. The canonical drift remains fixed by the rate \(1\)-form and its closure.

\paragraph{Canonical rates (recap).}
From Axiom \S\ref{axioms:path-rate}, the Poincar\'e–Cartan (rate) \(1\)-form and closure,
\begin{equation}
\Theta_{\mathcal R}(q,P,t)=P_i\,dq^i-H(q,P,t)\,dt,
\qquad
d\Theta_{\mathcal R}=0,
\label{corollary:qpi-template:eq:pc-form}
\end{equation}
yield the canonical system
\begin{equation}
\dot q^i=\partial_{P_i}H,
\qquad
\dot P_i=-\,\partial_{q^i}H ,
\label{corollary:qpi-template:eq:canonical}
\end{equation}
identical in content to \eqref{axioms:path-rate:eq:pc-form}–\eqref{axioms:path-rate:eq:canonical}.

\paragraph{Diffusive branch (MaxCal/Onsager–Machlup).}
On a window where local fluctuations around the canonical drift are modeled by a Gaussian kernel
with positive‑definite diffusivity matrix \(D(q,t)\) and a scalar rate field \(U(q,t)\),
maximizing path caliber with constraints on \(\int |\dot q-v|_{D^{-1}}^{2}\,dt\) and \(\int U\,dt\)
yields the Onsager–Machlup rate
\begin{equation}
\mathcal R_{\rm diff}(q,\dot q,t)
\;=\;
\frac{1}{2}\,(\dot q-v)^{\!T} D^{-1}(\dot q-v)\;+\;U(q,t),
\label{corollary:qpi-template:eq:OM}
\end{equation}
and the path weight
\begin{equation}
\mathbb P[\gamma]\ \propto\
\exp\!\left(-\int_{t_0}^{t_1}\mathcal R_{\rm diff}\,dt\right).
\label{corollary:qpi-template:eq:weight}
\end{equation}
The state density \(\rho(q,t)\) satisfies a Fokker–Planck form
\begin{equation}
\partial_t \rho
\;=\;
-\nabla\!\cdot(v\,\rho)\;+\;\tfrac12\,\nabla^{\!T}\!\bigl(D\,\nabla\rho\bigr),
\label{corollary:qpi-template:eq:fp}
\end{equation}
with \(D\) tied to the counting noise kernel of App.~\ref{app:noise-kernel} (see \S\ref{corollary:noise}).

\paragraph{Oscillatory branch (wave template).}
Keep the same local fields but rotate the quadratic kernel to an \emph{oscillatory} weight:
\begin{equation}
\mathcal A[\gamma]\ \propto\
\exp\!\left(i\,2\pi\!\int_{t_0}^{t_1}\!\big[P\!\cdot\!\dot q-H(q,P,t)\big]\,dt\right).
\label{corollary:qpi-template:eq:amp}
\end{equation}
Choose a quadratic generator with an inertial map \(M(q,t)\):
\begin{equation}
H(q,P,t)=\tfrac12\,P^{\!T} M^{-1}(q,t)\,P\;+\;U(q,t),
\label{corollary:qpi-template:eq:quadH}
\end{equation}
so that stationary phase reproduces the canonical drift \eqref{corollary:qpi-template:eq:canonical} with
\(\dot q=M^{-1}P\) and \(\dot P=-\nabla U\) (cf.\ \S\ref{corollary:thermo-newton}).

\paragraph{Short‑time kernel and a Schr\"odinger‑type evolution (template).}
Under the quadratic choice \eqref{corollary:qpi-template:eq:quadH}, a Fourier representation of the short‑time kernel
for \eqref{corollary:qpi-template:eq:amp} yields a linear evolution for a complex amplitude \(\psi(q,t)\):
\begin{equation}
i\,\partial_t \psi
\;=\;
\left[
-\frac{\alpha^{2}}{2}\,\nabla^{\!T}\!\bigl(M^{-1}\nabla\bigr)
\;+\;U(q,t)
\right]\psi,
\qquad \alpha>0,
\label{corollary:qpi-template:eq:schro}
\end{equation}
where \(\alpha\) is a positive wave scale and \(M\) an inertial map (both fixed once per sector when used).
No universal constant is introduced.

\paragraph{Minimal coupling (Info–EM specialization).}
If an informational \(1\)-form \(A_I\) is present (\S\ref{waves}), minimal coupling enters by
\begin{equation}
P\ \mapsto\ P-A_I(q,t)
\quad\text{in}\quad
\eqref{corollary:qpi-template:eq:amp}–\eqref{corollary:qpi-template:eq:schro},
\label{corollary:qpi-template:eq:min-coupling}
\end{equation}
preserving the Poincar\'e–Cartan structure and producing gauge‑covariant drift and waves.
(A sector charge matrix may be included as a modeling choice.)

\paragraph{Relations between the branches.}
Both closures reduce to the same canonical drift \eqref{corollary:qpi-template:eq:canonical} under stationary phase.
Heuristically, larger counting noise (larger \(D\) in \S\ref{corollary:noise}/App.~\ref{app:noise-kernel})
narrows resolvable interference, effectively restricting admissible \(\alpha\) in
\eqref{corollary:qpi-template:eq:schro}. The branch choice is empirical and boundary‑/channel‑dependent.

\paragraph{Units.}
Rates follow the global convention in App.~\ref{app:notation-units} (Kz).
Any sector‑specific scales (\(M,\alpha\)) are fixed once when this template is used; no bare constants are carried by the algebra.

\medskip
\noindent\emph{Summary.}
The Stokes identity on \((q,t)\) fixes the canonical drift. Adding a quadratic local kernel yields two
consistent fluctuation closures: a diffusive (Fokker–Planck) branch and, by oscillatory continuation,
a Schr\"odinger‑type template. Both sit on the same manifold‑first principles and introduce no bare constants.

\medskip
\noindent\emph{Literature note.}
Diffusive (Onsager–Machlup) and Fokker–Planck treatments are standard \cite{OnsagerMachlup1953,Risken1989,Gardiner2004}.
The oscillatory path‑integral template follows the classic construction \cite{Feynman1948,FeynmanHibbs1965}.
The stochastic/noise‑kernel connection is summarized in App.~\ref{app:noise-kernel}; see also \cite{HuVerdaguer2008}.
