% ======================================================================
% 03c12_qpi-template.tex — Quantum path–integral template (from fluctuations)
% ======================================================================

\subsection{Quantum path–integral template (from informational fluctuations)}
\label{corollary:qpi-template}

\paragraph{Setup.}
On configuration–time space \((q,t)\) as in \S\ref{axioms:path-rate}, the canonical drift is determined by
the Hamiltonian generator through \eqref{axioms:path-rate:eq:canonical}.
On a window \([t_0,t_1]\), fluctuations around this drift are modelled by local quadratic data:
a drift field \(v(q,t)\) (canonically linked via \eqref{axioms:path-rate:eq:canonical}),
a positive–definite diffusivity \(D(q,t)\), a scalar rate field \(U(q,t)\),
an inertial map \(M(q,t)\!>\!0\) for quadratic generators, and a wave scale \(\alpha>0\) used in the oscillatory branch.

\paragraph{Diffusive branch.}
A quadratic rate functional
\begin{equation}
\mathcal R_{\rm diff}(q,\dot q,t)
\;=\;
\frac{1}{2}\,(\dot q-v)^{\!T} D^{-1}(\dot q-v)\;+\;U(q,t),
\label{corollary:qpi-template:eq:OM}
\end{equation}
yields the path weight
\begin{equation}
\mathbb P[\gamma]\ \propto\
\exp\!\left(-\int_{t_0}^{t_1}\mathcal R_{\rm diff}\,dt\right),
\label{corollary:qpi-template:eq:weight}
\end{equation}
and the state density \(\rho(q,t)\) satisfies
\begin{equation}
\partial_t \rho
\;=\;
-\nabla\!\cdot(v\,\rho)\;+\;\tfrac12\,\nabla^{\!T}\!\bigl(D\,\nabla\rho\bigr).
\label{corollary:qpi-template:eq:fp}
\end{equation}

\paragraph{Oscillatory branch.}
An oscillatory kernel with action density \(P\!\cdot\!\dot q - H(q,P,t)\) defines the path amplitude
\begin{equation}
\mathcal A[\gamma]\ \propto\
\exp\!\left(i\,2\pi\!\int_{t_0}^{t_1}\!\big[P\!\cdot\!\dot q-H(q,P,t)\big]\,dt\right).
\label{corollary:qpi-template:eq:amp}
\end{equation}
For a quadratic generator
\begin{equation}
H(q,P,t)=\tfrac12\,P^{\!T} M^{-1}(q,t)\,P\;+\;U(q,t),
\label{corollary:qpi-template:eq:quadH}
\end{equation}
stationary phase reproduces the canonical drift in \eqref{axioms:path-rate:eq:canonical}.

\paragraph{Short–time kernel and Schr\"odinger–type evolution.}
Under \eqref{corollary:qpi-template:eq:quadH}, a Fourier representation of the short–time kernel gives a linear
evolution for a complex amplitude \(\psi(q,t)\):
\begin{equation}
i\,\partial_t \psi
\;=\;
\left[
-\frac{\alpha^{2}}{2}\,\nabla^{\!T}\!\bigl(M^{-1}\nabla\bigr)
\;+\;U(q,t)
\right]\psi,
\qquad \alpha>0 .
\label{corollary:qpi-template:eq:schro}
\end{equation}

\paragraph{Minimal coupling.}
With an informational \(1\)-form \(A_I\), minimal coupling is
\begin{equation}
P\ \mapsto\ P-A_I(q,t)
\quad\text{in}\quad
\eqref{corollary:qpi-template:eq:amp}–\eqref{corollary:qpi-template:eq:schro}.
\label{corollary:qpi-template:eq:min-coupling}
\end{equation}

\medskip
\noindent\emph{Literature note.}
Diffusive (Onsager–Machlup) and Fokker–Planck treatments: Onsager–Machlup~\cite{OnsagerMachlup1953},
Risken~\cite{Risken1989}, Gardiner~\cite{Gardiner2004}.
Oscillatory path–integral constructions: Feynman~\cite{Feynman1948}, Feynman–Hibbs~\cite{FeynmanHibbs1965}.
