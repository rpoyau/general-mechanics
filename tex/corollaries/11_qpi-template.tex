% ======================================================================
% 11_qpi-template.tex — Quantum path–integral template (AF–Pure)
% ======================================================================

\subsection{Quantum path–integral template (from informational fluctuations)}
\label{corollary:qpi-template}

\paragraph{Setup.}
On configuration–time space \((q,t)\), the canonical rate equations and Poincaré–Cartan form
hold by \eqref{axioms:path-rate:eq:canonical} and \eqref{axioms:path-rate:eq:pc-form}.
On a window \([t_0,t_1]\), fluctuations about the canonical drift are modeled by local quadratic data:
a drift \(v(q,t)\), diffusivity \(D(q,t)>0\), scalar rate \(U(q,t)\), and an inertial map \(M(q,t)>0\).
A positive wave scale \(\alpha>0\) is used in the oscillatory branch.

\paragraph{Derivation.}
From \eqref{axioms:path-rate:eq:canonical} (drift) and a quadratic deviation penalty, the
diffusive branch uses the Onsager–Machlup density
\begin{equation}
\mathcal R_{\mathrm{diff}}(q,\dot q,t)
\;=\;
\frac{1}{2}\,(\dot q-v)^{\!T} D^{-1}(\dot q-v)\;+\;U(q,t).
\label{corollary:qpi-template:eq:primary}
\end{equation}
The associated path weight is
\begin{equation}
\mathbb P[\gamma]\ \propto\
\exp\!\left(-\!\int_{t_0}^{t_1}\!\mathcal R_{\mathrm{diff}}\,dt\right),
\label{corollary:qpi-template:eq:weight}
\end{equation}
and the state density satisfies the Fokker–Planck equation
\begin{equation}
\partial_t \rho
\;=\;
-\nabla\!\cdot(v\,\rho)\;+\;\tfrac12\,\nabla^{\!T}\!\bigl(D\,\nabla\rho\bigr).
\label{corollary:qpi-template:eq:fp}
\end{equation}
For an oscillatory branch based on the rate \(1\)-form, the path amplitude uses
\begin{equation}
\mathcal A[\gamma]\ \propto\
\exp\!\left(i\,2\pi\!\int_{t_0}^{t_1}\!\big[P\!\cdot\!\dot q-H(q,P,t)\big]\,dt\right),
\label{corollary:qpi-template:eq:amp}
\end{equation}
with quadratic generator
\begin{equation}
H(q,P,t)=\tfrac12\,P^{\!T} M^{-1}(q,t)\,P\;+\;U(q,t).
\label{corollary:qpi-template:eq:quadH}
\end{equation}
A short‑time kernel (Fourier representation) yields the Schrödinger‑type evolution
\begin{equation}
i\,\partial_t \psi
\;=\;
\Big[-\,\tfrac{\alpha^{2}}{2}\,\nabla^{\!T}\!\big(M^{-1}\nabla\big)+U(q,t)\Big]\psi .
\label{corollary:qpi-template:eq:schro}
\end{equation}
Minimal coupling to an informational \(1\)-form \(A_I\) is
\begin{equation}
P\ \mapsto\ P-A_I(q,t)
\quad\text{in}\quad
\eqref{corollary:qpi-template:eq:amp}–\eqref{corollary:qpi-template:eq:schro}.
\label{corollary:qpi-template:eq:min-coupling}
\end{equation}

\paragraph{Specialization.}
For constant \(M\) and \(D\), \eqref{corollary:qpi-template:eq:fp} reduces to drift–diffusion
with uniform coefficients, and \eqref{corollary:qpi-template:eq:schro} becomes a constant‑mass
Schrödinger generator.

\paragraph{Calibration.}
(omitted)

\paragraph{Falsification.}
On a declared window with fixed \((v,D,U,M)\), if the measured transition densities deviate
from \eqref{corollary:qpi-template:eq:fp} beyond statistical tolerance on the same window, the
diffusive branch is falsified. If interferometric phase transport fails to obey
\eqref{corollary:qpi-template:eq:schro} under the same \(U,M,\alpha\), the oscillatory branch
is falsified.

\medskip
\noindent\emph{Provenance.}
\eqref{axioms:path-rate:eq:pc-form}, \eqref{axioms:path-rate:eq:canonical},
\eqref{corollary:qpi-template:eq:primary}–\eqref{corollary:qpi-template:eq:min-coupling}.

\medskip
\noindent\emph{Literature note.}
Onsager–Machlup; Fokker–Planck (Risken; Gardiner). Path integrals (Feynman; Feynman–Hibbs).
