% ======================================================================
% 02_confinement.tex — Confinement potentials (AF–Pure)
% ======================================================================

\subsection{Confinement potentials}
\label{corollary:conf}

\paragraph{Setup.}
Frame form with \(\partial_t i=0\) on an open annulus \(W\subset U\) where \(\sigma|_W=0\).
Let \(A_n(r)\) be the area of a radius–\(r\) boundary in \(n\) spatial dimensions. Linear channel
closure holds by \eqref{axioms:linear-closure:eq}.

\paragraph{Derivation.}
From the frame‑boundary relation \eqref{corollary:frame-boundary:eq:boundary-law} and
\eqref{axioms:linear-closure:eq},
\begin{equation}
\big|E_I(r)\big|
\;=\;\frac{1}{\kappa}\,\frac{\Phi}{A_n(r)} .
\label{corollary:conf:eq:primary}
\end{equation}
The potential drop along a radial segment satisfies
\begin{equation}
\Phi_I(r_2)-\Phi_I(r_1)
\;=\;
-\int_{r_1}^{r_2}\!E(\rho)\,d\rho ,
\label{corollary:conf:eq:drop}
\end{equation}
by \eqref{corollary:entropic-corrections:eq:potential}.

\paragraph{Specialization.}
\emph{(1D tube)} If \(A_1(r)=A_\perp\) is constant,
\begin{equation}
E(r)=E_{\mathrm{tube}}:=\frac{\Phi}{\kappa\,A_\perp}.
\label{corollary:conf:eq:1d}
\end{equation}
\emph{(3D spherical, unsaturated)} If \(A_3(r)=4\pi r^2\),
\begin{equation}
E(r)=\frac{\Phi}{4\pi\kappa}\,\frac{1}{r^{2}} .
\label{corollary:conf:eq:3d}
\end{equation}
\emph{(Ceiling and flattening)} With the capacity ceiling \eqref{corollary:entropic-corrections:eq:emax},
the intersection radius in \(n=3\) is
\begin{equation}
r_{\mathrm{flat}}=\sqrt{\frac{\Phi/\kappa}{4\pi\,E_{\max}}}\,,
\quad
E(r)=\min\!\left\{\frac{\Phi}{4\pi\kappa r^{2}},\ E_{\max}\right\}.
\label{corollary:conf:eq:ceiling}
\end{equation}

\paragraph{Calibration.}
With a one‑datum sector calibration \(U=c_{\mathrm{conf}}\Phi_I\),
\emph{(1D)} the separation potential is
\begin{equation}
U(r)=U_0+\sigma r,\qquad \sigma=c_{\mathrm{conf}}\,E_{\mathrm{tube}} .
\label{corollary:conf:eq:cal1d}
\end{equation}
\emph{(3D + ceiling)} using \eqref{corollary:conf:eq:ceiling} and \eqref{corollary:conf:eq:drop},
\begin{equation}
U(r)=U_0+c_{\mathrm{conf}}
\begin{cases}
E_{\max}\,r, & r\le r_{\mathrm{flat}},\\[4pt]
E_{\max}\,r_{\mathrm{flat}}+\dfrac{\Phi}{4\pi\kappa}
\!\left(\dfrac{1}{r_{\mathrm{flat}}}-\dfrac{1}{r}\right), & r>r_{\mathrm{flat}}.
\end{cases}
\label{corollary:conf:eq:3d-sat}
\end{equation}

\paragraph{Falsification.}
(1D) In a declared guide, if \(U(r)\) departs from linearity by \(>10^{-4}\) over \(10^3\) wavelengths
(on the same segment), the 1D specialization is falsified.
(3D) If the calibrated \(E(r)\) fails the \(1/r^2\) law outside the declared production region,
either \(\sigma\neq0\) on \(W\) or the ceiling assumption is violated.

\medskip
\noindent\emph{Provenance.}
\eqref{corollary:frame-boundary:eq:boundary-law}, \eqref{axioms:linear-closure:eq},
\eqref{corollary:entropic-corrections:eq:potential},
\eqref{corollary:entropic-corrections:eq:emax}.

\medskip
\noindent\emph{Literature note.}
Inverse‑power radial fields and sector areas: Jackson. Geometric flux/Stokes: Frankel.
