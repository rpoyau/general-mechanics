% ======================================================================
% 03c00_relational-info-gas.tex — C0: Relational info–gas on screens
% ======================================================================

\subsection{C0: Relational info–gas on screens}
\label{corollary:info-gas}

\paragraph{Literature scope.}
All corollaries below follow directly from the generalized Stokes identity \(d(*j)=\Sigma\) and the
snapshot closure \(\mathbf j=-\kappa\,\mathbf E_I\), plus the optional Info--EM specialization where stated.
Standard identities (Poynting, Kramers--Kronig, Raychaudhuri) are recalled in
Apps.~\ref{app:forms-stokes}--\ref{app:jacobson} and used only in their textbook regimes.

\paragraph{Screen bookkeeping.}
We model boundary counts as a \emph{info–gas} generalized to a ternary alphabet (trits): distinctions
cross \(\partial U\) and are counted per area per time. Let \(j:=J^\mu dx_\mu\) be the
informational/entropic current \(1\)-form (nats on \(\partial U\) unless noted),
\(j^{(n)}:=\!*j\) its Hodge dual, and \(\Sigma:=\sigma\,\mathrm{vol}\) the production form.
On a fixed region \(U\) the cut bookkeeping is
\begin{equation}
\dot I_{\mathrm{enc}}=\Phi_{\mathrm{in}}-\Phi_{\mathrm{out}}+\Pi,
\qquad
\Phi_{\mathrm{in/out}}=\int_{\partial U}(\mathbf j\!\cdot n)_{\mp}\,dA,
\qquad
\Pi=\int_{U}\sigma\,d^n x ,
\label{eq:c0-cut}
\end{equation}
which is \eqref{eq:cut-balance} in the axioms. The static, linear snapshot closure is
\begin{equation}
\mathbf E_I:=-\nabla\Phi_I,
\qquad
\mathbf j=-\kappa\,\mathbf E_I ,
\label{eq:c0-static-closure}
\end{equation}
identical to \eqref{eq:static-closure}.

\paragraph{Tube invariance (steady, source–free).}
Consider a narrow tube \(\mathcal T\) whose side is everywhere tangent to \(\mathbf j\).
For any cross–section \(\Sigma(r)\) orthogonal to the tube axis, define the directed flux
\begin{equation}
\Phi(r):=\int_{\Sigma(r)}\mathbf j\!\cdot n\,dA .
\label{eq:c0-flux}
\end{equation}
If \(\partial_t i=0\) and \(\sigma=0\) inside \(\mathcal T\), then \(d j^{(n)}=0\) implies constant tube flux,
\begin{equation}
\Phi(r)=\mathrm{const},\qquad
\big\langle j_n\big\rangle(r)=\frac{\Phi}{A(r)},
\label{eq:c0-tube}
\end{equation}
where \(A(r):=\int_{\Sigma(r)}dA\) and \(\langle j_n\rangle\) is the cross–sectional mean of the normal component.

\paragraph{Cones \& screens (dimension–dependent scaling).}
For a conical sector of fixed solid angle \(\Delta\Omega\) in \(n\) spatial dimensions,
\begin{equation}
A_n(r)=\Delta\Omega\,r^{\,n-1}\quad\Rightarrow\quad
\big\langle j_n\big\rangle(r)=\frac{\Phi}{\Delta\Omega}\,\frac{1}{r^{\,n-1}} .
\label{eq:c0-cone}
\end{equation}
With \eqref{eq:c0-static-closure},
\begin{equation}
\big|\mathbf E_I(r)\big|
=\frac{1}{\kappa}\,\big\langle j_n\big\rangle(r)
=\frac{1}{\kappa}\,\frac{\Phi}{A_n(r)}
\ \propto\ r^{-(n-1)} .
\label{eq:c0-inverse-area}
\end{equation}
For full spherical screens \(A_n(r)=S_{n-1}\,r^{\,n-1}\) (Appendix~\ref{app:nD-screens}, eq.~\eqref{appC:eq:areas}),
\begin{equation}
\big|\mathbf E_I(r)\big|=\frac{\Phi}{\kappa\,S_{n-1}}\;\frac{1}{r^{\,n-1}}.
\label{eq:c0-sphere}
\end{equation}

\paragraph{Low–dimensional limits (static).}
Equation \eqref{eq:c0-inverse-area} immediately gives
\[
n=1:\ \big|\mathbf E_I\big|\propto r^{0}\ (\text{no geometric decay}),\qquad
n=2:\ \big|\mathbf E_I\big|\propto r^{-1},\qquad
n=3:\ \big|\mathbf E_I\big|\propto r^{-2}.
\]
The \(n{=}1\) case captures confinement in tubes/guides (constant cross–section), while \(n{=}2\) captures
sheet/cylindrical spreading. These are revisited and generalized with waves in C8 (\S\ref{corollary:dimensional}).

\paragraph{Directional filtering.}
For apertures or one–sided diaphragms, replace the total flux by the inward component
\(\Phi_-(r)=\int_{\Sigma(r)}(\mathbf j\!\cdot n)_-\,dA\).
Formulas \eqref{eq:c0-tube}–\eqref{eq:c0-sphere} hold with \(\Phi\mapsto\Phi_-\).

\paragraph{Trits, nats, and units.}
If boundary counts are recorded in \emph{trits}, the entropy flux in nats is
\(j_{\rm nats}=(\ln 3)\,j_{\rm trits}\).
Equations \eqref{eq:c0-tube}–\eqref{eq:c0-sphere} are invariant under this log–base choice;
only the numerical value of \(\kappa\) changes accordingly. When a measured sector is desired, a single
datum in Kz (SI$/h$) fixes the linear calibration (Appendix~\ref{app:calibrations}).

\paragraph{What this establishes.}
C0 isolates the \emph{surface (area), not line} nature of conservation:
\emph{flux} \(\Phi\) is invariant across screens/tubes, while \emph{amplitudes} scale as \(1/A\).
Corollary~\ref{corollary:static-screen} specializes \eqref{eq:c0-sphere} to the \(3\)D static screen law and expresses
sector fields in constant–free, one–datum ratios; Corollary~\ref{corollary:info-em} treats wave export
(\(\langle|\mathbf E_I|\rangle\propto 1/\sqrt{A}\); cf.\ eq.~\eqref{eq:c4-amp-scaling}); and
Corollary~\ref{corollary:dimensional} collects the dimension–dependent static and wave scalings with confinement.

\medskip
\noindent\emph{Literature note.}
Background on differential forms/Stokes and potential theory appears in
Apps.~\ref{app:forms-stokes} and \ref{app:nD-screens}; see also Frankel \cite{Frankel2011}.
