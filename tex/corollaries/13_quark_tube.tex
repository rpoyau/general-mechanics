% ======================================================================
% 13_quark_tube.tex — QCD string (declared 1D tube; AF–Pure)
% ======================================================================

\subsection{QCD string (declared 1D tube)}
\label{corollary:qcd-string}

\paragraph{Setup.}
Frame form with \(\partial_t i=0\) on an open \(W\subset U\) where \(\sigma|_W=0\).
Declare an effective 1D flux‑tube with constant cross‑section \(A_\perp\).
Adopt the linear channel closure \eqref{axioms:linear-closure:eq}.

\paragraph{Derivation.}
From \eqref{corollary:conf:eq:1d}, the informational field inside the tube is uniform:
\begin{equation}
E_I\;=\;\frac{\Phi}{\kappa\,A_\perp}\;=:\;E_{\mathrm{string}}.
\label{corollary:qcd-string:eq:primary}
\end{equation}
With \(\mathbf E_I=-\nabla\Phi_I\) and a sector calibration \(U=c_{\mathrm{conf}}\Phi_I\),
the separation potential is
\begin{equation}
U(r)=U_0+\sigma r,\qquad \sigma=c_{\mathrm{conf}}\,E_{\mathrm{string}} .
\label{corollary:qcd-string:eq:calib}
\end{equation}

\paragraph{Specialization.}
For a lattice‑anchored \(A_\perp=\pi r_0^2\), any single datum for \(\sigma\) fixes
\(c_{\mathrm{conf}}E_{\mathrm{string}}\).

\paragraph{Calibration.}
A typical datum \(\sigma\approx(0.44\,\mathrm{GeV})^2\) sets the slope in the chosen sector.

\paragraph{Falsification.}
If the measured force \(F=dU/dr\) differs from a constant by \(>3\%\) for \(r\gtrsim1.5\,\mathrm{fm}\)
in the declared 1D regime, the tube specialization is falsified.

\medskip
\noindent\emph{Provenance.}
\eqref{axioms:linear-closure:eq}, \eqref{corollary:conf:eq:1d},
\eqref{corollary:qcd-string:eq:primary}–\eqref{corollary:qcd-string:eq:calib}.

\medskip
\noindent\emph{Literature note.}
External provenance only: lattice flux‑tube/string‑tension measurements (Wilson loops; Polyakov‑type formulations).
