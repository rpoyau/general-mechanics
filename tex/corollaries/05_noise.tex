% ======================================================================
% 05_noise.tex — Noise floor ("force cloud")
% ======================================================================

\subsection{Noise floor (``force cloud'')}
\label{corollary:noise}

This corollary quantifies the estimator floor on the informational field from boundary counting
and shows how it propagates to calibrated sectors. Let \(A\) be a \emph{boundary patch} with outward unit
normal \(n\), observed over a window \([t,t{+}\Delta t]\). Define the inward flux and the patch‑mean
normal flux density as
\begin{equation}
J_A \;:=\; \int_A \big(\mathbf j\!\cdot n\big)_{-}\, dA,
\qquad
\overline{j}_A \;:=\; \frac{J_A}{|A|}.
\label{corollary:noise:eq:flux}
\end{equation}

\paragraph{Windowed counting witness (definitions and operational bound).}
Let \(N_A(\Delta t)\) be the inward counts on \(A\) over the window \(\Delta t\). The mean count and
windowed Fano factor are
\begin{equation}
\mu \;:=\; \mathbb E[N_A]\;=\; J_A\,\Delta t,
\qquad
\mathrm{Var}\!\big(N_A\big)\;=\;F_A\,\mu .
\label{corollary:noise:eq:mean-fano}
\end{equation}
If arrivals exhibit temporal correlation with effective correlation time \(\tau>0\),
define an \emph{effective window} \(\Delta t_{\!\mathrm{eff}}\le \Delta t\) from the count
autocorrelation on the same record (App.~\ref{app:noise-kernel}). Any boundary estimator of the
patch‑mean flux density admits the operational benchmark
\begin{equation}
\mathrm{Std}\!\big(\widehat{\overline{j}}_A\big)
\;\approx\;
\sqrt{\frac{F_A\,\overline{j}_A}{|A|\,\Delta t_{\!\mathrm{eff}}}} .
\label{corollary:noise:eq:crb-jbar}
\end{equation}
Shot‑noise–limited records have \(F_A\!\approx\!1\) and \(\Delta t_{\!\mathrm{eff}}\!\approx\!\Delta t\);
overdispersed or correlated arrivals are captured by the measured \((F_A,\Delta t_{\!\mathrm{eff}})\).

\paragraph{From \(\mathbf j\) to the informational field \(\mathbf E_I\) (frame mapping).}
Using the frame closure \eqref{axioms:stokes:frame:eq:closure}, \(\mathbf j=-\kappa\,\mathbf E_I\),
the \emph{windowed} floor for \(\mathbf E_I\) is
\begin{equation}
\boxed{\quad
\mathrm{Var}(\widehat{E}_I)\ \gtrsim\ \frac{F_A\,\overline{j}_A}{\kappa^{2}\,|A|\,\Delta t_{\!\mathrm{eff}}},
\qquad
\sigma_{E_I}(\Delta t)\ \gtrsim\ \frac{1}{\kappa}\,
\sqrt{\frac{F_A\,\overline{j}_A}{|A|\,\Delta t_{\!\mathrm{eff}}}} \ .
\quad}
\label{corollary:noise:eq:var-EI}
\end{equation}
In the shot‑noise limit \((F_A\!\approx\!1,\ \Delta t_{\!\mathrm{eff}}\!\approx\!\Delta t)\),
\begin{equation}
\sigma_{E_I}(\Delta t)\ \gtrsim\ \frac{1}{\kappa}\sqrt{\frac{\overline{j}_A}{|A|\,\Delta t}} .
\label{corollary:noise:eq:var-EI-shot}
\end{equation}
Thus the \emph{force estimate} has a windowed cloud whose width decreases as \(\Delta t_{\!\mathrm{eff}}^{-1/2}\)
unless \(|A|\) or the collected inward flux \(J_A\) is increased.

\paragraph{Finite correlation time.}
If arrivals have an effective correlation time \(\tau>0\) (dead time, coherence, batching), the number
of independent samples in \(\Delta t\) is \(\approx \Delta t/\tau\), giving
\begin{equation}
\mathrm{Var}(\widehat{\overline{j}}_A)\ \gtrsim\ \frac{\overline{j}_A}{|A|}\,\frac{\tau}{\Delta t},
\qquad
\sigma_{E_I}\ \gtrsim\ \frac{1}{\kappa}\sqrt{\frac{\overline{j}_A}{|A|}\,\frac{\tau}{\Delta t}} .
\label{corollary:noise:eq:corr}
\end{equation}
This matches \eqref{corollary:noise:eq:crb-jbar} upon identifying
\(\Delta t_{\!\mathrm{eff}}\!\sim\!\Delta t/\tau\) up to order‑one factors set by the measured autocorrelation.

\paragraph{Spherical boundary (3D, far‑field).}
In the far field, for a full boundary \(A=4\pi r^2\) around a steady source with inward flux \(J_{4\pi}\),
\begin{equation}
\overline{j}_A=\frac{J_{4\pi}}{4\pi r^2},
\qquad
\sigma_{E_I}\ \gtrsim\ \frac{1}{\kappa}\,\frac{1}{4\pi r^2}\,\sqrt{\frac{J_{4\pi}}{\Delta t}} .
\label{corollary:noise:eq:spherical}
\end{equation}
Directional diaphragms use the aperture area for \(|A|\) and replace \(J_{4\pi}\) by the collected inward flux.

\paragraph{Measured sectors (one‑datum calibration).}
With the sector mappings (one‑datum calibration; see \S\ref{operationalization:calibration} and App.~\ref{app:calibrations})
\begin{equation}
\mathbf g=c_M\,\mathbf E_I,\qquad \mathbf E=c_Q\,\mathbf E_I,
\label{corollary:noise:eq:sector-maps}
\end{equation}
the estimator widths propagate as
\begin{equation}
\sigma_{g}\ \gtrsim\ \frac{c_M}{\kappa}\sqrt{\frac{F_A\,\overline{j}_A}{|A|\,\Delta t_{\!\mathrm{eff}}}},
\qquad
\sigma_{E}\ \gtrsim\ \frac{c_Q}{\kappa}\sqrt{\frac{F_A\,\overline{j}_A}{|A|\,\Delta t_{\!\mathrm{eff}}}} .
\label{corollary:noise:eq:measured}
\end{equation}
(Shot‑noise limit: replace \(F_A\to1\) and \(\Delta t_{\!\mathrm{eff}}\to\Delta t\).)
An acceleration estimate \(a(r)=|\mathbf g|\) inherits the same \(\Delta t_{\!\mathrm{eff}}^{-1/2}\) floor.

\paragraph{Trit‑density picture (boundary shot noise).}
For a spherical boundary with \emph{trit density} \(N(r)=\alpha\,4\pi r^2\) and mean per‑trit hit rate \(f\) (1/time),
the mean inward count in \(\Delta t\) is \(\mu=f\,N(r)\,\Delta t\) and
\[
\frac{\sigma_{\text{counts}}}{\mu}\ \approx\ \frac{1}{\sqrt{\mu}}
\ =\ \frac{1}{\sqrt{f\,\alpha\,4\pi r^2\,\Delta t}} .
\]
Mapping counts to \(\mathbf E_I\) through \eqref{corollary:noise:eq:var-EI-shot} reproduces the same
\(1/\sqrt{|A|\,\Delta t}\) scaling.

\paragraph{Relation to time–information and capacity.}
The counting term in the rate bound (\S\ref{corollary:time-info}) is precisely
\(\overline{J}_{A}=\int_A(\mathbf j\!\cdot n)_-\,dA/\Delta t\).
Equations \eqref{corollary:noise:eq:var-EI}–\eqref{corollary:noise:eq:corr} give the corresponding estimator
width for \(\mathbf E_I\), showing that shorter windows both require more informational flux and enlarge the
force cloud. In applications that form durable records and later erase/compress them, the \(P/T\) resource in
\S\ref{corollary:time-info} provides an additional operational limit; the variance floor here arises from counting statistics.

\paragraph{Noise kernel link (App.~\ref{app:noise-kernel}).}
The local bounds above define a minimal \emph{noise kernel} for flux fluctuations on the boundary.
App.~\ref{app:noise-kernel} uses such kernels \(\mathcal{N}\) to propagate covariances through linear response
(Einstein–Langevin–type) and to connect windowed shot noise to dynamical spectra.

\paragraph{Units and practical levers.}
Fluxes are in \emph{trits/time} by default (nats/time via a factor of \(\ln 3\)).
To tighten the cloud: enlarge \(|A|\), increase the inward flux \(J_A\) (e.g., via wider apertures or focusing),
extend \(\Delta t\), reduce correlation time \(\tau\), or multiplex independent patches and average.

\medskip
\noindent\emph{Summary.}
Windowed counting with or without finite correlation time implies an operational floor
\(\sigma_{E_I}\propto \sqrt{\overline{j}_A/(|A|\Delta t_{\!\mathrm{eff}})}\).
This sets a principled ``force cloud'' that narrows only with larger area, greater inward flux, or longer
integration time, and it propagates to calibrated sectors by a one‑datum mapping.

\medskip
\noindent\emph{Literature note.}
Background on Poisson/renewal counting and shot noise: Kingman~\cite{Kingman1993}.
Stochastic/Langevin and spectral viewpoints: Gardiner~\cite{Gardiner2004}.
App.~\ref{app:noise-kernel} records the windowed noise‑kernel and covariance‑propagation formulas.
