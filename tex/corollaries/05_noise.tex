% ======================================================================
% 05_noise.tex — Corollary: Boundary-window estimator floors & spectral forms (AF–Pure)
% ======================================================================

\subsection{Noise on boundary windows}
\label{corollary:noise}

\paragraph{Setup.}
A boundary patch \(A\subset\partial U\) is observed over a window \([t,t+\Delta t]\) (\S\ref{axioms:window}).
Window counts are denoted \(N_A(\Delta t)\).
The windowed Fano factor is \(F_A:=\mathrm{Var}(N_A)/\mathbb E[N_A]\).
An effective duration \(\Delta t_{\!\mathrm{eff}}\le\Delta t\) is obtained from the count autocorrelation on the same record.
A far‑field, low‑frequency regime is assumed (\(\omega r\ll c_I\)) outside the source.
Arrivals may be characterized by a correlation time \(\tau>0\).
For a source family labeled by total energy \(E_{\rm src}\), the boundary flux satisfies
\begin{equation}
\Phi\ \propto\ E_{\rm src}.
\label{corollary:noise:eq:label}
\end{equation}

\paragraph{Estimator floor (boundary window).}
Using the directional split \S\ref{axioms:directional-split}, the inward patch integral and patch‑mean density are
\[
J_A:=\int_A \big(\mathbf j\!\cdot \mathbf n\big)_{-}\,\mathrm dA,
\qquad
j_A:=\frac{\mathbb E[N_A]}{|A|\,\Delta t}=\frac{J_A}{|A|}.
\]
With the Linear Channel Closure \eqref{axioms:linear-closure:eq},
\begin{equation}
\mathrm{Var}(\widehat{E}_I)
\;\gtrsim\;
\frac{F_A\,j_A}{\kappa^{2}\,|A|\,\Delta t_{\!\mathrm{eff}}}.
\label{corollary:noise:eq:estimator-floor}
\end{equation}
Shot‑noise specialization: \(F_A\to 1\) and \(\Delta t_{\!\mathrm{eff}}\to \Delta t\).

\paragraph{Correlated arrivals (time constant \(\tau\)).}
If arrivals exhibit a correlation time \(\tau>0\), the variance bound \eqref{corollary:noise:eq:estimator-floor} is
\begin{equation}
\mathrm{Var}(\widehat{E}_I)
\;\gtrsim\;
\frac{j_A}{\kappa^{2}\,|A|}\,\frac{\tau}{\Delta t}.
\label{corollary:noise:eq:tau-floor}
\end{equation}

\paragraph{Energy‑rate fluctuations \(\Rightarrow\) acceleration noise (far field, frame).}
On a frame in a source‑free annulus, the inverse‑area law \eqref{corollary:info-gas:eq:sphere} implies
\(|\mathbf E_I(r)|=\frac{1}{\kappa}\,\frac{\Phi}{A(r)}\).
At fixed geometry \((r,A)\), \eqref{corollary:noise:eq:label} yields
\begin{equation}
\frac{\delta Y(r,t)}{Y(r)}=\frac{\delta\Phi}{\Phi}
=\frac{\delta E_{\rm src}(t)}{E_{\rm src}} .
\label{corollary:noise:eq:ratio}
\end{equation}
For an acceleration readout \(Y\equiv g\),
\begin{equation}
\boxed{\ \ \delta g(r,t)=g(r)\,\frac{\delta E_{\rm src}(t)}{E_{\rm src}}\ .\ }
\label{corollary:noise:eq:delta-g}
\end{equation}
In one‑sided PSD form,
\begin{equation}
\boxed{\ \ S_g(\omega)=\left(\frac{g(r)}{E_{\rm src}}\right)^{\!2}\,S_{E}(\omega)\ .\ }
\label{corollary:noise:eq:psd}
\end{equation}
For a canonical ensemble with heat capacity \(C\) within band (white \(S_E\)),
\begin{equation}
\boxed{\ \ g_{\mathrm{rms}}(r)\ \approx\ g(r)\,\frac{\sqrt{k_B T^2 C}}{E_{\rm src}}\ .\ }
\label{corollary:noise:eq:rms}
\end{equation}

\medskip
\noindent\emph{Literature note.}
Hu–Verdaguer~\cite{HuVerdaguer2008}; Kingman~\cite{Kingman1993}.
