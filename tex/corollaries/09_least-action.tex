% ======================================================================
% corollaries/09_least-action.tex — Least‑action equivalence (AF‑Pure)
% ======================================================================

\subsection{Least‑action equivalence}
\label{corollary:least-action}

\paragraph{Setup.}
On \((q,t)\) as in \S\ref{axioms:path-rate}, the canonical rate equations \eqref{axioms:path-rate:eq:canonical} hold.
Legendre regularity (strict convexity in \(P\)) is assumed on the sector.

\paragraph{Derivation.}
Define the Legendre pair
\begin{equation}
\mathcal R(q,\dot q,t)
\;:=\;
\sup_{P}\,\big(P_i\dot q^i-H(q,P,t)\big),
\qquad
H(q,P,t)=\sup_{\dot q}\,\big(P_i\dot q^i-\mathcal R(q,\dot q,t)\big).
\label{corollary:least-action:eq:Legendre}
\end{equation}
Using the Poincaré–Cartan form \(\Theta_{\mathcal R}=P_i\,dq^i- H\,dt\) (\S\ref{axioms:path-rate}),
closure on rectangular \(\Sigma\) implies \(\int_{\partial\Sigma}\Theta_{\mathcal R}=0\).
Hence the variation of
\begin{equation}
\mathcal S[q]\;:=\;\int_{t_0}^{t_1}\!\mathcal R\big(q,\dot q,t\big)\,dt
\label{corollary:least-action:eq:action}
\end{equation}
vanishes for fixed endpoints, \(\delta\mathcal S[q]=0\).
Therefore the Euler–Lagrange equations
\begin{equation}
\frac{d}{dt}\Big(\frac{\partial \mathcal R}{\partial \dot q^i}\Big)
\;-\;
\frac{\partial \mathcal R}{\partial q^i}
\;=\;0
\label{corollary:least-action:eq:primary}
\end{equation}
are equivalent to the canonical rates \eqref{axioms:path-rate:eq:canonical}.

\paragraph{Specialization.}
At the stationary pair,
\(\dot q^i=\partial_{P_i}H\) and \(P_i=\partial_{\dot q^i}\mathcal R\); with \(\mathcal R=P\!\cdot\dot q-H\),
one recovers \(\dot P_i=-\partial_{q^i}H\).

\paragraph{Falsification.}
On a short window with fixed endpoints, if \(\int_{\partial\Sigma}\Theta_{\mathcal R}\neq 0\) under the declared regularity and boundary conditions,
the equivalence between \eqref{corollary:least-action:eq:primary} and \eqref{axioms:path-rate:eq:canonical} is violated for the modelling choice at hand.

\medskip
\noindent\emph{Provenance.}
\eqref{axioms:path-rate:eq:pc-form}, \eqref{axioms:path-rate:eq:canonical},
\eqref{corollary:least-action:eq:Legendre}–\eqref{corollary:least-action:eq:primary}.

\medskip
\noindent\emph{Literature note.}
Equivalence of Hamilton/Euler–Lagrange via the Poincaré–Cartan form: Arnold~\cite{Arnold1989};
geometric mechanics (Abraham–Marsden)~\cite{AbrahamMarsden1978}; convex duality and Legendre transforms:
Rockafellar~\cite{Rockafellar1970}.
