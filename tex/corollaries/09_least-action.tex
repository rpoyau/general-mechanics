% ======================================================================
% corollaries/09_least-action.tex — Least‑action equivalence
% ======================================================================

\subsection{Least‑action equivalence}
\label{corollary:least-action}

\paragraph{Setup.}
On configuration–time space \((q,t)\) as in \S\ref{axioms:path-rate}, the canonical rate equations
\eqref{axioms:path-rate:eq:canonical} hold. Legendre regularity is assumed on the sector of interest
(strict convexity in \(P\)).

\paragraph{Legendre pair and rate functional.}
The rate functional \(\mathcal R\) is defined by the Legendre transform in \(P\):
\begin{equation}
\mathcal R(q,\dot q,t)
\;:=\;
\sup_{P}\,\big(P_i\dot q^i-H(q,P,t)\big),
\qquad
H(q,P,t)=\sup_{\dot q}\,\big(P_i\dot q^i-\mathcal R(q,\dot q,t)\big).
\label{corollary:least-action:eq:Legendre}
\end{equation}
At the stationary pair,
\begin{equation}
\dot q^i=\partial_{P_i}H,
\qquad
P_i=\frac{\partial \mathcal R}{\partial \dot q^i}.
\label{corollary:least-action:eq:R-def}
\end{equation}
Using the Poincaré–Cartan form \(\Theta_{\mathcal R}\) from \eqref{axioms:path-rate:eq:pc-form}, \(\Theta_{\mathcal R}\) on solutions becomes
\begin{equation}
\Theta_{\mathcal R}
\;=\;
\big(P_i\dot q^i-H\big)\,dt
\;=\;
\mathcal R(q,\dot q,t)\,dt .
\label{corollary:least-action:eq:theta-as-Rdt}
\end{equation}

\paragraph{Boundary closure implies stationarity.}
Evaluating Stokes on a rectangular \(2\)-surface \(\Sigma\) spanned by two nearby curves between \(t_0\) and \(t_1\)
(as in \S\ref{axioms:path-rate}) yields \(\int_{\partial\Sigma}\Theta_{\mathcal R}=0\).
With \eqref{corollary:least-action:eq:theta-as-Rdt}, the boundary integral is the first variation of
\begin{equation}
\mathcal S[q]\;:=\;\int_{t_0}^{t_1}\!\mathcal R\big(q,\dot q,t\big)\,dt ,
\label{corollary:least-action:eq:action}
\end{equation}
hence
\begin{equation}
\delta\mathcal S[q]\;=\;0 \quad \text{for fixed endpoints } q(t_0),\,q(t_1).
\label{corollary:least-action:eq:deltaS0}
\end{equation}

\paragraph{Euler–Lagrange \(\Leftrightarrow\) canonical rates.}
Stationarity of \eqref{corollary:least-action:eq:action} gives
\begin{equation}
\frac{d}{dt}\Big(\frac{\partial \mathcal R}{\partial \dot q^i}\Big)
\;-\;
\frac{\partial \mathcal R}{\partial q^i}
\;=\;0 .
\label{corollary:least-action:eq:EL}
\end{equation}
With \(P_i=\partial_{\dot q^i}\mathcal R\) and \(\mathcal R=P\!\cdot\dot q-H\), this yields
\(\dot P_i=-\,\partial_{q^i}H\); together with \(\dot q^i=\partial_{P_i}H\) one recovers
\eqref{axioms:path-rate:eq:canonical}. Thus the Euler–Lagrange equations and Hamilton’s equations are equivalent,
with \(\Theta_{\mathcal R}\) given by \eqref{axioms:path-rate:eq:pc-form}.

\paragraph{Constraints, nonconvexity, and dissipation.}
If \(H\) is not strictly convex in \(P\), the Legendre transform is replaced by a convex dual or constraints
are imposed with Lagrange multipliers; the equivalence holds on the constrained submanifold. Nonconservative or driven channels
arise from explicit \(t\)-dependence or nonpotential terms in \(H(q,P,t)\), producing the corresponding
Rayleigh‑type contributions.

\medskip
\noindent\emph{Literature note.}
Equivalence of Hamilton/Euler–Lagrange via the Poincaré–Cartan form: Arnold~\cite{Arnold1989};
geometric mechanics (Abraham–Marsden)~\cite{AbrahamMarsden1978}; convex duality and Legendre transforms:
Rockafellar~\cite{Rockafellar1970}.
