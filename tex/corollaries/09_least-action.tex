% ======================================================================
% corollaries/09_least-action.tex — Least‑action equivalence (derived)
% ======================================================================

\subsection{Least‑action equivalence (derived, not assumed)}
\label{corollary:least-action}

This corollary shows that the usual least‑action (Euler–Lagrange) equations are \emph{equivalent} to the
canonical rate equations obtained from a single boundary closure on a $2$‑surface in $(q,t)$.
No action postulate is assumed; the variational statement is a repackaging of the same closure.

\paragraph{Rate $1$‑form and canonical rates.}
Define the Poincar\'e--Cartan (rate) $1$‑form
\begin{equation}
\Theta_{\mathcal R}(q,P,t)\;=\;P_i\,dq^i\;-\;H(q,P,t)\,dt,
\label{corollary:least-action:eq:pc-form}
\end{equation}
and use the closure $d\Theta_{\mathcal R}=0$ on admissible rectangles in $(q,t)$.
In local coordinates this is equivalent to the canonical rate equations
\begin{equation}
\dot q^i=\partial_{P_i}H,
\qquad
\dot P_i=-\,\partial_{q^i}H ,
\label{corollary:least-action:eq:canonical}
\end{equation}
cf.\ the axiom‑level statement \eqref{axioms:path-rate:eq:pc-form}–\eqref{axioms:path-rate:eq:canonical}.

\paragraph{Legendre pair and rate functional (regularity).}
Assume standard regularity (strict convexity in $P$ on the sector of interest) so the Legendre map is well‑posed.
Introduce the rate functional by a Legendre transform in $P$:
\begin{equation}
\mathcal R(q,\dot q,t)
\;:=\;
\sup_{P}\,\big(P_i\dot q^i-H(q,P,t)\big),
\qquad
H(q,P,t)=\sup_{\dot q}\,\big(P_i\dot q^i-\mathcal R(q,\dot q,t)\big).
\label{corollary:least-action:eq:Legendre}
\end{equation}
At the stationary pair $(\dot q,P)$ one has the first‑order relations
\begin{equation}
\dot q^i=\partial_{P_i}H,
\qquad
P_i=\frac{\partial \mathcal R}{\partial \dot q^i}.
\label{corollary:least-action:eq:R-def}
\end{equation}
On solutions (at the maximizing $P$) the $1$‑form becomes
\begin{equation}
\Theta_{\mathcal R}\;=\;\big(P_i\dot q^i-H\big)\,dt\;=\;\mathcal R(q,\dot q,t)\,dt .
\label{corollary:least-action:eq:theta-as-Rdt}
\end{equation}

\paragraph{Boundary closure $\Rightarrow$ stationarity of the rate functional.}
Let $\Sigma$ be the rectangular $2$‑surface spanned by two nearby curves between $(t_0,t_1)$.
By Stokes,
\begin{equation}
\int_{\partial\Sigma}\Theta_{\mathcal R}\;=\;\int_{\Sigma}d\Theta_{\mathcal R}\;=\;0 .
\label{corollary:least-action:eq:stokes}
\end{equation}
The boundary integral is the first variation of the \emph{rate functional}
\begin{equation}
\mathcal S[q]\;:=\;\int_{t_0}^{t_1}\!\mathcal R\big(q,\dot q,t\big)\,dt ,
\label{corollary:least-action:eq:action}
\end{equation}
so \eqref{corollary:least-action:eq:stokes} implies
\begin{equation}
\delta\mathcal S[q]\;=\;0 \quad \text{for fixed endpoints } q(t_0),\,q(t_1).
\label{corollary:least-action:eq:deltaS0}
\end{equation}

\paragraph{Euler--Lagrange $\Leftrightarrow$ canonical rates.}
Stationarity of \eqref{corollary:least-action:eq:action} yields the Euler--Lagrange equations
\begin{equation}
\frac{d}{dt}\Big(\frac{\partial \mathcal R}{\partial \dot q^i}\Big)
\;-\;
\frac{\partial \mathcal R}{\partial q^i}
\;=\;0 .
\label{corollary:least-action:eq:EL}
\end{equation}
Using $P_i=\partial_{\dot q^i}\mathcal R$ and $\mathcal R=P\!\cdot\dot q-H$, one obtains
$\dot P_i=-\,\partial_{q^i}H$ and, together with $\dot q^i=\partial_{P_i}H$, recovers
the canonical system \eqref{corollary:least-action:eq:canonical}. Thus these are
Hamilton’s equations for the generator $H$:
\begin{equation}
\dot q^i=\partial_{P_i}H, \qquad \dot P_i=-\,\partial_{q^i}H ,
\label{corollary:least-action:eq:Hamilton}
\end{equation}
with $\Theta_{\mathcal R}=P_i\,dq^i-H\,dt$ the Poincar\'e--Cartan form.

\paragraph{Constraints, nonconvexity, and dissipation.}
If $H$ is not strictly convex in $P$, replace the Legendre transform by a convex dual or impose
constraints via Lagrange multipliers; the equivalence holds on the constrained submanifold.
Nonconservative or driven channels are included by adding explicit $t$‑dependence or nonpotential
terms to $H(q,P,t)$, leading to the corresponding generalized Euler--Lagrange equations with
Rayleigh‑type contributions.

\paragraph{Field analogue (pointer).}
For fields $\varphi^a(x)$ with polymomenta $P_a^{\ \mu}$, the multisymplectic form
$\Theta_{\!\mathrm{DW}}$ and the closure $d\Theta_{\!\mathrm{DW}}=0$ reproduce the covariant De Donder--Weyl
equations; a $3{+}1$ split reduces to \eqref{corollary:least-action:eq:canonical}. See App.~\ref{app:ddw}.

\medskip
\noindent\emph{Summary.}
The least‑action principle is not an extra postulate here: it is the boundary restatement of the same
closure that yields the canonical rate equations. Under convex duality, Euler--Lagrange, Hamilton, and
the closure $d\Theta_{\mathcal R}=0$ are fully equivalent.

\medskip
\noindent\emph{Literature note.}
The Poincar\'e--Cartan form and Hamilton/Euler--Lagrange equivalence are standard
(see Arnold~\cite{Arnold1989} and Abraham--Marsden~\cite{AbrahamMarsden1978});
convex duality and Legendre transforms are summarized in Rockafellar~\cite{Rockafellar1970}.
The field/multisymplectic analogue is reviewed in App.~\ref{app:ddw} (cf.\ Kanatchikov~\cite{Kanatchikov1998};
Gotay--Isenberg--Marsden~\cite{GotayIsenbergMarsden1998}; Marsden--Patrick--Shkoller~\cite{MarsdenPatrickShkoller1998}).
