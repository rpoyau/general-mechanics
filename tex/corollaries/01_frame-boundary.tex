% ======================================================================
% 01_frame-boundary.tex — Frame boundary (inverse-power fields)
% ======================================================================

\subsection{Frame boundary (inverse-power fields)}
\label{corollary:frame-boundary}

\paragraph{Setup (frame form; source-free annulus).}
Work on a \frame (\S\ref{axioms:frame}) with \(\partial_t i=0\).
On an open annulus \(W\subset U\) with \(\sigma|_{W}\equiv 0\) (source-free on \(W\)),
define the directed flux through a closed boundary \(\Sigma(r)\subset W\) of radius \(r\) by
\[
\Phi(r):=\int_{\Sigma(r)} \mathbf j\!\cdot \mathbf n\,\mathrm dA.
\]
By the generalized Stokes identity on \(W\)
\eqref{axioms:stokes:identity:eq}–\eqref{axioms:stokes:identity:eq:cut}
(and equivalently by the \emph{cap/side (cut) identity}, \S\ref{axioms:worldtube}),
\(\Phi(r)\) is independent of \(r\) across nested boundaries in \(W\).

\paragraph{Boundary area in \(n\) spatial dimensions.}
For a spherical boundary in \(n\) spatial dimensions,
\[
A_n(r)=S_{n-1}\,r^{\,n-1},
\]
hence the cross‑sectional mean of the normal current and the frame amplitude are
\begin{equation}
\bigl\langle j_n\bigr\rangle(r)=\frac{\Phi}{A_n(r)}=\frac{\Phi}{S_{n-1}\,r^{\,n-1}},
\qquad
\bigl|\mathbf E_I(r)\bigr|=\frac{1}{\kappa}\,\bigl\langle j_n\bigr\rangle(r).
\label{corollary:frame-boundary:eq:boundary-law}
\end{equation}
Thus \(\bigl|\mathbf E_I(r)\bigr|\propto r^{-(n-1)}\).

\paragraph{3D specialization.}
For \(n=3\), \(A_3(r)=4\pi r^2\) and \eqref{corollary:frame-boundary:eq:boundary-law} reduces to
\begin{equation}
\bigl|\mathbf E_I(r)\bigr|=\frac{\Phi}{4\pi\kappa}\,\frac{1}{r^{2}} .
\label{corollary:frame-boundary:eq:3d}
\end{equation}

\paragraph{Calibrated sectors (constant–free predictions).}
Adopt linear sector mappings \(g=c_M\,|\mathbf E_I|\) and \(E=c_Q\,|\mathbf E_I|\).
Then the measured \(3\)D fields are
\begin{equation}
g(r)=\frac{c_M\,\Phi}{4\pi\kappa}\,\frac{1}{r^{2}},\qquad
E(r)=\frac{c_Q\,\Phi}{4\pi\kappa}\,\frac{1}{r^{2}} .
\label{corollary:frame-boundary:eq:measured-3d}
\end{equation}
A single datum \((r_\star, g_\star)\) (or \((r_\star, E_\star)\)) for a given source family yields the
constant–free ratio form
\begin{equation}
g(r)=g_\star\left(\frac{\Phi}{\Phi_\star}\right)\left(\frac{r_\star}{r}\right)^{\!2},\qquad
E(r)=E_\star\left(\frac{\Phi}{\Phi_\star}\right)\left(\frac{r_\star}{r}\right)^{\!2}.
\label{corollary:frame-boundary:eq:ratio-MQ}
\end{equation}

\paragraph{Directional filtering and sectors.}
For apertures or diaphragms, the inward component
\[
\Phi_- \;:=\; \int_{\Sigma} (\mathbf j\!\cdot \mathbf n)_-\,\mathrm dA
\]
(using the directional split in \S\ref{axioms:directional-split}) specifies the flux.
Equations \eqref{corollary:frame-boundary:eq:boundary-law}–\eqref{corollary:frame-boundary:eq:ratio-MQ}
are valid under the substitution \(\Phi \mapsto \Phi_-\).
For a fixed solid angle \(\Delta\Omega\) in \(n\) spatial dimensions, the sector area is
\(A_\Delta(r)=\Delta\Omega\,r^{\,n-1}\).

\medskip
\noindent\emph{Literature note.}
Frankel~\cite{Frankel2011} and Jackson~\cite{Jackson1999}, and the
dimension–dependence summarized in Corollary~\ref{corollary:dimensional}.
