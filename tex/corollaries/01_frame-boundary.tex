% ======================================================================
% 01_frame-boundary.tex — Frame boundary (inverse-power fields, AF‑Pure)
% ======================================================================

\subsection{Frame boundary (inverse-power fields)}
\label{corollary:frame-boundary}

\paragraph{Setup.}
Work on a frame with \(\partial_t i=0\) on an open annulus \(W\subset U\) where \(\sigma|_W=0\).
Let \(\Sigma(r)\subset W\) be nested spherical boundaries of radius \(r\) with area \(A_n(r)\).
Define the directed flux \(\Phi(r):=\int_{\Sigma(r)} \mathbf j\!\cdot\!\mathbf n\,\mathrm dA\).

\paragraph{Derivation.}
From the Stokes identity in integral/cut form
\(\eqref{axioms:stokes:identity:eq:integral}\)–\(\eqref{axioms:stokes:identity:eq:cut}\) on \(W\),
\(\Phi(r)\) is independent of \(r\).
Therefore
\begin{equation}
\big\langle j_n\big\rangle(r)=\frac{\Phi}{A_n(r)}.
\label{corollary:frame-boundary:eq:primary}
\end{equation}\label{corollary:frame-boundary:eq:boundary-law}
With the Linear Channel Closure \(\eqref{axioms:linear-closure:eq}\), the frame amplitude satisfies
\begin{equation}
\big|\mathbf E_I(r)\big|=\frac{1}{\kappa}\,\frac{\Phi}{A_n(r)}.
\label{corollary:frame-boundary:eq:calib}
\end{equation}
Directional filtering: with the inward component
\(\Phi_-:=\int_{\Sigma}(\mathbf j\!\cdot\!\mathbf n)_-\,\mathrm dA\),
relations \eqref{corollary:frame-boundary:eq:primary}–\eqref{corollary:frame-boundary:eq:calib}
hold under the substitution \(\Phi\mapsto\Phi_-\).

\paragraph{Specialization.}
In \(n=3\), \(A_3(r)=4\pi r^{2}\), giving
\begin{equation}
\big|\mathbf E_I(r)\big|=\frac{\Phi}{4\pi\,\kappa}\,\frac{1}{r^{2}}.
\label{corollary:frame-boundary:eq:special}
\end{equation}\label{corollary:frame-boundary:eq:3d}

\paragraph{Calibration.}
With linear sector mappings \(g=c_M\,|\mathbf E_I|\) and \(E=c_Q\,|\mathbf E_I|\),
\begin{equation}
g(r)=\frac{c_M\,\Phi}{4\pi\,\kappa}\,\frac{1}{r^{2}},\qquad
E(r)=\frac{c_Q\,\Phi}{4\pi\,\kappa}\,\frac{1}{r^{2}}.
\label{corollary:frame-boundary:eq:measured-3d}
\end{equation}
A single datum \((r_\star,Y_\star)\) in a given sector yields the constant–free ratio
\begin{equation}
Y(r)=Y_\star\left(\frac{\Phi}{\Phi_\star}\right)\left(\frac{r_\star}{r}\right)^{\!2},\qquad Y\in\{g,E\}.
\label{corollary:frame-boundary:eq:ratio-MQ}
\end{equation}

\paragraph{Falsification.}
On a source‑free frame, if \(r^{2}\big|\mathbf E_I(r)\big|\) departs from a constant by more than a declared tolerance
over a stated \(r\)-range, the premises of this corollary are contradicted.

\medskip
\noindent\emph{Provenance.}
\(\eqref{axioms:stokes:identity:eq:integral}\),
\(\eqref{axioms:stokes:identity:eq:cut}\),
\(\eqref{axioms:linear-closure:eq}\),
\(\eqref{corollary:frame-boundary:eq:primary}\).

\medskip
\noindent\emph{Literature note.}
Frankel~\cite{Frankel2011}; Jackson~\cite{Jackson1999}.
