% ======================================================================
% _template_pure.tex — Corollary template (AF‑Pure, no label magic)
% ======================================================================

% ------------------------ Editor-only guidance (comments) --------------
% How to instantiate this template:
% 1) Replace [TITLE] and [LABEL] below.
% 2) In "Setup.", state initial conditions and geometric context DE-CLAR-A-TIVE-LY
%    (e.g., frame/window regime, source-free set, regularity). No instructions.
% 3) In "Derivation.", INSERT your governing identity and any closure by LABEL
%    inline (e.g., "... from \eqref{<your-identity>} and \eqref{<your-closure>} ...").
%    Do NOT restate axioms; only cite them. Keep the statement neutral (“From…, … obeys …”).
% 4) Insert the primary relation in the equation block. No motivational prose.
% 5) Optional: add a concise specialization (geometry/regime) and its equation.
% 6) Optional: add a calibration paragraph if the note defines a one-datum scale elsewhere.
% 7) Literature note: EXTERNAL references only (textbooks/papers). No self-citations.
% ----------------------------------------------------------------------

\subsection{[TITLE]}
\label{corollary:[LABEL]}

\paragraph{Setup.}
% State initial conditions and geometric context in one or two sentences.
% Example: Frame form with \(\partial_t i=0\) on an open \(W\subset U\) and \(\sigma|_W=0\).
% Example: Wave window with homogeneous constitutive map on the segment; sources excluded.

\paragraph{Derivation.}
% Cite identities/closures INLINE here (insert your labels); do not restate them.
% Example: From \eqref{<identity-label>} and, when linear response is assumed, \eqref{<closure-label>},
% the quantity satisfies:
\begin{equation}
% [Insert primary relation here, e.g., \langle j_n\rangle(r)=\Phi/A_n(r)]
\end{equation}

% ------------------------- Optional specialization ---------------------
\paragraph{Specialization.}
% One neutral sentence indicating the regime/geometry specialization.
% Example: For a spherical boundary in \(n\) spatial dimensions with area \(A_n(r)\),
% \eqref{corollary:[LABEL]:eq:primary} reduces to:
\begin{equation}
% [Insert specialized relation, e.g., |\mathbf E_I(r)|=\Phi/(\kappa S_{n-1})\,r^{-(n-1)}]
\end{equation}

% --------------------------- Optional calibration ----------------------
\paragraph{Calibration.}
% One neutral sentence tying measured readouts to the primary amplitude by a one-datum scale
% already defined elsewhere in the note (cite that label inline; do not repeat formulas).
% Example: With the one–datum calibration (cite label), measured sector readouts satisfy:
\begin{equation}
% [Insert calibrated/measured relation(s), e.g., g(r)=g_\star(A_\star/A_n(r))]
\end{equation}

\medskip
\noindent\emph{Literature note.}
% External sources only (no self-citations). Keep terse; provenance, not premises.
% Example: Frankel~[forms/Stokes]; Jackson~[inverse-power fields]; Callen~[macroscopic thermodynamics].
