% ======================================================================
% 03c08_dimensional-flows-confinement.tex — Dimensional flows & confinement
% ======================================================================

\subsection{Dimensional flows \& confinement}
\label{corollary:dimensional}

This corollary makes explicit how \emph{frame} and \emph{wave} amplitudes scale with the number of accessible spatial
dimensions and how boundaries reduce the \emph{effective} dimension. All statements follow from the Stokes
identity \eqref{axioms:stokes:identity:eq}, the frame linear closure \eqref{axioms:stokes:frame:eq:closure}, and
(for waves) the local balance (Poynting) identity \eqref{waves:eq:local-balance} (see §\ref{waves}; App.~\ref{app:forms-stokes}).
No new constants are introduced.

% ------------------------- Summary (frame vs waves) --------------------

\paragraph{Summary (frame vs.\ waves).}
Let \(A_n(r)\) be the area of a radius–\(r\) \emph{boundary} in \(n\) spatial dimensions
(App.~\ref{app:nD-boundaries}, eq.~\eqref{app:nD-boundaries:eq:areas}). On a steady, source‑free annulus the tube/boundary flux
\(\Phi\) is constant (§\ref{corollary:info-gas}), so the \textbf{frame} mean and amplitude follow
(§\ref{corollary:frame-boundary})
\begin{equation}
\big\langle j_n\big\rangle(r)=\frac{\Phi}{A_n(r)},\qquad
\big|\mathbf E_I(r)\big|=\frac{1}{\kappa}\,\frac{\Phi}{A_n(r)} .
\label{eq:c8-static}
\end{equation}
For spherical or conical boundaries with \(A_n(r)=S_{n-1}\,r^{\,n-1}\) this gives \(|\mathbf E_I|\propto r^{-(n-1)}\).
For \textbf{wave} segments in homogeneous media, time‑averaging the local balance on a source‑free annulus yields
radius‑independent exported power (see §\ref{waves}),
\begin{equation}
P(r):=\int_{\Sigma(r)}\!\big\langle \mathbf S_I\big\rangle\!\cdot n\,dA=\text{const in }r ,
\label{eq:c8-power}
\end{equation}
and with \(\langle \mathbf S_I\rangle\propto \langle|\mathbf E_I|^2\rangle\) the far‑field envelope scales as
\begin{equation}
\big\langle |\mathbf E_I(r)| \big\rangle \;\propto\; \frac{1}{\sqrt{A_n(r)}}
\quad\text{(cf.\ §\ref{waves}, eq.~\eqref{waves:eq:envelope})}.
\label{eq:c8-waves}
\end{equation}
For spherical/conical boundaries this gives \(\langle|\mathbf E_I|\rangle\propto r^{-(n-1)/2}\).

% ------------------------- Dimensional table ---------------------------

\paragraph{Dimensional summary (table).}
\begin{table}[t]
\centering
\caption{Geometric scaling of \textbf{frame amplitudes} and \textbf{wave envelopes} by effective spatial dimension.
Here \(A_n(r)\) is the \(n\)D boundary area (App.~\ref{app:nD-boundaries}). Frame scaling follows \eqref{eq:c8-static};
wave envelopes follow \eqref{eq:c8-power}–\eqref{eq:c8-waves}.}
\label{tab:c8-dim}
\begin{tabular}{lccc}
\hline
\textbf{Effective dimension} & \textbf{Area \(A_n(r)\)} & \textbf{Frame \(|\mathbf E_I|\)} & \textbf{Wave \(\langle|\mathbf E_I|\rangle\)} \\
\hline
1D (guide / tube)      & \(A_\perp\) (const.)   & \(\propto r^{0}\)   & \(\propto r^{0}\) \\
2D (sheet / interface) & \(A_2(r)\propto r\)    & \(\propto r^{-1}\)  & \(\propto r^{-1/2}\) \\
3D (spherical)         & \(A_3(r)=4\pi r^{2}\)  & \(\propto r^{-2}\)  & \(\propto r^{-1}\) \\
\hline
\end{tabular}
\end{table}

\noindent\emph{Sector note.} For a fixed solid‑angle \emph{boundary sector} \(\Delta\Omega\) in \(n\)D,
replace \(A_n(r)\) by \(A_\Delta(r)=\Delta\Omega\,r^{\,n-1}\) (App.~\ref{app:nD-boundaries}); the exponents are unchanged.

% ------------------------- Confinement cases ----------------------------

\paragraph{Confinement cases (effective dimension).}
Boundaries that constrain propagation reduce the accessible dimension \(n_{\rm eff}\):
\begin{itemize}\itemsep2pt
  \item \textbf{Waveguides (effective 1D).} Cross–section \(A_\perp\) fixed along the axis. \textbf{Frame:}
  \(\langle j_n\rangle=\Phi/A_\perp\) and \( |\mathbf E_I|=\Phi/(\kappa A_\perp)\) are independent of \(r\).
  \textbf{Waves:} \(\langle|\mathbf E_I|\rangle\sim r^0\) along the guide; mode power is exported with \eqref{eq:c8-power}.
  \item \textbf{Sheets / interfaces (effective 2D).} Fixed thickness \(\Rightarrow A_2(r)\propto r\).
  \textbf{Frame:} \( |\mathbf E_I|\propto 1/r\). \textbf{Waves:} \( \langle|\mathbf E_I|\rangle\propto 1/\sqrt{r}\).
\end{itemize}
Mode spectra and group velocities are set by boundary conditions and any adopted constitutive map \(\chi\).

% ------------------------- Directional sectors -------------------------

\paragraph{Directional sectors and apertures.}
For an aperture subtending \(\Delta\Omega\) (in \(n\)D), use \(A_\Delta(r)=\Delta\Omega\,r^{\,n-1}\) in
\eqref{eq:c8-static}–\eqref{eq:c8-waves}. This is the sector (tube) form of the same conservation rule
(§\ref{corollary:info-gas}).

% ------------------------- Remarks & scope -----------------------------

\paragraph{Remarks.}
(i) Near–field structure depends on sources and boundary detail; the listed wave scalings are far‑field envelopes.\\
(ii) Edge diffraction in 2D sheets and leakage in imperfect guides enter through attenuation/survival factors
(§\ref{corollary:entropic-corrections}) and modify only prefactors, not the exponents, in the asymptotic limits.

\medskip
\noindent\emph{Summary.}
Frame amplitudes fall as \(1/A_n(r)\); wave envelopes fall as \(1/\sqrt{A_n(r)}\).
Confinement replaces \(A_n(r)\) by the appropriate effective cross‑section or circumference.
These follow directly from the Stokes identity, the frame linear closure, and constant exported power,
with no additional postulates.

\medskip
\noindent\emph{Literature note.}
Background geometry/areas and cone/tube forms are summarized in App.~\ref{app:nD-boundaries}; the exported‑power
statement is recalled in §\ref{waves} and App.~\ref{app:forms-stokes}. For standard expositions, see
Frankel~\cite{Frankel2011} (geometry) and Jackson~\cite{Jackson1999} or
Landau–Lifshitz–Pitaevskii~\cite{LandauLifshitzEDCM1984}.
