% ======================================================================
% corollaries/08_thermo-newton.tex — Thermodynamics → Newtonian rate laws
% ======================================================================

\subsection{Thermodynamics to Newtonian rate laws (constant‑free)}
\label{corollary:thermo-newton}

\paragraph{Setup.}
On configuration–time space \((q,t)\) (\S\ref{axioms:path-rate}), a Hamiltonian with quadratic kinetic term and external potential is adopted:
\begin{equation}
H(q,P,t)=\tfrac12\,P\!\cdot M^{-1}P \;+\; U(q,t),
\qquad M>0 .
\label{corollary:thermo-newton:eq:kinetic}
\end{equation}

\paragraph{Newtonian update.}
From the canonical rate equations \eqref{axioms:path-rate:eq:canonical} applied to \eqref{corollary:thermo-newton:eq:kinetic},
\[
\dot q = M^{-1}P,\qquad \dot P = -\,\partial_q U .
\]
With \(\mathbf p:=M\dot q\) and \(\mathbf f:=-\,\partial_q U\), this gives
\begin{equation}
\dot{\mathbf p} \;=\; \mathbf f.
\label{corollary:thermo-newton:eq:newton}
\end{equation}
If \(M=M(q)\), additional geometric terms arise from \(\partial_q\!\bigl(P\!\cdot M^{-1}P/2\bigr)\).

\paragraph{Work–power identity.}
For \(K(P):=\tfrac12\,P\!\cdot M^{-1}P=\tfrac12\,\dot q\!\cdot M\dot q\),
\begin{equation}
\frac{dK}{dt}
= \dot q\!\cdot \dot P
= \dot q\!\cdot \mathbf f
= \mathbf f\!\cdot \mathbf v ,
\qquad \mathbf v:=\dot q .
\label{corollary:thermo-newton:eq:power}
\end{equation}
If \(U=U(q,t)\) has explicit time dependence, \(\dot H=\partial_t U\) with \(H:=K+U\); otherwise \(\dot H=0\).

\paragraph{REOS reading.}
With the Relational Equation of State \eqref{axioms:reos:eq}, a control field that induces a spatial price gradient produces a force \(\mathbf f=-\partial_q U\), matching \eqref{corollary:thermo-newton:eq:newton}. Non‑integrable cycles \(d\lambda\neq0\) entail nonnegative production \(\sigma\) as recorded by \eqref{axioms:stokes:identity:eq}–\eqref{axioms:stokes:identity:eq:cut}.

\paragraph{Variants.}
\begin{itemize}\itemsep2pt
\item \textbf{Anisotropic inertia.} For matrix \(M\), \(\mathbf p=M\dot q\) and \eqref{corollary:thermo-newton:eq:newton} hold; if \(M(q)\), Christoffel‑type terms appear in \(\dot{\mathbf p}\).
\item \textbf{Dissipation.} A Rayleigh term \(\mathcal R(\dot q)\) modifies the second canonical equation to
\(\dot P=-\partial_q U-\partial_{\dot q}\mathcal R\), introducing drag consistent with nonnegative production in \eqref{axioms:stokes:identity:eq}–\eqref{axioms:stokes:identity:eq:cut}.
\end{itemize}

\paragraph{Calibration (sector readouts).}
With the Linear Channel Closure \eqref{axioms:linear-closure:eq} and the one–datum sector calibration,
measured sector fields satisfy \(\mathbf g=c_M\,\mathbf E_I\) and \(\mathbf E=c_Q\,\mathbf E_I\).

\medskip
\noindent\emph{Literature note.}
Poincaré–Cartan form and Hamiltonian rates: Arnold~\cite{Arnold1989}. Convex duality and Legendre regularity:
Rockafellar~\cite{Rockafellar1970}.
