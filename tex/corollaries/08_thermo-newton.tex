% ======================================================================
% 03c09_thermo-newton-laws.tex — Thermodynamics → Newtonian rate laws
% ======================================================================

\subsection{Thermodynamics to Newtonian rate laws (constant‑free)}
\label{corollary:thermo-newton}

This corollary shows that the Newtonian update \(\dot{\mathbf p}=\mathbf f\) and the work–power identity
follow from: (i) the canonical rate equations obtained by evaluating Stokes on a rectangular \(2\)-surface
in \((q,t)\) (\S\ref{axioms:path-rate}), (ii) the \emph{frame} linear closure when relating forces to boundary fields,
and (iii) a single, sector‑specific calibration (App.~\ref{app:calibrations}). No bare constants are introduced.

\paragraph{Setup: canonical rates from Stokes.}
From the rate \(1\)-form (\eqref{axioms:path-rate:eq:pc-form}),
\begin{equation}
\Theta_{\mathcal R}(q,P,t)=P\!\cdot dq - H(q,P,t)\,dt,
\label{corollary:thermo-newton:eq:theta}
\end{equation}
Stokes on a rectangular \(2\)-surface spanned by two nearby paths yields the canonical rate equations
\begin{equation}
\dot q = \partial_{P}H,
\qquad
\dot P = -\,\partial_{q}H ,
\label{corollary:thermo-newton:eq:canonical}
\end{equation}
matching \eqref{axioms:path-rate:eq:pc-form}–\eqref{axioms:path-rate:eq:canonical}.

\paragraph{Minimal kinetic choice and Newton’s update.}
Choose a quadratic kinetic term and an external driving potential,
\begin{equation}
H(q,P,t)=\tfrac12\,P\!\cdot M^{-1}P \;+\; U(q,t),
\label{corollary:thermo-newton:eq:kinetic}
\end{equation}
with \(M\) a constant (or slowly varying) positive‑definite inertia matrix. Then
\eqref{corollary:thermo-newton:eq:canonical} gives
\begin{equation}
\dot q = M^{-1}P \quad\Rightarrow\quad P=M\,\dot q \;(:=\mathbf p),
\qquad
\dot P = -\,\partial_q U \;(:=\mathbf f),
\label{corollary:thermo-newton:eq:newton}
\end{equation}
i.e.\ the Newtonian rate law \(\dot{\mathbf p}=\mathbf f\).
If \(M=M(q)\), geometric terms appear in \(\dot{\mathbf p}\) via
\(-\partial_q\!\bigl(\tfrac12 P\!\cdot M^{-1}P\bigr)\).

\paragraph{Work–power identity.}
Let \(K(P):=\tfrac12 P\!\cdot M^{-1}P=\tfrac12\,\dot q\!\cdot M\dot q\).
Then, using \eqref{corollary:thermo-newton:eq:newton},
\begin{equation}
\frac{dK}{dt}
= \dot q\!\cdot \dot P
= \dot q\!\cdot \mathbf f
= \mathbf f\!\cdot \mathbf v ,
\qquad \mathbf v:=\dot q ,
\label{corollary:thermo-newton:eq:power}
\end{equation}
so the instantaneous power delivered by the driving equals the kinetic‑energy rate.
If \(U\) has explicit time dependence, \(\dot H=\partial_t U\) with \(H:=K+U\); for time‑independent \(U\),
\(\dot H=0\) and \(\mathbf f\!\cdot\mathbf v=-\,\dot U\).

\paragraph{Entropic/operational reading (via REOS).}
In the REOS language \S\ref{axioms:reos}, boundary/control variations \(R^{(a)}\) with price \(\lambda_a\) satisfy
\(d\rho=-\lambda_a\,dR^{(a)}\). When a control field induces a spatial price gradient, the force on a probe
reads as the \(q\)-gradient of a potential, i.e.\ \(\mathbf f=-\partial_q U\) in
\eqref{corollary:thermo-newton:eq:newton}. Non‑integrable control cycles (\(d\lambda\neq0\)) incur nonnegative
production \(\sigma\ge0\), which registers via \eqref{axioms:stokes:identity:eq}–\eqref{axioms:stokes:identity:eq:cut}.

\paragraph{Notes and variants.}
\begin{itemize}\itemsep2pt
\item \emph{Vector/tensor inertia.} For anisotropic agents, \(M\) is a matrix; \eqref{corollary:thermo-newton:eq:newton}
gives \(\mathbf p:=M\,\dot q\) and \(\dot{\mathbf p}=\mathbf f\) unchanged (with the usual Christoffel‑type terms if \(M(q)\)).
\item \emph{Other sectors.} The same construction carries to other sectors: replace \(U(q,t)\) with the sectoral
potential appropriate to the chosen readout (gravitational, electric, thermal) and read off \(\mathbf f=-\partial_q U\).
\item \emph{Dissipation.} Adding a Rayleigh term \(\mathcal R(\dot q)\) to the rate generator replaces
\(\dot P=-\partial_q U\) by \(\dot P=-\partial_q U-\partial_{\dot q}\mathcal R\), producing a drag consistent with
nonnegative production \(\sigma\) in \eqref{axioms:stokes:identity:eq}–\eqref{axioms:stokes:identity:eq:cut}.
\end{itemize}

\medskip
\noindent\emph{Summary.}
A minimal kinetic choice in the Stokes‑evaluated canonical system yields \(\dot{\mathbf p}=\mathbf f\) and the
work–power identity in a constant‑free form. Sectoral forces arise as \(-\partial_q U\); dissipative additions
match the production term captured by \eqref{axioms:stokes:identity:eq}–\eqref{axioms:stokes:identity:eq:cut}. The same pattern
transfers to other frame sectors by changing only the sectoral potential \(U\).

\medskip
\noindent\emph{Literature note.}
The Poincar\'e–Cartan form and Hamiltonian rates are standard (see Arnold~\cite{Arnold1989}); convex duality
and Legendre regularity are summarized in Rockafellar~\cite{Rockafellar1970}. This derivation uses only the
Stokes evaluation on \((q,t)\) (\S\ref{axioms:path-rate}) and one‑datum sector calibration (App.~\ref{app:calibrations}).
