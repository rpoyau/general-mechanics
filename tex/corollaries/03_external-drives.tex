% ======================================================================
% 02_external-drives-delay.tex — External drives, delay & production
% ======================================================================

\subsection{External drives, delay \& production}
\label{corollary:external-drives}

Open informational transport across a bounded region \(V\) along a narrow tube
\(\mathcal T\) with inlet cross–section \(\Sigma_{\rm in}\) and outlet \(\Sigma_{\rm out}\) admits
the inward and outward directed fluxes
\[
\Phi_{\rm in}(t):=\int_{\Sigma_{\rm in}}(\mathbf j\!\cdot n)_-\,dA,
\qquad
\Phi_{\rm out}(t):=\int_{\Sigma_{\rm out}}(\mathbf j\!\cdot n)_+\,dA .
\]
The cut identity \eqref{axioms:stokes:identity:eq:cut} on the tube volume reads
\begin{equation}
\dot I_{\mathcal T}(t) \;=\; \Phi_{\rm in}(t)-\Phi_{\rm out}(t) \;+\; \Pi_{\mathcal T}(t),
\qquad
\Pi_{\mathcal T}(t):=\int_{\mathcal T}\sigma\,d^n x ,
\label{eq:c3-cut}
\end{equation}
where \(\sigma\) is the local production density.

\paragraph{(i) Transparent transit (source‑free segment).}
On an open subsegment \(W\subset\mathcal T\) with \(\sigma|_W\equiv 0\) and directed speed profile
\(v(s)>0\) along the tube coordinate \(s\in[0,L]\), the transit time is
\begin{equation}
\tau \;=\; \int_{0}^{L}\frac{ds}{v(s)} .
\label{corollary:external-drives:eq:transit}
\end{equation}
With negligible dispersion, the outlet flux is a delayed copy of the inlet flux,
\begin{equation}
\Phi_{\rm out}(t)\;=\;\Phi_{\rm in}(t-\tau)
\qquad\text{(transparent transit)}.
\label{corollary:external-drives:eq:delay}
\end{equation}
For mild dispersion of speeds, \(\Phi_{\rm out}=K*\Phi_{\rm in}\) with a causal kernel
\(K(t)\) of unit mass \(\int K\,dt=1\) and mean \(\tau\).

\paragraph{(ii) Collisional attenuation/production.}
When interactions remove directed flux at a local rate \(\mu(s)\ge 0\) (per unit length),
write the directed tube flux \(\Phi(s,t):=\int_{\Sigma(s)}(\mathbf j\!\cdot n)_+\,dA\).
In an advective transport regime with speed \(v(s)\),
\begin{equation}
\partial_t \Phi(s,t) + v(s)\,\partial_s \Phi(s,t) \;=\; -\,\mu(s)\,\Phi(s,t) .
\label{corollary:external-drives:eq:transport}
\end{equation}
Along characteristics, \(\Phi\) decays exponentially with survival factor
\begin{equation}
S \;=\; \exp\!\Bigl(-\int_{0}^{L}\mu(s)\,ds\Bigr),
\qquad
\tau \;=\; \int_{0}^{L}\frac{ds}{v(s)},
\label{corollary:external-drives:eq:survival}
\end{equation}
so that
\begin{equation}
\Phi_{\rm out}(t)\;=\; S\,\Phi_{\rm in}(t-\tau)
\qquad\text{(attenuated transit)}.
\label{corollary:external-drives:eq:attenuated-delay}
\end{equation}
The loss from the directed export appears in \eqref{eq:c3-cut} as increased stored content
\(I_{\mathcal T}\) and/or as nonnegative production \(\Pi_{\mathcal T}\ge 0\)
(cf.\ Axiom~\ref{axioms:reos} for cyclic–cost monotonicity).

\paragraph{Cross–sectional amplitudes.}
Let \(A(s)\) be the cross–section area of \(\Sigma(s)\) and
\(\langle j_n\rangle(s,t):=\Phi(s,t)/A(s)\) the mean normal \emph{flux density}.
In regime (i) (\(S=1\)), \(\Phi\) is constant along \(s\) and amplitudes follow the inverse–area rule
(\S\ref{corollary:info-gas}): \(\langle j_n\rangle\propto 1/A(s)\).
In regime (ii), \(\partial_s\Phi=-\mu\Phi\) so
\begin{equation}
\langle j_n\rangle(s,t)
\;=\;
\frac{\Phi(0,t-\tau_s)}{A(s)}\,
\exp\!\Bigl(-\int_{0}^{s}\mu(u)\,du\Bigr),
\qquad
\tau_s=\int_{0}^{s}\frac{du}{v(u)} .
\label{corollary:external-drives:eq:amplitude}
\end{equation}

\paragraph{Composition rules (series segments).}
For piecewise constant segments \(\{(L_k,v_k,\mu_k)\}\),
\begin{equation}
\tau \;=\; \sum_k \frac{L_k}{v_k},
\qquad
S \;=\; \prod_k e^{-\mu_k L_k}
\;=\;\exp\!\Bigl(-\sum_k \mu_k L_k\Bigr).
\label{corollary:external-drives:eq:composition}
\end{equation}
Temporal kernels convolve in series; survival factors multiply.

\paragraph{Moving tube (kinematic invariance).}
If the tube translates as \(\Sigma(s,t)=\Sigma(s-\xi(t))\) with \(\dot\xi=U(t)\),
the Stokes identity \eqref{axioms:stokes:identity:eq} on the \emph{product region}
\(\mathcal T\times[t_0,t_1]\) yields the same cut identity \eqref{eq:c3-cut}.
A co–moving frame replaces \(\partial_t\) by a material derivative along \(U(t)\)
but leaves \eqref{corollary:external-drives:eq:delay}–\eqref{corollary:external-drives:eq:composition} unchanged.

\paragraph{Calibration and sectors.}
Counts on \(\partial V\) may be kept in \emph{trits}; the entropy flux in nats is
\(j_S=(\ln 3)\,j_{\rm trits}\).
When a measured sector is calibrated to \(\mathbf E_I\)
(see \S\ref{operationalization}; App.~\ref{app:calibrations}), the corresponding power/time units
are in Kz (App.~\ref{app:kick}). The frame closure \(\mathbf j=-\kappa\,\mathbf E_I\) then maps
\(\langle j_n\rangle\) to field amplitudes along the tube.

\paragraph{Remarks.}
(i) The transparent‑transit limit provides a practical way to estimate \(\tau\) by cross‑correlating
\(\Phi_{\rm in}\) and \(\Phi_{\rm out}\).
(ii) In frame applications, the survival factor \(S(r)=\exp(-\!\int \mu\,ds)\) reduces to the
attenuation factor multiplying the inverse–area rule (\S\ref{corollary:entropic-corrections}).
(iii) Apertures and cones are handled by replacing \(A(s)\) with the sector area \(A_\Delta(r)\)
(App.~\ref{app:nD-boundaries}).

\medskip
\noindent\emph{Summary.}
A transparent transit copies the inlet flux to the outlet after a transit time \(\tau\).
Collisions introduce a survival factor \(S\le 1\) and optional temporal blur via a causal kernel.
Cross–sectional amplitudes follow the \(1/A\) rule, modulated by exponential survival along the tube.
Series segments add delays and multiply survivals, independent of the choice of frame.

\medskip
\noindent\emph{Literature note.}
The advection–attenuation model \eqref{corollary:external-drives:eq:transport} is a standard first‑order transport
(Beer–Lambert–type) equation; see, e.g., Evans \cite{Evans2010}. Stokes/cut forms used here are recalled in
App.~\ref{app:forms-stokes}; sector geometry (areas) in App.~\ref{app:nD-boundaries}; units and rate conventions in
App.~\ref{app:kick}.
