% ======================================================================
% 03c01_static-screen-law.tex — C1: Static screen law (inverse-power fields)
% ======================================================================

\subsection{C1: Static screen law (inverse-power fields)}
\label{corollary:static-screen}

\paragraph{Setup (static snapshot, source-free annulus).}
Work on a time-frozen snapshot with $\partial_t i=0$. On an open annulus $W\subset U$ with $\sigma|_{W}\equiv 0$
(source-free on $W$), define the directed flux through a closed screen $\Sigma(r)\subset W$ of radius $r$ by
\[
\Phi(r):=\int_{\Sigma(r)} \mathbf j\!\cdot n\,dA.
\]
By the Stokes identity on $W$ (App.~\ref{app:forms-stokes})—equivalently, by the cut identity with $\partial_t i=0$ and $\Pi=0$
(see \eqref{eq:cut-identity})—$\Phi(r)$ is independent of $r$ across nested screens in $W$.

\paragraph{Static linear closure.}
Using the static closure \eqref{eq:static-closure},
\[
\mathbf E_I:=-\,\nabla\Phi_I,
\qquad
\mathbf j=-\,\kappa\,\mathbf E_I .
\]

\paragraph{Screen area in $n$ spatial dimensions.}
For a spherically symmetric screen in $n$ spatial dimensions (App.~\ref{app:nD-screens}),
\[
A_n(r)=S_{n-1}\,r^{\,n-1}\quad\text{(see eq.~\eqref{appC:eq:areas})},
\]
hence the cross–sectional mean of the normal current is
\begin{equation}
\bigl\langle j_n\bigr\rangle(r)=\frac{\Phi}{A_n(r)}=\frac{\Phi}{S_{n-1}\,r^{\,n-1}},
\qquad
\bigl|\mathbf E_I(r)\bigr|=\frac{1}{\kappa}\,\bigl\langle j_n\bigr\rangle(r)
=\frac{\Phi}{\kappa\,S_{n-1}}\,\frac{1}{r^{\,n-1}} .
\label{eq:c1-screen-law}
\end{equation}
Thus static informational-field amplitudes scale as $r^{-(n-1)}$.

\paragraph{3D specialization.}
For $n=3$, $A_3(r)=4\pi r^2$ and \eqref{eq:c1-screen-law} reduces to
\begin{equation}
\bigl|\mathbf E_I(r)\bigr|=\frac{\Phi}{4\pi\kappa}\,\frac{1}{r^{2}} .
\label{eq:c1-3d}
\end{equation}

\paragraph{Calibrated sectors (constant–free predictions).}
Using the one–datum calibrations of App.~\ref{app:calibrations} (eq.~\eqref{appF:eq:cal-def}),
\begin{equation}
\mathbf g=c_M\,\mathbf E_I,\qquad \mathbf E=c_Q\,\mathbf E_I,\qquad \mathbf j_{\rm th}=\kappa_T\,\mathbf E_I ,
\label{eq:c1-calibrations}
\end{equation}
the measured $3$D fields follow immediately:
\begin{equation}
g(r)=\frac{c_M\,\Phi}{4\pi\kappa}\,\frac{1}{r^{2}},\qquad
E(r)=\frac{c_Q\,\Phi}{4\pi\kappa}\,\frac{1}{r^{2}} .
\label{eq:c1-measured-3d}
\end{equation}
Fixing a single datum $(r_\star, g_\star)$ (or $(r_\star, E_\star)$) for a given source family yields the
constant–free ratio form
\begin{equation}
g(r)=g_\star\left(\frac{\Phi}{\Phi_\star}\right)\left(\frac{r_\star}{r}\right)^{\!2},\qquad
E(r)=E_\star\left(\frac{\Phi}{\Phi_\star}\right)\left(\frac{r_\star}{r}\right)^{\!2}.
\label{eq:c1-ratio}
\end{equation}
When the enclosed flux $\Phi$ is empirically linear in a source label (e.g.\ mass $M$ or charge $Q$)
within the family, \eqref{eq:c1-ratio} specializes to
\begin{equation}
g(r)=g_\star\left(\frac{M}{M_\star}\right)\left(\frac{r_\star}{r}\right)^{\!2},\qquad
E(r)=E_\star\left(\frac{Q}{Q_\star}\right)\left(\frac{r_\star}{r}\right)^{\!2}.
\label{eq:c1-ratio-MQ}
\end{equation}
No bare constants enter beyond the one–datum calibration.

\paragraph{Directional filtering (apertures/sectors).}
For a one–sided screen (open cone or diaphragm), replace $\Phi$ by the inward component
$\Phi_-=\int_{\Sigma}( \mathbf j\!\cdot n)_-\,dA$; formulas
\eqref{eq:c1-screen-law}–\eqref{eq:c1-ratio-MQ} hold with $\Phi\mapsto\Phi_-$.
For a fixed solid angle $\Delta\Omega$ in $n$D, use $A_\Delta(r)=\Delta\Omega\,r^{\,n-1}$ (App.~\ref{app:nD-screens}).

\paragraph{Units and trits/nats.}
The default alphabet is ternary; counts are in \emph{trits} and map to nats by
\[
S_{\rm nats}=(\ln 3)\,S_{\rm trits}.
\]
The law \eqref{eq:c1-screen-law} is unit–agnostic; only the numerical value of $\kappa$ changes with the
trit/nat convention (global unit policy is stated once elsewhere).

\medskip
\noindent\textit{Remark.}
The conical/tube argument in Corollary~\ref{corollary:info-gas} (C0) gives the same
$1/A_n(r)$ amplitude rule on sectors of fixed solid angle;
\eqref{eq:c1-screen-law} is the full–screen specialization.
Confinement (guides/sheets) replaces $A_n(r)$ by an effective cross–section or
circumference; see Corollary~\ref{corollary:dimensional} (C8).

\medskip
\noindent\emph{Literature note.}
Screen areas and Green–function normalizations appear in App.~\ref{app:nD-screens};
see also Frankel~\cite{Frankel2011} and Jackson~\cite{Jackson1999}, and the
dimension–dependence summarized in Corollary~\ref{corollary:dimensional} (C8).
