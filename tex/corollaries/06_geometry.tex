% ======================================================================
% 03c07_geometry-equivalence.tex — Geometry as tool (flat ↔ curved)
% ======================================================================

\subsection{Geometry as tool—flat \texorpdfstring{$\leftrightarrow$}{↔} curved}
\label{corollary:geometry}

This corollary uses geometry purely as a \emph{tool} to relate boundary amplitudes to the change of
cross–sectional area along a thin tube of trajectories on a (Lorentzian or Riemannian) manifold.
Let \(k^a\) be a future–directed null or timelike congruence threading a narrow tube, with affine
(or proper) parameter \(\lambda\) and cross–sectional area \(A(\lambda)\) orthogonal to \(k^a\).

\paragraph{Expansion and area.}
The congruence expansion is
\begin{equation}
\theta \;:=\; \nabla_a k^a,
\qquad
\frac{d}{d\lambda}\ln A \;=\; \theta .
\label{corollary:geometry:eq:expansion}
\end{equation}
On a steady, source–free tube the directed boundary flux \(\Phi\) is constant, and the mean normal
\emph{flux density} satisfies \(\langle j_n\rangle=\Phi/A\). Therefore
\begin{equation}
\frac{d}{d\lambda}\,\ln \langle j_n\rangle \;=\; -\,\theta .
\label{corollary:geometry:eq:spread-frame}
\end{equation}
For wave amplitudes in homogeneous segments (see §\ref{waves}, eq.~\eqref{waves:eq:envelope}),
\(\langle |\mathbf E_I|\rangle \propto [A(\lambda)]^{-1/2}\), hence
\begin{equation}
\frac{d}{d\lambda}\,\ln \langle |\mathbf E_I|\rangle \;=\; -\,\frac{1}{2}\,\theta .
\label{corollary:geometry:eq:spread-wave}
\end{equation}
Equations \eqref{corollary:geometry:eq:spread-frame}–\eqref{corollary:geometry:eq:spread-wave}
express the flat \(\leftrightarrow\) curved \emph{equivalence}: spreading in flat regions
(\(\theta>0\)) and focusing in curved regions (\(\theta<0\)) are the same area statement, with
amplitudes set solely by \(A(\lambda)\).

\paragraph{Raychaudhuri focusing.}
For a geodesic, hypersurface–orthogonal congruence (\(\omega_{ab}=0\)), the Raychaudhuri equation gives
\begin{equation}
\frac{d\theta}{d\lambda}
\;=\;
-\,\frac{1}{2}\,\theta^2
\;-\;\sigma_{ab}\sigma^{ab}
\;-\;R_{ab}\,k^a k^b ,
\label{corollary:geometry:eq:raychaudhuri}
\end{equation}
where \(\sigma_{ab}\) is the shear tensor and \(R_{ab}\) the Ricci tensor.\footnote{For timelike
congruences the numerical coefficient in front of \(\theta^2\) is dimension–dependent; the focusing
term \(R_{ab}k^a k^b\) retains the same sign.} \emph{Positive} \(R_{ab}k^a k^b\) increases focusing
(drives \(\theta\) downward), tightening \(A(\lambda)\) and raising amplitudes via
\eqref{corollary:geometry:eq:spread-frame}–\eqref{corollary:geometry:eq:spread-wave}.

\paragraph{Calibrated equality (consistency tool, not an axiom).}
On the \emph{same} region used for flux measurements, a single sector calibration
(§\ref{subsec:op-calibration}) ties the focusing term to a calibrated stress component:
\begin{equation}
R_{ab}\,k^a k^b
\;\calibeq\;
8\pi\,\mathcal T^{(\mathrm{cal})}_{ab}\,k^a k^b .
\label{corollary:geometry:eq:info-einstein}
\end{equation}
The symbol \(\calibeq\) denotes equality \emph{after} one calibration (no bare coupling).
Equation \eqref{corollary:geometry:eq:info-einstein} is used only as a \emph{consistency check}:
the same datum that fixes statics (§\ref{corollary:frame-boundary}) fixes the focusing response
inferred from \(A(\lambda)\).

\paragraph{Flat check.}
In flat geometry with a radial congruence in \(n\) spatial dimensions,
\(A(\lambda)\propto r^{\,n-1}\) and \(\theta=(n{-}1)\,\dot r/r\).
Then \eqref{corollary:geometry:eq:spread-frame} gives
\(\langle j_n\rangle\propto r^{-(n-1)}\) and \eqref{corollary:geometry:eq:spread-wave} gives
\(\langle |\mathbf E_I|\rangle\propto r^{-(n-1)/2}\), matching the boundary inverse–area law
(§\ref{corollary:frame-boundary}) and the wave envelope in §\ref{waves}.

\paragraph{Remarks (shear and vorticity).}
Shear \(\sigma_{ab}\) can broaden or squeeze the tube independently of \(R_{ab}k^a k^b\);
vorticity \(\omega_{ab}\neq 0\) would enter Raychaudhuri with a sign that counteracts focusing.
These kinematics combine through \eqref{corollary:geometry:eq:raychaudhuri} to determine \(A(\lambda)\),
hence amplitudes via \eqref{corollary:geometry:eq:spread-frame}–\eqref{corollary:geometry:eq:spread-wave}.

\medskip
\noindent\emph{Summary.}
Geometry fixes how \(A(\lambda)\) evolves; Stokes fixes the conserved quantities.
With one calibration shared with statics, the focusing term can be related to a calibrated stress
along the tube. No new constants are introduced, and all amplitude rules reduce to the area
relations \eqref{corollary:geometry:eq:spread-frame}–\eqref{corollary:geometry:eq:spread-wave}.

\medskip
\noindent\emph{Literature note.}
Geodesic congruences and Raychaudhuri: Wald~\cite{Wald1984}, Raychaudhuri~\cite{Raychaudhuri1955},
Poisson~\cite{Poisson2004}. The calibrated focusing equality
\eqref{corollary:geometry:eq:info-einstein} follows the local‑horizon logic summarized in
App.~\ref{app:jacobson} (cf.\ Jacobson~\cite{Jacobson1995}). Boundary geometry in \(n\)D appears
in App.~\ref{app:nD-boundaries}. Static and wave amplitude scalings referenced here are established in
§\ref{corollary:frame-boundary} and §\ref{waves}.
