% ======================================================================
% sections/07_worldtube-IED.tex — Worldtube dynamics & IED (v1.0.2)
% ======================================================================

\section{Worldtube dynamics and information electrodynamics (IED)}
\label{sec:worldtube-IED}

\paragraph{Setup (forms).}
Let \(\mathcal A_I\) be the information potential \(1\)-form and
\(\mathcal F_I:=\mathrm d\mathcal A_I\) the field \(2\)-form.
A medium map \(\chi\) produces the excitation \(2\)-form
\(\mathcal H_I:=\chi\!\cdot\!\mathcal F_I\).
The information current \(1\)-form is \(j=J_{I\mu}\,\mathrm dx^\mu\),
and its Hodge dual is the \(3\)-form \(j_3:=*j\).

\paragraph{Field equations (worldtube Stokes).}
IED is the pair
\begin{equation}
\mathrm d\mathcal F_I=0,
\qquad
\mathrm d\mathcal H_I=j_3,
\label{eq:ied-forms}
\end{equation}
i.e.\ Bianchi identity plus sourced response.
Integrating \eqref{eq:ied-forms} over a fixed worldtube
\(\Omega=V\times[t_0,t_1]\) gives
\[
\int_{\partial\Omega}\!\!\mathcal F_I=0,
\qquad
\int_{\partial\Omega}\!\!\mathcal H_I=\int_{\Omega}\! j_3,
\]
so the lateral flux across \(\partial V\times[t_0,t_1]\) equals interior production plus the difference of the top and bottom caps. No moving boundary is required.

\paragraph{Gauge and potential.}
\(\mathcal A_I\) is defined up to a gauge \( \mathcal A_I \mapsto \mathcal A_I + \mathrm d\lambda\).
In a \((3{+}1)\) split with coordinates \((t,\mathbf x)\),
\begin{equation}
\mathcal A_I = -\phi_I\,\mathrm dt + \mathbf A_I\cdot \mathrm d\mathbf x,
\qquad
\mathcal F_I = \mathbf E_I\cdot \mathrm dt\wedge \mathrm d\mathbf x
+ \mathbf B_I\cdot \frac{1}{2}\,\epsilon_{ijk}\,\mathrm dx^i\wedge \mathrm dx^j,
\label{eq:split-AF}
\end{equation}
with \(\mathbf E_I=-\nabla\phi_I-\partial_t\mathbf A_I\) and
\(\mathbf B_I=\nabla\times\mathbf A_I\).

\paragraph{Maxwell-type system (3+1).}
Writing \(\mathcal H_I\) in the same split as
\(\mathcal H_I=\mathbf D_I\cdot \mathrm d\mathbf x\wedge \mathrm dt
+ \mathbf H_I\cdot \tfrac12\,\epsilon_{ijk}\,\mathrm dx^i\wedge \mathrm dx^j\),
\eqref{eq:ied-forms} yields
\begin{equation}
\begin{aligned}
\nabla\cdot \mathbf D_I &= i, & \qquad \nabla\cdot \mathbf B_I &= 0,\\
\nabla\times \mathbf H_I &= \mathbf j + \partial_t \mathbf D_I, & \qquad
\nabla\times \mathbf E_I &= -\,\partial_t \mathbf B_I,
\end{aligned}
\label{eq:maxwell-like}
\end{equation}
where \(i\) and \(\mathbf j\) are the information density and flux of \(J_I^\mu=(i,\mathbf j)\) from §\ref{sec:conventions}.
This is the dynamic counterpart of the continuity law \(\nabla_\mu J_I^\mu=\sigma\): here \(J_I\) acts as the \emph{source} of the IED field.

\paragraph{Constitutive laws and wave speed.}
For homogeneous, isotropic media we take
\begin{equation}
\mathbf D_I = \varepsilon_I\,\mathbf E_I,
\qquad
\mathbf B_I = \mu_I\,\mathbf H_I,
\label{eq:constitutive}
\end{equation}
with \(\varepsilon_I,\mu_I>0\) constant (in general \(\chi\) may be anisotropic, dispersive, or nonlinear).
Combining \eqref{eq:maxwell-like}–\eqref{eq:constitutive} gives the wave equations
\begin{equation}
\Box_{v_I}\,\mathbf E_I = -\,\frac{1}{\varepsilon_I}\,\partial_t \mathbf j - \frac{1}{\varepsilon_I}\,\nabla i,
\qquad
\Box_{v_I}\,\mathbf B_I = \mu_I\,\nabla\times \mathbf j,
\quad
\Box_{v_I}:=\frac{1}{v_I^2}\,\partial_t^2 - \nabla^2,
\label{eq:wave-E-B}
\end{equation}
with signal speed \(v_I=\dfrac{c}{\sqrt{\varepsilon_I \mu_I}}\le c\) (causal).
In Lorenz gauge \(\nabla\cdot \mathbf A_I + \tfrac{1}{v_I^2}\partial_t \phi_I=0\),
the potentials satisfy
\begin{equation}
\Box_{v_I}\,\phi_I = -\,\frac{i}{\varepsilon_I},
\qquad
\Box_{v_I}\,\mathbf A_I = -\,\mu_I\,\mathbf j.
\label{eq:wave-potentials}
\end{equation}

\paragraph{Poynting-type balance on a worldtube.}
Define the IED energy density and flux (Poynting vector)
\begin{equation}
u_I := \tfrac{1}{2}\big(\mathbf E_I\cdot \mathbf D_I + \mathbf B_I\cdot \mathbf H_I\big),
\qquad
\mathbf S_I := \mathbf E_I\times \mathbf H_I.
\label{eq:poynting-def}
\end{equation}
From \eqref{eq:maxwell-like}–\eqref{eq:constitutive} one obtains the local balance
\begin{equation}
\partial_t u_I + \nabla\cdot \mathbf S_I \;=\; -\,\mathbf j\cdot \mathbf E_I,
\label{eq:poynting-local}
\end{equation}
and integrating \eqref{eq:poynting-local} over \(\Omega=V\times[t_0,t_1]\) yields
\begin{equation}
\int_{\partial V\times[t_0,t_1]}\!\!\mathbf S_I\!\cdot d\mathbf A\,dt
= \int_{V}\! u_I\,d^3x\Big|_{t_0}^{t_1}
\;-\;\int_{\Omega}\! \mathbf j\!\cdot\!\mathbf E_I\,d^4x.
\label{eq:poynting-worldtube}
\end{equation}
Thus the \emph{lateral} radiative export equals the cap energy change minus source work.
This makes the worldtube balance explicit with fixed boundaries.

\paragraph{Static limit and connection to §\ref{sec:static-gauss-nd}.}
In static situations (\(\partial_t=0\), \(\mathbf B_I=\mathbf H_I=0\)),
\eqref{eq:maxwell-like} reduces to
\[
\nabla\cdot(\varepsilon_I \mathbf E_I)= i,\qquad \nabla\times \mathbf E_I=0,
\]
so \(\mathbf E_I=-\nabla\Phi_I\) and \(\varepsilon_I\,\Delta\Phi_I=-i\).
This matches §\ref{sec:static-gauss-nd} with \(\kappa=1/\varepsilon_I\) and \(\dot I_{\rm enc}=\int_V i\,d^3x\).
Hence the static \(1/r^{\,2}\) profiles (or \(1/r^{\,n-1}\) in \(n\)D) appear as the steady snapshot of IED.

\paragraph{Retarded solutions (sketch).}
From \eqref{eq:wave-potentials}, the potentials admit the retarded integral
\begin{equation}
\phi_I(t,\mathbf x) = \frac{1}{4\pi \varepsilon_I}\int \frac{i(t_r,\mathbf x')}{R}\,d^3x',
\qquad
\mathbf A_I(t,\mathbf x) = \frac{\mu_I}{4\pi}\int \frac{\mathbf j(t_r,\mathbf x')}{R}\,d^3x',
\label{eq:retarded}
\end{equation}
with \(R=|\mathbf x-\mathbf x'|\) and \(t_r=t-\dfrac{R}{v_I}\).
These generate the Liénard–Wiechert–type information fields \((\mathbf E_I,\mathbf B_I)\), ensuring causal propagation of “surprise’’ at speed \(v_I\).

\paragraph{Remarks.}
(1) \(\varepsilon_I,\mu_I\) (or the full \(\chi\)) are \emph{calibrations} that set wave speed and impedance; no universal constants appear in algebra.  
(2) The worldtube identity \eqref{eq:poynting-worldtube} is exact for fixed boundaries and cleanly separates lateral export, cap change, and source work.  
(3) The static limit reproduces §\ref{sec:static-gauss-nd}; time dependence adds retarded radiation of information.
