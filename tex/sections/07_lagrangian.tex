% sections/07_lagrangian.tex  —  Anchor Corollary II: Least-Action & Hamiltonian Formulation

\section{Anchor Corollary II: Least–Action \& Hamiltonian}
\label{sec:lagrangian}

\paragraph{Setup.}
Let $\rho(\mathbf x,t)$ be the Shannon-entropy density and
$R^{(a)}\,(\mathbf x,t)$ the resource densities that satisfy the REOS
differential (Eq.\,\ref{eq:reos_diff}).  
Choose a positive–definite mobility tensor $A^{ij}(\mathbf x)$,\footnote{%
Examples: $A^{ij}=D\delta^{ij}$ (diffusion), $A^{ij}=-c^{2}\delta^{ij}$
(wave propagation).} and enforce the axiom with a Lagrange multiplier~$\lambda$.

\begin{boxedeq}[eq:L]
\mathcal L(\rho,R,\mathbf J;\lambda)
   \;=\;
   \frac12 A^{ij} J_i J_j
   \;+\;
   \lambda \,\Phi\!\bigl(\rho,R^{(a)}\bigr).
\end{boxedeq}

\paragraph{Euler–Lagrange equations (Tier-0 recap).}
Varying $\mathcal L$ under
$\partial_t\rho + \nabla\!\cdot\!\mathbf J = \sigma$
(with multiplier $\eta$) yields

\begin{align}
A^{ij} J_j + \nabla^i \eta &= 0, \label{eq:EL_J}\\
\lambda\,\partial_\rho\Phi - \partial_t\eta &= 0, \label{eq:EL_rho}\\
\lambda\,\partial_{R^{(a)}}\Phi - \partial_t\xi^{(a)} &= 0, \label{eq:EL_R}
\end{align}
reproducing the diffusive or wave-like transport law once $A^{ij}$ is fixed.

\paragraph{Canonical momentum and Hamiltonian density.}
The canonical momentum conjugate to $\rho$ is
$\pi = \partial\mathcal L / \partial(\partial_t\rho) = \eta$.
Legendre transforming over $\partial_t\rho$ gives

\begin{boxedeq}[eq:H]
\mathcal H
   \;=\;
   \frac12 A^{ij}J_iJ_j
   \;-\;\lambda\,\Phi(\rho,R)
   \;-\;\pi\,\sigma.
\end{boxedeq}

Hamilton’s equations reproduce Eqs.\,\eqref{eq:EL_J}–\eqref{eq:EL_R},
so extremising the action

\[
S[\rho,R] =
\int_{t_1}^{t_2}\!\!\int_{\Omega}\!
\mathcal L\,d^{\,n}x\,dt
\]

selects the flow of \emph{least information‐capacity expenditure}
compatible with the REOS constraint.

\paragraph{Interpretation.}
Equation (\ref{eq:L}) provides a single functional that unifies the
continuity identity, the informational force
$\mathbf f=-R^{(a)}\nabla\Lambda_a$ (derived in
Sec.\,\ref{sec:info_force}), and the Tier-1 capacity bounds:
choosing a specific $A^{ij}$ or resource list $R^{(a)}$ automatically
fixes the dynamical law, the admissible flux ceiling, and
the work cost of any measurement sink $\sigma<0$.

\paragraph{Literature note.}
The structure mirrors Onsager–Machlup for diffusion
\cite{OnsagerMachlup1953} and, in the Kick gauge,
matches the information-geometry actions reviewed by Amari
\cite{Amari2000}.  Here it emerges \emph{solely} from the REOS axiom
and the choice of mobility tensor—no stochastic or thermodynamic
postulates are imported.

