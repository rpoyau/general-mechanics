% ======================================================================
% 04_operationalization.tex — Section 4: Operationalization & Calibration
% ======================================================================

\section{Operationalization \& Calibration}
\label{operationalization}

This section records minimal steps to turn the Stokes identity on a region
\(U\subset M\) with boundary \(\partial U\) into numerical predictions for a chosen
measured sector. The spine is: generalized Stokes on \(U\) and
on \(U\times[t_0,t_1]\) (see \eqref{axioms:stokes:identity:eq}–\eqref{axioms:stokes:identity:eq:cut}),
the \emph{frame} linear closure \eqref{axioms:stokes:frame:eq:closure}, and (when used) a
\emph{wave transport closure} recorded in \S\ref{waves}
that provides a Poynting identity on source–free segments
(App.~\ref{app:forms-stokes}, eq.~\eqref{app:forms-stokes:eq:poynting}).

% --- Thermodynamic dictionary (boundary-first) ---
\paragraph{Thermodynamic dictionary (boundary-first).}
Piston face area \(A\) \(\to\) spherical \emph{boundary} \(S^{n-1}(r)\) with area \(A_n(r)\);
pressure \(p\) \(\to\) normal entropic flux density \(j_n\) on the boundary;
volume \(V\) \(\to\) enclosed \(V_n(r)\) (with \(dV_n=A_n(r)\,dr\));
mechanical work \(\delta W\) \(\to\) \(j_n\) times boundary displacement \(A_n(r)\,\delta r\);
heat \(\delta Q\) \(\to\) optional boundary energy mapping \(\int_A T\,\delta S_I\) (one datum per sector).
This dictionary is used operationally in (\S\ref{corollary:info-gas}, \S\ref{corollary:frame-boundary}).

% --- Geometry note for n+1 and moving boundaries ---
\paragraph{Geometry in \(n{+}1\) (boundaries and worldtubes).}
In \(n\) spatial dimensions use the spherical boundary area \(A_{n}(r)\) (or the effective cross‑section/circumference under confinement; \S\ref{corollary:dimensional}).
For frame tube/\emph{boundary} arguments work on an \emph{open annulus} \(W\subset U\) with \(\sigma|_W\equiv0\).
For moving boundaries replace \(A_{n}(r)\) by the instantaneous \(A_{n}(r,t)\) and use the worldtube \emph{cap+side (cut) identity} with side integrand \(\mathbf j\!\cdot\!\mathbf n - i\,v_n\)
(App.~\ref{app:forms-stokes}).

% --- Operational witnesses (local, optional) ---
\paragraph{Operational witnesses (local, optional).}
\emph{W‑FluxConst[\(\varepsilon\)]}: on a frame annulus with negligible production, the directed \emph{boundary} flux is invariant across nested boundaries (C1, \S\ref{corollary:frame-boundary}).\newline
\emph{W‑Narrow[\(\varepsilon\)]}: on a propagation segment, exported power is spectrally concentrated near a carrier (\S\ref{waves}).\newline
\emph{W‑Lin[\(\varepsilon\)]}, \emph{W‑Stat[\(\varepsilon\)]}, \emph{W‑Ret[\(\varepsilon\)]}: local linearity, stationarity, and retardedness checks on the window (used when forming transfer functions; if satisfied, the Causality–Dispersion corollary may be invoked, \S\ref{waves:causality-dispersion}).\newline
Consequences in (\S\ref{corollary:frame-boundary}, \S\ref{waves}) are invoked on windows where the corresponding witness is observed.

% ----------------------------- 4.1 -------------------------------------
\subsection{Inputs and outputs}
\label{subsec:op-inputs}

To make the framework predictive, the observer supplies:
\begin{enumerate}\itemsep2pt
  \item \textbf{Boundary choice.} A concrete \(\partial U\): closed \emph{boundary} or open \emph{boundary sector/aperture}
        (cf.\,\S\ref{corollary:info-gas}; \S\ref{corollary:frame-boundary}). Confinement choices
        (guides/sheets) produce effective lower dimensions (C8, \S\ref{corollary:dimensional}).
  \item \textbf{Channel class.} \emph{Frame} (use \(\mathbf j=-\kappa\,\mathbf E_I\)) or a \emph{wave
        segment} (use the general wave template in \S\ref{waves};
        the Poynting identity \eqref{app:forms-stokes:eq:poynting} applies on source–free segments).
        If waves, state the constitutive description (e.g.\ \(\chi\)) on the segment where it is treated as homogeneous.
  \item \textbf{One datum (sector scale).} A single measurement \(Y_\star\) on a reference
        boundary/sector (radius \(r_\star\), area \(A_\star\)) for the chosen readout
        \(Y\in\{g,E,j_{\rm th}\}\); see \S\ref{subsec:op-calibration}.
  \item \textbf{Resource budgets.} Area \(A\), time window \(\Delta t\), power budget \(P\),
        boundary temperature \(T\) (\textbf{Kz}; App.~\ref{app:kick}), and a target reliability \(\varepsilon\)
        for coding/decoding.
  \item \textbf{Noise model (optional).} If needed, a noise kernel \(\mathcal N\) for
        fluctuations (App.~\ref{app:noise-kernel}) or simply the Poisson shot–noise assumption
        used in (\S\ref{corollary:noise}).
\end{enumerate}
\emph{Outputs} are calibrated fields and bounds over the same manifold:
frame profiles \(Y(r)\), wave envelopes \(\langle|Y(r)|\rangle\), and operational
limits from time–information and noise.

% ----------------------------- 4.2 -------------------------------------
\subsection{One–datum calibration (constant–free scale setting)}
\label{subsec:op-calibration}

On a frame, source–free annulus the directed flux \(\Phi\) across any boundary is
constant and
\[
\big\langle j_n\big\rangle(r)=\frac{\Phi}{A_n(r)},\qquad
|\mathbf E_I(r)|=\frac{1}{\kappa}\,\frac{\Phi}{A_n(r)}
\quad\text{(C0–C1; \S\ref{corollary:info-gas}, \S\ref{corollary:frame-boundary}).}
\]
A measured sector is a linear readout of \(\mathbf E_I\):
\[
\mathbf g=c_M\,\mathbf E_I,\qquad
\mathbf E=c_Q\,\mathbf E_I,\qquad
\mathbf j_{\rm th}=\kappa_T\,\mathbf E_I.
\]
Fix \(c_{\rm sec}\) by a single datum \(Y_\star\) on a reference boundary/sector \((r_\star,A_\star)\)
with the same flux \(\Phi\):
\begin{equation}
c_{\rm sec}
=\frac{Y_\star}{|\mathbf E_I(r_\star)|}
=Y_\star\,\frac{\kappa\,A_\star}{\Phi}.
\label{eq:op-cal-const}
\end{equation}
Thereafter, in the lossless \emph{frame} regime,
\begin{equation}
Y(r)=c_{\rm sec}\,|\mathbf E_I(r)|
=Y_\star\,\frac{A_\star}{A_n(r)}.
\label{eq:op-one-datum-ratio}
\end{equation}
If a flux label \(\mathcal Q\) is used operationally (e.g.\ mass/charge within a source
family) with \(\Phi\propto\mathcal Q\), then \(Y(r)=Y_\star(\mathcal Q/\mathcal Q_\star)\,
[A_\star/A_n(r)]\) (App.~\ref{app:calibrations}).

\paragraph{Cones and confinement.}
For a conical \emph{boundary sector} of solid angle \(\Delta\Omega\), replace \(A_n(r)\) by
\(A_\Delta(r)=\Delta\Omega\,r^{\,n-1}\). For guides/sheets use the effective constant
cross–section/circumference (C8, \S\ref{corollary:dimensional}). No new constants are introduced.

% ----------------------------- 4.3 -------------------------------------
\subsection{Wave segments: envelopes and export}
\label{subsec:op-waves}

On homogeneous source–free segments, the wave template in \S\ref{waves} plus the
Poynting identity (App.~\ref{app:forms-stokes}, \eqref{app:forms-stokes:eq:poynting}) implies constant exported power
through \emph{boundaries}. Time–averaging gives
\[
P(r)=\int_{\Sigma(r)}\!\langle \mathbf S\rangle\!\cdot n\,dA=\text{const},\qquad
\langle \mathbf S\rangle\propto \langle|\mathbf E_I|^2\rangle,
\]
hence the far–field envelope
\begin{equation}
\big\langle |\mathbf E_I(r)| \big\rangle \propto \frac{1}{\sqrt{A_n(r)}}
\quad\Rightarrow\quad
\big\langle |Y(r)| \big\rangle = c_{\rm sec}\,\big\langle |\mathbf E_I(r)| \big\rangle .
\label{eq:op-wave-envelope}
\end{equation}
In short windows \(\Delta t\), interference can suppress instantaneous export on sub‑windows,
while the \emph{window‐integrated} export \(\int_t^{t+\Delta t}\!P(\tau)\,d\tau\) remains radius‑independent
on the same segment. Resolution is bounded by \eqref{eq:op-rate}–\eqref{eq:op-time} and the noise floor
\eqref{eq:op-noise}. Confinement replaces \(A_n(r)\) by the effective cross–section/circumference (\S\ref{corollary:dimensional}).
Apply \S\ref{corollary:entropic-corrections} if attenuation or capacity ceilings are relevant (C11).

% ----------------------------- 4.4 -------------------------------------
\subsection{Operational limits: time, capacity, and erasure}
\label{subsec:op-limits}

Let \(\overline{J}_{A}\) be the inward \emph{counting rate} over a boundary patch \(A\) in a window
\(\Delta t\) (\S\ref{corollary:time-info}). With a reliability target \(\varepsilon\)
and a channel class choice (coding, alphabet, constraints), let
\(C_{\rm chan}^{(\mathrm{trits})}(\varepsilon)\) denote the corresponding \(\varepsilon\)–reliable
capacity (trits per time). Then the resource–limited bounds are
\begin{equation}
\frac{\Delta I_{\rm trits}}{\Delta t}
\le \min\!\left\{\frac{\overline{J}_{A}}{\ln 3},\
C_{\rm chan}^{(\mathrm{trits})}(\varepsilon),\
\frac{P}{T\,\ln 3}\right\},
\label{eq:op-rate}
\end{equation}
\begin{equation}
\Delta t
\ge \max\!\left\{\frac{\Delta I_{\rm trits}}{\overline{J}_{A}/\ln 3},\
\frac{\Delta I_{\rm trits}}{C_{\rm chan}^{(\mathrm{trits})}(\varepsilon)},\
\frac{\Delta I_{\rm trits}\,\ln 3\,T}{P}\right\}.
\label{eq:op-time}
\end{equation}

\paragraph{Erasures/unknowns (resolved content).}
If a fraction \(\varepsilon_{\bot}\in[0,1)\) of symbols observed on the same window are erasures \((\bot)\),
and one wishes to bound \emph{resolved} content only, make the substitution
\begin{equation}
\overline{J}_{A}\ \mapsto\ (1-\varepsilon_{\bot})\,\overline{J}_{A},
\qquad
C_{\rm chan}^{(\mathrm{trits})}\ \mapsto\ (1-\varepsilon_{\bot})\,C_{\rm chan}^{(\mathrm{trits})}
\end{equation}
in \eqref{eq:op-rate}–\eqref{eq:op-time}. The \(P/T\) term is unchanged.

% ----------------------------- 4.5 -------------------------------------
\subsection{Noise floor and estimator width (operational counting witness)}
\label{subsec:op-noise}

Let \(N_A(\Delta t)\) be the inward \emph{counts} on a boundary patch \(A\) over a window \(\Delta t\).
Define the \emph{measured} mean inward current density
\[
\overline{j}_A:=\frac{\mathbb E[N_A]}{|A|\,\Delta t},
\]
the windowed \emph{Fano factor} \(F_A:=\mathrm{Var}(N_A)/\mathbb E[N_A]\), and an
\emph{effective window} \(\Delta t_{\!\mathrm{eff}}\le \Delta t\) obtained from the
count autocorrelation on the same record (App.~\ref{app:noise-kernel}).
Any boundary estimator \(Y\) built from these counts admits the \emph{operational}
benchmark width
\begin{equation}
\sigma_{Y}\ \approx\ \frac{c_{\rm sec}}{\kappa}\,
\sqrt{\frac{F_A\,\overline{j}_A}{|A|\,\Delta t_{\!\mathrm{eff}}}}.
\label{eq:op-noise}
\end{equation}
Shot–noise–limited records have \(F_A\!\approx\!1\) and \(\Delta t_{\!\mathrm{eff}}\!\approx\!\Delta t\);
overdispersed or correlated arrivals are captured by the measured \((F_A,\Delta t_{\!\mathrm{eff}})\).
This is a windowed, data–driven estimate.

% ----------------------------- 4.6 -------------------------------------
\subsection{Procedure checklist (frame and wave cases)}
\label{subsec:op-checklist}

\noindent\emph{Frame.}
\begin{enumerate}\itemsep2pt
  \item Choose \(\partial U\) (boundary/sector) and, if needed, confinement geometry (\S\ref{corollary:dimensional}).
  \item Apply \eqref{axioms:stokes:identity:eq} and the \emph{cut identity} \eqref{axioms:stokes:identity:eq:cut} with \(\sigma|_W\equiv0\) on an open annulus \(W\subset U\) to identify
        a constant directed flux \(\Phi\).
  \item Fix \(c_{\rm sec}\) from one datum \(Y_\star\) on \((r_\star,A_\star)\) via \eqref{eq:op-cal-const}.
  \item Predict \(Y(r)\) from the ratio law \eqref{eq:op-one-datum-ratio}; apply corrections if attenuation
        or capacity ceilings are relevant (\S\ref{corollary:entropic-corrections}).
  \item If a finite window/area is used, report estimator width from \eqref{eq:op-noise}.
\end{enumerate}

\noindent\emph{Waves.}
\begin{enumerate}\itemsep2pt
  \item Specify the segment and constitutive description where it is treated homogeneous; ensure sources are excluded (\S\ref{waves}).
  \item Use the Poynting identity to obtain constant power \(P(r)\) and the envelope
        \eqref{eq:op-wave-envelope}; apply confinement if appropriate (C8, \S\ref{corollary:dimensional}).
  \item Calibrate with the same \(c_{\rm sec}\) and quantify bounds via
        \eqref{eq:op-rate}–\eqref{eq:op-time} for the chosen reliability \(\varepsilon\).
  \item Apply corrections if attenuation or capacity ceilings are relevant (\S\ref{corollary:entropic-corrections}).
\end{enumerate}

% ----------------------------- 4.7 -------------------------------------
\subsection{Micro–example (3D frame boundary)}
\label{subsec:op-example}

A spherical boundary of radius \(r\) in \(n=3\) has area \(4\pi r^2\).
If \(\Phi\) is the measured tube flux on a frame annulus with negligible production,
\begin{equation}
\langle j_n(r)\rangle=\frac{\Phi}{4\pi r^2} .
\end{equation}
With the frame closure \eqref{axioms:stokes:frame:eq:closure}, the informational field amplitude scales as
\begin{equation}
\big|\mathbf E_I(r)\big|=\frac{1}{\kappa}\,\frac{\Phi}{4\pi r^2}.
\end{equation}
This furnishes the frame \(1/r^2\) profile used in examples.

% ----------------------------- 4.8 -------------------------------------
\subsection{Units (Kz) and trits/nats}
\label{subsec:op-units}

Kz units are used for rates; base logs convert by \(\ln 3\):
\begin{equation}
S_{\rm nats}=(\ln 3)\,S_{\rm trits}.
\end{equation}
Formulas above are invariant under the trit/nat choice except for the explicit factors
of \(\ln 3\) in the time–information bounds \eqref{eq:op-rate}–\eqref{eq:op-time}.
(Recovering binary units replaces \(\ln 3\) by \(\ln 2\) only.)

\medskip
\noindent\emph{Literature note.}
Capacity and reliability follow Shannon and Gallager~\cite{Shannon1948,Gallager1968};
erasure cost follows Landauer and Bennett~\cite{Landauer1961,Bennett2003};
for wave transport/export see \S\ref{waves} and App.~\ref{app:forms-stokes};
for boundary counting statistics, the windowed Fano factor and autocorrelation–based \(\Delta t_{\!\mathrm{eff}}\) are standard empirical tools
(App.~\ref{app:noise-kernel}).
