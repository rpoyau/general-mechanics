% ======================================================================
% sections/06_entropic-gravity-screen.tex — Entropic gravity (screen) v1.0.2
% ======================================================================

\section{Entropic gravity: spherical screen}
\label{sec:entropic-gravity-screen}

\paragraph{Setup (Kz convention).}
Consider a static spherical screen of radius \(r\) enclosing sources with total rest–Kick \(M=\sum_i m_i c^2/h\).
Let the screen carry a \emph{bit density} \(\alpha\) (bits per unit area), so the number of bits on the screen is
\begin{equation}
N(r)=\alpha\,S_2\,r^2=\alpha\,(4\pi r^2).
\label{eq:bits-on-screen}
\end{equation}
We will connect the acceleration field outside the screen to \(M\) by counting plus causality.

\paragraph{Equipartition + Unruh (all in Kz).}
Assume equipartition of the enclosed rest–Kick over the screen bits:
\begin{equation}
M \;=\; \frac12\,N(r)\,T(r).
\label{eq:equipartition}
\end{equation}
For an observer with proper acceleration \(a(r)\), the Unruh temperature in Kz is
\begin{equation}
T(r)\;=\;\frac{a(r)}{4\pi^2 c}.
\label{eq:unruh-kz}
\end{equation}
Combining \eqref{eq:equipartition} and \eqref{eq:unruh-kz} with \eqref{eq:bits-on-screen} yields the mean exterior acceleration
\begin{equation}
a(r)
\;=\;\frac{2\pi c}{\alpha}\,\frac{M}{r^2}.
\label{eq:a-mean}
\end{equation}
Thus the inverse–square law emerges purely from (i) screen area growth, (ii) equipartition, and (iii) Unruh in Kz.

\paragraph{Calibration to measured fields.}
In the informational picture, measured acceleration is a \emph{calibration} of the informational field:
\[
\mathbf g = c_M\,\mathbf F_I,\qquad \mathbf F_I=-\nabla\Phi_I.
\]
Equation~\eqref{eq:a-mean} fixes an overall constant once a single datum is chosen.
Equivalently, one may pick \(\alpha\) so that \eqref{eq:a-mean} reproduces a reference acceleration at \((M_*,r_*)\);
all other radii then follow by \(\propto r^{-2}\).

\paragraph{Consistency with static Gauss.}
From \S\ref{sec:static-gauss-nd}, spherical symmetry with the static closure \(\mathbf j=\kappa\nabla\Phi_I\) gives
\(|\nabla\Phi_I|=\dot I_{\rm enc}/(4\pi\kappa r^2)\).
Identifying the enclosed informational rate with the equipartition/Unruh combination,
\(\dot I_{\rm enc} \propto M\,T\), one obtains \(\mathbf F_I\propto (M/r^2)\,\hat{\mathbf r}\).
Calibration (\(c_M\)) then matches \(\mathbf g\) to \eqref{eq:a-mean}.

\paragraph{Finite-count noise (shot-noise floor).}
Screen bits are finite. Any unbiased estimate of \(a(r)\) built from boundary counts has a Poisson relative uncertainty
\begin{equation}
\frac{\delta a}{a}\;\sim\;\frac{1}{\sqrt{N(r)}}\;=\;\frac{1}{\sqrt{\alpha\,4\pi r^2}}.
\label{eq:rel-noise}
\end{equation}
A minimal causal relaxation (retarded equilibration over light‑crossing time \(\tau\sim r/c\)) suggests an OU/Langevin model
\[
\dot a = -\frac{1}{\tau}\big(a-a_{\rm mean}(r)\big)+\sqrt{2D_a}\,\xi(t),\qquad
D_a \approx \frac{\sigma_a^2}{\tau},\ \ \sigma_a \approx a\,\frac{1}{\sqrt{N(r)}}.
\]
Quantitative decoherence implications for matter‑wave interferometry are developed in \S\ref{sec:examples-noise-decoherence} and App.~\ref{app:noise-kernel}.

\paragraph{Interior/exterior fields (constant density).}
If sources fill a ball of radius \(R\) with uniform rest–Kick density \(\rho\) (Kz per m\(^3\)), then
\(M(r)=\rho\,\tfrac{4\pi}{3}r^3\) for \(r<R\) and \(M(r)=\rho\,\tfrac{4\pi}{3}R^3\) for \(r\ge R\).
Equation~\eqref{eq:a-mean} gives
\[
a(r)=
\begin{cases}
\dfrac{2\pi c}{\alpha}\,\dfrac{\rho\,\tfrac{4\pi}{3}r^3}{r^2}
= \dfrac{8\pi^2 c}{3\alpha}\,\rho\,r, & r<R,\\[1.0ex]
\dfrac{2\pi c}{\alpha}\,\dfrac{\rho\,\tfrac{4\pi}{3}R^3}{r^2}, & r\ge R,
\end{cases}
\]
i.e.\ linear inside and inverse–square outside, matching the informational Gauss result.

\paragraph{Remarks.}
(1) No gravitational constant appears in the algebra; one datum fixes the single calibration/bit‑density parameter.  
(2) The \(1/\sqrt{N}\) floor is a principled limit from finite counting; Unruh adds an extremely small thermal floor in Kz.  
(3) The derivation uses only boundary area scaling and local acceleration thermality; it is compatible with the worldtube Stokes balance in \S\ref{sec:worldtube-IED}.
