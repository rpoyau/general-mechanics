% sections/02_kick_units.tex  —  Units: the Compton energy Kick  (⟂plan §02)
% Replaces earlier draft; pulls content from planck-tick-mass/note-v2.

\section{Units: The Compton Energy Kick \texorpdfstring{\(\Kk\)}{Kk}}
\label{sec:kick_units}

% ------------------------------------------------------------------
\subsection*{2.1  Frequency \,≡\, energy}
The mass–energy identity
\[
  E = mc^{2} = h\,f_{\mathrm C}(m)
  \tag{2.1}
\]
already equates every rest-mass with a \emph{frequency}.  
We therefore \emph{define} the \textbf{Kick}\,\cite{Compton1923}:
\[
  1\,\Kk \;\coloneqq\; 1\;\text{Hz}.
\]

% ------------------------------------------------------------------
\subsection*{2.2  Kilogram relegated}
The SI kilogram becomes a derived label:
\[
  1\;\text{kg} \;=\;
  1.356392489652\times10^{50}\;\Kk,
\]
retained solely for commercial continuity; all calculations proceed in
\(\Kk\).

% ------------------------------------------------------------------
\subsection*{2.3  Planck frequency: a natural ceiling}
The Planck tick \(t_{P}\approx 5.391\times10^{-44}\,\text{s}\)
sets a maximal natural frequency:
\[
f_{P}=t_{P}^{-1}\approx
1.854\times10^{43}\;\text{Hz}
= 1.854\times10^{43}\;\Kk.
\]

% ------------------------------------------------------------------
\subsection*{2.4  Prefix ladder}
\[
\begin{array}{@{}llc@{}}
\text{Name} & \text{Symbol} & \text{Definition (Hz)} \\ \hline
\text{Kick}          & \Kk  & 10^{0} \\
\text{Kilokick}      & \text{k}\Kk & 10^{3} \\
\text{Megakick}      & \text{M}\Kk & 10^{6} \\
\text{Gigakick}      & \text{G}\Kk & 10^{9} \\
\text{Terakick}      & \text{T}\Kk & 10^{12} \\
\text{Planck freq.}  & f_{P} & 1.854\times10^{43}
\end{array}
\]

% ------------------------------------------------------------------
\subsection*{2.5  Quick reference}
\[
\begin{aligned}
f_{\mathrm C}(m_{e}) &\approx 1.2356\times10^{20}\;\Kk,\\
f_{\mathrm C}(m_{p}) &\approx 2.2687\times10^{23}\;\Kk,\\
f_{\mathrm C}\bigl(1\,\text{g}\bigr) &\approx 1.3564\times10^{47}\;\Kk.
\end{aligned}
\]

\begin{tcolorbox}[title=Kick–SI round-trip,width=\linewidth]
\begin{enumerate}\itemsep0pt
  \item \textbf{Electron mass.}\quad
        $m_{e}=2.27\times10^{23}\,\Kk
        \;\longrightarrow\;
        9.11\times10^{-31}\,\text{kg}
        \;\longrightarrow\;
        2.27\times10^{23}\,\Kk$.
  \item \textbf{One joule.}\quad
        $1\,\text{J}=1.51\times10^{33}\,\Kk
        \;\longrightarrow\;
        1\,\text{J}$ (exactly).
\end{enumerate}
\end{tcolorbox}

\paragraph{Practical consequence.}
Expressing energy in \(\Kk\)

\begin{itemize}\itemsep3pt
  \item removes \(h\) and \(\hbar\) from classical formulae and sets \(k_{B}=1\);
  \item leaves the speed of light \(c\) as the bridge between space and time;
  \item demotes the \textbf{joule} to a fixed multiplier:
        \(1\,\text{J} = \tfrac{1}{h}\,\Kk \approx 1.51\times10^{33}\,\Kk\);
  \item makes mechanical work \(P V\) and all “energy” terms the same \(\Kk\) unit;
  \item converts temperature to a Kick density \(T\) and entropy to a pure Shannon tally.
\end{itemize}

Thus kilogram, joule and kelvin become veneers; once stripped away,
the thermodynamic identities in Sec.~\ref{sec:landauer} reduce to
straight arithmetic.
