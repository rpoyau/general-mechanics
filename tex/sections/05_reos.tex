% sections/05_reos.tex  —  Differential form of the REOS axiom

\section{The Differential REOS Equation}
\label{sec:reos_equation}

\paragraph{Recap of the axiom.}
From the boxed statement
\[
\Phi\!\bigl(\rho,R^{(a)}\bigr)=0\,, \qquad a=1,\dots,m,
\]
we take the total differential under the sole assumption that
$\Phi$ is $C^{1}$ and strictly monotone in each argument.

\begin{boxedeq}[eq:reos_diff]
d\rho \;=\; -\Lambda_{a}\,dR^{(a)}, 
\qquad
\Lambda_{a}
:= \frac{\partial\Phi/\partial R^{(a)}}{\partial\Phi/\partial\rho}.
\end{boxedeq}

\paragraph{Informational potentials \(\Lambda_{a}\).}
Each \(\Lambda_{a}\) is a \emph{dimensionless price per bit} for the
corresponding resource density \(R^{(a)}\).  Typical specialisations:

\begin{itemize}
\item Energy \(R^{(E)}=E\) \quad\(\Rightarrow\quad\Lambda_{E}=1/T\) in
      SI units, or simply \(\Lambda_{E}=1\) in Kick gauge.
\item Momentum \(R^{(p_i)}=p_i\)\quad\(\Rightarrow\)
      \(\Lambda_{p_i}\) has units of position in SI, dimensionless in Kick.
\item Charge–squared \(R^{(Q)}=Q^{2}\)\quad\(\Rightarrow\)
      \(\Lambda_{Q}\) resembles an inverse length (self-energy radius).
\end{itemize}

\paragraph{Why this matters.}
Equation~\eqref{eq:reos_diff} is the algebraic link that powers every
Tier-1 corollary:

\begin{enumerate}[label=(\roman*)]
\item inserting \(d\rho=-\Lambda_{a}\,dR^{(a)}\) into the continuity
      identity gives the informational force
      \(\mathbf f=-R^{(a)}\nabla\Lambda_{a}\)
      (Sec.~\ref{sec:force});
\item replacing \(dR^{(a)}\) by finite differences yields the
      resource–clock uncertainty bound
      \(\Delta R^{(a)}\,\Delta\Theta_{(a)}\gtrsim 1/(2\ln2)\)
      (Appendix~\ref{app:covariant});
\item supplying an explicit \(R^{(a)}\!\mapsto\!C(R^{(a)})\) constraint
      recovers classical capacity ceilings (Sec.~\ref{sec:tier1}).
\end{enumerate}

\paragraph{Scale-factor invariance.}
Because \(\Lambda_{a}\) is a \emph{ratio of derivatives}, the overall
entropy scale \(S=C\ln W\) cancels; all subsequent equations are valid
for \(C=1/\ln2\) (bits), \(C=1\) (nats), or Boltzmann’s \(C=k_{B}\).

The remainder of the note repeatedly plugs
Eq.\,\eqref{eq:reos_diff} into specific geometries and resource lists;
every result about forces, capacities, or uncertainty budgets traces
straight back to this differential form.
