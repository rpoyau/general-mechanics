% ======================================================================
% 02_axioms.tex — Axioms (manifold-first, Stokes-only core)
% ======================================================================

\section{Axioms}
\label{axioms}

\paragraph{Three pillars (thermodynamics, information–relational, Stokes).}
(i) \emph{Thermodynamics as content.} The piston thought–experiment generalizes via Stokes; the operational
\emph{thermodynamic dictionary} is used in \S\ref{corollary:info-gas}–\S\ref{corollary:frame-boundary} and recorded once
in \S\ref{operationalization}. Identification of informational and entropic quantities is stated below.\\
(ii) \emph{Information theory / relational.} Counting and priors are Shannon‑based over a \emph{ternary}
alphabet (\emph{yes/no/unknown}); observables are \emph{relational}. Trit/nat choices differ by a factor of
\(\ln 3\) and do not affect geometric statements.\\
(iii) \emph{Geometric calculus (Stokes identity).} A single generalized Stokes identity is applied on a chosen region
and its time extension, with \worldtube\ decomposition (\S\ref{axioms:stokes:identity};
App.~\ref{app:forms-stokes}). Frame \emph{boundary} invariants use open subsets with \(\sigma|_W\equiv0\).

% ---------------------------- Relational -------------------------------

\subsection{Relational}
\label{axioms:relational}

\subsubsection{Relational primacy}
\label{axioms:relational:primacy}
Relations are the primitives; observables are relational. A model specifies a region \(U\subset M\) and its boundary
\(\partial U\).

\subsubsection{Counting \& priors (trits/nats)}
\label{axioms:relational:counting}
A configuration determined by \(n\) ternary distinctions has \(W=3^{\,n}\) microstates. With a prior \(p(x)\) on an
alphabet \(\mathcal A\) of size \(3\), the uncertainty is the Shannon entropy
\(H(p)=-\sum_{x\in\mathcal A} p(x)\ln p(x)\) (nats). Units are a log‑base choice:
\[
H_{\text{trits}}=\frac{H_{\text{nats}}}{\ln 3}.
\]
“Unknown’’ may be represented explicitly by a symbol \(\bot\in\mathcal A\) without altering any geometric/Stokes
statements. All later flux/current formulas are invariant under nat/trit choice up to factors of \(\ln 3\).

\paragraph{Where Stokes is applied.}
The same generalized Stokes identity is applied on
\begin{enumerate}\itemsep0.25em
  \item a \emph{frame} spatial region \(U\) (boundaries/sectors), and
  \item the time‑extended region \(U\times[t_0,t_1]\) (\emph{\worldtube}, dynamics).
\end{enumerate}
For frame \emph{boundary} arguments an \emph{open set} \(W\subset U\) with \(\sigma|_W\equiv0\) is used; this yields
tube invariants and inverse‑area rules. Conventions for forms, orientations, and Hodge duals are summarized
in App.~\ref{app:forms-stokes}.

% ---------------------------- Generalized Stokes -----------------------

\subsection{Generalized Stokes}
\label{axioms:stokes}

\subsubsection{Identity (continuity)}
\label{axioms:stokes:identity}
Let \(j:=J^\mu dx_\mu\) be the informational/entropic \emph{flux} \(1\)-form and \(j^{(n)}:=\!*j\) its Hodge‑dual
\(n\)-form on \(U\times[t_0,t_1]\). With production \(\Sigma:=\sigma\,\mathrm{vol}\),
\begin{equation}
d\,j^{(n)}=\Sigma
\quad\Longleftrightarrow\quad
\partial_\mu J^\mu=\sigma .
\label{axioms:stokes:identity:eq}
\end{equation}
The integral identity is
\begin{equation}
\int_{\partial(U\times[t_0,t_1])}\! j^{(n)}
\;=\;
\int_{U\times[t_0,t_1]}\! \Sigma .
\label{axioms:stokes:identity:eq:integral}
\end{equation}
On a \emph{fixed} \(U\) with outward unit normal \(\mathbf n\), define the enclosed information
\(I_{\mathrm{enc}}(t):=\int_U i(x,t)\,\mathrm d^n x\), the inward/outward boundary fluxes
\(\Phi_{\mathrm{in/out}}(t):=\int_{\partial U}\!\big(\mathbf j\!\cdot\!\mathbf n\big)_{\mp}\,\mathrm dA\),
and the interior production \(\Pi(t):=\int_U \sigma\,\mathrm d^n x\). Then
\begin{equation}
\dot I_{\mathrm{enc}}(t)=\Phi_{\mathrm{in}}(t)-\Phi_{\mathrm{out}}(t)+\Pi(t),
\qquad\text{(the \emph{cut identity}).}
\label{axioms:stokes:identity:eq:cut}
\end{equation}
In local coordinates this is the continuity equation
\begin{equation}
\partial_t i(x,t)+\nabla\!\cdot \mathbf j(x,t)=\sigma(x,t).
\label{axioms:stokes:identity:eq:local-continuity}
\end{equation}

\paragraph{Boundary Identification: Information, Entropy, and Units.}
On boundaries, the information current is identified with the entropy current: \(j \equiv j_S\).
The normal component is \(j_n = \mathbf{j} \cdot \mathbf{n}\). Units are counts per unit area per unit time, with the
specific count unit (e.g., nats, trits) fixed by the choice of logarithm base. The corresponding volume
densities are also identified, \(i \equiv s\), with units of nats per unit volume. Window integrals, which provide
total counts and window‑averaged currents, are defined in \S\ref{operationalization}.

\paragraph{Carriers.}
The statements hold for any carrier. Additional informational/entropic sources (e.g.\ charge, mass, energy/momentum)
admit the same Stokes formulation on the same manifold; \emph{boundary} invariants apply on open subsets where the
corresponding production vanishes.

\paragraph{\worldtube\ note.}
If the boundary moves, the side contribution uses the integrand \(\mathbf j\!\cdot\!\mathbf n - i\,v_n\); the same
decomposition applies on the time‑extended boundary; see App.~\ref{app:forms-stokes}.

\subsubsection{Frame closure}
\label{axioms:stokes:frame}
On a \emph{frame} in linear, isotropic, near‑equilibrium regimes,
\begin{equation}
\mathbf E_I:=-\,\nabla\Phi_I,
\qquad
\mathbf j=-\,\kappa\,\mathbf E_I .
\label{axioms:stokes:frame:eq:closure}
\end{equation}
Here \(\Phi_I\) is the informational potential, \(\mathbf E_I\) its field, and \(\kappa\) a constitutive scale of the
channel. This frame closure is used to derive boundary/sector relations; time‑extended segments and
wave transport are treated separately (see \S\ref{waves}).

\subsubsection{Relational Equation of State (REOS)}
\label{axioms:reos}
Let \(R^{(a)}\) denote admissible boundary/control variations (e.g., apertures, partitions, admissible alphabets,
connectivity), and let \(\rho=\rho(R)\) be a \emph{calibrated rate density (informational cost)} on the control manifold;
“calibrated’’ here only means that a single sector scale has been fixed once outside the axioms
(see \S\ref{operationalization}). Define the price one‑form \(\lambda:=\lambda_a\,dR^{(a)}\) with components
\[
\lambda_a:=-\,\frac{\partial\rho}{\partial R^{(a)}}
\]
when \(\rho\) is differentiable. The REOS statement is
\begin{equation}
d\rho \;=\; -\,\lambda_a\,dR^{(a)} .
\label{axioms:reos:eq}
\end{equation}
If \(d\lambda=0\) (integrable sector), \(\rho\) is path‑independent and Maxwell‑type identities hold.
If \(d\lambda\neq 0\) (non‑integrable sector), cyclic variations of \(R^{(a)}\) entail nonnegative production \(\sigma\)
in the identities \eqref{axioms:stokes:identity:eq}–\eqref{axioms:stokes:identity:eq:cut}.

\paragraph{Boundary heat mapping (windows and frames).}
With the identification \(j\equiv j_S\) on boundaries, define the \emph{total-heat} on a boundary patch
\(A\subset\partial U\) over a window \([t_0,t_1]\) by
\begin{equation}
Q_A\big([t_0,t_1]\big)\;:=\;\int_{t_0}^{t_1}\!\!\int_A T\,j_S\,\mathrm dA\,\mathrm dt .
\label{axioms:reos:eq:total-heat}
\end{equation}
On a \emph{frame}, this reduces to the instantaneous \emph{heat‑flux} density
\begin{equation}
q_n \;:=\; T\,j_n .
\label{axioms:reos:eq:flux-heat}
\end{equation}
Units (Kz): see App.~\ref{app:kick}.

\subsubsection{Canonical rates from Stokes on \((q,t)\)}
\label{axioms:path-rate}
Define the rate (Poincaré–Cartan) \(1\)-form on configuration–time space
\begin{equation}
\Theta_{\mathcal R}=P_i\,dq^i- H(q,P,t)\,dt .
\label{axioms:path-rate:eq:pc-form}
\end{equation}
Evaluating Stokes on a rectangular \(2\)-surface \(\Sigma\) spanned by two nearby paths between \(t\) and \(t+dt\) gives
\[
\int_{\partial\Sigma}\Theta_{\mathcal R}\;=\;\int_{\Sigma} d\Theta_{\mathcal R}\;=\;0,
\]
which is equivalent to the canonical rate equations
\begin{equation}
\dot q^i=\partial_{P_i}H,
\qquad
\dot P_i=-\,\partial_{q^i}H .
\label{axioms:path-rate:eq:canonical}
\end{equation}
The Euler–Lagrange form follows under convex duality (see \S\ref{corollary:least-action}).

\medskip
\noindent\emph{Notes.}
(1) Trits versus nats are related by a factor \(\ln 3\); the axioms are invariant under this unit choice.\\
(2) Propagation closures and wave transport are in \S\ref{waves} (with dimension/confinement
summarized in \S\ref{corollary:dimensional}).\\
(3) Operational sector mappings and units (Kz) are specified once outside this section (App.~\ref{app:kick});
the boundary dictionary appears in \S\ref{operationalization}.

\medskip
\noindent\emph{Literature note (provenance).}
The geometric/Stokes ingredients are recalled in Apps.~\ref{app:forms-stokes}--\ref{app:nD-boundaries}; see also standard
references on forms and Stokes in physics \cite{Frankel2011,Spivak1965,BottTu1982}. The Poincaré–Cartan/Hamiltonian rate
structure underlying \S\ref{axioms:path-rate} follows \cite{Arnold1989}; convex duality/Legendre transform background appears in
\cite{Rockafellar1970}.
