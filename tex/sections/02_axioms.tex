% ======================================================================
% 02_axioms.tex — Axioms (manifold-first, Stokes-only core)
% ======================================================================

\section{Axioms}
\label{axioms}

\paragraph{Three pillars (thermodynamics, information–relational, Stokes).}
(i) \emph{Thermodynamics as content.} The piston thought–experiment generalizes via Stokes.\\
(ii) \emph{Information theory / relational.} Counting and priors are Shannon‑based over a \emph{ternary}
alphabet (\emph{yes/no/unknown}); observables are relational.\\
(iii) \emph{Geometric calculus (Stokes identity).} A single generalized Stokes identity \eqref{axioms:stokes:identity:eq}
is applied on a chosen region and its time extension, with cap/side (\worldtube) decomposition
(\S\ref{axioms:worldtube}). Frame boundary invariants use open subsets with \(\sigma|_W\equiv0\).

% ---------------------------- Glossary -------------------------------
\subsection{Glossary of agnostic terms}
\label{axioms:glossary}

\paragraph{Window (definition).}
\label{axioms:window}
A \emph{window} is a finite interval \([t_0,t_1]\) on the same region.

\paragraph{Frame (definition).}
\label{axioms:frame}
A \emph{frame} is a time‑frozen spatial slice \(V_t:=U\cap\{t=\mathrm{const}\}\).
The \frame form of any window statement is the limit \(\Delta t\!\to\!0\) evaluated on \(V_t\).

\paragraph{Worldtube (cap/side; moving boundary).}
\label{axioms:worldtube}
The time‑extended boundary \(\partial(U\times[t_0,t_1])\) decomposes into a top cap, a bottom cap, and a side.
For moving boundaries with normal speed \(v_n\), the side integrand is
\[
\mathbf j\!\cdot\!\mathbf n \;-\; i\,v_n .
\]

\paragraph{Directional split.}
\label{axioms:directional-split}
For any scalar \(a\), \(a_\pm:=\tfrac12(a\pm|a|)\); inward/outward boundary fluxes use
\((\mathbf j\!\cdot\!\mathbf n)_-\) and \((\mathbf j\!\cdot\!\mathbf n)_+\).

\paragraph{Boundary identification and heat mapping.}
\label{axioms:boundary-heat}
On boundaries, the information current is identified with the entropy current: \(j \equiv j_S\) (counts / area / time).
The volume density is \(i \equiv s\) (nats / vol). With a boundary temperature field \(T\), the heat flux is \(q := T\,j_S\) (Kz).

% ---------------------------- Relational -------------------------------
\subsection{Relational}
\label{axioms:relational}

\subsubsection{Relational primacy}
\label{axioms:relational:primacy}
Relations are the primitives; observables are relational. A model specifies a region \(U\subset M\) and its boundary
\(\partial U\).

\subsubsection{Counting \& priors (trits/nats)}
\label{axioms:relational:counting}
A configuration determined by \(n\) ternary distinctions has \(W=3^{\,n}\) microstates. With a prior \(p(x)\) on an
alphabet \(\mathcal A\) of size \(3\), the uncertainty is the Shannon entropy
\(H(p)=-\sum_{x\in\mathcal A} p(x)\ln p(x)\) (nats), with
\[
H_{\text{trits}}=\frac{H_{\text{nats}}}{\ln 3}.
\]

% ---------------------------- Mechanical Axioms -----------------------
\subsection{Mechanical Axioms}
\label{axioms:stokes}

\subsubsection{Identity (continuity)}
\label{axioms:stokes:identity}
The informational/entropic flux \(1\)-form \(j\) (with Hodge‑dual \(j^{(n)}\))
and production \(\Sigma:=\sigma\,\mathrm{vol}\) satisfy
\begin{equation}
d\,j^{(n)}=\Sigma
\quad\Longleftrightarrow\quad
\partial_\mu J^\mu=\sigma .
\label{axioms:stokes:identity:eq}
\end{equation}
The integral form is
\begin{equation}
\int_{\partial(U\times[t_0,t_1])}\! j^{(n)}
\;=\;
\int_{U\times[t_0,t_1]}\! \Sigma .
\label{axioms:stokes:identity:eq:integral}
\end{equation}
On a fixed \(U\), the \emph{cut identity} is
\begin{equation}
\dot I_{\mathrm{enc}}(t)=\Phi_{\mathrm{in}}(t)-\Phi_{\mathrm{out}}(t)+\Pi(t),
\label{axioms:stokes:identity:eq:cut}
\end{equation}
where \(I_{\mathrm{enc}}\), \(\Phi_{\mathrm{in/out}}\), and \(\Pi\) denote the volume/boundary integrals of
\(i\), \((\mathbf j\!\cdot\!\mathbf n)_{\mp}\), and \(\sigma\).
The local coordinate form is
\begin{equation}
\partial_t i(x,t)+\nabla\!\cdot \mathbf j(x,t)=\sigma(x,t).
\label{axioms:stokes:identity:eq:local-continuity}
\end{equation}

% ----------------------------- Constitutive closures (manifold properties) -------------------
\subsubsection{Constitutive closures (manifold properties)}
\label{axioms:closures}

\paragraph{Linear channel closure.}
\label{axioms:linear-closure}
In linear, isotropic, near‑equilibrium regimes,
\begin{equation}
\mathbf E_I:=-\,\nabla\Phi_I,
\qquad
\mathbf j=-\,\kappa\,\mathbf E_I .
\label{axioms:linear-closure:eq}
\end{equation}

\paragraph{Wave‑supporting closure.}
\label{axioms:waves:closure}
On a window \([t_0,t_1]\), fields \((U,V)\) with a linear, causal constitutive map
\begin{equation}
V=\chi:U,
\label{axioms:waves:eq:chi}
\end{equation}
have stored energy
\begin{equation}
u \;:=\; \tfrac12\,U{:}V ,
\label{axioms:waves:eq:energy}
\end{equation}
and a spatial flux \(\mathbf S\) satisfying
\begin{equation}
\partial_t u \;+\; \nabla\!\cdot\mathbf S \;=\; -\,\Pi,
\qquad \Pi\ge 0.
\label{axioms:waves:eq:balance}
\end{equation}

% ------------------------------- REOS ----------------------------------
\subsubsection{Relational Equation of State (REOS)}
\label{axioms:reos}
For admissible boundary/control variations \(R^{(a)}\) and calibrated rate density \(\rho=\rho(R)\),
the price one‑form \(\lambda:=\lambda_a\,dR^{(a)}\) (with \(\lambda_a:=-\,\partial\rho/\partial R^{(a)}\)) satisfies
\begin{equation}
d\rho \;=\; -\,\lambda_a\,dR^{(a)} .
\label{axioms:reos:eq}
\end{equation}
If \(d\lambda=0\), \(\rho\) is path‑independent. If \(d\lambda\neq 0\), cyclic variations entail nonnegative production \(\sigma\)
in \eqref{axioms:stokes:identity:eq}–\eqref{axioms:stokes:identity:eq:cut}.

% ------------------------ Canonical rates on (q,t) ---------------------
\subsubsection{Canonical rates from Stokes on \((q,t)\)}
\label{axioms:path-rate}
For a rate (Poincaré–Cartan) \(1\)-form on configuration–time space
\begin{equation}
\Theta_{\mathcal R}=P_i\,dq^i- H(q,P,t)\,dt ,
\label{axioms:path-rate:eq:pc-form}
\end{equation}
evaluating Stokes on a rectangular \(2\)-surface \(\Sigma\) spanned by two nearby paths gives
\[
\int_{\partial\Sigma}\Theta_{\mathcal R}\;=\;\int_{\Sigma} d\Theta_{\mathcal R}\;=\;0,
\]
which is equivalent to
\begin{equation}
\dot q^i=\partial_{P_i}H,
\qquad
\dot P_i=-\,\partial_{q^i}H .
\label{axioms:path-rate:eq:canonical}
\end{equation}
