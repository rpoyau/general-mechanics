% sections/16_tier1_jacobson.tex  —  Tier-1: Null-tube → Einstein equation
% Plan ref: §16  (≈1 page)

\section{Tier-1 Corollary III: Null-Tube to Einstein Field Equation}
\label{sec:tier1_jacobson}

\paragraph{Setup (Jacobson 1995 template).}
Choose a space-time point $p$ and a local Rindler horizon patch
generated by a bundle of null geodesics $k^{\mu}$ normalised so that
$k^{\mu}\nabla_{\mu}\lambda = 1$, with $\lambda=0$ at $p$.
Take the \emph{null world-tube} $V$ defined by $0\le\lambda\le\lambda_0$
and cross-sectional area element $dA$.

\paragraph{Entropy–flux through the tube.}
The REOS continuity identity on the null volume gives
\[
\Delta H = \int_{V}\sigma\,dV
         = \int_{\Sigma_\lambda}\!\!\rho\,k^{\mu}d\Sigma_{\mu}.
\]
Using the Raychaudhuri equation
$d(\sqrt{h})/d\lambda = \theta\sqrt{h}$ and expanding
$\theta(\lambda)=\theta_0+\lambda R_{\mu\nu}k^{\mu}k^{\nu}+O(\lambda^2)$
one finds (to first order)
\[
\Delta H
   = \frac{\theta_0\,\lambda_0}{2\ln 2}\,dA
     + \frac{\lambda_0^{2}}{2\ln 2}
       R_{\mu\nu}k^{\mu}k^{\nu}dA .
\]

\paragraph{Heat flow (boost energy).}
With boost Killing vector $\chi^{\mu}=2\pi\lambda k^{\mu}$,
the heat flow across the horizon is
\[
\delta Q
  = \int_{0}^{\lambda_0}\!\!\!
    T_{\mu\nu}\chi^{\mu}k^{\nu}\,d\lambda\,dA
  = 2\pi\,\lambda_0^{2}T_{\mu\nu}k^{\mu}k^{\nu}\frac{dA}{2}.
\]

\paragraph{REOS clock bound as Clausius relation.}
The resource–clock uncertainty
$\Delta E\,\Delta\lambda \gtrsim 1/(2\ln 2)$
sets the “local Unruh temperature”
$T = 1/2\pi\lambda_0$ (Kick gauge).  
Demanding the Clausius condition
$\delta Q = T\,\Delta H$
for \emph{all} null vectors $k^{\mu}$ and choices of $dA$
implies
\[
R_{\mu\nu}-\frac12 R\,g_{\mu\nu}
  = 8\pi\,T_{\mu\nu},
\]
i.e.\ the Einstein field equation with Newton constant
$G=1$ in Kick units.

\paragraph{Interpretation.}
No horizon entropy proportional to area was postulated; instead,
\emph{Shannon entropy density} $\rho$ and the REOS clock bound provided
the missing scale.  The derivation therefore supports the reading:
“Einstein = informational Clausius,” valid whenever the REOS axiom holds
and the null‐tube uncertainty bound is saturated.

\paragraph{References.}
Original argument: \cite{Jacobson1995}.  
Information-theoretic swaps of thermodynamic $S$ for Shannon $H$:
\cite{Padmanabhan2010,Verlinde2011}.  The present derivation differs by
removing the $S\!\propto\!A$ postulate; the area term emerges from the
Raychaudhuri expansion alone given the REOS clock bound.
