% ======================================================================
% 03_wave.tex — Universal wave transport & refraction (Stokes + closure)
% ======================================================================

\section{Universal wave transport \& refraction}
\label{waves}

\subsection{From Stokes to waves: closure $\Rightarrow$ hyperbolic transport}
\label{waves:closure}
On any \emph{window} where the channel admits a pair $(U,V)$ related by a linear, causal constitutive map
$V=\chi:U$ (possibly dispersive on the window) and supports a quadratic stored energy
\begin{equation}
u \;:=\; \tfrac12\,U\!:\!V ,
\label{waves:eq:potential-energy}
\end{equation}
together with a spatial flux $\mathbf S$ satisfying the local \emph{identity}
\begin{equation}
\partial_t u \;+\; \nabla\!\cdot\mathbf S \;=\; -\,\Pi,
\qquad \Pi\ge 0,
\label{waves:eq:local-balance}
\end{equation}
the Stokes identity yields \emph{radius‑independent export} on any source‑free, lossless subwindow $(\Pi=0)$:
\begin{equation}
P(r):=\int_{\Sigma(r)}\!\langle \mathbf S\rangle\!\cdot n\,dA \;=\; \text{const in } r,
\label{waves:eq:const-power}
\end{equation}
and therefore the far‑field envelope \emph{scales as}
\begin{equation}
\big\langle |U(r)| \big\rangle \;\propto\; [A_n(r)]^{-1/2},
\qquad A_n(r)=S_{n-1}\,r^{\,n-1}.
\label{waves:eq:envelope}
\end{equation}
With attenuation $\mu$ along a worldtube, $P(r)\propto \exp\!\big[-\int\mu\,ds\big]$ and the envelope acquires
\begin{equation}
\big\langle |U(r)| \big\rangle \;\propto\; [A_n(r)]^{-1/2}\,
\exp\!\left[-\tfrac12\!\int \mu\,ds\right].
\label{waves:eq:envelope-loss}
\end{equation}
Confinement replaces $A_n(r)$ by an effective cross‑section/circumference:
\begin{equation}
A_\Delta(r)=\Delta\Omega\,r^{\,n-1},
\qquad
\big\langle |U(r)| \big\rangle \;\propto\; [A_\Delta(r)]^{-1/2}.
\label{waves:eq:confinement}
\end{equation}

\paragraph{Remark (generality and Info–EM example).}
Equation \eqref{waves:eq:local-balance} is the \emph{wave‑transport closure template}. Info–EM is one realization with
$U\!\equiv\!F_I$, $V\!\equiv\!H_I$, $H_I=\chi:F_I$, and the same identity \eqref{waves:eq:local-balance}. Under this
realization, the informational field amplitude $E_I$ follows the envelope law \eqref{waves:eq:envelope} and loss factor
\eqref{waves:eq:envelope-loss}.

\subsection{Eikonal limit, Fermat principle, and tube transport}
\label{waves:eikonal}
Assume a high‑frequency ansatz on a homogeneous or slowly varying window,
\begin{equation}
U(x,t)\;\sim\; \Re\!\Big\{ A(x)\,e^{i\omega S(x)-i\omega t}\Big\},
\qquad
\omega\gg\text{(variation scales)},
\label{waves:eikonal:eq:ansatz}
\end{equation}
with slowly varying amplitude $A$ and phase $S$ (eikonal).
The constitutive description defines an \emph{effective index} $n(x)$ (equivalently, a local speed $c_{\rm eff}=c_I/n$)
from the dispersion implied by $V=\chi:U$. At leading order, $S$ satisfies the Hamilton–Jacobi equation
\begin{equation}
|\nabla S(x)|^2 \;=\; n^2(x)\,\frac{\omega^2}{c_I^2},
\label{waves:eikonal:eq}
\end{equation}
and rays are the integral curves of $\mathbf k:=\nabla S$; they extremize the optical time
\begin{equation}
\delta\,\int_\gamma \frac{n(x)}{c_I}\,ds \;=\; 0,
\label{waves:eikonal:eq:fermat}
\end{equation}
yielding Snell’s law at interfaces.
Tube transport for the slowly varying amplitude gives
\begin{equation}
\frac{d}{ds}\!\Big(|A|^2 A_\perp\Big) \;=\; -\,|A|^2 A_\perp\,\mu(s),
\qquad\Rightarrow\qquad
|A|^2 A_\perp \propto \exp\!\left[-\!\int\mu\,ds\right],
\label{waves:eikonal:eq:tube-transport}
\end{equation}
so $|A|\propto A_\perp^{-1/2}$ (lossless), consistent with \eqref{waves:eq:envelope}.

\subsection{Nonlinear refraction from constitutive feedback}
\label{waves:nonlinear}
If the constitutive map $\chi$ depends on a local invariant of the field, e.g.\ the energy density
$u=\tfrac12\,U\!:\!V$ or occupancy on the window, then the effective index becomes \emph{amplitude‑dependent}:
\begin{equation}
n\;=\;n_0 \;+\; \alpha\,u \;+\; \mathcal O(u^2) \qquad (\text{weak feedback}).
\label{waves:eq:nonlinear}
\end{equation}
Using \eqref{waves:eikonal:eq:tube-transport}, $u\propto |A|^2\propto A_\perp^{-1}$ (lossless), so the ray equation acquires a
self‑consistent bending term:
\begin{equation}
\frac{d}{ds}\!\left(\frac{\nabla S}{n}\right)
\;=\;
\nabla\!\big(\ln n(u)\big)
\;\;\Rightarrow\;\;
\text{\emph{focusing/defocusing} driven by}\;\ \nabla u,
\label{waves:nonlinear:eq:ray-bending}
\end{equation}
closing the loop \(\text{geometry}\to |A|\to u\to n(u)\to \text{geometry}\).
In differential‑geometric language, the expansion $\theta:=d(\ln A_\perp)/ds$ along a congruence of rays follows a
Raychaudhuri‑type identity
\begin{equation}
\frac{d\theta}{ds}
\;=\;
-\,\tfrac12\,\theta^2 \;-\; \varsigma^2 \;-\; \mathcal R_{\rm eff}[n(u)]
\;+\; \text{attenuation terms},
\label{waves:nonlinear:eq:raychaudhuri-analogue}
\end{equation}
where $\varsigma$ denotes the shear magnitude and $\mathcal R_{\rm eff}[n(u)]$ encodes inhomogeneity of $n(u)$.
No gravitational postulate is invoked: \eqref{waves:nonlinear:eq:raychaudhuri-analogue} is the refractive focusing identity
induced by the feedback \(n\!\circ\!u\!\circ\!A_\perp\).

\subsection{Causality–Dispersion (KK‑window corollary)}
\label{waves:causality-dispersion}
When a channel segment is linear, time‑invariant, and causal on a given window
(i.e., the constitutive map $\chi(\omega)$ is retarded and stationary there),
the measured dispersion on that window satisfies the Kramers–Kronig Hilbert–pair relation.
We refer to this observed local rule as the \emph{Causality–Dispersion (KK‑window) corollary} and use it
as a consistency witness on that window.

\subsection{Worldtubes, moving boundaries, and boundary sectors}
\label{waves:worldtube}
When boundaries move, use the cap–side (worldtube) form with side integrand
$\mathbf j\!\cdot\! \mathbf n - i\,v_n$ (App.~\ref{app:forms-stokes}).
Export is constant for fixed solid‑angle boundary sectors $\Delta\Omega$; envelopes
follow \([A_\Delta(r)]^{-1/2}\) with $A_\Delta(r)=\Delta\Omega\,r^{\,n-1}$.

\subsection{Summary}
\label{waves:summary}
Waves are a \emph{consequence} of Stokes plus a hyperbolic closure: they carry constant exported power on lossless
segments, have far‑field envelopes $\propto [A_n(r)]^{-1/2}$, follow Fermat/Snell in the eikonal limit, and exhibit
\emph{nonlinear refraction} whenever the index depends on the locally stored energy. The feedback loop
$A_\perp \leftrightarrow u \leftrightarrow n(u)$ provides a mechanism for self‑focusing/defocusing entirely
within the informational transport template. Capacity and counting bounds limit the sharpness and inferability of
these effects on finite windows.
