% ======================================================================
% 03_wave.tex — Universal wave transport & refraction (consequence of Stokes + closure)
% ======================================================================

\section{Universal wave transport \& refraction}
\label{sec:waves}

\subsection{From Stokes to waves: closure $\Rightarrow$ hyperbolic transport}
\label{subsec:waves-closure}
On any window where the channel admits a pair $(U,V)$ related by a linear, causal map $V=\chi:U$
(possibly dispersive on the window) and supports a quadratic stored energy
\begin{equation}
u \;:=\; \tfrac12\,U\!:\!V ,
\label{eq:w-u}
\end{equation}
together with a spatial flux $\mathbf S$ obeying the local balance
\begin{equation}
\partial_t u \;+\; \nabla\!\cdot\mathbf S \;=\; -\,\Pi, \qquad \Pi\ge 0,
\label{eq:w-local-balance}
\end{equation}
the Stokes identity implies \emph{radius‑independent export} on any source‑free, lossless subwindow $(\Pi=0)$:
\begin{equation}
P(r):=\int_{\Sigma(r)}\!\langle \mathbf S\rangle\!\cdot n\,dA \;=\; \text{const in } r,
\label{eq:w-const-power}
\end{equation}
and thus the far‑field envelope
\begin{equation}
\big\langle |U(r)| \big\rangle \;\propto\; [A_n(r)]^{-1/2},
\qquad A_n(r)=S_{n-1}\,r^{\,n-1}.
\label{eq:w-envelope}
\end{equation}
With attenuation $\mu$ along a worldtube, $P(r)\propto e^{-\int\mu\,ds}$ and the envelope acquires
$e^{-\tfrac12\int\mu\,ds}$. Confinement replaces $A_n(r)$ by the effective cross‑section/circumference
(App.\,\ref{app:nD-screens}; cf.\ C4).

\paragraph{Remark (generality).}
Equation \eqref{eq:w-local-balance} is precisely the \emph{wave transport closure template} used elsewhere:
Info--EM (C4) is one instance with $U\!\equiv\!F_I$, $V\!\equiv\!H_I$. Channels that are purely diffusive
(parabolic) do not admit \eqref{eq:w-local-balance} in hyperbolic form; however, any finite relaxation/inertia
(Maxwell–Cattaneo/telegraph‑type response) regularizes diffusion into \emph{damped} wave transport on short
windows, bringing the channel under \eqref{eq:w-local-balance}.

\subsection{Eikonal limit, Fermat principle, and Snell law}
\label{subsec:waves-eikonal}
Assume a high‑frequency ansatz on a homogeneous window,
\begin{equation}
U(x,t)\;\sim\; \Re\!\Big\{ A(x)\,e^{i\omega S(x)-i\omega t}\Big\}, \qquad \omega\gg \text{(variation scales)},
\label{eq:w-ansatz}
\end{equation}
with slowly varying amplitude $A$ and phase $S$ (eikonal).
The constitutive description defines an \emph{effective index} $n(x)$ (equivalently, a local speed $c_{\rm eff}=c_I/n$)
from the dispersion implied by $V=\chi:U$. At leading order, $S$ satisfies the Hamilton–Jacobi (eikonal) equation
\begin{equation}
|\nabla S(x)|^2 \;=\; n^2(x)\,\frac{\omega^2}{c_I^2}.
\label{eq:w-eikonal}
\end{equation}
Rays are the integral curves of $\mathbf k:=\nabla S$; they extremize the optical time
\begin{equation}
\delta\,\int_\gamma \frac{n(x)}{c_I}\,ds \;=\; 0,
\label{eq:w-fermat}
\end{equation}
giving Snell/Fermat refraction at interfaces and graded‑index bending in inhomogeneous $n(x)$.
At next order, amplitude transport and Poynting export yield the familiar tube law
\begin{equation}
\frac{d}{ds}\Big(|A|^2 A_\perp\Big) \;=\; -\,\int_\perp \mu\,d\ell,
\qquad A_\perp\ \text{cross‑sectional area of the ray tube},
\label{eq:w-tube-transport}
\end{equation}
so $|A|^2\propto A_\perp^{-1}$ on lossless segments, consistent with \eqref{eq:w-envelope}.

\subsection{Nonlinear refraction from constitutive feedback}
\label{subsec:waves-nonlinear}
If the constitutive map $\chi$ depends on a local invariant of the field, e.g.\ the energy density
$u=\tfrac12\,U\!:\!V$ or occupancy on the window, then the effective index becomes \emph{amplitude‑dependent}:
\begin{equation}
n\;=\;n_0 \;+\; \alpha\,u \;+\; \mathcal O(u^2) \qquad (\text{weak feedback}).
\label{eq:w-n-of-u}
\end{equation}
Using \eqref{eq:w-tube-transport}, $u\propto |A|^2\propto A_\perp^{-1}$ (lossless), so the ray equation acquires a
self‑consistent bending term:
\begin{equation}
\frac{d}{ds}\!\left(\frac{\nabla S}{n}\right)
\;=\;
\nabla\!\big(\ln n(u)\big)
\;\;\Rightarrow\;\;
\text{\emph{focusing/defocusing} driven by}\;\ \nabla u,
\label{eq:w-ray-bending}
\end{equation}
closing the loop \(\text{geometry}\to |A|\to u\to n(u)\to \text{geometry}\).
In differential‑geometric language, the expansion $\theta:=d(\ln A_\perp)/ds$ along a congruence of rays obeys a
Raychaudhuri‑type balance
\begin{equation}
\frac{d\theta}{ds}
\;=\;
-\,\tfrac12\,\theta^2 \;-\; \sigma^2 \;-\; \mathcal R_{\rm eff}[n(u)]
\;+\; \text{attenuation terms},
\label{eq:w-raychaudhuri-analogue}
\end{equation}
where $\sigma$ is shear and $\mathcal R_{\rm eff}[n(u)]$ encodes inhomogeneity of $n(u)$. No gravitational
postulate is invoked: \eqref{eq:w-raychaudhuri-analogue} is the refractive focusing identity induced by the
feedback \(n\!\circ\!u\!\circ\!A_\perp\).

\paragraph{Operational bounds (capacity and noise).}
Finite capacity and counting noise bound the achievable $|A|$ and the resolvable curvature of $n(u)$ on a window.
With the C5 rate bound and C6 estimator floor, refraction strength and its inference obey
\[
\Delta t\ \gtrsim\ \max\!\left\{
\frac{\Delta I_{\rm trits}}{C_{\rm chan}(\varepsilon)},\;
\frac{\Delta I_{\rm trits}\,\ln 3\,T}{P}
\right\},
\qquad
\sigma_{U}\ \propto\ \sqrt{\frac{F_A\,\overline{j}_A}{|A|\,\Delta t_{\!\mathrm{eff}}}},
\]
so large gradients $\nabla n(u)$ require either longer integration or larger apertures.

\subsection{Worldtubes, moving screens, and sectors}
\label{subsec:waves-worldtube}
When screens move, use the cap–side (worldtube) form with side integrand $\mathbf j\!\cdot\! \mathbf n - i\,v_n$
(App.~\ref{app:forms-stokes}). Sector (conical) export is constant for fixed solid angle $\Delta\Omega$; envelopes
follow \([A_\Delta(r)]^{-1/2}\) with $A_\Delta(r)=\Delta\Omega\,r^{\,n-1}$.

\subsection{Summary}
\label{subsec:waves-summary}
Waves are a \emph{consequence} of Stokes plus a hyperbolic closure: they carry constant exported power on lossless
segments, have far‑field envelopes $\propto [A_n(r)]^{-1/2}$, obey Fermat/Snell in the eikonal limit, and exhibit
\emph{nonlinear refraction} whenever the index depends on the locally stored energy. The feedback loop
$A_\perp \leftrightarrow u \leftrightarrow n(u)$ provides a clean mechanism for self‑focusing/defocusing entirely
within the informational transport template, independently of any gravitational specialization. Capacity and counting
bounds limit the sharpness and inferability of these effects on finite windows.
