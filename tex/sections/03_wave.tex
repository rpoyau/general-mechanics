% ======================================================================
% 03_wave.tex — Universal wave transport & refraction (+ aperture witness)
% ======================================================================

\section{Universal wave transport \& refraction}
\label{waves}

\subsection{Wave‑supporting closure and exported power}
\label{waves:closure}
Integrating the local balance \eqref{axioms:waves:eq:balance} on a source‑free, lossless
subregion \(W\) and applying Stokes on its spatial boundary \(\Sigma(r)\) yields radius‑independent export
\begin{equation}
P(r):=\int_{\Sigma(r)}\!\langle \mathbf S\rangle\!\cdot \mathbf n\,\mathrm dA \;=\; \text{const in } r,
\label{waves:eq:const-power}
\end{equation}
and therefore the far‑field envelope scales as
\begin{equation}
\big\langle |U(r)| \big\rangle \;\propto\; [A(r)]^{-1/2},
\label{waves:eq:envelope}
\end{equation}
where \(A(r)\) is the area of the boundary \(\Sigma(r)\). With attenuation \(\mu\) along a \worldtube,
\(P(r)\propto \exp\!\big[-\int\mu\,\mathrm ds\big]\) and the envelope acquires
\begin{equation}
\big\langle |U(r)| \big\rangle \;\propto\; [A(r)]^{-1/2}\,
\exp\!\left[-\tfrac12\!\int \mu\,\mathrm ds\right].
\label{waves:eq:envelope-loss}
\end{equation}
For confined propagation, replace \(A(r)\) by the effective cross‑section \(A_\perp\) of the transport tube, giving
\begin{equation}
\big\langle |U| \big\rangle \;\propto\; [A_\perp]^{-1/2}.
\label{waves:eq:confinement}
\end{equation}

\paragraph{Remark (Info–EM example).}
Under \eqref{axioms:waves:eq:chi}–\eqref{axioms:waves:eq:balance}, Info–EM is realized by \(U\!\equiv\!F_I\), \(V\!\equiv\!H_I\),
\(H_I=\chi:F_I\). The informational field amplitude \(E_I\) then follows \eqref{waves:eq:envelope} and \eqref{waves:eq:envelope-loss}.

\subsection{Eikonal limit, Fermat principle, and tube transport}
\label{waves:eikonal}
Assume a high‑frequency limit on a homogeneous or slowly varying window: the amplitude and index vary slowly
relative to the carrier,
\[
\|\nabla A\|/|A| \ll \omega, \qquad \|\nabla n\|/n \ll \omega/c_I .
\]
Use the ansatz
\begin{equation}
U(x,t)\;\sim\; \Re\!\Big\{ A(x)\,e^{i\omega S(x)-i\omega t}\Big\},
\label{waves:eikonal:eq:ansatz}
\end{equation}
with slowly varying amplitude \(A\) and phase \(S\) (eikonal).
The constitutive description \eqref{axioms:waves:eq:chi} defines an \emph{effective index} \(n(x)\) from dispersion and a local speed
\(c_{\mathrm{eff}}:=c_I/n(x)\). In locally isotropic media (or for an effectively isotropic eigenmode), \(n\) is a scalar; otherwise
\(n\) denotes the scalar slowness along the ray. At leading order, \(S\) satisfies the Hamilton–Jacobi equation
\begin{equation}
|\nabla S(x)|^2 \;=\; n^2(x)\,\frac{\omega^2}{c_I^2},
\label{waves:eikonal:eq}
\end{equation}
and rays are the integral curves of \(\mathbf k:=\nabla S\) with unit tangent \(\hat{\mathbf s}:=\mathbf k/|\mathbf k|\); they extremize the optical time
\begin{equation}
\delta\,\int_\gamma \frac{n(x)}{c_I}\,\mathrm ds \;=\; 0,
\label{waves:eikonal:eq:fermat}
\end{equation}
yielding Snell’s law at interfaces. The ray equation is, equivalently,
\begin{equation}
\frac{\mathrm d}{\mathrm ds}\!\big(n\,\hat{\mathbf s}\big)\;=\;\nabla n,
\qquad\Longleftrightarrow\qquad
\frac{\mathrm d\hat{\mathbf s}}{\mathrm ds}\;=\;\nabla_{\!\perp}\big(\ln n\big).
\label{waves:eikonal:eq:ray}
\end{equation}
Tube transport for the slowly varying amplitude gives
\begin{equation}
\frac{\mathrm d}{\mathrm ds}\!\Big(|A|^2 A_\perp\Big) \;=\; -\,|A|^2 A_\perp\,\mu(s),
\qquad\Rightarrow\qquad
|A|^2 A_\perp \propto \exp\!\left[-\!\int\mu\,\mathrm ds\right],
\label{waves:eikonal:eq:tube-transport}
\end{equation}
so \(|A|\propto A_\perp^{-1/2}\) (lossless), consistent with \eqref{waves:eq:envelope}.

\subsection{Nonlinear refraction from constitutive feedback}
\label{waves:nonlinear}
If the constitutive map \(\chi\) depends on a local invariant of the field, e.g.\ the energy density
\(u=\tfrac12\,U{:}V\) or occupancy on the window, then the effective index becomes \emph{amplitude‑dependent}:
\begin{equation}
n\;=\;n_0 \;+\; \alpha\,u \;+\; \mathcal O(u^2) \qquad (\text{weak feedback}).
\label{waves:eq:nonlinear}
\end{equation}
Using \eqref{waves:eikonal:eq:tube-transport}, \(u\propto |A|^2\propto A_\perp^{-1}\) (lossless), so the ray equation acquires a
self‑consistent bending term:
\begin{equation}
\frac{\mathrm d}{\mathrm ds}\!\big(n\,\hat{\mathbf s}\big)
\;=\;
\nabla n\!\big(u(x)\big)
\quad\Longleftrightarrow\quad
\frac{\mathrm d\hat{\mathbf s}}{\mathrm ds} \;=\; \nabla_{\!\perp}\!\big(\ln n(u)\big),
\label{waves:nonlinear:eq:ray-bending}
\end{equation}
closing the loop \(\text{geometry}\to |A|\to u\to n(u)\to \text{geometry}\).
In differential‑geometric language, the expansion \(\theta:=\mathrm d(\ln A_\perp)/\mathrm ds\) along a congruence of rays follows a
Raychaudhuri‑type identity
\begin{equation}
\frac{\mathrm d\theta}{\mathrm ds}
\;=\;
-\,\tfrac12\,\theta^2 \;-\; \varsigma^2 \;-\; \mathcal R_{\rm eff}[n(u)]
\;+\; \text{attenuation terms},
\label{waves:nonlinear:eq:raychaudhuri-analogue}
\end{equation}
where \(\varsigma\) denotes the shear magnitude and \(\mathcal R_{\rm eff}[n(u)]\) collects the transverse curvature induced by
gradients of \(\ln n(u)\) along the congruence (vanishes for uniform \(n\)).
No gravitational postulate is invoked: \eqref{waves:nonlinear:eq:raychaudhuri-analogue} is the refractive focusing identity
induced by the feedback \(n\!\circ\!u\!\circ\!A_\perp\).

\subsection{Causality–Dispersion (KK‑window corollary)}
\label{waves:causality-dispersion}
On a window that is linear, time‑invariant, and causal (retarded, stationary \(\chi(\omega)\) on that window),
the measured dispersion on that window may be compared with the Kramers–Kronig Hilbert–pair relation.

\subsection{\worldtube s, moving boundaries, and boundary sectors}
\label{waves:worldtube}
For moving boundaries, the side integrand is \(\mathbf j\!\cdot\! \mathbf n - i\,v_n\) (\S\ref{axioms:worldtube}).
For fixed boundary sectors, export remains constant; envelopes follow \([A_\perp]^{-1/2}\) with \(A_\perp\) the sector’s cross‑section.
