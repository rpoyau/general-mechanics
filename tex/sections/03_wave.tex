% ======================================================================
% 03_wave.tex — Universal wave transport & refraction (+ aperture witness)
% ======================================================================

\section{Universal wave transport \& refraction}
\label{waves}

\subsection{From Stokes to waves: closure on a window}
\label{waves:closure}
Let fields $(U,V)$ be related on a window by a linear, causal constitutive map
$V=\chi:U$ (possibly dispersive on that window). Define the quadratic stored energy
\begin{equation}
u \;:=\; \tfrac12\,U\!:\!V ,
\label{waves:eq:potential-energy}
\end{equation}
and a spatial flux $\mathbf S$ such that
\begin{equation}
\partial_t u \;+\; \nabla\!\cdot\mathbf S \;=\; -\,\Pi,
\qquad \Pi\ge 0.
\label{waves:eq:local-balance}
\end{equation}
Integrate \eqref{waves:eq:local-balance} on a source‑free, lossless
subregion $W$ and apply Stokes on its spatial boundary $\Sigma(r)$: the exported power
\begin{equation}
P(r):=\int_{\Sigma(r)}\!\langle \mathbf S\rangle\!\cdot \mathbf n\,\mathrm dA \;=\; \text{const in } r,
\label{waves:eq:const-power}
\end{equation}
hence the far‑field envelope scales as
\begin{equation}
\big\langle |U(r)| \big\rangle \;\propto\; [A_n(r)]^{-1/2},
\qquad A_n(r)=S_{n-1}\,r^{\,n-1}.
\label{waves:eq:envelope}
\end{equation}
With attenuation $\mu$ along a \worldtube, $P(r)\propto \exp\!\big[-\int\mu\,\mathrm ds\big]$ and the envelope acquires
\begin{equation}
\big\langle |U(r)| \big\rangle \;\propto\; [A_n(r)]^{-1/2}\,
\exp\!\left[-\tfrac12\!\int \mu\,\mathrm ds\right].
\label{waves:eq:envelope-loss}
\end{equation}
Confinement replaces $A_n(r)$ by an effective cross‑section/circumference:
\begin{equation}
A_\Delta(r)=\Delta\Omega\,r^{\,n-1},
\qquad
\big\langle |U(r)| \big\rangle \;\propto\; [A_\Delta(r)]^{-1/2}.
\label{waves:eq:confinement}
\end{equation}

\paragraph{Remark (generality and Info–EM example).}
Equation \eqref{waves:eq:local-balance} is a window‑closure template. Info–EM is one realization with
$U\!\equiv\!F_I$, $V\!\equiv\!H_I$, $H_I=\chi:F_I$, satisfying the same local identity; then the informational
field amplitude $E_I$ follows \eqref{waves:eq:envelope} and \eqref{waves:eq:envelope-loss}.

\subsection{Eikonal limit, Fermat principle, and tube transport}
\label{waves:eikonal}
Assume a high‑frequency limit on a homogeneous or slowly varying window: the amplitude and index vary slowly
relative to the carrier,
\[
\|\nabla A\|/|A| \ll \omega, \qquad \|\nabla n\|/n \ll \omega/c_I .
\]
Use the ansatz
\begin{equation}
U(x,t)\;\sim\; \Re\!\Big\{ A(x)\,e^{i\omega S(x)-i\omega t}\Big\},
\label{waves:eikonal:eq:ansatz}
\end{equation}
with slowly varying amplitude $A$ and phase $S$ (eikonal).
The constitutive description $V=\chi:U$ defines an \emph{effective index} $n(x)$ from dispersion and a local speed
$c_{\mathrm{eff}}:=c_I/n(x)$. In locally isotropic media (or for an effectively isotropic eigenmode), $n$ is a scalar; otherwise
$n$ denotes the scalar slowness along the ray. At leading order, $S$ satisfies the Hamilton–Jacobi equation
\begin{equation}
|\nabla S(x)|^2 \;=\; n^2(x)\,\frac{\omega^2}{c_I^2},
\label{waves:eikonal:eq}
\end{equation}
and rays are the integral curves of $\mathbf k:=\nabla S$ with unit tangent $\hat{\mathbf s}:=\mathbf k/|\mathbf k|$; they extremize the optical time
\begin{equation}
\delta\,\int_\gamma \frac{n(x)}{c_I}\,\mathrm ds \;=\; 0,
\label{waves:eikonal:eq:fermat}
\end{equation}
yielding Snell’s law at interfaces. The ray equation is, equivalently,
\begin{equation}
\frac{\mathrm d}{\mathrm ds}\!\big(n\,\hat{\mathbf s}\big)\;=\;\nabla n,
\qquad\Longleftrightarrow\qquad
\frac{\mathrm d\hat{\mathbf s}}{\mathrm ds}\;=\;\nabla_{\!\perp}\big(\ln n\big).
\label{waves:eikonal:eq:ray}
\end{equation}
Tube transport for the slowly varying amplitude gives
\begin{equation}
\frac{\mathrm d}{\mathrm ds}\!\Big(|A|^2 A_\perp\Big) \;=\; -\,|A|^2 A_\perp\,\mu(s),
\qquad\Rightarrow\qquad
|A|^2 A_\perp \propto \exp\!\left[-\!\int\mu\,\mathrm ds\right],
\label{waves:eikonal:eq:tube-transport}
\end{equation}
so $|A|\propto A_\perp^{-1/2}$ (lossless), consistent with \eqref{waves:eq:envelope}.

\paragraph{Diffraction (aperture witness).}
On a homogeneous, source‑free window supporting carrier wavenumber $k_0:=\omega n/c_I$, the far‑field pattern from a bounded aperture
$a(\mathbf x_\perp)$ scales as the Fourier transform of the aperture:
\[
U(\mathbf r)\;\propto\;\frac{e^{ik_0 r}}{r}\,\widehat{a}\!\big(k_0\,\hat{\mathbf r}_\perp\big)
\quad\text{(witness; no derivation here).}
\]

\subsection{Nonlinear refraction from constitutive feedback}
\label{waves:nonlinear}
If the constitutive map $\chi$ depends on a local invariant of the field, e.g.\ the energy density
$u=\tfrac12\,U\!:\!V$ or occupancy on the window, then the effective index becomes \emph{amplitude‑dependent}:
\begin{equation}
n\;=\;n_0 \;+\; \alpha\,u \;+\; \mathcal O(u^2) \qquad (\text{weak feedback}).
\label{waves:eq:nonlinear}
\end{equation}
Using \eqref{waves:eikonal:eq:tube-transport}, $u\propto |A|^2\propto A_\perp^{-1}$ (lossless), so the ray equation acquires a
self‑consistent bending term:
\begin{equation}
\frac{\mathrm d}{\mathrm ds}\!\big(n\,\hat{\mathbf s}\big)
\;=\;
\nabla n\!\big(u(x)\big)
\quad\Longleftrightarrow\quad
\frac{\mathrm d\hat{\mathbf s}}{\mathrm ds} \;=\; \nabla_{\!\perp}\!\big(\ln n(u)\big),
\label{waves:nonlinear:eq:ray-bending}
\end{equation}
closing the loop \(\text{geometry}\to |A|\to u\to n(u)\to \text{geometry}\).
In differential‑geometric language, the expansion $\theta:=\mathrm d(\ln A_\perp)/\mathrm ds$ along a congruence of rays follows a
Raychaudhuri‑type identity
\begin{equation}
\frac{\mathrm d\theta}{\mathrm ds}
\;=\;
-\,\tfrac12\,\theta^2 \;-\; \varsigma^2 \;-\; \mathcal R_{\rm eff}[n(u)]
\;+\; \text{attenuation terms},
\label{waves:nonlinear:eq:raychaudhuri-analogue}
\end{equation}
where $\varsigma$ denotes the shear magnitude and $\mathcal R_{\rm eff}[n(u)]$ collects the transverse curvature induced by
gradients of $\ln n(u)$ along the congruence (vanishes for uniform $n$).
No gravitational postulate is invoked: \eqref{waves:nonlinear:eq:raychaudhuri-analogue} is the refractive focusing identity
induced by the feedback \(n\!\circ\!u\!\circ\!A_\perp\).

\subsection{Causality–Dispersion (KK‑window corollary)}
\label{waves:causality-dispersion}
On a window that is linear, time‑invariant, and causal (retarded, stationary $\chi(\omega)$ on that window),
the measured dispersion on that window \emph{may be compared} with the Kramers–Kronig Hilbert–pair relation.
This is used as a \emph{local comparison witness} (not a premise).

\subsection{\worldtube s, moving boundaries, and boundary sectors}
\label{waves:worldtube}
When boundaries move, use the \worldtube\ form with side integrand
$\mathbf j\!\cdot\! \mathbf n - i\,v_n$ (App.~\ref{app:forms-stokes}).
Export is constant for fixed solid‑angle boundary sectors $\Delta\Omega$; envelopes
follow \([A_\Delta(r)]^{-1/2}\) with $A_\Delta(r)=\Delta\Omega\,r^{\,n-1}$.

\subsection{Summary}
\label{waves:summary}
Waves follow from Stokes on a window plus a linear, causal closure: lossless segments carry constant exported power
\eqref{waves:eq:const-power}; far‑field envelopes scale as $[A_n(r)]^{-1/2}$ \eqref{waves:eq:envelope}, with the weak‑loss
factor \eqref{waves:eq:envelope-loss}; confinement follows \eqref{waves:eq:confinement}. Eikonal windows yield
Fermat/Snell \eqref{waves:eikonal:eq}–\eqref{waves:eikonal:eq:fermat} with the ray law \eqref{waves:eikonal:eq:ray};
constitutive feedback produces nonlinear refraction \eqref{waves:eq:nonlinear}–\eqref{waves:nonlinear:eq:raychaudhuri-analogue};
and apertures admit a far‑field \emph{aperture witness} (pattern $\propto$ Fourier transform) while preserving the same $[A_n(r)]^{-1/2}$ envelope.
