% sections/14_tier1_maxwell.tex  —  Tier-1: Maxwell-style field form

\section{Tier-1 Corollary I: Maxwell–Style Form}
\label{sec:tier1_maxwell}

\noindent
Choose the “vacuum” resource list \(R^{(a)}=\{P^{\mu}\}\) (canonical list
2 in Table \ref{tab:canonical}).  The differential REOS
\(d\rho=-\Lambda_{\mu}\,dP^{\mu}\) and the continuity identity combine
to produce a tensor field that obeys Maxwell-type equations.

\paragraph{Field definitions.}
\[
F_{\mu\nu}
   \;:=\;
   \partial_\mu \Lambda_\nu - \partial_\nu \Lambda_\mu,
\qquad
\mathcal J_\nu = \star J^\mu dx_\mu \quad
(\text{$n$-form current}).
\]

\paragraph{“Source” equation (informational analogue of Gauss–Ampère).}
Take the divergence of \(F_{\mu\nu}\):
\[
\partial^\mu F_{\mu\nu}
  \;=\;
  -\,\partial^\mu\partial_\nu\Lambda_\mu
  \;=\;
  \partial_\nu \rho
  \;=\;
  -\,\sigma_\nu,
\qquad
\sigma_\nu := \partial_\nu\rho .
\]
Thus \(\sigma_\nu\) plays the role of a four-current source for the
informational field.

\paragraph{“Bianchi” identity (informational Faraday).}
By construction
\[
\partial_{[\alpha}F_{\mu\nu]} = 0 ,
\]
identical to the electromagnetic Bianchi identity.

\paragraph{Interpretation.}
With vacuum resources \(\{P^{\mu}\}\), the REOS potentials
\(\Lambda_\mu\) become a four-potential; their curl is an
information-flux field; its divergence yields the sink/source
four-current \(\sigma_\nu\).  No additional physical postulates, gauge
choices, or Lorentz structures are introduced—the Maxwell form drops
straight out of the REOS differential and continuity.

\paragraph{Comparison.}
Setting \(\rho\!\to\!\rho_{\text{charge}}\),
\(\Lambda_\mu\!\to\!A_\mu\),
\(F_{\mu\nu}\!\to\!F_{\mu\nu}^{\text{EM}}\) reproduces standard
electrodynamics, but here the “charge” is informational entropy and the
“field” drives gradients of uncertainty rather than electric force.

\paragraph{Next step.}
Section~\ref{sec:tier1_pathint} leverages this field form to construct a
UV-finite path integral over information vector fields.
