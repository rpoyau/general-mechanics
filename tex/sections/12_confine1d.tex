\section{Tier‑1: Confinement in One Spatial Dimension}
\label{sec:confine1d}

\subsection{Setup}
Let $(\tau,x)$ be a local chart on $V=[\tau_1,\tau_2]\times[0,L]\subset\mathcal M$
with volume form $d\tau\wedge dx$. Write
\[
  J \;=\; j^{\tau}\, dx \;-\; j^{x}\, d\tau,
  \qquad
  \Sigma \;=\; \sigma\, d\tau\wedge dx .
\]
Then \eqref{eq:stokes-balance} reads
\begin{equation}
  \partial_{\tau} j^{\tau} + \partial_{x} j^{x} \;=\; \sigma .
  \label{eq:conf1d-balance}
\end{equation}

\subsection{Impenetrable walls (no‑flux)}
Boundary conditions at $x=0$ and $x=L$:
\begin{equation}
  j^{x}(\tau,0)=0, \qquad j^{x}(\tau,L)=0 .
  \label{eq:conf1d-noflux}
\end{equation}
Integrating \eqref{eq:conf1d-balance} over $x\in[0,L]$ and using \eqref{eq:conf1d-noflux} gives
\begin{equation}
  \frac{d}{d\tau}\,\int_{0}^{L} j^{\tau}(\tau,x)\,dx
  \;=\;
  \int_{0}^{L} \sigma(\tau,x)\,dx .
  \label{eq:conf1d-integrated}
\end{equation}

\subsection{Partially permeable wall}
If $j^{x}(\tau,L)=\phi(\tau)$ while $j^{x}(\tau,0)=0$, then
\begin{equation}
  \frac{d}{d\tau}\,\int_{0}^{L} j^{\tau}\,dx
  \;=\;
  \int_{0}^{L} \sigma\,dx \;-\; \phi(\tau) .
  \label{eq:conf1d-leaky}
\end{equation}

\subsection{Discrete 1D chain}
Let cells be intervals $c_i=[x_i,x_{i+1}]$ with oriented edges $e_i$ at $x_i$.
Assign $J_{e_i}$ to edges and $\sigma_{c_i}$ to cells. With incidence $B$,
\begin{equation}
  (B^{\mathsf T}J)_{c_i} \;=\; \sigma_{c_i}
  \quad\Longleftrightarrow\quad
  J_{e_{i+1}} - J_{e_i} \;=\; \sigma_{c_i}.
  \label{eq:conf1d-discrete}
\end{equation}
No‑flux at the ends means $J_{e_0}=J_{e_N}=0$, yielding
$\sum_{i=0}^{N-1} \sigma_{c_i} = 0$ for a steady state.
For a leaky right end with $J_{e_N}=\phi(\tau)$ one recovers the discrete form of
\eqref{eq:conf1d-leaky}.
