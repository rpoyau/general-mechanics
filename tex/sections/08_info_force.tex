% sections/08_info_force.tex  —  Anchor Corollary III: Informational Force

\section{Anchor Corollary III: Informational Force}
\label{sec:info_force}

\paragraph{Force density.}
From the differential REOS equation
$d\rho=-\Lambda_a\,dR^{(a)}$
and the spatial gradient
$\nabla\rho = -\Lambda_a\,\nabla R^{(a)}-R^{(a)}\nabla\Lambda_a$,
we isolate the \emph{force density}
\begin{boxedeq}[eq:force]
\mathbf f
  \;:=\;
  -\,R^{(a)}\,\nabla\Lambda_a .
\end{boxedeq}
It acts on the informational “fluid” exactly as a pressure or electric
field acts on matter, but the driving potential is
$\Lambda_a$—the local \emph{price per bit} of the $a$-th resource.

\paragraph{Work–flux relation.}
Contracting $\mathbf f$ with the transport velocity
$\mathbf v := \mathbf J/\rho$ gives
\[
\mathbf f\!\cdot\!\mathbf v
  \;=\;
  -\,\frac{R^{(a)}}{\rho}\,
    \mathbf J\!\cdot\!\nabla\Lambda_a
  \;=\;
  -\,\frac{dR^{(a)}}{dt},
\]
so the work done by $\mathbf f$ converts resource \(R^{(a)}\) into or
out of Shannon entropy at a rate fixed by the continuity identity.

\subsection{Example 1: Inhomogeneous Diffusive Bus}

\begin{itemize}
\item \textbf{Data bus.}\; 1-D line, bandwidth profile
      $B(x)$ [Kick s\(^{-1}\) m\(^{-1}\)].
\item \textbf{Mobility.}\; Choose
      $A^{xx}=D(x)=B^{-1}(x)$ (bits travel slower where bandwidth is low).
\item \textbf{Resource.}\; $\rho$ itself ($R^{(1)}=\rho$) so
      $\Lambda_\rho=1$ and $\mathbf f=-\rho\,\nabla(1/D)$.
\end{itemize}
Hence bits drift from congested ($B\!\downarrow$) to roomy
($B\!\uparrow$) regions exactly as charge carriers drift down a
resistivity gradient.

%% \subsection{Example 2: Electrostatic Self-Energy Pull}

%% \begin{itemize}
%% \item \textbf{Resource list.}\; $R^{(1)}=E$ (rest+kinetic),
%%       $R^{(2)}=Q^{2}$ (charge squared).
%% \item \textbf{Potentials.}\; With
%%       $U=\tfrac{\alpha}{2r}$ and Kick units,
%%       $\Lambda_{Q^{2}} = 1/(2r)$.
%% \item \textbf{Force.}\; Keeping $E$ uniform but letting
%%       $r(x)$ vary (dielectric spacer),
%%       $\mathbf f = -Q^{2}\nabla(1/2r) =
%%       -\tfrac12 Q^{2}\,\nabla r^{-1}$,
%%       pulling charge toward narrower regions—micro-analog of the
%%       capacitor edge effect.
%% \end{itemize}

\paragraph{Relation to Tier-1 capacities.}
In Secs.~\ref{sec:tier1_maxwell}–\ref{sec:tier1_jacobson} the channel capacity ceiling
$J^{\mu}n_{\mu}\le C(R^{(a)})$ is enforced by setting
$\nabla\Lambda_{a}$ so steep that $\mathbf f$ opposes any attempt to
exceed $C$; the force formalism therefore re-expresses capacity limits
as \emph{hard potential walls} in the information landscape.

\paragraph{Take-away.}
Equation~\eqref{eq:force} translates every resource gradient into a real
spatial push or pull on uncertainty density, making the REOS axiom not
just a bookkeeping device but a driver of dynamics.

