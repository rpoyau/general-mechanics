% sections/12_multi_clock.tex  —  Multi-clock generalisation
% Plan ref: §12  • Tbl 12-1 resource-clock pairs

\section{Multi-Clock Generalization}
\label{sec:multi_clock}

\noindent
The baseline single-clock form keeps \emph{one} physical time coordinate~$t$.
More generally, every conserved resource $R^{(a)}$ can be paired with a
\emph{conjugate clock} $\Theta_{(a)}$ such that
$\Delta R^{(a)}\,\Delta\Theta_{(a)}\gtrsim 1/(2\ln 2)$
(Sec.~\ref{app:covariant}).  
Table~\ref{tab:clocks} lists representative pairs.

\begin{table}[ht]
\centering
\caption{Canonical resource–clock pairs (Kick gauge).}
\label{tab:clocks}
\begin{tabular}{lll}
\toprule
Resource $R^{(a)}$ & Clock $\Theta_{(a)}$ & Typical experiment \\ 
\midrule
Energy \(E\)         & Time \(t\)               & Ramsey fringes \\[2pt]
Momentum \(p_i\)      & Position \(x_i\)         & Diffraction \\[2pt]
%Charge-squared \(Q^{2}\) & Exposure radius \(r\) & Capacitor edge \\[2pt]
Entanglement pairs \(E_{\mathrm{PR}}\) & Gate cycles \(\Theta_{\mathrm{PR}}\) & Repeater chain \\[2pt]
Heat entropy \(S_{\text{phys}}\) & Log-volume \(\ln V\) & Adiabatic compression \\[2pt]
Photon number \(N_\gamma\) & Optical path length \(L/c\) & Cavity QED \\[2pt]
ATP molecules \(N_{\mathrm{ATP}}\) & Reaction time \(t_{\mathrm{rxn}}\) & Molecular motor \\[2pt]
\bottomrule
\end{tabular}
\end{table}

\paragraph{Extended manifold.}
Including all $\Theta_{(a)}$ enlarges the space–
time manifold to
\(\widehat{M}=M^{n+1}\!\times\!\prod_{a}S^{1}_{\Theta_{(a)}}\).
The informational current acquires extra components
\(J^{\Theta_{(a)}}=-\Lambda_{a}\), and the generalised Stokes balance
still holds:
\[
\int_{\widehat\Sigma}\!\widehat{\mathcal J}
  \;=\;
  \int_{\widehat V}\!\widehat\sigma.
\]

\subsection{Vacuum-Limited Channel}\label{subsec:vacuum_channel}\label{sec:vacuum_channel}
Choosing the vacuum world-tube $\Sigma$ as the unique physical channel (cf.\ Sec.~\ref{sec:info_currents})
lets the generalised Stokes balance act as a capacity ledger. Logical sub-channels share the
same resource vector $R^{(a)}$ and therefore compete under the REOS constraint.

\paragraph{Why useful?}
Multi-clock form cleanly separates trade-offs:
squeezing energy bandwidth (short \(\Delta t\))
has no algebraic coupling to, say, entanglement bandwidth
(\(\Delta\Theta_{\mathrm{PR}}\)) unless the REOS itself links \(E\)
and \(E_{\mathrm{PR}}\).  Sections~\ref{sec:tier1_maxwell}–\ref{sec:tier1_jacobson}
exploit this independence to derive capacity bounds under mixed
resource budgets.

