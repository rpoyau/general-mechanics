% sections/04_info_currents.tex  —  Informational current & world-tube sketch

\section{Informational Currents and the World-Tube}
\label{sec:info_currents}

\paragraph{Informational $n$-form.}
Given the Shannon–entropy density $\rho(\mathbf x,t)$ and spatial flux
$\mathbf J(\mathbf x,t)$ we package the $4$-current
$J^{\mu}=(\rho,\mathbf J)$ into the $n$-form
$\displaystyle \mathcal J := \star J^{\mu}\,dx_{\mu}$ introduced in
Sec.~\ref{sec:axiom}.  The continuity identity
\[
d\mathcal J = \sigma\,d^{\,n+1}x
\]
(see Eq.\,(C\,0)) holds everywhere, with
$\sigma$ the local source/sink term.

\paragraph{World-tube as universal channel.}
Choose any time-like hypersurface~$\Sigma$ that encloses the domain of
interest (e.g.\ the vacuum boundary used in
Sec.~\ref{sec:vacuum_channel}).  The generalised Stokes balance
\[
\int_{\Sigma}\!\mathcal J \;=\; \int_{V}\!\sigma
\]
equates the net information flux through~$\Sigma$ with the integrated
production inside~$V$.  If $\sigma=0$ (no measurement or erasure) the
world-tube acts as a \emph{loss-less channel}: every bit that enters its
mouth exits its tail.

\begin{figure}[H]
  \centering
  % figs/world_tube.tikz  – compiles independently
\documentclass[tikz,border=0pt]{standalone}
\usetikzlibrary{decorations.pathmorphing,patterns}

\begin{document}
\begin{tikzpicture}[>=stealth,scale=0.9]

  % control points for the reference world-line
  \coordinate (P0) at ( 0.00,0);
  \coordinate (P1) at ( 0.25,1);
  \coordinate (P2) at (-0.20,2);
  \coordinate (P3) at ( 0.15,3);
  \coordinate (P4) at (-0.10,4);
  \def\eps{0.25}

  % shaded tube -------------------------------------------------
  \fill[blue!12,opacity=.6]
        plot[smooth] coordinates
          {($(P0)+(\eps,0)$) ($(P1)+(\eps,0)$) ($(P2)+(\eps,0)$)
           ($(P3)+(\eps,0)$) ($(P4)+(\eps,0)$)}
      -- plot[smooth] coordinates
          {($(P4)-(\eps,0)$) ($(P3)-(\eps,0)$) ($(P2)-(\eps,0)$)
           ($(P1)-(\eps,0)$) ($(P0)-(\eps,0)$)}
      -- cycle;

  % tube boundaries
  \foreach \s in {\eps,-\eps}{
    \draw[blue!60,thick,dashed]
      plot[smooth] coordinates
        {($(P0)+(\s,0)$) ($(P1)+(\s,0)$) ($(P2)+(\s,0)$)
         ($(P3)+(\s,0)$) ($(P4)+(\s,0)$)};
  }

  % informational current 𝒥
  \draw[thick,->] plot[smooth] coordinates
        {(P0)(P1)(P2)(P3)(P4)}
        node[pos=.78,right] {$\mathcal J$};

  % stippled sink region (σ<0)
  \fill[pattern=north west lines,pattern color=gray!60,opacity=.6]
        ($(P1)!0.45!(P2)+(-\eps,0)$) rectangle
        ($(P1)!0.45!(P2)+(\eps,0.9)$);

  % light-cone hints
  \draw[densely dotted,gray] (0,0) -- +( 1,1) -- +( 1,1.3);
  \draw[densely dotted,gray] (0,0) -- +(-1,1) -- +(-1,1.3);

  % axes
  \draw[->] (-2,0) -- ( 2,0) node[right] {$x$};
  \draw[->] ( 0,-.25) -- (0,5.2) node[above] {$t$};

  % ε label
  \draw[<->,blue!60] (-\eps,2.45) -- (\eps,2.45)
        node[midway,above,blue!60] {$\varepsilon$};

\end{tikzpicture}
\end{document}
%  % adjust path if needed; no .tex extension
  \caption{Illustrative world-tube~$\Sigma$ (shaded). Arrows denote the
           informational current~$\mathcal J$; the stippled region marks a
           possible measurement sink ($\sigma<0$).}
  \label{fig:world_tube}
\end{figure}



\paragraph{Local vs.\ logical channels.}
Partitioning $\Sigma$ into disjoint patches
$\{\Sigma_k\}_{k=1}^N$ yields $N$ \emph{logical channels}.  Their fluxes
sum to the total, but each competes for the same resource vector
$R^{(a)}$ in the REOS:
\[
\sum_{k}\!\int_{\Sigma_k}\!\mathcal J
  \;=\;\int_{\Sigma}\!\mathcal J,
\qquad
R^{(a)}_{\text{total}}=\sum_{k}R^{(a)}_k.
\]
This partition-independence underpins the capacity trade-offs developed
in Secs.~\ref{sec:tier1_maxwell}–\ref{sec:tier1_jacobson}
