% sections/10_synthesis.tex  —  Pillars mosaic (summary bridge)
% Plan ref: §10  ▲ Fig 10-1 pillars_mosaic.pdf

\section{Synthesis: Four Pillars at a Glance}\label{sec:synthesis}

\begin{figure}[H]
  \centering
  % figs/pillars_mosaic.tex  – perfectly centred version
\begin{tikzpicture}[
      font=\small,
      pillar/.style={
        minimum width=2.4cm, minimum height=.9cm,
        draw, thick, align=center}]
  % Put the Bridge at (0,0) and build around it
  \node[pillar,fill=gray!25]            (bridge) {Bridge};

  % Canonical pillar (to the *left* of bridge)
  \node[pillar,fill=cyan!12,left=1.3cm of bridge] (c0) {Canonical\\Tier 0};
  \node[pillar,fill=cyan!20,above=1pt  of c0]     (c1) {Tier 1};
  \node[pillar,fill=cyan!30,above=1pt  of c1]     (c2) {Tier 2};

  % Synthesis pillar (to the *right* of bridge)
  \node[pillar,fill=orange!12,right=1.3cm of bridge] (s0) {Synthesis\\Tier 0};
  \node[pillar,fill=orange!20,above=1pt of s0]       (s1) {Tier 1};
  \node[pillar,fill=orange!30,above=1pt of s1]       (s2) {Tier 2};
\end{tikzpicture}
 % no .tex extension
  \caption{The four boxed identities that structure the note.
           Left–to–right: (I) continuity, (II) differential REOS,
           (III) informational force, (IV) least-action / Hamiltonian.
           Arrows indicate logical dependence; shaded tiles mark the
           anchor corollaries derived so far.}
  \label{fig:pillars}
\end{figure}


Figure~\ref{fig:pillars} recaps the logical flow in one page.
\begin{enumerate}[label=\Roman*.]
\item \textbf{Continuity (\S\ref{subsec:axiom-continuity})} encapsulates
      bookkeeping for Shannon entropy under sources $\sigma$.
\item \textbf{REOS differential (\S\ref{sec:reos_equation})}
      links $\rho$ to any resource list $R^{(a)}$.
\item \textbf{Informational force (\S\ref{sec:info_force})}
      converts potential gradients into real spatial push–pull.
\item \textbf{Least-action (\S\ref{sec:lagrangian})}
      provides dynamics; choosing $A^{ij}$ picks diffusion,
      wave-like propagation, or heterogeneous-mobility transport.
\end{enumerate}


Every Tier-1 corollary that follows—Landauer cost, energy–time capacity,
Maxwell form, UV-finite path integral, null-tube $\leftrightarrow$ Einstein equation—
+is an immediate algebraic offspring of this four-pillar scaffold.

\vspace{2ex}
\noindent\textbf{Reading guide.}
\begin{itemize}
\item Newcomers can stop here; the remainder of the note shows worked examples.
\item Readers interested in measurement theory may skip to
      Sec.~\ref{sec:landauer}; gravitational readers to
      Sec.~\ref{sec:tier1_jacobson}.
\item The falsification ladder (\S\ref{sec:falsify}) lists empirical
      challenges in order of experimental difficulty.
\end{itemize}
