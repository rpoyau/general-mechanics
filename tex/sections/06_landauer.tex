% sections/06_landauer.tex  —  Anchor Corollary: Landauer Bound

\section{Anchor Corollary I: Landauer Cost}
\label{sec:landauer}

\paragraph{Statement.}
Erasing $\Delta H$ bits of Shannon uncertainty in a reservoir at
temperature $T$ requires a minimum work
\begin{equation}
\boxed{\displaystyle
W_{\min}
   \;=\;
   T\,\Delta H\,\ln 2
   \quad
   (\text{joule}).
}
\end{equation}
In Kick units ($k_{B}=\hbar=c=1$) this is simply
$W_{\min} = T\,\Delta H$.

\paragraph{Derivation from the REOS axiom.}

\begin{enumerate}
\item \textbf{Choose resources.}  
      Take the resource vector to be
      $R^{(1)}=E$ (internal energy) and
      $R^{(2)}=S_{\text{therm}}$ (physical entropy).
\item \textbf{Invoke Eq.\,(\ref{eq:reos_diff}).}  
      With $C=\tfrac1{\ln 2}$ we have
      $\rho = S_{\text{therm}}/k_{B}\ln 2$ so
      $d\rho = dS_{\text{therm}}/(k_{B}\ln 2)$.
\item \textbf{Isothermal process.}  
      For a reversible isothermal path at temperature $T$
      the first law gives $dE = T\,dS_{\text{therm}}$.
      Insert both differentials into
      $d\rho=-\Lambda_E\,dE-\Lambda_S\,dS_{\text{therm}}$
      and note $\Lambda_S=1$ by construction; solve for $\Lambda_E$:
      $\Lambda_E = 1/(T\ln 2)$.
\item \textbf{Integrate.}  
      Remove $\Delta H$ bits ($\Delta\rho=-\Delta H$) while the bath
      absorbs energy $\Delta E=W_{\min}$.  Using
      $\Delta E = -\Delta H / \Lambda_E$ yields
      $\Delta E = T\,\Delta H\,\ln 2$, completing the proof.
\end{enumerate}

\paragraph{Interpretation.}
No additional postulates (irreversibility, thermodynamic limit, etc.)
enter—the bound is a direct algebraic consequence of the REOS
differential and the isothermal leg of the first law.  In later
sections we treat erasure as a point sink
$\sigma=-\Delta H\,\delta(x-x_{m})$; the same algebra recovers the cost
locally.

\paragraph{References.}
Landauer’s original argument appears in \cite{Landauer1961}.  Our
derivation matches recent algebraic proofs
(e.g.\ \cite{ReebWolf2014}) but removes the need for explicit bath
Hamiltonians or coarse-graining assumptions.

