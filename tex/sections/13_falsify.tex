\section{Falsification Ladders}
\label{sec:falsify}

\subsection{Tier‑0: Balance on \texorpdfstring{$\mathcal M$}{M}}
\begin{enumerate}\itemsep0pt
  \item \textbf{Boundary–interior equality.}
        For any admissible $V\subset \mathcal M$, test
        \[
          \int_{\partial V} J \stackrel{?}{=} \int_V \Sigma
          \quad\text{(Eq.~\eqref{eq:stokes-balance})}.
        \]
  \item \textbf{Partition invariance.}
        For $V=\bigsqcup_k V_k$ (disjoint interiors), verify
        $\int_{\partial V}J=\sum_k\int_{\partial V_k}J$ and
        $\int_V\Sigma=\sum_k\int_{V_k}\Sigma$.
  \item \textbf{Discrete chain test.}
        On a finite $n$‑complex with incidence $B$, record edge currents $J$ and cell sources $\sigma$;
        test $B^{\mathsf T}J\stackrel{?}{=}\sigma$ on all $n$‑cells.
\end{enumerate}

\subsection{Resource–space integrability}
Let $\alpha(R)=\Lambda_a(R)\,dR^{(a)}$.
\begin{enumerate}\itemsep0pt
  \item \textbf{Loop test.} On small rectangles $\Pi$ in a resource patch,
        $\oint_{\partial\Pi}\alpha \stackrel{?}{=} 0$.
  \item \textbf{Maxwell symmetry.} Cross‑partials:
        $\partial_{R^{(b)}}\Lambda_a \stackrel{?}{=} \partial_{R^{(a)}}\Lambda_b$.
  \item \textbf{Path independence.} For two paths $\Gamma_1,\Gamma_2$ with same endpoints,
        \[
          \int_{\Gamma_1}\Lambda_a\,dR^{(a)}
          \stackrel{?}{=}
          \int_{\Gamma_2}\Lambda_a\,dR^{(a)} .
        \]
\end{enumerate}

\subsection{Tier‑0 consequences}
\begin{enumerate}\itemsep0pt
  \item \textbf{Reversible path–cost.}
        For a protocol $\Gamma$,
        \[
          \int_{\Gamma}\Lambda_a\,dR^{(a)} \stackrel{?}{=} -\,\Delta\rho
          \quad\text{(Eq.~\eqref{eq:rev-cost})}.
        \]
  \item \textbf{Force covector.}
        For a field $R^{(a)}(x)$, compute $\mathcal F:=-\,R^{(a)}\,d\Lambda_a$
        and compare with measured tendencies under small controlled spatial variations.
  \item \textbf{Variational scaffold.}
        Stationarity of $\mathcal S$ in Eq.~\eqref{eq:tier0-action}
        reproduces \eqref{eq:stokes-balance} and \eqref{eq:reos}.
\end{enumerate}

\subsection{Thermodynamic representation}
With $R=(E,V,N_i,\ldots)$ and $\rho=S$:
\begin{enumerate}\itemsep0pt
  \item \textbf{Gibbs identity.}
        \(
          dS \stackrel{?}{=} \tfrac{1}{T}dE + \tfrac{p}{T}dV - \sum_i \tfrac{\mu_i}{T} dN_i
          \quad\text{(Eq.~\eqref{eq:gibbs-entropy-form})}.
        \)
  \item \textbf{Landauer equality (reversible).}
        At fixed $(V,N_i)$,
        \(
          \Delta E_{\min} \stackrel{?}{=} T\,\Delta S
          \quad\text{(Eq.~\eqref{eq:landauer-equality})}.
        \)
\end{enumerate}

\subsection{Tier‑1 (only when the extra structure is chosen)}
\begin{enumerate}\itemsep0pt
  \item \textbf{Channel throughput/capacity.}
        For each channel $k$, test $\mathcal M_k(\theta_k)=\frac{d}{d\theta_k}\int_{\Sigma_k(\theta_k)}J$
        and the bound $\lvert \mathcal M_k\rvert \stackrel{?}{\le} C_k$.
  \item \textbf{Maxwell‑style analogue.}
        If $(\mathcal A,F)$ and a metric are chosen, test $dF=0$ and
        $d\,\star F=J-\Pi$ with $d\Pi=\Sigma$.
  \item \textbf{Kernel stationarity.}
        With the quadratic action \eqref{eq:t1-action}, check that variations reproduce the imposed constraints and adjoint equations.
\end{enumerate}
