% sections/09_confine1d.tex  —  1 + 1 confinement demonstration

\section{Demo: 1\,+\,1 Confinement}
\label{sec:confine1d}

\paragraph{Set-up.}
We simulate a one-dimensional channel of length $L=1$ Kick-metre
discretised into $N=256$ cells.  
The mobility tensor is piece-wise:

\[
A^{xx}(x)=
\begin{cases}
D_{\text{in}}        & x\in[0,\tfrac14L]\cup[\tfrac34L,L],\\[4pt]
D_{\text{barrier}}   & x\in(\tfrac14L,\tfrac34L),
\end{cases}
\qquad
D_{\text{barrier}}\;\ll\;D_{\text{in}} .
\]

Hence $\mathbf J=-A^{xx}\partial_x\rho$ produces a kinetic “barrier”
in the middle half of the line.

\paragraph{Numerics.}
Script \texttt{src/lattice\_M1.py} solves
\[
\partial_t\rho = \partial_x
 \!\bigl( A^{xx}\partial_x\rho \bigr)
\qquad (\sigma=0)
\]
with Crank–Nicolson time-stepping, initial condition
$\rho(x,0)=\delta_{x,\,N/8}$, and reflective boundaries.
The output CSV is cached under \texttt{data/} and the
steady‐state profile is plotted in
Fig.\,\ref{fig:lattice_potential}.

\paragraph{Interpretation.}
Because $A^{xx}$ is small in the central region, the informational force
$\mathbf f=-\rho\nabla(1/A^{xx})$ pushes uncertainty outwards, confining
the density to the two end-pockets and reproducing the textbook “double-well”
picture without invoking potential energy—only the mobility tensor.
All mass, charge, or bandwidth resources remain free parameters; the
entire effect follows from the REOS differential plus the chosen $A^{xx}$.

\begin{figure}[ht]
  \centering
  \includegraphics[width=.72\linewidth]{lattice_potential.pdf}
  \caption{Steady‐state Shannon entropy density
           $\rho(x)$ for $D_{\text{barrier}}/D_{\text{in}}=10^{-2}$.
           Uncertainty accumulates in the low-resistance pockets at
           $x
