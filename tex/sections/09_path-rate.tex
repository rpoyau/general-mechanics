% ======================================================================
% sections/09_path-rate.tex — Path-rate mechanics (v1.0.2)
% ======================================================================

\section{Path-rate mechanics}
\label{sec:path-rate}

All energy-like quantities are in Kick (Kz), cf.\ \S\ref{sec:conventions}. We take the \emph{rate} (per unit time) from counting as primitive.

\paragraph{Rate, polymomentum, dual.}
Given a configuration \(q\in Q\) with velocity \(\dot q\), define the \emph{information rate}
\begin{equation}
\mathcal R=\mathcal R(q,\dot q,t)\quad[\text{Kz}],
\qquad
P:=\partial_{\dot q}\mathcal R,\qquad
\Phi(q,P,t):=P\!\cdot\!\dot q-\mathcal R
\label{eq:rate-defs}
\end{equation}
(the Legendre–Fenchel dual). Assume \(\mathcal R\) is convex in \(\dot q\) so that the map \(\dot q\mapsto P\) is invertible on the domain of interest.

\paragraph{Information Poincar\'e--Cartan 1-form and strip–Stokes.}
Define
\begin{equation}
\Theta_{\mathcal R}=P\!\cdot\mathrm dq-\Phi\,\mathrm dt,
\qquad
\omega:=\mathrm d\Theta_{\mathcal R}=\mathrm dP\wedge \mathrm dq - \mathrm d\Phi\wedge \mathrm dt.
\label{eq:theta-omega}
\end{equation}
Let \(\gamma_\epsilon:[t_0,t_1]\to T^*Q\) be a variation strip with tangent \(X=\partial_t+\dot q^i\partial_{q^i}+\dot P_i\partial_{P_i}\).
Stokes on the strip yields \(\int_{\partial \Sigma}\Theta_{\mathcal R}=\int_{\Sigma}\iota_X\omega\), and stationarity for fixed endpoints implies
\begin{equation}
\dot q=\partial_P\Phi,\qquad \dot P=-\partial_q\Phi.
\label{eq:canonical}
\end{equation}
Thus the \emph{canonical equations} arise from boundary calculus, not from an assumed action.

\paragraph{Hamilton--Jacobi (information eikonal).}
If a scalar \(S(q,t)\) exists with \(\mathrm dS=\Theta_{\mathcal R}\) on solutions, then
\begin{equation}
P=\partial_q S,\qquad \partial_t S + \Phi\!\big(q,\partial_q S,t\big)=0.
\label{eq:hj}
\end{equation}
The quantity \(S\) is an \emph{information eikonal}; along a trajectory, \(\Delta S=\int (P\!\cdot\dot q-\Phi)\,dt=\int\Theta_{\mathcal R}\).

\paragraph{Symplectic structure and Noether statements.}
The 2-form \(\omega=\mathrm dP\wedge\mathrm dq\) on each time slice is preserved by \eqref{eq:canonical} (Liouville theorem).
If \(\Phi\) is invariant under a continuous group, the corresponding moment map is conserved (time translations \(\Rightarrow \Phi=\) const., spatial translations \(\Rightarrow P=\) const., etc.).

\paragraph{Examples (closures from counting).}
\begin{enumerate}[itemsep=0.3em,leftmargin=1.3em]
\item \textbf{Free particle closure.}
Take the quadratic rate
\begin{equation}
\mathcal R_{\rm free}(q,\dot q)=\frac{M}{2}\,\frac{\dot q^2}{c^2},
\qquad M:=\frac{mc^2}{h}\ \text{(rest–Kick)}.
\label{eq:Rfree}
\end{equation}
Then \(P=\partial_{\dot q}\mathcal R=M\,\dot q/c^2\) and the dual is
\begin{equation}
\Phi_{\rm free}(q,P)=\frac{c^2}{2M}\,P^2.
\label{eq:Phifree}
\end{equation}
The canonical equations recover uniform motion. In ordinary units this is the usual kinetic energy, divided by \(h\).

\item \textbf{Adding a relation price (REOS coupling).}
If changing a relation \(R^a(q,t)\) carries a \emph{price} \(\Lambda_a(q,t)\) (from \(\mathrm d\rho=-\Lambda_a\,\mathrm dR^a\)), incorporate it as an additive term in the dual:
\begin{equation}
\Phi(q,P,t)=\Phi_{\rm free}(q,P) + \Psi(q,t),
\qquad
\Psi(q,t):=\Lambda_a(q,t)\,R^a(q,t).
\label{eq:Phi-plus-psi}
\end{equation}
Then \(\dot P=-\partial_q\Psi\) acts as an informational force derived from the REOS.

\item \textbf{Minimal coupling to an informational 1-form.}
If a background 1-form \(\mathcal A_I=-\phi_I\,\mathrm dt + \mathbf A_I\cdot \mathrm d\mathbf x\) (Sec.~\ref{sec:worldtube-IED}) mediates interactions, use
\begin{equation}
\Theta_{\mathcal R}^{(\mathcal A)}=\Theta_{\mathcal R} + \alpha\,\mathcal A_I,
\label{eq:theta-plus-A}
\end{equation}
with coupling \(\alpha\) (a calibration constant). This reproduces standard minimal coupling when \(\mathcal A_I\) is identified with an electromagnetic‑like potential in Kz; the field part evolves by IED.
\end{enumerate}

\paragraph{Dissipation / production.}
If information is produced/erased along the path at rate \(\sigma(q,\dot q,t)\ge 0\), one may use a contact extension
\[
\Theta_c = \Theta_{\mathcal R} + \mathrm dS,\qquad \dot S = \mathcal R - P\!\cdot\!\dot q - \Upsilon,
\]
or equivalently augment \eqref{eq:canonical} by a Rayleigh‑type term \(D(q,\dot q,t)\) in Kz:
\begin{equation}
\dot q=\partial_P\Phi,\qquad \dot P=-\partial_q\Phi - \partial_{\dot q}D.
\label{eq:dissipative}
\end{equation}
At steady state, \(\Upsilon\) equals the average production rate \(\sigma\).

\paragraph{Field (De Donder--Weyl) variant.}
For fields \(\phi^a(x)\) on spacetime, let the rate density (Lagrangian density in Kz) be \(\mathcal R(\phi,\partial_\mu\phi,x)\).
Define polymomenta \(P_a^{\ \mu}:=\partial\mathcal R/\partial(\partial_\mu\phi^a)\) and the dual density
\begin{equation}
\Phi(\phi,P,x)=P_a^{\ \mu}\,\partial_\mu\phi^a - \mathcal R.
\label{eq:ddw-dual}
\end{equation}
The De Donder–Weyl equations are
\begin{equation}
\partial_\mu \phi^a = \frac{\partial \Phi}{\partial P_a^{\ \mu}},
\qquad
\partial_\mu P_a^{\ \mu} = -\,\frac{\partial \Phi}{\partial \phi^a},
\label{eq:ddw-eq}
\end{equation}
and the multisymplectic Cartan form generalizes \eqref{eq:theta-omega}.
Couplings to the informational gauge sector proceed via \(\mathcal R(\phi,\partial\phi-\alpha\,\mathcal A_I, x)\) or, equivalently, \(\Theta\mapsto\Theta+\alpha\,\mathcal A_I\).

\paragraph{Remarks.}
(1) \(\mathcal R\) comes from counting micro‑moves; \(\Phi\) is its convex dual—\emph{these are the only primitives}.  
(2) The traditional Lagrangian/Hamiltonian pair are merely names for \((\mathcal R,\Phi)\); no separate postulates are needed.  
(3) Kram anchors and the Kz convention keep all expressions dimensionally transparent; calibrations enter only when mapping \(\mathbf F_I\) or sources to measured sectors.
