% ======================================================================
% 01_introduction.tex — Introduction
% ======================================================================

\section{Introduction}
\label{introduction}

\subsection{Aim and stance.}
We take \emph{General Mechanics} to mean: dynamics derived from a single \emph{geometric identity}
(Stokes' theorem) applied to an \emph{informational current} on a chosen region of a manifold.
Relations are primary; observables are relational. An observer specifies a region $U\subset M$ with
boundary $\partial U$; all statements below are read from the boundary, not added as extra postulates.

\subsection{One geometric identity, two uses.}
The generalized Stokes identity relates the exterior derivative of a current to its flux through the
boundary. With $j:=J^\mu dx_\mu$ the informational/entropic current $1$-form and $*j$ its Hodge dual,
the identity is $d(*j)=\sigma\,\mathrm{vol}$ (continuity).\footnote{Local form and the \emph{cut (cap–side)
decomposition} are stated in \eqref{eq:stokes-identity}–\eqref{eq:cut-balance} once the axioms are introduced.}
We \emph{apply} this identity in two standard ways:
(i) on a \emph{time‑frozen} spatial region (static screens/sectors; see C0–C1:
\S\ref{corollary:info-gas}, \S\ref{corollary:static-screen}); and
(ii) on a \emph{product region} $U\times[t_0,t_1]$ to discuss change over a time window (C3:
\S\ref{corollary:external-drives}).

\bigskip
\noindent\fbox{\begin{minipage}{0.97\linewidth}
\textbf{Cone/Tube principle (time frozen).}
Let $\sigma=0$ in a neighborhood, and let the side of a narrow tube be everywhere tangent to $\mathbf j$.
Then the directed tube flux
$\displaystyle \Phi=\int_{\Sigma} \mathbf j\!\cdot n\,dA$ is independent of the cross‑section $\Sigma$.
For any screen or conical sector of area $A(r)$ at radius $r$,
\[
\big\langle j_n\big\rangle(r)=\frac{\Phi}{A(r)}.
\]
With the snapshot closure $\mathbf E_I:=-\nabla\Phi_I$ and $\mathbf j=-\kappa\,\mathbf E_I$
(see \eqref{eq:static-closure}), the informational field amplitude obeys
\[
\big|\mathbf E_I(r)\big|=\frac{1}{\kappa}\,\frac{\Phi}{A(r)}.
\]
\emph{Conservation is a surface statement:} the flux $\Phi$ is constant; amplitudes scale as $1/A$.
\end{minipage}}

\subsection{Closures (use and scope).}
We employ two minimalist ingredients when appropriate:
\emph{(a) Static linear closure} for near‑equilibrium, time‑frozen statics,
$\mathbf E_I:=-\nabla\Phi_I$ and $\mathbf j=-\kappa\,\mathbf E_I$ (Axiom D2; \eqref{eq:static-closure});
\emph{(b) Wave‑supporting closures} when propagation/export are relevant on a source‑free segment.
In that case the channel supplies a quadratic energy density and flux $(u,\mathbf S)$ with a
\emph{Poynting identity}\footnote{Statement in forms: see App.~\ref{app:forms-stokes}, eq.~\eqref{appB:eq:poynting}.}
so exported power through screens is radially constant and far‑field envelopes obey
$\langle|\mathbf E_I|\rangle\propto A^{-1/2}$ in homogeneous media (C4: \S\ref{corollary:info-em}).
\emph{Info–EM} is one such specialization; elastic/metric (gravitational‑wave), acoustic/hydrodynamic,
or plasma closures are equally admissible when they supply $(u,\mathbf S)$. Only D2 is an axiom;
wave closures are invoked \emph{only} in their regimes (C4).

\subsection{Units and calibration (once).}
We use Kz units (see \eqref{eq:appA-kz-map}) so energies and powers are rates. The abstract informational field
is mapped to measured sectors (gravitational, electric, thermal) by \emph{one‑datum} linear calibrations
stated once outside the axioms (App.~\ref{app:calibrations}); no bare constants are carried through the algebra.

\subsection{How to use the framework (recipe).}
\begin{enumerate}\itemsep2pt
  \item Declare the region $U\subset M$ and its boundary $\partial U$ (screen or sector).
  \item Write the \emph{Stokes identity} $d(*j)=\sigma\,\mathrm{vol}$ and, when needed, the \emph{cut form} on $U$
        (Axiom D1; \eqref{eq:stokes-identity}–\eqref{eq:cut-balance}).
  \item Choose a closure suited to the regime: \textbf{static} (D2) for time‑frozen problems (C0–C1:
        \S\ref{corollary:info-gas}, \S\ref{corollary:static-screen}); \textbf{wave‑supporting} closure for
        transport/export on source‑free segments (e.g.\ Info–EM in C4: \S\ref{corollary:info-em}).
  \item Count distinctions on $\partial U$ (bits or nats) and relate them to $j$ (R1; used in C5:
        \S\ref{corollary:time-info} and C6: \S\ref{corollary:noise}).
  \item If comparison with a measured sector is required, fix a single datum to calibrate (App.~\ref{app:calibrations}).
  \item Read off statics from the inverse‑area rule ($\propto 1/A$; C0–C1:
        \S\ref{corollary:info-gas}, \S\ref{corollary:static-screen}), and wave envelopes from power export
        ($\propto 1/\sqrt{A}$ in homogeneous media; C4: \S\ref{corollary:info-em}; with confinement C8:
        \S\ref{corollary:dimensional}).
  \item Apply operational limits: time–information complementarity (C5: \S\ref{corollary:time-info}) and
        the counting‑noise floor (C6: \S\ref{corollary:noise}) when finite windows/areas are relevant.
\end{enumerate}

\subsection{Scope, reductions, and non‑claims.}
The spine is manifold‑theoretic and informational. \emph{Dimensional reductions} (1D/2D confinement)
are handled by replacing $A_n(r)$ with the effective cross‑section/circumference; the resulting static
$r^{-(n-1)}$ and wave $r^{-(n-1)/2}$ laws appear later (C8: \S\ref{corollary:dimensional}). The
\emph{least‑action statement} is a derived equivalence of boundary rate equations (C10:
\S\ref{corollary:least-action}), not an assumption. A \emph{quantum path‑integral template} can be
overlaid as a modelling choice tied to a quadratic local kernel and the same noise floor (C12:
\S\ref{corollary:qpi-template}); it is optional. External drives, delay, and production are treated in
C3 (\S\ref{corollary:external-drives}); calibrated static corrections (attenuation/capacity) are in
C11 (\S\ref{corollary:entropic-corrections}); thermodynamic/Newtonian rate laws are in C9
(\S\ref{corollary:thermo-newton}).

\subsection{Roadmap.}
Section~\ref{axiom:root} states the axioms: continuity (D1: \eqref{eq:stokes-identity}), static closure (D2: \eqref{eq:static-closure}),
REOS (D3: \eqref{eq:reos}), and canonical rates (D4: \eqref{eq:canonical}). The corollary ladder then develops: relational bit‑gas and static tube/screen
laws (C0–C1: \S\ref{corollary:info-gas}, \S\ref{corollary:static-screen}); continuity/Clausius (C2:
\S\ref{corollary:continuity}); external drives with delay/production (C3: \S\ref{corollary:external-drives});
waves and exported power (C4: \S\ref{corollary:info-em}); time–information complementarity (C5:
\S\ref{corollary:time-info}); the counting‑noise floor (C6: \S\ref{corollary:noise}); geometry as a tool
(C7: \S\ref{corollary:geometry}); dimensional flows and confinement (C8: \S\ref{corollary:dimensional});
and optional extensions (C9–C12: \S\ref{corollary:thermo-newton}, \S\ref{corollary:least-action},
\S\ref{corollary:entropic-corrections}, \S\ref{corollary:qpi-template}). Appendices collect forms/Stokes
details (App.~\ref{app:forms-stokes}), $n$‑D screen geometry (App.~\ref{app:nD-screens}), calibrated focusing
(App.~\ref{app:jacobson}), noise kernels and stochastic response (App.~\ref{app:noise-kernel}), the calibration
procedure (App.~\ref{app:calibrations}), the De Donder–Weyl field version (App.~\ref{app:ddw}), and a notation/unit
sheet (App.~\ref{app:notation-units}).
