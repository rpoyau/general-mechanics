% ======================================================================
% 01_introduction.tex — Introduction
% ======================================================================

\section{Introduction}
\label{introduction}

\subsection{Aim and stance}
This work takes \emph{General Mechanics} to mean: familiar mechanics appear as \emph{relational motifs}
read from a single geometric identity (the generalized Stokes identity) on a chosen \emph{informational} manifold.
Relations are primary; observables are relational motifs. An observer specifies a region \(U\subset M\) with
boundary \(\partial U\); motifs are read from the boundary and its \emph{frame(s)}.

\paragraph{Backbone (identity and open-set clause).}
Generalized Stokes’ theorem together with its \emph{cut identity} (cap\,+\,side) and production density \(\sigma\)
is the bookkeeping spine. Tube/boundary invariants are read only on an \emph{open set} \(W\subset U\) with
\(\sigma|_W\equiv 0\) (source‑free); no global premise is assumed.

\paragraph{Terminology.}
A \emph{frame} is a time‑frozen spatial slice at label \(t_0\).
A \emph{window} is a finite time interval \([t_0,t_1]\).
A \emph{worldtube} is the product region \(U\times[t_0,t_1]\) whose boundary decomposes into a top cap, a bottom cap,
and a side; for moving boundaries the side integrand is \(\mathbf j\!\cdot\!\mathbf n - i\,v_n\).

Measurements are phrased in information theory (Shannon measures with Bayesian updating) over a ternary evidence
alphabet: \emph{yes}, \emph{no}, and \emph{unknown}.

\paragraph{Boundary identification (information \(\equiv\) entropy; units).}
We identify \(j\) with the \emph{entropy flux} on the boundary (\(j\equiv j_S\)); its normal component \(j_n\) has units
\([\text{count}]/(\text{area}\cdot\text{time})\), with the log base fixing the count unit (nats or trits;
\(\text{nats}=\ln 3\times\text{trits}\)). The corresponding density is \(i\equiv s\) (nats/vol).
A heat mapping is optional: on a boundary, \(q_n = T\,j_n\) and over a window \(Q=\int T\,j_n\) (introduced formally in C2).

\subsection{One geometric identity, two uses}
With \(j:=J^\mu dx_\mu\) the informational/entropic \(1\)-form and \(*j\) its Hodge dual, the Stokes identity is
\(d(*j)=\sigma\,\mathrm{vol}\) (continuity).\footnote{Local and cut forms:
\eqref{axioms:stokes:identity:eq:local-continuity} and \eqref{axioms:stokes:identity:eq:cut}.}
It is used in two complementary ways:
\begin{enumerate}\itemsep2pt
  \item on a \emph{frame} to read boundary/tube relations and sector amplitudes
        (\S\ref{corollary:info-gas}, \S\ref{corollary:frame-boundary}); and
  \item on a \emph{worldtube} \(U\times[t_0,t_1]\) to record change across a time window
        (\S\ref{corollary:external-drives}).
\end{enumerate}

\paragraph{Caps and side (worldtube).}
The time‑extended boundary \(\partial(U\times[t_0,t_1])\) decomposes into a top cap, a bottom cap, and a side.
The cap–side statement is recorded in App.~\ref{app:forms-stokes}, Eq.~\eqref{app:forms-stokes:eq:cap-side};
the side integrand for moving boundaries is \(\mathbf j\!\cdot\!\mathbf n - i\,v_n\).

\bigskip
\noindent\fbox{\begin{minipage}{0.97\linewidth}
\textbf{Cone/Tube principle (frame).}
On an open set \(W\subset U\) with \(\sigma|_W\equiv 0\) (source‑free), and with the side of a narrow tube everywhere
tangent to \(\mathbf j\), the directed tube flux
\[
\Phi=\int_{\Sigma} \mathbf j\!\cdot n\,dA
\]
is independent of the cross‑section \(\Sigma\subset W\).
For any boundary (or conical sector) of area \(A(r)\) at radius \(r\),
\[
\big\langle j_n\big\rangle(r)=\frac{\Phi}{A(r)}.
\]
With the \emph{frame linear closure} (\S\ref{axioms:stokes:frame}, eq.~\eqref{axioms:stokes:frame:eq:closure}),
\(\mathbf E_I:=-\nabla\Phi_I\) and \(\mathbf j=-\kappa\,\mathbf E_I\), the informational field amplitude reads
\[
\big|\mathbf E_I(r)\big|=\frac{1}{\kappa}\,\frac{\Phi}{A(r)}.
\]
\emph{Conservation is a boundary statement:} the flux \(\Phi\) is constant; amplitudes scale as \(1/A\).
\end{minipage}}

\subsection{Closures (use and scope)}
Two minimalist ingredients are employed when appropriate:
\emph{(a) frame linear closure} for near‑equilibrium (\S\ref{axioms:stokes:frame};
eq.~\eqref{axioms:stokes:frame:eq:closure}); and
\emph{(b) wave‑supporting closures} when transport/export is relevant on a source‑free segment.
\emph{Wave transport and envelopes—see \S\ref{waves}.}
On windows that are linear, time‑invariant, and causal, we also use the \emph{Causality–Dispersion}
(KK‑window) corollary as a local comparison witness (§\ref{waves:causality-dispersion}).

\subsection{Units and calibration (once)}
Kz units are used (App.~\ref{app:kick}) so energies and powers are rates. The abstract informational field is mapped
to measured sectors (gravitational, electric, thermal) by a \emph{one‑datum} linear calibration (App.~\ref{app:calibrations});
no bare constants are carried through the algebra.

\subsection{How to use the framework (recipe)}
\begin{enumerate}\itemsep2pt
  \item Declare \(U\subset M\) and \(\partial U\) (boundary or sector).
  \item Write the Stokes identity \(d(*j)=\sigma\,\mathrm{vol}\) and, when needed, the \emph{cut identity} on \(U\)
        (\eqref{axioms:stokes:identity:eq:local-continuity}–\eqref{axioms:stokes:identity:eq:cut}).
  \item Choose a closure suited to the regime: \textbf{frame} (\S\ref{axioms:stokes:frame}) for frame problems
        (\S\ref{corollary:info-gas}–\S\ref{corollary:frame-boundary}); or a \textbf{wave‑supporting} closure for export on
        source‑free segments (see §\ref{waves}). If the window is LTI and causal, the Causality–Dispersion corollary may be
        applied locally (§\ref{waves:causality-dispersion}).
  \item Count distinctions on \(\partial U\) in a ternary alphabet (\emph{yes/no/unknown}) using information‑theoretic
        bookkeeping, and relate them to \(j\) (\S\ref{axioms:relational}; used in \S\ref{corollary:frame-boundary}–\S\ref{corollary:noise}).
  \item If comparison with a measured sector is required, fix a single datum to calibrate (App.~\ref{app:calibrations}).
  \item Read statics from the inverse‑area rule (\(\propto 1/A\); \S\ref{corollary:info-gas}–\S\ref{corollary:frame-boundary});
        for wave export and far‑field envelopes, follow \S\ref{waves} (with confinement summarized in \S\ref{corollary:dimensional}).
  \item Apply operational limits: time–information complementarity (\S\ref{corollary:time-info}) and the counting‑noise floor
        (\S\ref{corollary:noise}) for finite windows/areas.
\end{enumerate}

\subsection{Scope and reductions}
Dimensional reductions (1D/2D confinement) are handled by replacing \(A_n(r)\) with an effective
cross‑section/circumference; the resulting static and wave summaries appear in \S\ref{corollary:dimensional}.
Detailed wave envelopes and exported‑power statements are collected in \S\ref{waves}.

\medskip
\noindent\emph{Literature note.}
Differential forms/Stokes background: Frankel~\cite{Frankel2011}. Poynting/export in EM: Jackson~\cite{Jackson1999},
Landau--Lifshitz--Pitaevskii~\cite{LandauLifshitzEDCM1984}. Information/capacity: Shannon~\cite{Shannon1948},
Gallager~\cite{Gallager1968}. Erasure cost: Landauer~\cite{Landauer1961}, Bennett~\cite{Bennett2003}.
