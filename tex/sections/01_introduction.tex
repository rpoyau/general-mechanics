% ======================================================================
% 01_introduction.tex — Introduction
% ======================================================================

\section{Introduction}
\label{introduction}

\subsection{Aim and stance.}
This work takes \emph{General Mechanics} to mean: dynamics read from a single \emph{geometric identity}
(the Stokes identity) applied to an \emph{informational current} on a chosen region of a manifold.
Relations are primary; observables are relational. An observer specifies a region $U\subset M$ with
boundary $\partial U$; statements below are read from the boundary and its time extension.

Measurements are framed in information theory (Shannon measures with Bayesian updating) over a
ternary evidence alphabet: \emph{yes}, \emph{no}, and \emph{yesno} (unknown).

\subsection{One geometric identity, two uses.}
With $j:=J^\mu dx_\mu$ the informational/entropic current $1$-form and $*j$ its Hodge dual, the
Stokes identity is $d(*j)=\sigma\,\mathrm{vol}$ (continuity).\footnote{Local and cut forms appear
in \eqref{eq:stokes-identity}–\eqref{eq:cut-balance} once the axioms are stated.}
It is used in two ways:

(i) on a \emph{time‑frozen} spatial region (static screens/sectors; C0–C1:
\S\ref{corollary:info-gas}, \S\ref{corollary:static-screen}); and

(ii) on a \emph{product region}
$U\times[t_0,t_1]$ to discuss change over a window (C3: \S\ref{corollary:external-drives}).

\bigskip
\noindent\fbox{\begin{minipage}{0.97\linewidth}
\textbf{Cone/Tube principle (time frozen).}
On an open set $W\subset U$ with production density $\sigma|_W\equiv 0$ (source‑free), and with the side
of a narrow tube everywhere tangent to $\mathbf j$, the directed tube flux
$\displaystyle \Phi=\int_{\Sigma} \mathbf j\!\cdot n\,dA$ is independent of the cross‑section $\Sigma\subset W$.
For any screen or conical sector of area $A(r)$ at radius $r$,
\[
\big\langle j_n\big\rangle(r)=\frac{\Phi}{A(r)}.
\]
With the snapshot closure $\mathbf E_I:=-\nabla\Phi_I$ and $\mathbf j=-\kappa\,\mathbf E_I$
(see \eqref{eq:static-closure}), the informational field amplitude is read as
\[
\big|\mathbf E_I(r)\big|=\frac{1}{\kappa}\,\frac{\Phi}{A(r)}.
\]
\emph{Conservation is a surface statement:} the flux $\Phi$ is constant; amplitudes scale as $1/A$.
\end{minipage}}

\subsection{Closures (use and scope).}
Two minimalist ingredients are employed when appropriate:
\emph{(a) Static linear closure} for near‑equilibrium, time‑frozen statics,
$\mathbf E_I:=-\nabla\Phi_I$ and $\mathbf j=-\kappa\,\mathbf E_I$ (D2; \eqref{eq:static-closure});
\emph{(b) Wave‑supporting closures} when transport/export is relevant on a source‑free segment.
In $n$ spatial dimensions, a channel with quadratic energy density and flux $(u,\mathbf S)$ satisfying a
Poynting identity yields radius‑independent exported power across spherical screens $S^{n-1}(r)$ and
far‑field envelopes
\[
\langle|\mathbf E_I|\rangle \propto [A_n(r)]^{-1/2},\qquad A_n(r)\propto r^{\,n-1}
\]
(C4: \S\ref{corollary:info-em}; App.~\ref{app:nD-screens}).
Constrained geometries replace $A_n(r)$ by an effective cross‑section $A_\Delta(r)=\Delta\Omega\,r^{\,n-1}$
(C8: \S\ref{corollary:dimensional}). Special cases: $n{=}1$ ($A_1\!\equiv\!2$; no geometric decay),
$n{=}2$ ($A_2=2\pi r$; envelope $\propto r^{-1/2}$), $n{=}3$ ($A_3=4\pi r^2$; envelope $\propto r^{-1}$).
For moving screens use the worldtube (cap–side) form with side integrand $\mathbf j\!\cdot\!\mathbf n - i\,v_n$
(App.~\ref{app:forms-stokes}).

\subsection{Units and calibration (once).}
Kz units are used (App.~\ref{app:kick}) so energies and powers are rates. The abstract field is mapped to
measured sectors (gravitational, electric, thermal) by a \emph{one‑datum} linear calibration stated once
(App.~\ref{app:calibrations}); no bare constants are carried through the algebra.

\subsection{How to use the framework (recipe).}
\begin{enumerate}\itemsep2pt
  \item Declare $U\subset M$ and $\partial U$ (screen or sector).
  \item Write the Stokes identity $d(*j)=\sigma\,\mathrm{vol}$ and, when needed, the cut identity on $U$
        (D1; \eqref{eq:stokes-identity}–\eqref{eq:cut-balance}).
  \item Choose a closure suited to the regime: \textbf{static} (D2) for time‑frozen problems (C0–C1); or a
        \textbf{wave‑supporting} closure for export on source‑free segments (e.g., Info–EM in C4).
  \item Count distinctions on $\partial U$ in a ternary alphabet (\emph{yes/no/yesno}) using information‑theoretic
        accounting, and relate them to $j$ (R1; used in C5–C6).
  \item If comparison with a measured sector is required, fix a single datum to calibrate (App.~\ref{app:calibrations}).
  \item Read statics from the inverse‑area rule ($\propto 1/A$; C0–C1), and wave envelopes from power export
        ($\propto 1/\sqrt{A}$ in homogeneous media; C4; with confinement C8).
  \item Apply operational limits: time–information complementarity (C5) and the counting‑noise floor (C6) for
        finite windows/areas.
\end{enumerate}

\subsection{Scope and reductions.}
Dimensional reductions (1D/2D confinement) are handled by replacing $A_n(r)$ with an effective cross‑section/circumference;
the resulting static $r^{-(n-1)}$ and wave $r^{-(n-1)/2}$ laws appear in C8. The least‑action statement is a derived
equivalence of boundary rate equations (C10). A quantum path‑integral template can be overlaid as a modelling choice
tied to a quadratic local kernel and the same noise floor (C12).

\subsection{Roadmap.}
Section~\ref{axiom:root} states the axioms: continuity (D1: \eqref{eq:stokes-identity}),
static closure (D2: \eqref{eq:static-closure}), REOS (D3: \eqref{eq:reos}),
and canonical rates (D4: \eqref{eq:canonical}). The ladder then develops: relational bit‑gas and static tube/screen
laws (C0–C1: \S\ref{corollary:info-gas}, \S\ref{corollary:static-screen}); continuity/Clausius (C2:
\S\ref{corollary:continuity}); external drives with delay/production (C3: \S\ref{corollary:external-drives});
waves and exported power (C4: \S\ref{corollary:info-em}); time–information complementarity (C5:
\S\ref{corollary:time-info}); the counting‑noise floor (C6: \S\ref{corollary:noise}); geometry as a tool
(C7: \S\ref{corollary:geometry}); dimensional flows and confinement (C8: \S\ref{corollary:dimensional});
and optional extensions (C9–C12: \S\ref{corollary:thermo-newton}, \S\ref{corollary:least-action},
\S\ref{corollary:entropic-corrections}, \S\ref{corollary:qpi-template}). Appendices collect forms/Stokes
details (App.~\ref{app:forms-stokes}), $n$‑D screen geometry (App.~\ref{app:nD-screens}), calibrated focusing
(App.~\ref{app:jacobson}), noise kernels and stochastic response (App.~\ref{app:noise-kernel}), the calibration
procedure (App.~\ref{app:calibrations}), the De Donder–Weyl field version (App.~\ref{app:ddw}), and a notation/unit
sheet (App.~\ref{app:notation-units}).

\medskip
\noindent\emph{Literature note.}
Differential forms/Stokes background: Frankel~\cite{Frankel2011}. Poynting/export in EM: Jackson~\cite{Jackson1999},
Landau--Lifshitz--Pitaevskii~\cite{LandauLifshitzEDCM1984}. Information/capacity: Shannon~\cite{Shannon1948},
Gallager~\cite{Gallager1968}. Erasure cost: Landauer~\cite{Landauer1961}, Bennett~\cite{Bennett2003}.
