% sections/18_falsify.tex  —  Falsification ladder for the REOS axiom

\section{Falsification Ladder}
\label{sec:falsify}

\noindent
Because the note is deliberately “claims‐light,” its scientific value rests
on how easily the REOS axiom could be \emph{empirically} contradicted.
Table~\ref{tab:falsify} lists a spectrum of potential tests, ordered
from least to most challenging.  Failure at \emph{any} rung falsifies the
axiom.

\begin{table}[ht]
\centering
\caption{Falsification ladder --- each rung contradicts a specific boxed
         REOS corollary.}
\label{tab:falsify}
\begin{tabular}{cll}
\toprule
Level & Experimental scenario & Contradicted equation \\
\midrule
F\,1  & Bit‐erasure with work $W < T\Delta H\ln 2$ (quasistatic) &
       Landauer bound (Sec.~\ref{sec:landauer}) \\[2pt]
F\,2  & Energy–time product $\Delta E\,\Delta t < 1/(2\ln2)$
       in a Ramsey‐type interference & URC inequality
       (Appendix~\ref{app:covariant}) \\[2pt]
F\,3  & Information flux through a wave-guide exceeding
       capacity $J^0 > C(R)$ for known power budget &
       Tier-1 capacity ceiling (Sec.~\ref{sec:tier1}) \\[2pt]
F\,4  & Spatial confinement of $\rho$ inside a
       mobility barrier (Sec.~\ref{sec:confine1d})
       fails to match simulated force &
       Informational force law (Sec.~\ref{sec:info_force}) \\[2pt]
F\,5  & Observation of $\partial_\mu J^\mu \neq \sigma$
       in a closed, measurement-free region &
       Continuity identity C\,0 (Sec.~\ref{sec:continuity}) \\
\bottomrule
\end{tabular}
\end{table}

\paragraph{Notes on feasibility.}
\begin{itemize}
\item \textbf{F\,1} is routinely satisfied by state-of-the-art
      single-electron boxes within experimental error; any verified
      violation would be headline news.
\item \textbf{F\,2} approaches current atomic-clock precision limits
      ($10^{-4}$ below the bound).  A dedicated Ramsey or
      spin-echo protocol could close that gap.
\item \textbf{F\,3} applies to optical fibres, microwave wave-guides, or
      biological ion channels—whichever offers the cleanest resource
      audit (power in ↔ bit-rate out).
\item \textbf{F\,4} can be tested on a nanophotonic lattice with spatially
      patterned refractive index; Sec.~\ref{sec:confine1d}
      details the simulation recipe.
\item \textbf{F\,5} is the “nuclear option”: every known measurement‐free
      system respects local continuity, so falsifying C\,0 would upend
      not just the REOS axiom but standard probability theory.
\end{itemize}

\paragraph{Recommended strategy.}
Start at F\,1 and climb.  Each successive rung demands either tighter
metrology (F\,2), stricter resource bookkeeping (F\,3), or more refined
control over inhomogeneous mobilities (F\,4).  The ladder thus
provides a clear experimental roadmap for evaluating the relational
model.

