% ======================================================================
% 00_title.tex — Title + Abstract
% ======================================================================

\title{General Mechanics}
\author{Reginald Poyau}
\date{\today}

\maketitle

\labtitle

% ---------- Epigraph (kept) ----------
\vspace{1.5em}
\begin{center}\small\itshape
God does not play dice, but models must.\\[2pt]
God makes laws; mortals create models.
\end{center}
% -------------------------------------

\begin{abstract}
We start from a single geometric statement on a manifold: a generalized Stokes identity for an informational current. Extending the region with a time dimension turns it into a dynamic information channel, where boundary integrals equal interior integrals. From this informational viewpoint, familiar motifs of mechanics follow as necessities of the identity: continuity and conservation laws, inverse‑area scaling on screens in statics, canonical rate equations from a boundary variation, and wave‑like propagation under an electrodynamics‑style closure. The same structure yields operational limits—time–information complementarity, finite channel capacity, and erasure cost—without introducing bare constants. A constant‑free, one‑datum calibration then maps the informational field to measured sectors (e.g., gravitational, electric, thermal), keeping the spine purely manifold‑theoretic while making the framework predictive.
\end{abstract}
