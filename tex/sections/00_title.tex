% sections/title.tex  —  Title page + abstract  (≈250 words)

\title{\bfseries
  Relational Information Equation of State (REOS)\\[2pt]}

\author{Reginald Poyau}
\date{\today}

\maketitle

% ---------- Epigraph ----------
\vspace{1.5em}
\begin{center}\small\itshape
God does not play dice, but models must.\\[2pt]
God makes laws; mortals create models.
\end{center}
% -------------------------------


\begin{abstract}
\noindent
\textbf{Scope.}  
We introduce the \emph{Relational Information Model} (RIM) and state a single
\emph{Relational Equation-of-State} (REOS) axiom that links local Shannon entropy
density~$\rho$ (bits) to a vector of conserved resources~$R^{(a)}$.
From this axiom and the generalised Stokes theorem we collect, without further
assumptions, a chain of corollaries: (i)~a continuity equation for an
informational $n$-form~$\mathcal J$, (ii)~Landauer-type cost bounds,
(iii)~a least-action / Hamiltonian formulation, (iv)~an informational force
law, and (v)~a universal resource--clock uncertainty principle
$\Delta R^{(a)}\,\Delta\Theta_{(a)}\gtrsim 1/(2\ln 2)$.  

\textbf{Specialisation.}  
Choosing the vacuum world-tube as the unique physical channel and expressing
all units in either kicks ($c=\hbar=k_B=4\pi\varepsilon_0=1$) or explicit SI
constants, we derive an energy--time throughput ceiling that reproduces
classical results (Bekenstein, Landauer) as algebraic identities.  No claims
are made beyond these direct consequences of the REOS axiom.

\textbf{Organisation.}  
Section~\ref{sec:motivation} motivates the relational stance;  
Section~\ref{sec:kick} fixes units;  
Section~\ref{sec:axiom} states the axiom;  
Sections~\ref{sec:tier0}--\ref{sec:tier1} list foundational and anchor
corollaries;  
Section~\ref{sec:confine1d} gives a $1+1$ confinement demo;  
Section~\ref{sec:falsify} outlines empirical falsification ladders;  
appendices gather unit tables, covariant extensions and code.

The aim is a compact ledger: all derivations are algebraic, all references
point to established literature, and every statement is traceable to the
single REOS axiom or to standard definitions of Shannon entropy.
\end{abstract}
