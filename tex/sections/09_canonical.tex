% sections/09_canonical.tex  —  Canonical resource lists & shorthand forms
% Plan ref: §9  • Tbl 9-1 canonical cases

\section{Canonical Resource Lists}
\label{sec:canonical}

\noindent
Table \ref{tab:canonical} compiles four resource sets that recur
throughout the note.  For each set we show the shorthand symbol used in
later equations and note which boxed corollaries first invoke it.

\begin{table}[!htbp]
\centering
\caption{Canonical resource lists. “First use” refers to boxed results in Tier-0 or Tier-1 sections.}
\label{tab:canonical}
\begin{tabular}{lclp{5cm}}
\toprule
Label & Resources \(R^{(a)}\) & Shorthand & First use \\
\midrule
\textbf{Thermo-pair} & \(E,\,S_{\text{phys}}\) & \(\{E,S\}\) & Landauer bound (\S\ref{sec:landauer}) \\[2pt]
\textbf{Vacuum 4-mom.} & \(P^{\mu}\) & \(P^{\mu}\) & Energy--time capacity (\S\ref{sec:tier1_maxwell}) \\[2pt]
\textbf{Entanglement} & \(E_{\mathrm{PR}},\,E\) & \(\{E_{\mathrm{PR}},E\}\) & Path--integral UV cut-off (\S\ref{sec:tier1_pathint}) \\
\bottomrule
\end{tabular}
\end{table}


\paragraph{Why canonical?}
Each list closes under the REOS differential
\(d\rho=-\Lambda_{a}dR^{(a)}\) with \emph{no} extra multipliers.
For instance, adding photons to the thermo-pair would introduce
chemical potential terms that complicate Tier-0 proofs; the minimalist
lists keep the algebra transparent.

\paragraph{Usage note.}
When a later section quotes “canonical vacuum list,” it means the
\(\{P^{\mu}\}\) entry above; likewise “thermo-pair” implies
\(\{E,S\}\).  This single‐table convention avoids repeating resource
definitions.
