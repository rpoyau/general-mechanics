% ======================================================================
% sections/08_geometry-response.tex — Metric response (info–Einstein) v1.0.2
% ======================================================================

\section{Geometric response: info–Einstein dynamics}
\label{sec:geometry-response}

\paragraph{Postulate (metric field equation).}
Spacetime responds locally and covariantly to informational sources via
\begin{equation}
G_{\mu\nu} + \Lambda\, g_{\mu\nu} \;=\; \chi_M\, T_{\mu\nu},
\label{eq:info-einstein}
\end{equation}
where \(G_{\mu\nu}\) is the Einstein tensor, \(\Lambda\) a cosmological constant (optional), and \(T_{\mu\nu}\) is the stress tensor expressed in Kick (Kz) units (energy per \(h\)). The constant \(\chi_M\) is a \emph{single calibration} that ties the informational scale to the metric response.

\paragraph{Conservation and consistency.}
Bianchi identities give \(\nabla^\mu G_{\mu\nu}=0\), hence from \eqref{eq:info-einstein} we obtain
\(\nabla^\mu T_{\mu\nu}=0\) provided \(\chi_M\) and \(\Lambda\) are constants. We assume the \emph{mostly‑plus} signature and units with all energy‑like quantities in Kz (Sec.~\ref{sec:conventions}).

\paragraph{Sources in Kz.}
For a perfect fluid (plus possible shear),
\begin{equation}
T^{\mu\nu} = (\rho + p/c^2)\,u^\mu u^\nu + p\,g^{\mu\nu} + \Pi^{\mu\nu},
\label{eq:stress-perfect}
\end{equation}
where \(\rho\) is total energy density in Kz (rest \(+\) kinetic \(+\) thermal), \(p\) is pressure in Kz, \(u^\mu\) the $4$‑velocity, and \(\Pi^{\mu\nu}\) traceless shear. In the low‑pressure, slow‑motion regime, \(\rho\simeq\rho_{\rm rest}\) dominates; in hot/dense media, \(p\) and shear contribute non‑negligibly.

\paragraph{Weak‑field / quasi‑Newtonian limit.}
Linearize about Minkowski: \(g_{\mu\nu}=\eta_{\mu\nu}+h_{\mu\nu}\), \(|h_{\mu\nu}|\ll 1\).
In harmonic gauge, the trace‑reversed perturbation \(\bar h_{\mu\nu}\) satisfies
\begin{equation}
\square\,\bar h_{\mu\nu} \;=\; -\,2\,\chi_M\,T_{\mu\nu},
\qquad
\square := -\frac{1}{c^2}\,\partial_t^2 + \nabla^2.
\label{eq:lin-einstein}
\end{equation}
For static, slow sources (\(\partial_t \to 0\), \(T_{0i}\approx 0\)), the time–time component gives a Poisson‑type law for the Newtonian potential \(\psi\) defined by \(g_{00}=-(1+2\psi/c^2)\):
\begin{equation}
\nabla^2 \psi \;=\; \kappa_N \,\rho_{\rm eff},
\quad
\rho_{\rm eff} \simeq \rho + \zeta\,\frac{3p}{c^2},
\quad
\kappa_N := \frac{c^2}{2}\,\chi_M,
\label{eq:poisson-effective}
\end{equation}
with \(\zeta\simeq 1\) capturing the familiar pressure correction in hot/dense regimes.%
\footnote{In standard GR normalization \(\chi_M=8\pi G/c^4\), \eqref{eq:poisson-effective} reduces to \(\nabla^2\psi=4\pi G\,(\rho_{\rm SI}+3p_{\rm SI}/c^2)\). Here \(\rho,p\) are Kz versions (SI divided by \(h\)); we keep \(\kappa_N\) free as a \emph{calibration} fixed by one datum.}
Thus in cool, non‑relativistic systems \(\nabla^2\psi \approx \kappa_N\,\rho\).

\paragraph{Calibration to the informational field.}
The measurable acceleration is
\begin{equation}
\mathbf g = -\,\nabla \psi = c_M\,\mathbf F_I,
\label{eq:g-calibration}
\end{equation}
where \(\mathbf F_I:=-\nabla\Phi_I\) is the informational field (Secs.~\ref{sec:static-gauss-nd}, \ref{sec:worldtube-IED}).
One may fix the pair \((\kappa_N,c_M)\) by matching a single reference system (e.g.\ a spherical screen at \((M_*,r_*)\) from Sec.~\ref{sec:entropic-gravity-screen}), after which all other cases follow by linearity and symmetry.

\paragraph{Interior/exterior consistency.}
For a static sphere of radius \(R\) with uniform \(\rho\) and negligible \(p\),
\eqref{eq:poisson-effective} gives
\[
\psi(r)=
\begin{cases}
-\dfrac{\kappa_N \rho}{6}\,\bigl(3R^2 - r^2\bigr), & r<R,\\[0.7ex]
-\dfrac{\kappa_N \rho\,R^3}{3r}, & r\ge R,
\end{cases}
\quad
\mathbf g(r)=-\nabla\psi=
\begin{cases}
-\dfrac{\kappa_N \rho}{3}\,r\,\hat{\mathbf r}, & r<R,\\[0.7ex]
-\dfrac{\kappa_N \rho\,R^3}{3r^2}\,\hat{\mathbf r}, & r\ge R,
\end{cases}
\]
matching the linear‑inside / inverse‑square‑outside pattern of \S\ref{sec:static-gauss-nd} and the entropic screen scaling of \S\ref{sec:entropic-gravity-screen} after calibration.

\paragraph{Stochastic extension (Einstein–Langevin).}
Finite counting implies fluctuations of the source around its mean:
\begin{equation}
T_{\mu\nu} = \langle T_{\mu\nu}\rangle + \tau_{\mu\nu},
\qquad
\langle \tau_{\mu\nu}\rangle=0.
\label{eq:split-T}
\end{equation}
Linearizing \eqref{eq:info-einstein} about a background \(g\) yields the stochastic equation
\begin{equation}
\mathcal E[h]_{\mu\nu} \;=\; \chi_M\,\tau_{\mu\nu},
\label{eq:einstein-langevin}
\end{equation}
where \(\mathcal E\) is the linearized Einstein operator. The \emph{noise kernel}
\begin{equation}
N_{\mu\nu\alpha\beta}(x,x') \;:=\; \tfrac12\,\big\langle \{\tau_{\mu\nu}(x),\,\tau_{\alpha\beta}(x')\} \big\rangle
\label{eq:noise-kernel}
\end{equation}
determines the two‑point variance of \(h_{\mu\nu}\) via the retarded Green function of \(\mathcal E\).
In weak fields this produces tiny, principle‑limited metric fluctuations; in $3$D static problems they manifest as the \(1/\sqrt{N}\) acceleration noise floor anticipated in \S\ref{sec:entropic-gravity-screen}.
Details and examples appear in App.~\ref{app:noise-kernel}.

\paragraph{Remarks.}
(1) Equation~\eqref{eq:info-einstein} is the unique local, covariant, divergence‑free choice built from \(g_{\mu\nu}\) and its first two derivatives; all non‑universality is captured by the single calibration \(\chi_M\).  
(2) The weak‑field mapping \eqref{eq:poisson-effective} is \emph{calibration‑first}: we avoid hard‑coding SI constants; one datum fixes \(\kappa_N\).  
(3) The informational route (\(\mathbf F_I\to \mathbf g\)) and the geometric route (\(T_{\mu\nu}\to g_{\mu\nu}\)) agree after calibration and produce identical static profiles; dynamics add causal propagation and small stochastic corrections.
